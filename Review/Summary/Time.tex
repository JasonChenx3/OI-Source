\section{时间轴有关问题}
\subsection{静态问题}
\begin{itemize}
    \item 对固定区间的结果有影响:将存在转化为扫描线插入+删除,变为动态问题。
\end{itemize}
\subsection{动态问题}
在询问中带有修改操作,修改对后续的询问有影响,修改之间对询问的贡献独立。

\begin{itemize}
    \item 仅插入:
    \begin{itemize}
        \item 离线:对于后半部分的询问,前半部分的所有修改均对它们有影响。使用$\lg$的
        代价进行CDQ分治。
        \item 强制在线:二进制分组
    \end{itemize}
    \item 离线+插入+删除
    \begin{itemize}
        \item 将其转化为在固定区间内存在,变为静态问题,使用时间线段树解决。
        \item 对于后半部分的询问,前半部分在该询问前未被删除的元素有贡献。考虑
        从右到左扫描后半部分的询问,使用时光倒流将前半部分的删除转化为插入。再
        套用仅插入的做法,以$\lg^2$的代价除去动态插入删除操作。
    \end{itemize}
\end{itemize}
\subsection{时间分治技巧}
\begin{itemize}
    \item 分治前预排序,分治到子区间时线性划分
    \item 分治后尽可能线性合并子区间的信息
    \item 分治计算一边对另一边的贡献时,考虑不建立数据结构,使用双指针法线性计算
\end{itemize}

该内容参考了国家集训队2013论文集许浩然的《浅谈数据结构题的几个非经典解法》。
