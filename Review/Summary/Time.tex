\section{时间轴有关问题}
\subsection{静态问题}
\begin{itemize}
    \item 对固定区间的结果有影响:将存在转化为扫描线插入+删除,变为动态问题。
    \item 可持久化操作:所有版本形成了一棵树,可以使用DFS序将操作展开为序列。
\end{itemize}
\subsection{动态问题}
在询问中带有修改操作,修改对后续的询问有影响,修改之间对询问的贡献独立。

\begin{itemize}
    \item 仅插入:
    \begin{itemize}
        \item 离线:对于后半部分的询问,前半部分的所有修改均对它们有影响。使用$\lg$的
        代价进行CDQ分治。
        \item 强制在线:二进制分组
    \end{itemize}
    \item 离线+插入+删除
    \begin{itemize}
        \item 将其转化为在固定区间内存在,变为静态问题,使用时间线段树解决。
        \item 对于后半部分的询问,前半部分在该询问前未被删除的元素有贡献。考虑
        从右到左扫描后半部分的询问,使用时光倒流将前半部分的删除转化为插入。再
        套用仅插入的做法,以$\lg^2$的代价除去动态插入删除操作。
    \end{itemize}
    \item 操作将区间内的数变为同一个数

    下面介绍将此类操作改为添加与撤销操作的方法。

    建立一棵以位置为下标的线段树,将$[l,r]$的元素修改时,向$[l,r]$处插入时间+新元素
    标记,若当前节点已有标记则下推标记并清除,当递归到整层时取出子树内的所有标记并清除。

    记当前时间为$j$,对于取出的时间为$i$的标记,将其转化为两个操作:
    \begin{itemize}
        \item 在时间$i$处执行覆盖操作
        \item 在时间$j$处撤销操作
    \end{itemize}

    构建时间复杂度$O(m\lg^2 n)$,实测时间复杂度接近$O(m\lg n)$。

    有了撤销操作后,覆盖操作可以转化为区间加减与区间和查询解决。
\end{itemize}
\subsection{时间分治技巧}
\begin{itemize}
    \item 分治前预排序,分治到子区间时线性划分
    \item 分治后尽可能线性合并子区间的信息
    \item 分治计算一边对另一边的贡献时,考虑不建立数据结构,使用双指针法线性计算
\end{itemize}

该内容参考了国家集训队2013论文集许浩然的《浅谈数据结构题的几个非经典解法》与
国家集训队2014论文集中徐寅展的《线段树在一类分治问题上的应用》。
