\section{树上连通块问题}
下文参考了IOI2018中国国家候选队论文集任轩笛的《解决树上连通块问题的一些技巧和工具》。

\subsection{无色连通块}
\subsubsection{树形DP}
以$dp_u$表示点$u$的子树内以$u$为根的连通块的答案。

时间复杂度取决于合并儿子的复杂度。
\begin{itemize}
    \item 若合并复杂度为$O(1)$,时间复杂度为$O(n)$
    \item 若合并复杂度为$O(siz_a*siz_b)$,时间复杂度为$O(n^2)$
\end{itemize}

\subsubsection{DFS序单点合并}
有时子树合并的复杂度过高,而单点合并较为廉价。

假定连通块必须包含根节点,以根节点生成DFS序,那么一个连通块就是DFS序列上
去掉若干段。

记$dp_i$为考虑DFS序上第$i$个位置以前的答案。考虑是否选择位置$i$,若选择则从$dp_{i}$
转移至$dp_{i+1}$,否则转移至$dp_{R[i]+1}$,表示跳过整个子树,其中$R[i]$表示以$i$为
根的子树区间右端点。

以背包问题为例,若使用树形DP则需要$O(nm^2)$的复杂度,其中$m$为背包大小。
然而DFS序单点合并添加物品只需$O(m)$,时间复杂度$O(nm)$。

一般题目不要求连通块必须包含根节点,那么可以考虑点分治,对每个重心管辖的连通块做一遍DP,
时间复杂度$O(nm\lg n)$。
\subsubsection{点数-边数}
利用树的``点数-边数=1''的性质,枚举计算包含每个点/每条边的方案数,然后将点的答案减去
边的答案,可以使得所有合法连通块都恰好被计数一次。

这个技巧常用于求若干个连通块的交的场合。
\subsubsection{线段树合并DP}
若DP数组大小不超过子树大小(比如颜色),且合并时仅仅简单地将对应位合并(位与位独立),
可以使用线段树合并进行DP转移,时间复杂度$O(n\lg n)$。
\subsubsection{链分治动态DP}
\begin{itemize}
    \item 树链剖分+矩阵乘法+线段树
    \item 全局平衡二叉树
    \item LCT
\end{itemize}
\subsection{黑白连通块}
给黑白颜色各开一个有根LCT维护连通性,然后若该节点为黑色则其在黑色LCT中向父亲(给根节点
一个虚父亲)连边,白色亦然。查询连通块时,在所处LCT中找到根,根是连通块最浅节点的异色父亲
,连通块的信息存储在splay链头右儿子中(注意不能是整链信息-链头信息,考虑白连通块-黑父亲-
白连通块的情况)。

\subsubsection{多色连通块处理}
给每种颜色建一个数据结构维护,以自身颜色为黑,其它颜色为白。时间复杂度多一个颜色数的因子。

\index{*TODO!Link Cut Memphis}
