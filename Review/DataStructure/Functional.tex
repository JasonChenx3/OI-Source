\section{可持久化数据结构}
可持久化数据结构的核心思想是{\bfseries Copy On Write}(写时复制),当一个对象将
被改变时,简单地复制其整体,未修改的部分仍引用原对象的数据,达到节省拷贝时间与
空间的目的。可持久化还容易支持历史查询操作。

可持久化数据结构有主席树(可持久化线段树),可持久化可并堆,可持久化Trie,
可持久化数组,可持久化并查集,可持久化平衡树等。
\subsection{主席树}
用主席树做的经典模型有:
\begin{itemize}
    \item 差分
    \item 对于每一个节点为左节点,维护其右边节点为右节点时的答案
    \item 将某一维离散化后不断插入新数据进行预处理以回答在线询问
\end{itemize}
\subsection{可持久化Trie}
若遇到求区间xor最大值之类的问题,使用可持久化Trie。
对于and,or最大值问题,可以在插入完数后把整棵子树加到另一棵子树上去,
查询时只需考虑一边的子树。
\subsection{可持久化数组}
可持久化数组有两种实现:
\begin{itemize}
    \item 块状数组
    \item 主席树
\end{itemize}
可持久化并查集可使用可持久化数组实现。
\subsection{优化}
\subsubsection{标记永久化}
将对整个区间的操作记录在管理此区间的节点,标记不下传,统计时标记参与计算。
此法节约了$push$的时间且对可持久化友好。
\subsubsection{克隆开关}
若已知按照原方法有一部分数据不再被任何时间的数据结构引用时,直接在该数据上修改
(当然也可以gc,实现比较麻烦)。
可以在操作前设置一个$enableClone$开关,若为$false$则直接返回原节点。

代码如下:
\begin{lstlisting}[title=cloneA]
bool enableClone=true;
int cloneNode(int src) {
    if(enableClone) {
        int id=allocNode();
        T[id]=T[src];
        return id;
    }
    return src;
}
\end{lstlisting}
对于可持久化并查集,若使用路径压缩优化(实践中不太好用),则不好判断是否需要$clone$。
可以在每个节点上记录其被创建时的时间戳,与当前版本时间戳比较。

代码如下:
\lstinputlisting{Source/Templates/FDSU.cpp}
此法节约了复制节点时的时间与空间,但是路径压缩增加了修改的时间和空间,考场上最好写
按秩合并。
