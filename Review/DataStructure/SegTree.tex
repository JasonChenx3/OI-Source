\section{线段树}
\subsection{技巧}
\subsubsection{全局最优值剪枝}
可以使用全局变量维护自己当前遍历到的最优值,若父节点维护的信息表明管辖范围内不可能
出现更优值,则直接返回减少递归深度。(在kd-tree中比较有效)
\subsubsection{标记永久化}
直接将对整个区间的操作存到标记中而不下放,统计时加回去,减少常数。
\subsection{zkw线段树}
\index{Z!zkw's Segment Tree}
留坑待填。
\index{*TODO!zkw线段树}
\subsection{势能分析线段树}
对于某种无法打标记的区间操作(例如区间开根号),若该操作对某个元素施加少数次该操作就会使其
趋于稳定或区间内的值相等,同样可以使用线段树。每次区间操作暴力修改,合并时维护下次操作是否
可以跳过/缩点。

更多应用需要SegmentTreeBeats,留坑待填。
\index{*TODO!Segment Tree Beats}
\index{S!Segment Tree Beats}
\subsection{线段树分治}
留坑待填。
\index{*TODO!线段树分治}
\subsection{线段树优化建图}
对于一个点到一个或多个连续区间内的点有连边且区间内的边权相等,考虑使用线段树优化建图。
即使用线段树的上层节点代表管辖区间内的所有节点,建树时上层节点向左右儿子连权值为0的边
({\bfseries 注意上层节点要拆点分别连有向边到儿子}),
建图时类似modify操作连边,边数由$O(mn)$降为$O(m\lg n)$。

这里的上层节点相当于``捆线带''。

例题:CF786B Legacy

线段树优化建图+裸最短路。

\lstinputlisting{Source/Source/SegmentTree/CF786B.cpp}
