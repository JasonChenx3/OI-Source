\section{划分树}
划分树是一种类似于线段树但很少使用的数据结构,用来求解静态区间第k大问题。
划分树可用主席树代替,权当了解。
\subsection{构建}
和线段树类似,在下一层中将每一段区间分为两个子区间,以这段区间的中位数为划分依据,
并且被划分到同一边的数的相对次序不变。为了支持查询,还需要记录区间内每一个数及之前的数
有多少个划分到左区间中。
\subsubsection{注意事项}
\begin{itemize}
    \item 由于每一层都只有n个数,所以只要每一层开n个数的数组即可,记$A[d]$为
    第$d$层的划分情况,$cnt[d][i]$为第$d$层的第$i$个数及同区间之前的数有多
    少个数进入了左子树。
    \item 中位数可预先对数组排序获得,记排序后数组为$B$。
    \item 注意有多个中位数的情况(否则左区间的数将覆盖到右区间的存储区域上)。
\end{itemize}
\subsection{查询}

\begin{enumerate}
    \item 当到达叶子节点时直接返回$A[d][i]$或$B[i]$。
    \item 利用$cnt$数组差分可以查询到$[l,r]$之间有多少数进入了左子树,
    决定往哪棵子树走。
    \item 根据$cnt$数组重新计算目标范围在下一层的区间,递归查询。
\end{enumerate}

\subsection{板子}

SP3946 MKTHNUM - K-th Number

\lstinputlisting[title=PartitionedTree]
{DataStructure/PartitionedTree.cpp}

以上内容参考了hchlqlz\footnote{划分树讲解 - hchlqlz
\url{https://www.cnblogs.com/hchlqlz-oj-mrj/p/5744308.html}}的博客。
