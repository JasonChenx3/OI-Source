\section{珂朵莉树/老司机树}
珂朵莉树主要用来解决序列上区间元素的复杂变换问题(比如区间幂等无法整体修改的操作)。

使用它的前置条件有:
\begin{itemize}
    \item 有区间赋值操作
    \item 数据随机(区间赋值操作的比例稳定且赋值区间均匀)
\end{itemize}

其主要思路是用std::set维护连续权值相同的块,执行区间操作时暴力切分维护。由于数据随机与区间
赋值操作,set的大小会快速下降。若有$n$个元素,$m$次操作,时间复杂度趋于$O(m\lg n)$。

\subsection{存储}
单个节点表示区间内的值都等于它的值。存储在set中的节点定义如下:
\begin{lstlisting}
struct Node {
    int l;
    mutable int val;
    bool operator<(const Node& rhs) const {
        return l<rhs.l;
    }
};
std::set<Node> seq;
\end{lstlisting}
这里只存储左端点比较方便,其$l$表示区间为$[l,l_{nxt})$。
初始化时在$n+1$处加一个哨兵。

因为set的键值不允许改变,decltype(*std::set<Node>::iterator())
是const Node\&,所以要将$val$设为mutable。
\subsection{切分}
$split(pos)$在$pos$处切分,并返回区间$[pos,)$的迭代器。

查找第一个$l$不小于$pos$的位置,如果其左端点恰好为$pos$,直接返回。否则$pos$在
前一个区间中,将其切分后返回迭代器。

\begin{lstlisting}
typedef std::set<Node>::iterator IterT;
IterT split(int pos) {
    IterT it=seq.lower_bound(pos);
    if(it->l==pos)
        return it;
    --it;
    return seq.insert(Node(pos,it->val)).first;
}
\end{lstlisting}
\subsection{区间赋值}
区间赋值是保证时间复杂度的关键操作。只要切分出区间,然后删掉该区间,插入
新区间就可以了。
\begin{lstlisting}
void assign(int l,int r,int val) {
    IterT beg=split(l),end=split(r+1);
    seq.erase(beg,end);
    seq.insert(Node(l,val));
}
\end{lstlisting}
由于split中只有插入操作,迭代器保证有效,不需要注意split(l)和split(r+1)的顺序。
\subsection{应用}
和区间赋值一样,$split(l),split(r+1)$后对$[beg,end)$的迭代器指向的值执行暴力
修改或查询。在查询时有时需要得到区间的大小,有三种解决方案:
\begin{itemize}
    \item 遍历时维护当前迭代器$it$,同时维护$nxt$。
    \item 在$Node$中维护$r$,split时删除原区间再加入前半段区间。注意此法要先
    $split(r+1)$再$split(l)$。
    \item 在$Node$中维护$mutable int r$。
\end{itemize}

上述内容参考了zyhang的博客\footnote{
    珂朵莉树详解
    \url{https://www.cnblogs.com/yzhang-rp-inf/p/9443659.html}
}。
