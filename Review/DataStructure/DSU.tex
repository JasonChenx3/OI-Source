\section{并查集}\label{DSU}
\index{D!Disjoint Set Union}
\subsection{路径压缩}
路径压缩的原理很简单,把找到的最新的祖先存储下来即可,把深度缩为1。
注意路径压缩会使树的形状改变,若有限制则只能使用LCT或本节~\ref{RankMerge}
所述的按秩合并。
\subsection{按秩合并}\label{RankMerge}
对于每个节点维护秩,代表该节点高度的上界。合并时按照启发式策略将较小秩的联通块
并到较大的联通块即可。
\subsection{复杂度证明}
留坑待填,参见\emph{Introduction to Algorithms Third Edition}\cite{ITA3}
中的21.3与21.4节。
\index{TODO!并查集复杂度证明}
\subsection{并查集的分裂}
如果要将集合中的某个点从原集合剥离,且不需要可持久化(即不查询历史信息),
则可以为每个点分配一个$id$,指向当前所指向的点,分裂时修改$id$指向新的点
即可,同时消除对原集合的影响。

\subsubsection{例题}

UVA11987 Almost Union-Find \footnote{
    【UVA11987】Almost Union-Find - 洛谷
    \url{https://www.luogu.org/problemnew/show/UVA11987}}

步骤同上所述,看了算法导论后才知道路径压缩和按秩合并同时使用才能达到
$O(m\alpha(n))$的复杂度。

代码如下:
\lstinputlisting[title=UVA11987]{DataStructure/UVA11987.cpp}
