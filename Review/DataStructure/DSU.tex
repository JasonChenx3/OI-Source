\section{并查集}\label{DSU}
\index{D!Disjoint Set Union}
\subsection{路径压缩}
路径压缩的原理很简单,即把找到的最新的祖先存储下来,于是该节点的深度被缩为1。
注意路径压缩会使树的形状改变,若维护的数据与树的形状有关则只能使用
LCT或本节~\ref{RankMerge}所述的按秩合并。设$find$次数为$f$,时间复杂度
$O(n+f\cdot (1+\log_{2+\frac{f}{n}}n))$。
\subsection{按秩合并}\label{RankMerge}
对于每个节点维护秩,代表该节点高度的上界。合并时按照启发式策略将较小秩的连通块
并到较大的连通块。设总操作数为$m$,时间复杂度为$O(m+n\lg n)$。
\subsection{复杂度证明}
留坑待填,参见算法导论\cite{ITA3}中的21.3与21.4节。
\index{*TODO!并查集复杂度证明}
\subsection{并查集的分裂}
若要将集合中的某个点从原集合剥离,且不需要可持久化(即不查询历史信息),则可以
考虑``金蝉脱壳'',即保留原节点不动,但消除其对集合的影响,另建新点代表该点。
具体步骤为为每个点维护一个$id$,指向当前实际所指向的点,分裂时先消除原$id$指向的点
对原集合的影响,再创建新的点并更新$id$。

\subsubsection{例题}

UVA11987 Almost Union-Find \footnote{
    【UVA11987】Almost Union-Find - 洛谷
    \url{https://www.luogu.org/problemnew/show/UVA11987}}

步骤同上所述,注意只有同时使用路径压缩和按秩合并才能达到$O(m\alpha(n))$的复杂度。

代码如下:
\lstinputlisting[title=UVA11987]{DataStructure/UVA11987.cpp}

\subsection{并查集重构树}
类似于Kruskal重构树,当两个联通块相连时建一个新的父亲节点并连边,
可以发现两个节点第一次被连接的时间点就是它们的LCA建立的时间点。

\paragraph{例题 BZOJ 3712: [PA2014]Fiolki}

可以发现两种物质在同一瓶内当它们在并查集重构树上的LCA建立时。
以LCA深度为第一关键字,反应优先级为第二关键字对反应进行排序。
按照排序后的顺序模拟反应。

代码:
\lstinputlisting{Source/Source/DSU/BZOJ3712.cpp}

该内容参考了小蒟蒻yyb的博客\footnote{
    【BZOJ3712】Fiolki(并查集重构树)
    \url{https://www.cnblogs.com/cjyyb/p/9368629.html}
}。
