\section{平衡二叉树}
\subsection{FHQTreap}\label{FHQTreap}
\index{F!FHQTreap}
FHQTreap是一种比较好写的平衡二叉树,虽然效率不太高
(不如子节~\ref{splay}\\的splay),但其易于理解,不需要旋转,
且对可持久化友好。
FHQTreap的使用基于两个基本操作:split与merge。

对于二叉搜索树的操作,由于FHQTreap本来就能够满足二叉搜索树的定义
,因此操作方法相同(在对效率要求不高的情况下灵活地使用split和merge
还可以减少代码量)。

对于序列操作,使用子节~\ref{split}所述的splitKth即可完成区间提取。

\subsubsection{split}\label{split}

split函数的作用是将一棵树按照权值或位置划分成两棵子树。

\begin{itemize}
\item
按位置划分:$split(rt,k,x,y)$表示将以$rt$为根的树分为以$x$为根的
左子树和以$y$为根的右子树,其中左子树内的节点是原树的前$k$个,需要维护
每个节点的子树大小$siz$。

代码如下:
\begin{lstlisting}[title=splitKth]
    void split(int u,int k,int& x,int& y) {
        if(u) {
            push(u);
            int lsiz=T[T[u].ls].siz;
            //`决定当节点u为第k个时被分到哪棵子树`
            if(k<=lsiz) {
                y=u;
                split(T[u].ls,k,x,T[u].ls);
            }
            else{
                x=u;
                split(T[u].rs,k-lsiz-1,T[u].rs,y);
            }
            update(u);
        }
        else x=y=0;
    }
\end{lstlisting}
\item
按权值划分:$split(rt,k,x,y)$表示将以$rt$为根的树分为以$x$为根的
左子树和以$y$为根的右子树,其中左子树内的节点值均小于等于$k$。

代码如下:
\begin{lstlisting}[title=splitKey]
    void split(int u,int k,int& x,int& y) {
        if(u) {
            push(u);
            //`=决定当T[u].val==k时被分到哪棵子树`
            if(T[u].val<=k) {

                x=u;
                split(T[u].rs,k,T[u].rs,y);
            }
            else{
                y=u;
                split(T[u].ls,k,x,T[u].ls);
            }
            update(u);
        }
        else x=y=0;
    }
\end{lstlisting}
\end{itemize}

根据树的实际意义(二叉搜索树还是序列)以及实际需要来决定使用哪种split。

\subsubsection{merge}

merge将两棵树按照左右顺序(中序遍历)合并。
和treap一样,merge使用随机权重来保持树的平衡。

代码如下:

\begin{lstlisting}[title=merge]
    int merge(int u,int v) {
        if(u && v) {
            if(T[u].pri<T[v].pri) {
                push(u);
                T[u].rs=megre(T[u].rs,v);
                update(u);
                return u;
            }
            else{
                push(v);
                T[v].ls=merge(u,T[v].ls);
                update(v);
                return v;
            }
        }
        return u|v;
    }
\end{lstlisting}

\subsubsection{指示权重的伪随机数生成器}\label{WRG}

显然FHQTreap也是一个Treap,所以需要一个表现良好的伪随机数生成器
来指示该节点的权。

最易于实现的伪随机数生成算法就是线性同余法(LCG)了。
\index{L!Linear Congruential\\ Generator}
C++11中<random>的$std::linear\_congruential\_engine$给出
了两组预置的参数:
\begin{itemize}
    \item $minstd\_rand0:(a=16807, c=0, m=2147483647)$

    Discovered in 1969 by Lewis, Goodman and Miller, adopted as
    "Minimal standard" in 1988 by Park and Miller.
    \item $minstd\_rand:(a=48271, c=0, m=2147483647)$

    Newer "Minimum standard", recommended by Park, Miller, and Stockmeyer in 1993.

\end{itemize}

通常选择第2个即$a=48271$,代码如下:

\begin{lstlisting}[title=minstd\_rand]
int getRand() {
    static int seed = 347;
    return seed = seed * 48271LL % 2147483647;
}
\end{lstlisting}

现在来试试说明它的优越性:

\begin{itemize}
    \item 根据节~\ref{PrimitiveRoot}所述,如果$g$是模数$P$的一个原根,
    则$g$的幂模$P$可以取到$[1,P-1]$内的每一个数,且循环周期长度为$P-1$。
    由于2147483647是梅森素数,所以它一定存在原根。
    下列程序可证明48271是2147483647的一个原根:
    \lstinputlisting[title=RandomTestA.cpp]{DataStructure/RandomTestA.cpp}
    \item 48271可以较早地使int溢出,从而避免出现$a$过小而导致``锯齿波''。
    下列程序可证明48271可以满足OI考试的需要:
    程序输出minc=7884 maxc=8515 except=8192 s2=88.492,可见数据还是蛮均匀的。
\end{itemize}

参见cppreference\footnote{
    \url{https://en.cppreference.com/w/cpp/numeric/random/
    linear\_congruential\_engine}}与
    Wikipedia-EN\footnote{
    \url{https://en.wikipedia.org/wiki/Linear\_congruential\_generator}}。

如果需要更均匀的随机数,可以使用如下方案(质量从低到高):
\begin{enumerate}
    \item 使用比LCG更好的梅森旋转算法
    \footnote{std::mersenne\_twister\_engine - cppreference.com
    \url{https://en.cppreference.com/w/cpp/numeric/random/
    mersenne\_twister\_engine}};
    \item Intel指令集内置RDRAND;
    \item 使用由一些机构提供的真随机数生成器SDK,如\url{https://www.random.org/};
    \item 在需要蒙特卡洛采样的场合使用低差异序列如Halton,Sobol等。
\end{enumerate}

\subsection{splay}\label{splay}
\index{S!splay}

由于Treap做LCT复杂度多一个log(而且我还看不懂),所以还是学一下好了。

splay主要由$rotate$和$splay$函数组成:

\subsubsection{rotate}

$rotate(u)$表示将节点$u$旋转到$u$的父亲上。



在实践中可使用$connect(u,p,c)$把节点$u$挂到节点$p$的位置$c$下,$getPos(u)$获得
节点$u$相对于父亲的位置。

代码如下:

\begin{lstlisting}[title=rotate]
int getPos(int u) {
    return u == T[T[u].p].c[1];
}
void connect(int u, int p, int c) {
    T[u].p = p;
    T[p].c[c] = u;
}
void rotate(int u) {
    int ku = getPos(u);
    int p = T[u].p;
    int kp = getPos(p);
    int pp = T[p].p;
    int t = T[u].c[ku ^ 1];
    T[u].p = pp;
    if (!isRoot(p))
        connect(u, pp, kp);
    connect(t, p, ku);
    connect(p, u, ku ^ 1);
    update(p);
    update(u);
}
\end{lstlisting}

\subsubsection{splay}

\subsubsection{具体应用}

对于二叉搜索树的操作,同子节~\ref{FHQTreap}相同,直接用二叉搜索树的
操作即可。

对于序列操作,可使用splay来提取区间:

\begin{enumerate}
    \item
\end{enumerate}

上述内容参考了自为风月马前卒的博客\footnote{splay详解(一) - 自为风月马前卒
\url{http://www.cnblogs.com/zwfymqz/p/7896036.html}}。
