\section{堆}
\subsection{左偏树}
\index{L!Leftist Tree}
左偏树(Leftist Tree)也是一个二叉堆,核心操作是$merge$函数,
它可以以$O(\lg n)$合并两个左偏树。

定义外节点为没有左子树或右子树的节点。对于左偏树的每一个节点,维护其到子树外节
点的最近距离,其中外节点的$dist=0$,$null$的$dist=-1$(其实没必要太严格,
差不多平衡就够了)。

左偏树具有左偏性质:
\begin{character}\label{LTC}
    $dist_l \geq dist_r$
\end{character}

由此定义可得到一个引理:

\begin{lemma}
    $dist_u=min(dist_l,dist_r)+1=dist_r+1$
\end{lemma}

考虑一棵距离为$k$的左偏树的节点数至少有多少,得到以下定理:

\begin{theorem}
    一棵距离为$k$的左偏树为满二叉树时节点数最少,有$2^{k+1}-1$个节点。
\end{theorem}

由此得到推论~\ref{LTI}:

\begin{inference}\label{LTI}
    一棵节点数为$n$的左偏树,距离最多为$[lg(n+1)-1]$。
\end{inference}

先给出引理~\ref{LTL}:

\begin{lemma}\label{LTL}
    左偏树的最右链恰好有一个外节点。
\end{lemma}

证明:由于左偏树是一棵树,因此至少有一个外节点;若存在两个以上外节点,
对于非深度最深的点,必有右子树(否则链就断了),却没有左子树(由外节点定义可知),
与性质~\ref{LTC}矛盾。

由推论~\ref{LTI}与引理~\ref{LTL}可得如下定理:

\begin{theorem}\label{LTT}
    一棵由$n$个节点组成的左偏树最右链最多有$[lg(n+1)]$个节点。
\end{theorem}

以上证明参考了阿波罗2003的博客\footnote{
    浅谈左偏树 - 阿波罗2003
    \url{https://www.cnblogs.com/yyf0309/p/LeftistTree.html}
}。

我原来的简单理解:对于左偏树的每一个节点,维护其子树高度。每次$merge$时往右边塞,
若右边的深度比左边的深度更大,就把左边换过来加。以此保证树的高度尽可能小。

$merge(u,v)$的操作如下:

\begin{enumerate}
    \item 如果$u$或$v$有一个为$null$则返回另一个节点;
    \item 若$v$应该在$u$的上一层则$swap(u,v)$;
    \item 递归将节点$v$与$u$的右子树合并;
    \item 若$dist_l<dist_r$则$swap$左右子树;
    \item 更新节点$u$的距离$dist_u=dist_r+1$;
    \item 返回该树的根$u$。
\end{enumerate}

根据定理~\ref{LTT}可得$merge$的复杂度为$O(lgn)$。

代码如下(以大根堆为例):

\begin{lstlisting}[title=merge]
int merge(int u, int v) {
    if (u && v) {
        if (T[u].val < T[v].val)
            std::swap(u, v);
        T[u].r = merge(T[u].r, v);
        if (T[T[u].l].dis < T[T[u].r].dis)
            std::swap(T[u].l, T[u].r);
        T[u].dis = T[T[u].r].dis + 1;
    }
    return u | v;
}
\end{lstlisting}

\subsection{斜堆}
斜堆的操作与左偏树差不多,唯一的区别就是不再维护到外节点的最近距离,
取而代之的是在每一次$merge$时$swap$左右子树。
