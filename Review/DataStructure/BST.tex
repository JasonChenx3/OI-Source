\section{二叉搜索树}
\index{B!Binary Search Tree}
\subsection{二叉搜索树性质}
\begin{property}
    向二叉搜索树中插入新节点$u$(节点关键字不重复),该节点
    要么是其前驱的右儿子,要么是其后继的左儿子。
\end{property}
\paragraph{证明} $u$的前驱后继在插入节点$u$前互为前驱后继,
若前驱有右儿子则说明前驱比后继高,后继没有左儿子,节点$u$成为后继的左儿子。
反之亦然。
\subsection{FHQTreap}\label{FHQTreap}
\index{F!FHQTreap}
FHQTreap是一种比较好写的二叉搜索树,虽然效率不太高
(不如子节~\ref{splay}的splay),但其易于理解,不需要旋转,且对可持久化友好。
FHQTreap的实现基于两个操作:split与merge。

对于二叉搜索树的操作,由于FHQTreap本来就是一棵二叉搜索树,因此操作方法相同
(若对效率要求不高,可以灵活地使用split和merge以减少代码量)。

对于序列操作,使用子节~\ref{split}所述的splitKth完成区间提取。

\subsubsection{split}\label{split}

split函数的作用是将一棵树按照权值或位置划分成两棵子树。

\begin{itemize}
	\item
	      按位置划分:$split(rt,k,x,y)$表示将以$rt$为根的树分为以$x$为根的
	      左子树和以$y$为根的右子树,其中左子树内的节点是原树的前$k$个,需要维护
	      每个节点的子树大小$siz$。

	      代码如下:
	      \begin{lstlisting}[title=splitKth]
    void split(int u,int k,int& x,int& y) {
        if(u) {
            push(u);
            int lsiz=T[T[u].ls].siz;
            //lsiz(+1)决定当节点u为第k个时被分到哪棵子树
            if(k<=lsiz) {
                y=u;
                split(T[u].ls,k,x,T[u].ls);
            }
            else{
                x=u;
                split(T[u].rs,k-lsiz-1,T[u].rs,y);
            }
            update(u);
        }
        else x=y=0;
    }
\end{lstlisting}
	\item
	      按权值划分:$split(rt,k,x,y)$表示将以$rt$为根的树分为以$x$为根的
	      左子树和以$y$为根的右子树,其中左子树内的节点值均小于等于$k$。

	      代码如下:
	      \begin{lstlisting}[title=splitKey]
    void split(int u,int k,int& x,int& y) {
        if(u) {
            push(u);
            //=决定当T[u].val==k时被分到哪棵子树
            if(T[u].val<=k) {
                x=u;
                split(T[u].rs,k,T[u].rs,y);
            }
            else{
                y=u;
                split(T[u].ls,k,x,T[u].ls);
            }
            update(u);
        }
        else x=y=0;
    }
\end{lstlisting}
\end{itemize}

根据树的实际意义(二叉搜索树还是序列)以及实际需要来决定使用哪种split。

\subsubsection{merge}

merge将两棵树按照左右顺序(中序遍历)合并。
和treap一样,merge使用随机权重来保持树的平衡。

代码如下:

\begin{lstlisting}[title=merge]
    int merge(int u,int v) {
        if(u && v) {
            if(T[u].pri<T[v].pri) {
                push(u);
                T[u].rs=megre(T[u].rs,v);
                update(u);
                return u;
            }
            else{
                push(v);
                T[v].ls=merge(u,T[v].ls);
                update(v);
                return v;
            }
        }
        return u|v;
    }
\end{lstlisting}

\subsubsection{指示权重的伪随机数生成器}\label{WRG}

FHQTreap是一个Treap,所以需要一个表现良好的伪随机数生成器来指示节点的权重。

最易于实现的伪随机数生成算法就是线性同余法(LCG)了。
\index{L!Linear Congruential\\ Generator}
C++11中<random>的$std::linear\_congruential\_engine$给出
了两组预置的参数:
\begin{itemize}
	\item $minstd\_rand0:(a=16807, c=0, m=2147483647)$

	      Discovered in 1969 by Lewis, Goodman and Miller, adopted as
	      "Minimal standard" in 1988 by Park and Miller.
	\item $minstd\_rand:(a=48271, c=0, m=2147483647)$

	      Newer "Minimum standard", recommended by Park, Miller, and Stockmeyer in 1993.

\end{itemize}

\index{*Constant!线性同余随机数生成器\\$a=48271,c=0,$\\ $m=2147483647$}
通常选择第2个即$a=48271$,代码如下:

\begin{lstlisting}[title=minstd\_rand]
int getRand() {
    static int seed = 347;
    return seed = seed * 48271LL % 2147483647;
}
\end{lstlisting}

现在尝试说明它的优越性:

\begin{itemize}
	\item 根据节~\ref{PrimitiveRoot}所述,如果$g$是模数$P$的一个原根,
	      则$g$的幂模$P$可以取到$[1,P-1]$内的每一个数,且循环周期长度为$P-1$。
	      由于2147483647是梅森素数,所以它一定存在原根。
	      下列程序可证明48271是2147483647的一个原根:
	      \lstinputlisting[title=RandomTestA.cpp]{DataStructure/RandomTestA.cpp}
	\item 48271是2147483647的一个大原根,可以较早地使int溢出,从而避免出现因
	      $a$过小而导致产生``锯齿波''。

          下列程序证明了48271可以满足OI考试的需要:
          程序输出minc=7884 maxc=8515 except=8192 s2=88.492,
          可见该算法生成的随机数还是蛮均匀的。

          \lstinputlisting[title=RandomTestB.cpp]{DataStructure/RandomTestB.cpp}
    \item 模数为2147483647,随机数值域广。
\end{itemize}

参见cppreference\footnote{
	\url{https://en.cppreference.com/w/cpp/numeric/random/
		linear\_congruential\_engine}}与
Wikipedia-EN\footnote{
	\url{https://en.wikipedia.org/wiki/Linear\_congruential\_generator}}。

如果需要更均匀的随机数,可以使用如下方案(质量从低到高):
\begin{enumerate}
	\item 使用比LCG更好的梅森旋转算法
	      \footnote{std::mersenne\_twister\_engine - cppreference.com
		      \url{https://en.cppreference.com/w/cpp/numeric/random/
			      mersenne\_twister\_engine}};
	\item RDRAND指令与$std::random\_device$;
	\item 使用由一些机构提供的真随机数生成器SDK,如\url{https://www.random.org/};
	\item 在需要蒙特卡洛采样的场合使用低差异序列如Halton,Sobol等。
\end{enumerate}

\subsubsection{用于可持久化时的注意事项}
在可持久化操作中涉及复制整棵子树,使其子树的随机优先级与原来相同,形态保持不变,容易导致
后续操作的树不再平衡。

简单的解决方法是不存储随机优先级,而是维护子树大小,根据两棵子树的大小\CJKsout{建立离散分布}随机
决定哪个节点在上方。

该方法来自2018年IOI2018国家候选队论文集董炜隽的论文《浅谈Splay和Treap的性质及其应用》。
\subsection{splay}\label{splay}
\index{S!Splay}

由于Treap做LCT复杂度多一个log(而且我看不懂),常数较大,splay仍然无法被完全替代。

splay主要由$rotate$和$splay$函数组成:

\subsubsection{rotate}

$rotate(u)$表示将节点$u$旋转到$u$的父亲上。

主要思想是提取出父亲所在的子树,然后把节点$u$提为子树的根,最后把一个原来的儿子
挂到原来的父亲下以保持二叉平衡树的性质。

具体步骤:
\begin{enumerate}
	\item 把自己挂到父亲的位置上;
	\item 根据相对位置把父亲挂到自己下方;
	\item 把父亲占用位置所在的节点(原来的儿子)挂到原来自己的位置上;
	\item 依次更新原父亲与自己的信息。
\end{enumerate}

$rotate$前需要提前将标记下传(在$splay$中$push$)。在实践中可使用辅助函数
$connect(u,p,c)$把节点$u$挂到节点$p$的位置$c$下,$getPos(u)$获得
节点$u$相对于父亲的位置。

代码如下:

\begin{lstlisting}[title=rotate]
int getPos(int u) {
    return u == T[T[u].p].c[1];
}
void connect(int u, int p, int c) {
    T[u].p = p;
    T[p].c[c] = u;
}
void rotate(int u) {
    int ku = getPos(u);
    int p = T[u].p;
    int kp = getPos(p);
    int pp = T[p].p;
    int t = T[u].c[ku ^ 1];
    connect(u, pp, kp);
    connect(t, p, ku);
    connect(p, u, ku ^ 1);
    update(p);
    update(u);
}
\end{lstlisting}

\subsubsection{splay}

$splay(u,p)$表示将节点$u$旋转到$p$的儿子的位置上。
$splay$操作有单旋与双旋之分,其中单旋会被卡(原因见下面所述)。

\paragraph{单旋}

单旋的思路很简单,就是不断地把自己旋转到父亲的位置上。

但是当原树是一条链时,单旋$splay$后仍然是一条链,树的高度并没有得到优化,
因此可以构造数据使得每次$splay$都达到最坏时间复杂度$O(n)$。

\paragraph{双旋}

当自己,父亲,爷爷在同一条直线上时,会多旋转一次父亲来尽可能地减小树高。
这种做法在链上表现良好。

代码如下:

\begin{lstlisting}[title=splay]
void splay(int u,int p) {
    //push p->u
    while(T[u].p!=p) {
        int pu=T[u].p;
        if(T[pu].p!=p)rotate(getPos(u)==getPos(pu)?pu:u);
        rotate(u);
    }
}
\end{lstlisting}

\subsubsection{具体应用}

\paragraph{树根}

显式维护树根是一件麻烦事,容易发现$getPos(root)$总是返回0,所以$T[0].c[0]$
即为$root$。

\paragraph{二叉搜索树}

对于二叉搜索树的操作,同子节~\ref{FHQTreap}相同,直接当做二叉搜索树
进行操作。为了优化树高(splay的复杂度是均摊$\lg n$而不是严格的),每次执行完操作后
对最近访问节点$u$调用$splay(u,0)${\bfseries(血泪史:[HNOI2002]营业额统计在insert
重复时也要splay)}。

\paragraph{序列操作}

对于序列操作,可使用splay来提取区间。如果需要操作区间$[l,r]$,则:

\begin{enumerate}
	\item 将节点$l-1 splay$到根节点;
	\item 将节点$r+1 splay$到$l-1$的右儿子上;
	\item 此时$r+1$的左子树就是目标区间。
\end{enumerate}

为了操作区间$[1,n]$,可以在头尾分别添加一个哨兵。

\paragraph{splay合并}

$join(T1,T2)$表示将树$T1$与$T2$合并为一棵树,其中$T1$的所有元素权值
均小于$T2$的任意元素权值。

做法很简单:把$T1$的最大节点$splay$到根,此时该节点没有右子树,然后把$T2$
的根挂到该节点右儿子的位置上。

上述内容参考了自为风月马前卒的博客\footnote{splay详解(一) - 自为风月马前卒
	\url{http://www.cnblogs.com/zwfymqz/p/7896036.html}}。

\paragraph{splay启发式合并}
维护splay树的size,合并时将size小的拆开插入size大的,时间复杂度$O(n\lg^2n)$。
据说拆树时使用中序遍历,插入复杂度均摊为O(1),时间复杂度O(nlgn)。该说法已被证实,
参见Cppreference\footnote{
    std::set::insert - cppreference.com
    \url{https://en.cppreference.com/w/cpp/container/set/insert}
}中带hint插入元素的复杂度说明。

{\bfseries 注意FHQTreap的merge必须保证两树无交且参数顺序按照合并后的序列顺序,
不能简单地将两树的根merge。}
