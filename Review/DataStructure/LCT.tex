\section{动态树}
\index{L!Link-Cut Tree}

{\bfseries 警告:首先考虑使用DFS序解决!}
\subsection{常规操作}
动态树是一堆splay组成的森林,主要以$access$和$makeRoot$操作为基础,
可以实现$link,cut,split,find$等功能。
LCT的splay用来维护以深度为关键字的重链信息。

\subsubsection{splay部分}
与常规splay的不同之处在于LCT中的splay的根也是有父亲的,指向另一棵splay的
节点,在此使用辅助函数$isRoot(u)$来判断节点$u$是不是splay的根。为了实现LCT
的$access$功能,需要支持区间翻转。

\begin{lstlisting}[title=splay]
int getPos(int u) {
    return u == T[T[u].p].c[1];
}
bool isRoot(int u) {
    int p = T[u].p;
    return T[p].c[0] != u && T[p].c[1] != u;
}
#define ls T[u].c[0]
#define rs T[u].c[1]
void pushDown(int u) {
    if (T[u].rev) {
        std::swap(ls, rs);
        T[ls].rev ^= 1;
        T[rs].rev ^= 1;
        T[u].rev = false;
    }
}
void update(int u);
void connect(int u, int p, int c) {
    T[u].p = p;
    T[p].c[c] = u;
}
void rotate(int u) {
    int ku = getPos(u);
    int p = T[u].p;
    int kp = getPos(p);
    int pp = T[p].p;
    int t = T[u].c[ku ^ 1];
    T[u].p = pp;
    if (!isRoot(p))
        T[pp].c[kp] = u;
    connect(t, p, ku);
    connect(p, u, ku ^ 1);
    update(p);
    update(u);
}
void push(int u) {
    if (!isRoot(u)) push(T[u].p);
    pushDown(u);
}
void splay(int u) {
    push(u);
    while (!isRoot(u)) {
        int p = T[u].p;
        if (!isRoot(p))
            rotate((getPos(u) == getPos(p)) ? p : u);
        rotate(u);
    }
}
\end{lstlisting}
\subsubsection{access}
$access(u)$的功能是使节点$u$与其LCT的根在同一棵$splay$内,
此时节点$u$是根节点重链中深度最大的点(已经和原来到儿子的边断开)。

具体操作如下:

\begin{enumerate}
    \item 将节点$u$翻转到其所在$splay$的根;
    \item 删除节点$u$的右子树,即断开与重儿子的连边,更新节点$u$;
    \item 跳到节点$u$的父亲节点,并翻转其至根;
    \item 把以$u$为根的链挂到父节点的右子树位置上,
        该操作自动断开父节点到原来重儿子的连边,更新;
    \item 重复步骤3合并重链直至合并到根节点为止。
\end{enumerate}

代码:
\begin{lstlisting}[title=access]
void access(int u) {
    int v = 0;
    do {
        splay(u);
        rs = v;
        update(u);
        v = u;
        u = T[u].p;
    } while(u);
}
\end{lstlisting}

\subsubsection{makeRoot}
$makeRoot(u)$的功能是把节点$u$翻转为整棵LCT的根。

具体操作如下:

\begin{enumerate}
    \item 打通节点$u$到根节点的路径,并使节点$u$成为splay的根;
    \item 由于节点$u$是splay中深度最大的节点,翻转整棵splay后就可以使
    节点$u$成为根了。
\end{enumerate}

代码:

\begin{lstlisting}[title=makeRoot]
void makeRoot(int u) {
    access(u);
    splay(u);
    T[u].rev ^= 1;
    pushDown(u);
}
\end{lstlisting}
\subsubsection{split}
若要提取$u-v$的路径,$makeRoot(u),access(v),splay(v)$后,节点$v$的子树
就保存了$u-v$的路径信息。
\subsubsection{find}\label{LCTFind}
首先$access(u),splay(u)$让$u$成为LCT根所在的splay的根。
根节点就是splay的最小节点(注意推送翻转标记)。

坑:有些毒瘤出题人可能会卡找最小节点的过程,因此在find找到根节点$u$后再执行一次
$splay(u)$。{\bfseries 注意如果$find$中使用了$splay(u)$并且$cut$中使用了$find$,
调用$find$后$u$在根位置,$cut$的后续代码需要稍微修改。}

该坑源自FlashHu的博客\footnote{
    LCT总结——概念篇+洛谷P3690[模板]Link Cut Tree(动态树)(LCT,Splay)
    \url{https://www.cnblogs.com/flashhu/p/8324551.html}
}。
\subsubsection{link}
$makeRoot(u)$后使$T[u].p=v$,注意在link前要进行连通性检测。
\subsubsection{cut}

\begin{itemize}
    \item 若保证$u,v$连接,则split后令$T[u].p=T[v].c[0]=0$
    (节点$v$深度最大,因此只需和左子树断开),注意要更新节点$v$。
    \item 若不保证$u,v$连接,则在$makeRoot(u)$后检查以下条件:
    \begin{itemize}
        \item 节点$v$所在LCT的根是$u$;
        \item 节点$u$的父亲是$v$(查根时执行完$find(v)$后节点$v$已经是
        节点$u$所在splay的根);
        \item 节点$v$的左儿子是$u$;
        \item 节点$u$没有右儿子(若有则说明$u-v$中有其他节点)。
    \end{itemize}
    由于$find(v)$隐式执行了$access(v),splay(v)$,只需令
    $T[u].p=T[v].c[0]=0$。
\end{itemize}

以上内容参考了Saramanda的博客\footnote{
    LCT(Link-Cut Tree)详解(蒟蒻自留地)
    \url{https://blog.csdn.net/saramanda/article/details/55253627}}。

\subsection{技巧/常见方法}
\subsubsection{DSU优化连通性检测}
如果可以保证处于同一连通块内的点不会再次分离(允许临时分离,比如动态MST在环上换边时),
可以使用本章~\ref{DSU}节所述的并查集代替~\ref{LCTFind}中昂贵的$find$。
\subsubsection{使用access进行从某节点到根的路径的染色}
$access(u)$后节点$u$到根节点的路径上的点在同一棵splay内,可用于
模拟染色过程。

参见[SDOI2017]树点涂色\footnote{【P3703】[SDOI2017]树点涂色 - 洛谷
    \url{https://www.luogu.org/problemnew/show/P3703}}。
\subsubsection{动态LCA}
注意link/cut操作不能换根,换根会使树的形态改变导致LCA不同。

求节点$u,v$的LCA时,首先$access(u)$开辟一条链,类似地$access(v)$再开辟一条链,
$access(v)$时会与$access(u)$的链相交,交点即为LCA。在实现时可以直接令$access$函数
返回$v$值作为与上一条链的交点(当$while(u)$为false时,说明在$u=T[u].p$之前的$u$是
链的交汇点,然而过程退出前将$u$赋给了$v$)。

板子:SP8791 DYNALCA
\lstinputlisting{Source/Source/LCT/SP8791.cpp}

事实上该方法的适用范围还可以再扩展。考虑每次$access$后将提取出的链打一个标记,那么在
下次$access$时,每个$u$都意味着这可能是与之前某个节点的LCA。检查标记可以得到对应的节点标号
(一般结合题目性质贪心保留某个特殊节点)。例题:「雅礼集训 2017 Day7」事情的相似度。
\subsubsection{LCT维护边权信息}
把边当做节点,与两端点相连,端点点权为0。边的节点编号最好为边编号+总端点数,
以便于快速得到原边编号并断开端点连接。
\subsubsection{LCT维护子树信息}
该方法主要用于解决{\bfseries 不断换根且子树无修改}的问题。若子树有修改则用TopTree
向轻儿子下放标记。

普通LCT只维护了一条链的信息,即splay内信息。维护子树信息的关键在于显式地将
虚子树(虚边儿子的子树)信息累加入自身信息中。每个节点维护2个数据,一个是
虚子树贡献,另一个是贡献总和。update时顺带累加虚子树贡献。

考虑哪些操作会改变虚实子树:
\begin{itemize}
    \item access:在$rs=v$处将虚子树贡献中$v$的贡献改为$rs$的贡献。
    \item makeRoot、find、split等变换树的形态的操作除access部分外无影响,
    无需特殊处理。
    \item link:令$T[u].p=v$会导致$v$及其祖先都需要更新贡献,可以``$split(u,v)$''
    后将$v$换到链顶后把$u$的贡献累加到$v$的虚子树贡献中,最后执行$T[u].p=v,update(v)$。
    如此只影响到了$v$的信息。
    \item cut:$u,v$在同一棵splay内,除access部分外无影响。
\end{itemize}

{\bfseries 注意节点与虚儿子一定连接,但不一定与实儿子连接。}对于简单的情况
不必考虑这个事实,比如「BJOI2014」大融合中的子树节点数。但像LOJ\#558
「Antileaf's Round」我们的 CPU 遭到攻击~这种题,子树黑点数的变化会导致子树
黑点距离和发生变化,因为实儿子不一定与自己相连。

接下来以此题为例讨论解决方法,下文的``链''指代splay序列:

在查询时需要$access(u),splay(u)$,此时$u$必定在splay链尾。虽然自己的实儿子与自己不一定相连,
但是自己的左实儿子和虚儿子连到到链的右端以及右实儿子和虚儿子到链的左端一定经过自己。考虑维护
自己的子树到splay中{\bfseries 实子树对应的链}的左端/右端的距离和。
\begin{itemize}
    \item 查询时把$access(u),splay(u)$后$u$自己就是链的右端,并且$u$在splay的根,子树是完整的,
    $u$到链右端的距离和就是答案。
    \item 维护子树信息仍然使用上述方法。
    \item 访问到左右端点的信息时注意提前下传翻转标记,下传翻转标记时同时把{\bfseries
    左右端信息交换}。

    {\bfseries 血泪史:用LCT刚「PA 2017」Banany 点分治题时,由于忘了下传翻转标记而
    加入错误信息,导致无法删除信息(用set维护虚子树信息)。在访问u的子节点的信息前一定要
    记得下传!!!一般在access替入rs到链左端的信息与update读取实儿子信息中需要使用。不过
    值得欣慰的是我用LCT拿了LOJ的rank1,时间是第二快的50\%,空间是第二小的30\%,代码长度是
    第二短的80\%。}
    \item 为了编码方便,可以将自身的贡献塞到虚子树贡献中。
    \item LCT初始化时使用常规DFS处理信息,而不是用link。
\end{itemize}

代码如下:
\lstinputlisting{Source/Source/LCT/LOJ558.cpp}

\subsubsection{LCT维护图上信息}
若动态图只加边不删边,仍旧可以使用LCT维护图上信息,关键是``可缩点''。

额外维护2个DSU,一个维护连通性,另一个维护{\bfseries 双连通性}。
当且仅当两个点双连通时才缩为一点,即第二个DSU指示该点的$id$。

每次访问父亲时都使用父亲的缩点$id$代替,其它操作不变。

关键在于$link$操作的处理,$link$前若两点未连通,执行普通$link$并标记第一个DSU。
否则$split(u,v)$拆出$u-v$的链,遍历整棵splay将其所有点的贡献累加到点$v$上,然后把
第二个DSU中的$id$置为$v$。{\bfseries 注意设置位置是原$id$,缩点后$v$的左右儿子要置0。}

上述内容参考了neither\_nor\footnote{
    LCT维护子树信息(子树信息LCT) LCT维护边权(边权LCT) 知识点讲解\\
    \url{https://blog.csdn.net/neither\_nor/article/details/52979425}
}和GuessYCB\footnote{
    LCT 模板及套路总结
    \url{https://www.cnblogs.com/GuessYCB/p/8330024.html}
}的博客,以及AntiLeaf's~Round的题解\footnote{
    AntiLeaf's Round 题解
    \url{https://loj.ac/article/304}
}。

\subsubsection{共价大爷游长沙技巧}
例题:UOJ\#207.共价大爷游长沙

需要不断给一棵树换边,判断给定路径是否全部经过某条边。

我原先的想法是每次增删路径时,在端点处打标记,查询时统计两边子树端点数。
这个想法是错的:考虑有两条路径都经过一条边,如果将这两条路径的一个端点交换,
在LCT上的标记还是一样的。

但是这个想法离正解已经很接近了。注意这里的问题是``判定‘'而不是``统计'',
考虑给每条路径随机一个权值,增删路径时在端点处异或该权值,查询任意一端子树异或和,
判断其是否为路径异或和。
\subsubsection{LCT维护黑白连通块}
该需求来自SP16549 QTREE6 - Query on a tree VI。
这题当然也可以使用LCT维护子树信息的做法,但是讨论比较多,性能不佳
(\CJKsout{树上单点修改+换根统计万金油倒是没错,复杂度保证全部交给LCT})。

考虑维护两个森林,一个森林上只有黑点,另一个森林上只有白点,更改颜色时暴力
断开原有连边,连到另一棵森林上,查询时只要询问其所在连通块的大小。大小比连通性
更容易维护。

但是如果这是一个菊花图,断开中间的点后,可能会导致$O(n)$的边修改量。那么考虑DFS
一遍给整棵树定型,每次只修改与父亲的连边。那么就保证了连通块内所有边的儿子都是该颜色
的点,只有连通块的深度最浅点不是该颜色的点(为了支持根节点的染色,给根节点一个虚父亲),
找到深度最浅的点后返回右子树大小(不能返回子树大小-1是因为左子树可能有其它连通块)。

因为整棵树被定型,不能使用makeRoot操作,需要对link和cut特殊处理。

\begin{itemize}
    \item $link(u,f_u)$:$f_u$要接受$u$子树大小的虚子树贡献,所以要
    $access(f_u),splay(f_u)$;为了将$u$挂在下面,要$splay(u)$;最后
    连接$fa[u]=f_u$、贡献$isiz[f_u]+=siz[u]$、更新$update(f_u)$。
    \item $cut(u,f_u)$:$access(u)$后要$splay(f_u)$而不是$splay(u)$,然后
    按照普通$cut$操作。若使用$splay(u)$,$f_u$并不一定是$u$的左儿子,不能令
    $fa[f_u]=0$,而是要令$fa[lson[u]]=0$。
\end{itemize}

该方法参考了cdsszjj的博客\footnote{
    bzoj3637: Query on a tree VI【LCT】\\
    \url{https://blog.csdn.net/cdsszjj/article/details/80332588}
}。

\subsubsection{LCT离线维护动态图连通性}
给定加边与删边操作序列,期间多次询问两点的连通性。

关键在于当连边后会出现环时如何将其变为树。考虑一种贪心的做法,将环上最早被删除的边删掉,
这样做将不会影响到图的连通性。而删边时间可以离线预处理。

该方法参考了国家集训队2014论文集黄志翱的《浅谈动态树的相关问题及简单扩展》。
\subsubsection{链权翻转}
给LCT的每条链再使用一个splay来维护链权。
均摊复杂度仍为$O(\lg n)$。
