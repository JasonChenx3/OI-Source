\section{动态树}
\index{L!Link-Cut Tree}
\subsection{常规操作}
动态树是一堆splay组成的森林,主要以$access$和$makeRoot$操作为基础,
可以实现$link,cut,split,find$等功能。
LCT的splay用来维护以深度为关键字的重链信息。

\subsubsection{splay部分}
与常规splay的不同之处在于LCT中的splay的根也是有父亲的,指向另一棵splay的
节点,在此使用辅助函数$isRoot(u)$来判断节点$u$是不是splay的根。为了实现LCT
的$access$功能,需要支持区间翻转。

\begin{lstlisting}[title=splay]
int getPos(int u) {
    return u == T[T[u].p].c[1];
}
bool isRoot(int u) {
    int p = T[u].p;
    return T[p].c[0] != u && T[p].c[1] != u;
}
#define ls T[u].c[0]
#define rs T[u].c[1]
void pushDown(int u) {
    if (T[u].rev) {
        std::swap(ls, rs);
        T[ls].rev ^= 1;
        T[rs].rev ^= 1;
        T[u].rev = false;
    }
}
void update(int u);
void connect(int u, int p, int c) {
    T[u].p = p;
    T[p].c[c] = u;
}
void rotate(int u) {
    int ku = getPos(u);
    int p = T[u].p;
    int kp = getPos(p);
    int pp = T[p].p;
    int t = T[u].c[ku ^ 1];
    T[u].p = pp;
    if (!isRoot(p))
        connect(u, pp, kp);
    connect(t, p, ku);
    connect(p, u, ku ^ 1);
    update(p);
    update(u);
}
void push(int u) {
    if (!isRoot(u)) push(T[u].p);
    pushDown(u);
}
void splay(int u) {
    push(u);
    while (!isRoot(u)) {
        int p = T[u].p;
        if (!isRoot(p))
            rotate((getPos(u) == getPos(p)) ? p : u);
        rotate(u);
    }
}
\end{lstlisting}
\subsubsection{access}
$access(u)$的功能是使节点$u$与其LCT的根在同一棵$splay$内,
此时节点$u$是根节点重链中深度最大的点(已经和原来到儿子的重链断开)。

具体操作如下:

\begin{enumerate}
    \item 将节点$u$翻转到其所在$splay$的根;
    \item 删除节点$u$的右子树,即断开与重儿子的连边,更新节点$u$;
    \item 跳到节点$u$的父亲节点,并翻转其至根;
    \item 把以$u$为根的链挂到父节点的右子树位置上,
        该操作自动断开父节点到原来重儿子的连边,更新;
    \item 重复步骤3合并重链直至合并到根节点为止。
\end{enumerate}

代码:
\begin{lstlisting}[title=access]
void access(u) {
    int v = 0;
    while (u) {
        splay(u);
        rs = v;
        update(u);
        v = u;
        u = T[u].p;
    }
}
\end{lstlisting}

\subsubsection{makeRoot}
$makeRoot(u)$的功能是把节点$u$翻转为整棵LCT的根。

具体操作如下:

\begin{enumerate}
    \item 打通节点$u$到根节点的路径,并使节点$u$成为splay的根;
    \item 由于节点$u$是splay中深度最大的节点,翻转整棵splay后就可以使
    节点$u$成为根了。
\end{enumerate}

代码:

\begin{lstlisting}[title=makeRoot]
void makeRoot(int u) {
    access(u);
    splay(u);
    T[u].rev ^= 1;
    pushDown(u);
}
\end{lstlisting}
\subsubsection{split}
若要提取$u-v$的路径,$makeRoot(u),access(v),splay(v)$后,节点$v$的子树
就保存了$u-v$的路径信息。
\subsubsection{find}\label{LCTFind}
首先$access(u),splay(u)$让$u$成为LCT根所在的splay的根。
然后查询splay的最小节点就是根节点(注意推送翻转标记)。
\subsubsection{link}
$makeRoot(u)$后使$T[u].p=v$即可,注意在link前要进行连通性检测。
\subsubsection{cut}

\begin{itemize}
    \item 若保证$u,v$连接,则split后令$T[u].p=T[v].c[0]=0$
    (节点$v$深度最大,因此只需和左子树断开),注意要更新节点$v$。
    \item 若不保证$u,v$连接,则在$makeRoot(u)$后检查以下条件:
    \begin{itemize}
        \item 节点$v$所在LCT的根是$u$;
        \item 节点$u$的父亲是$v$(查根时执行完$find(v)$后节点$v$已经是
        节点$u$所在splay的根);
        \item 节点$v$的左儿子是$u$;
        \item 节点$u$没有右儿子(若有则说明$u-v$中有其他节点)。
    \end{itemize}
    由于$find(v)$隐式执行了$access(v),splay(v)$,只需令
    $T[u].p=T[v].c[0]=0$。
\end{itemize}

以上内容参考了Saramanda的博客\footnote{
    LCT(Link-Cut Tree)详解(蒟蒻自留地)
    \url{https://blog.csdn.net/saramanda/article/details/55253627}}。

\subsection{技巧}
\subsubsection{DSU优化连通性检测}
如果可以保证处于同一连通块内的点不会再次分离(允许临时分离),
可以使用本章~\ref{DSU}节所述的并查集代替~\ref{LCTFind}中昂贵的$find$。
\subsubsection{使用access进行从某节点到根的路径的染色}
$access(u)$后节点$u$到根节点的路径上的点在同一棵splay内,可用于
模拟染色过程。

参见[SDOI2017]树点涂色\footnote{【P3703】[SDOI2017]树点涂色 - 洛谷
    \url{https://www.luogu.org/problemnew/show/P3703}}。
