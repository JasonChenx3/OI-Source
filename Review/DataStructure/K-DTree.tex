\section{K-D Tree}
K-D Tree是一棵二叉树,每一层按照某个轴将本空间内的所有点分为较为均匀的两部分,
该节点保存划分的中点。查询时依靠不断剪枝来提高查询速度。
\subsection{构树}
具体步骤如下:
\begin{enumerate}
	\item 对于当前子空间,选取一个轴来划分(使用$std::nth\_element$)出中点;
	\item 将中点存储在当前节点上;
	\item 递归建左右子树;
	\item 更新子树信息。
\end{enumerate}
$std::nth\_element$的复杂度为$O(n)$,因此构树的复杂度为$O(nlgn)$。
\subsection{插入}
\subsubsection{离线标记}
构树时将所有点加入,记录每个点的id,然后加入点时打标记一路更新即可,
不过这样做影响了查询的复杂度。
\subsubsection{替罪羊树}
与二叉搜索树的插入相同,注意需要确定每一个节点的划分轴。当二叉树不平衡时会影响
查询复杂度,采用替罪羊树的策略,维护每棵子树的size,如果
$max(siz_l,siz_r)>=siz_u \cdot fac$则暴力重构子树(注意每次插入只要找到最高
的不平衡子树重构即可),一般$fac$取0.75。
\subsection{删除}
删除节点后的处理方法与插入相同。
注意被删除的节点可以gc。

\begin{lstlisting}[title=gc]
    std::vector<int> pool;
    int newNode() {
        static int cnt=0;
        int id;
        if(pool.size()) {
            id=pool.back();
            pool.pop_hack();
        }
        else id=++cnt;
        return id;
    }
    void freeNode(int u) {
        pool.push_back(u);
    }
\end{lstlisting}

\subsection{查询}
\begin{enumerate}
	\item 如果整棵子树均不满足要求,就直接返回;
	\item 如果整棵子树均满足要求且可以不需要继续递归,就记录答案(或者打标记)后返回;
	\item 计算当前节点;
	\item 递归左右子树。
\end{enumerate}
在随机数据下,查询的时间复杂度是$O(lgn)$,在构造数据下复杂度约
是$O(n^\frac{d-1}{d})$。
证明待补充。
\index{TODO!K-D Tree查询复杂度证明}
\subsection{估值}
下列为一些常见估值函数:

由于每个方向上是的独立的,对每个方向贪心后加起来即可。
\subsubsection{曼哈顿距离最小}
$w=\sum_{i=1}^d{max(mind_i-p_i,0)+max(p_i-maxd_i,0)}$
当$p_i$在区域内时估值为0,在一边时估值为到最近一边的值(另一边由于符号问题值为0)。
\subsubsection{曼哈顿距离最大}
$w=\sum_{i=1}^d{max(abs(mind_i-p_i),abs(maxd_i-p_i))}$
选择距离最大的一边。
\subsubsection{欧几里得距离最小}
$w=\sum_{i=1}^d{max(mind_i-p_i,p_i-maxd_i,0)^2}$
\subsubsection{欧几里得距离最大}
$w=\sum_{i=1}^d{max((mind_i-p_i)^2,(p_i-maxd_i)^2)}$
\subsection{技巧}
\subsubsection{全局最优值剪枝}
如果通过该节点维护的子树信息可以确定子树内不存在更优解,搜索该子树
已经没有意义了。还可以搭配另一个优化:先求两棵子树的估价函数值,
选择最优的先进入(更有可能获得最优值然后减少在另一棵子树上的计算)。
\subsubsection{预处理降维}
如果插入与查询离线,则可以对某一维排序,边插入边查询,降低kd-Tree查询复杂度。

以上内容参考了n+e的课件\emph{K-D Tree 在信息学竞赛中的应用}\cite{kdTree}。
