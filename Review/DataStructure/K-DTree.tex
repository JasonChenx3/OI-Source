\section{K-D Tree}
\index{K!K-D Tree}
K-D Tree是一棵二叉树,每一层按照某个轴将本节点辖域内的所有点分为较为均匀的两部分,
该节点自身代表用于划分的中点。查询时依靠剪枝来提高查询速度。
\subsection{构树}
具体步骤如下:
\begin{enumerate}
	\item 对于当前子空间,选取一个轴来划分(使用$std::nth\_element$)出中点;
	\item 将中点存储在当前节点上;
	\item 递归建左右子树;
	\item 更新子树信息。
\end{enumerate}
$std::nth\_element$的复杂度为$O(n)$,因此构树的复杂度为$O(n\lg n)$。
\subsection{插入}
\subsubsection{离线标记}
构树时将所有点加入,记录每个点对应的id,但仅维护初始存在的点的信息。加入点时对该点打标记,
同时自底向上更新信息。这种做法虽然保持了最终树的形态良好,但是在当前存在的点很少时性能不佳。
\subsubsection{替罪羊树}
注意此时每一个节点的划分轴应该是固定的。当二叉树不平衡时会影响查询复杂度,采用替罪羊树的
平衡策略,维护每棵子树的size,如果$max(siz_l,siz_r)\geq siz_u \cdot fac$则暴力重构
子树(注意每次插入后只要找到最高的不平衡子树进行重构),一般$fac$取0.75。

不过实践表明每固定修改/查询次数暴力重建整棵树跑得更快\CJKsout{(面向测试数据编程)}。
\subsection{删除}
删除节点后的处理方法与插入相同。
注意被删除的节点可以gc。

\begin{lstlisting}[title=gc]
    std::vector<int> pool;
    int newNode() {
        static int cnt=0;
        int id;
        if(pool.size()) {
            id=pool.back();
            pool.pop_hack();
        }
        else id=++cnt;
        return id;
    }
    void freeNode(int u) {
        pool.push_back(u);
    }
\end{lstlisting}

\subsection{查询}
\begin{enumerate}
	\item 如果整棵子树均不满足要求,就直接返回;
	\item 如果整棵子树均满足要求且没有继续递归的必要,就记录答案或打标记后直接返回;
	\item 计算当前节点,更新答案;
	\item 递归查询左右子树。
\end{enumerate}
在随机数据下,查询的时间复杂度是$O(\lg n)$,在构造数据下复杂度约
是$O(n^\frac{d-1}{d})$。
证明待补充。
\index{*TODO!K-D Tree查询复杂度证明}
\subsection{估值}
下列为一些常见估值函数:

因为不同轴上的估值两两独立,所以可以对每个方向贪心计算后求和。
\subsubsection{曼哈顿距离最小}
$\displaystyle w=\sum_{i=1}^d{max(mind_i-p_i,0)+max(p_i-maxd_i,0)}$
当$p_i$在区域内时估值为0,在区域外时估值为到最近一边的值(另一边由于符号为负,值为0)。
\subsubsection{曼哈顿距离最大}
$\displaystyle w=\sum_{i=1}^d{max(abs(mind_i-p_i),abs(maxd_i-p_i))}$
选择距离最大的一边。
\subsubsection{欧几里得距离最小}
$\displaystyle w=\sum_{i=1}^d{max(mind_i-p_i,p_i-maxd_i,0)^2}$
\subsubsection{欧几里得距离最大}
$\displaystyle w=\sum_{i=1}^d{max((mind_i-p_i)^2,(p_i-maxd_i)^2)}$
\subsection{技巧}
以下三种技巧可兼容使用。
\subsubsection{全局最优值剪枝}
如果仅通过该节点维护的子树信息就可以确定子树内不存在更优解,搜索该子树
已经没有意义了,直接退出。
\subsubsection{贪心搜索路径选择}
与Alpha-Beta剪枝的思路相同。先求出两棵子树的估价函数值,先进入估价更优的子树搜索,
更有可能在该子树中获得最优值,然后在另一棵子树上获得更多的剪枝机会。
\subsubsection{预处理降维}
如果插入与查询离线,则可以对某一维排序,边插入边查询,降低K-D Tree的查询复杂度。

以上内容参考了n+e的课件《K-D Tree 在信息学竞赛中的应用》\cite{kdTree}。
