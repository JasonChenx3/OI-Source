\section{树状数组}
树状数组用于支持区间加减法的数据维护。若只需要前缀和,可不满足区间减法。
\subsection{标号管辖范围}

画图可发现,节点$i$保存的是$[i-lowbit(i)+1,i]$的和。在修改时,$+lowbit(i)$会使自
己移动到父节点。在查询时$-lowbit(i)$会使自己向前移动$lowbit(i)$位,离开自己的管辖
范围,移动到管辖范围前的更高层的节点。

\subsection{线性预处理}

\begin{lstlisting}[title=LinearBuild]
    for(int i=1;i<=n;++i) {
        int j=i+(i&-i);
        if(j<=n)A[j]+=A[i];
    }
\end{lstlisting}

\subsection{lowbit函数原理}

lowbit函数定义为:$lowbit(i)=i\&-i$。

由于$i$始终为正,所以$-i$的补码表示是$i$的位取反再加1。因此末尾形如$0100$的位会变为
$1111$,可以发现$i$与$-i$在末尾1前后的位均不同,该部分位与后为0。因此$i\&-i$仅保留
末尾1的位。
\subsection{区间加法区间求和}
记原数组为$A_i$,差分数组$D_i=A_i-A_{i-1}$,则有
$\displaystyle A_i=\sum_{j=1}^i{D_j}$。将区间求和转化为前缀和之差,记前缀和
为$S_i$,有$\displaystyle S_i=\sum_{j=1}^i{A_j}=\sum_{j=1}^i{(i-j+1)D_j}$。
将该式拆为$\displaystyle (i+1)\sum_{j=1}^i{D_i}-\sum_{j=1}^i{jD_i}$,
做区间加法时在差分数组上打标记,同时维护数组$iD_i$。
