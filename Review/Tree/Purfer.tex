\section{Purfer Sequence}\label{PurferSeq}
\index{P!Purfer Sequence}
Purfer Sequence用来表示树的结构({\bfseries 带标号})。
\subsection{构造Purfer Sequence}
\begin{enumerate}
	\item 在当前树中找到标号最小的叶子节点;
	\item 向序列加入与该叶子节点相连的节点标号,并将该叶子节点删除。
	\item 重复步骤1直至只剩2个节点。
\end{enumerate}
由此可得两个性质:
\begin{property}[唯一性]
	一个大小为$n$的树对应一个长度为$n-2$的Purfer Sequence。
\end{property}
\begin{property}
	一个度数为$d$的节点在Purfer Sequence中出现$d-1$次。
\end{property}
\subsection{恢复原树}
\begin{enumerate}
	\item 统计节点在Purfer Sequence中出现的次数得到每个点的度数,记为$d[u]$;
	\item 对于序列中的每一个编号$v$,选取$d[u]=1$且标号最小的节点$u$,连接$(u,v)$,
		  然后$--d[u],--d[v]$(实际上只需用平衡树维护度数为1的节点,
		  在平衡树上删除点$u$,$--d[v]$后判断是否要插入点$v$);
	\item 将剩下两个$d[u]=1$的点连边。
\end{enumerate}
\subsection{计数应用}
可根据度数与出现次数的关系计算满足度数要求的树的数目。
{\bfseries 注意无解和$n\leq 2$时的情况。}

以上内容参考了JMJST的博客\footnote{
	BZOJ 1005 [HNOI2008] 明明的烦恼(组合数学 Purfer Sequence) - jianzhang.zj
	\url{http://www.cnblogs.com/zhj5chengfeng/archive/2013/08/23/3278557.html}
}。
