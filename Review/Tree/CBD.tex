\section{链剖分}
以下方法只适用于静态树。
\subsection{轻重链剖分}
对于每个节点,选取其子树最大的儿子作为重儿子,其余节点为轻儿子。
然后将连续的重儿子串成一条链(由DFS序体现),每条链上最浅的节点为链头,
链内每个节点都指向链头方便跨越轻边跳跃到上一条链。

\begin{lstlisting}
int d[size],p[size],siz[size],son[size];
void buildTree(int u) {
    siz[u]=1;
    for(int i=last[u];i;i=E[i].nxt) {
        int v=E[i].to;
        if(v!=p[u]) {
            p[v]=u;
            d[v]=d[u]+1;
            buildTree(v);
            siz[u]+=siz[v];
            if(siz[son[u]]<siz[v])
                son[u]=v;
        }
    }
}
int top[size];
void buildChain(int u) {
    if(son[u]) {
        top[son[u]]=top[u];
        buildChain(son[u]);
    }
    for(int i=last[u];i;i=E[i].nxt) {
        int v=E[i].to;
        if(!top[v]) {
            top[v]=v;
            buildChain(v);
        }
    }
}
\end{lstlisting}

{\bfseries 注意调用buildChain前要先$top[1]=1$。}

\begin{property}
    剖分后的任意点到根经过的轻边与重链的数目为$O(\lg n)$级别。
\end{property}
证明:
\begin{itemize}
    \item 由于每个轻儿子的大小不超过父亲大小的一半,所以每次经过一次
    轻边时,所在子树大小增加一半,所以最多经过$O(\lg n)$条轻边。
    \item 重链是由轻边连接的,所以重链数=轻边数+1,因此经过重链数也是
    $O(\lg n)$级别的。
\end{itemize}

\subsubsection{在链上统计中的应用}
在线段树中需要保证操作链上的节点尽可能连续,
使用轻重链剖分后的DFS序保证了经过重链数为$O(\lg n)$级别,
搭配线段树可在$O(\lg^2 n)$内单次修改/查询。

模板与求LCA类似:
\begin{lstlisting}
int top[size],id[size],pid[size],icnt=0;
void buildChain(int u) {
    id[u]=++icnt;
    pid[icnt]=u;//for build
    //...
}
void build(int l,int r,int id) {
    if(l==r) {
        int u=pid[l];
        //...
    }
    //...
}
typedef void (*Func)(int,int,int);
template<Func func>
void applyImpl(int l,int r) {
    nl=l,nr=r;
    func(1,icnt,1);
}
template<Func func>
void apply(int u,int v) {
    while(top[u]!=top[v]) {
        if(d[top[u]]<d[top[v]])
            std::swap(u,v);
        applyImpl<Func>(id[top[u]],id[u]);
        u=p[top[u]];
    }
    if(d[u]>d[v])
        std::swap(u,v);
    applyImpl<Func>(id[u],id[v]);
}
\end{lstlisting}

\subsection{长链剖分}

顾名思义就是把子树深度最深的儿子当做重儿子进行剖分。

\subsubsection{快速合并以深度为下标的消息}
令同一条链上的点共享一块dp存储区,统计当前节点信息时,直接继承重儿子的信息,
然后暴力合并轻儿子的链的信息。

复杂度证明:注意每条轻儿子所在的链合并到长链上的复杂度是$O(链长)$,
而父亲继承重儿子消息的复杂度为$O(1)$,所以最终复杂度为$O(n)$。

为了能够让父亲$O(1)$继承重儿子的消息,按照重儿子优先的DFS序分配dp数组起始位置即可。
原因为:
\begin{itemize}
    \item 合并后子树dp数组已经不再需要,直接覆盖。
    \item 父亲恰好在重儿子前一位,满足深度上的相对关系,可以$O(1)$继承数组。
\end{itemize}
因此空间复杂度也是$O(n)$的。

\subsubsection{快速求k级祖先}
\paragraph{预处理}
\begin{enumerate}
    \item 使用树上倍增$O(n\lg n)$计算第$2^k$级祖先。
    \item 对每条长链预处理长链顶的前``链长''个祖先,以及链上的所有点的编号。
\end{enumerate}

\paragraph{查询}

\begin{theorem}\label{DBCBD}
    任意一点的$k$级祖先所在的链长一定大于等于$k$。
\end{theorem}
证明:
\begin{itemize}
    \item 若该点与$k$级祖先在同一长链,显然定理成立。
    \item 否则,$k$级祖先所在长链的叶节点肯定不比自己浅,定理同样成立。
\end{itemize}

步骤如下:
\begin{enumerate}
    \item 用倍增数组跳$k$的最高位,设剩余层数为$k'$,有$k'<\frac{k}{2}$;
    \item 由定理~\ref{DBCBD}得当前节点所在链的链长严格大于$k'$,利用链头
    向上/向下的数组即可$O(1)$查询。
\end{enumerate}

预处理$O(n\lg n)$,查询$O(1)$。

长链剖分参考了MoebiusMeow的博客\footnote{长链剖分随想 - MoebiusMeow
    \url{https://www.cnblogs.com/meowww/p/6403515.html}
}
与后缀自动机·张的文章\footnote{长链剖分之O(nlgn)-O(1)求k级祖先
    \url{https://zhuanlan.zhihu.com/p/25984772}
}。
