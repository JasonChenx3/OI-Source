\section{Dsu On Tree}
Dsu On Tree的功能是求解{\bfseries 静态树}的{\bfseries 子树统计}问题。
主要思路是在重链剖分后,令父亲直接继承重儿子的信息,
统计轻儿子时暴力计算。以达到优化时间复杂度的目的。

步骤如下:
\begin{enumerate}
    \item 递归调用轻儿子计算轻儿子子树节点的子树信息,统计完毕后消去影响;
    \item 调用重儿子计算重儿子子树节点的子树信息,统计完毕后保留影响;
    \item 暴力统计轻儿子的子树;
    \item 计算当前节点自身的影响;
    \item 此时维护的信息为该节点的子树信息,记录答案;
    \item 若标记为消去影响,则重置维护的信息。
\end{enumerate}

\begin{lstlisting}[title=Dsu On Tree]
void DFS(int u, int p, bool erase) {
    for(int i = last[u]; i; i = E[i].nxt) {
        int v = E[i].to;
        if(v != p && v != son[u])
            DFS(v, u, true);
    }
    if(son[u])
        DFS(son[u], u, false);
    for(int i = last[u]; i; i = E[i].nxt) {
        int v = E[i].to;
        if(v != p && v != son[u]) {
            for(int j = L[v]; j <= R[v]; ++j)
                add(idc[j]);//`idc数组为已在DFS序上展开的数据`
        }
    }
    add(c[u]);
    //`记录信息`
    if(erase) {
        for(int i = L[u]; i <= R[u]; ++i)
            erase(idc[i]);//`为了保证复杂度按需清空`
        //`重置其余维护信息`
    }
}
\end{lstlisting}

若可以$O(1)$计算加上/删除一个点的影响,则时间复杂度为$O(n\lg n)$。

证明:因为每个节点到根的距离为$O(\lg n)$,所以每个节点被作为轻儿子的
子树节点合并的次数为$O(\lg n)$,最终时间复杂度为$O(n\lg n)$。

以上内容参考了Arpa的博客\footnote{[Tutorial] Sack (dsu on tree) - Codeforces
    \url{http://codeforces.com/blog/entry/44351}
}。
