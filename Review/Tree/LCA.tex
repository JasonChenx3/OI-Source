\section{最小公共祖先}
\subsection{倍增法}
预处理:DFS时计算每个节点的深度和$2^k$级祖先。

查询:首先将较深节点跳到同一高度,若原节点在一条链上,则
较浅的点为LCA,算法结束。否则按$k$从大到小保持在不跳到同一祖先的前提下尽量向上同时跳。
最后这两个节点的父亲就是原节点的LCA。

\begin{lstlisting}
int d[size],p[size][20];
void DFS(int u) {
    for(int i=1;i<20;++i)
        p[u][i]=p[u][i-1][i-1];
    for(int i=last[u];i;i=E[i].next) {
        int v=E[i].to;
        if(p[u][0]!=v) {
            p[v][0]=u;
            d[v]=d[u]+1;
            DFS(v);
        }
    }
}
int lca(int u,int v){
    if(d[u]<d[v])std::swap(u,v);
    int delta=d[u]-d[v];
    for(int i=0;i<20;++i)
        if(delta&(1<<i))
            u=p[u][i];
    if(u==v)
        return u;
    for(int i=19;i>=0;--i)
        if(p[u][i]!=p[v][i])
            u=p[u][i],v=p[v][i];
    return p[u][0];
}
\end{lstlisting}
预处理$O(n\lg n)$,单次查询$O(\lg n)$。
\subsection{树链剖分}
树链剖分内容参见第~\ref{CBDW}节。
树链剖分后,如果在同一条链上则返回较浅者,否则令链头较深的节点向上跳。

\begin{lstlisting}
int lca(int u,int v) {
    while(top[u]!=top[v]) {
        if(d[top[u]]>d[top[v]])u=p[top[u]];
        else v=p[top[v]];
    }
    return d[u]<d[v]?u:v;
}
\end{lstlisting}

两趟DFS预处理$O(n)$。
由于树链剖分后重链不超过$\lg n$条,所以单次查询也是$O(\lg n)$的,但树链剖分的常数比倍增法小。
\subsection{欧拉序+ST表}
考虑DFS序,显然两个来自节点$i$的不同子树的点的LCA为节点$i$,那么可以在
DFS完一棵子树后加入节点$i$的深度作为隔板,维护ST表查询区间内深度最小的节点编号。为了
应对点对在一条到根的链上的情况,同时也为了节点$i$给一个访问时间戳,
在遍历节点$i$之初就插入一个隔板。
\begin{lstlisting}
int icnt = 0, A[size * 2][18], L[size], d[size];
void DFS(int u) {
    A[++icnt][0] = u;
    L[u] = icnt;
    for (int i = last[u]; i; i = E[i].nxt) {
        int v = E[i].to;
        if (!L[v]) {
            d[v] = d[u] + 1;
            DFS(v);
            A[++icnt][0] = u;
        }
    }
}
int choose(int u, int v) {
    return d[u] < d[v] ? u : v;
}
void buildST() {
    for (int i = 1; i < 18; ++i) {
        int off = 1 << (i - 1),
            end = icnt - (1 << i) + 1;
        for (int j = 1; j <= end; ++j)
            A[j][i] = choose(A[j][i - 1],
                A[j + off][i - 1]);
    }
}
int ilg2(int x) {
    return 31 - __builtin_clz(x);
}
int getLca(int u, int v) {
    int l = L[u], r = L[v];
    if (l > r) std::swap(l, r);
    int p = ilg2(r - l + 1);
    return choose(A[l][p], A[r - (1 << p) + 1][p]);
}
\end{lstlisting}

预处理$O(n\lg n)$,单次查询$O(1)$。
\subsection{Tarjan}
当查询离线时,可使用Tarjan算法。

代码如下:
\lstinputlisting[title=Tarjan]{Tree/Tarjan.cpp}

算法正确性证明:
\begin{itemize}
    \item 若询问的两个节点在一条从根节点出发的链上,则第二被访问的
    节点为深度较深的节点,而深度较浅的节点的父亲仍然是自己;
    \item 若询问的两个节点在LCA的左右子树上,则第二被访问的节点
    被访问时,第一被访问节点的祖先已被置为LCA。
\end{itemize}

\paragraph{优化} 可以把find改为用队列实现,迭代更新父亲。
