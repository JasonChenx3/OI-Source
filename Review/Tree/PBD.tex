\section{点分治}
点分治就是每次选择树的重心(儿子的子树大小的最大值最小的节点)作为分治点,
以分治点为根将整棵树分为多棵子树,统计当前节点的答案,
然后递归每棵子树分治。这种方法一般用来解决{\bfseries 路径统计问题}。
\subsection{常规点分治}
\subsubsection{重心性质}
\begin{property}\label{WPP}
    重心的儿子子树大小不超过整棵树大小的一半。
\end{property}
\begin{property}
    所有点到重心的距离和最小,到两个重心的距离和相等。
\end{property}
\begin{property}
    两棵树合并后的重心在这两棵树的重心的路径上。
\end{property}
\begin{property}
    添加或减少一个叶子节点,重心最多偏移一条边。
\end{property}
\subsubsection{重心选择}
以当前连通块内任意一点为根(当然是分治点的儿子)DFS,然后计算
其儿子的子树大小的最大值,然后与除自己子树外的节点数求最大值
(儿子为有根树意义下的父亲时的子树),
就得到当前节点的权重了。

使用$vis$数组来标记是否已成为分治点,顺便原来分割连通块。

\begin{lstlisting}[title=getRoot]
bool vis[size];
int root,tsiz,msiz,siz[size];
void getRootImpl(int u,int p) {
    int maxs=0;
    siz[u]=1;
    for(int i=last[u];i;i=E[i].nxt) {
        int v=E[i].to;
        if(!vis[v] && v!=p) {
            getRootImpl(v,u);
            siz[u]+=siz[v];
            maxs=std::max(maxs,siz[v]);
        }
    }
    maxs=std::max(maxs,tsiz-siz[u]);
    if(maxs<msiz) {
        msiz=maxs;
        root=u;
    }
}
int getRoot(int u,int csiz) {
    msiz=1<<30;
    tsiz=csiz;
    getRootImpl(u,0);
    return root;
}
\end{lstlisting}

\subsubsection{分治与统计}
每次把重心作为分治点,统计从分治点出发的信息,两两合并后对于每棵子树去除
来自同一棵子树的信息(因为这已经不是简单路径了)。

\begin{lstlisting}[title=divide]
void divide(int u) {
    //count u->child
    vis[u]=true;
    for(int i=last[u];i;i=E[i].nxt) {
        int v=E[i].to;
        if(!vis[v]) {
            //minus v->u->v
            divide(getRoot(v,siz[v]));
        }
    }
}
\end{lstlisting}

\subsubsection{时间复杂度}
点分治会带来$O(\lg n)$的复杂度,证明:

根据性质~\ref{WPP},点分治的层数为$O(\lg n)$,而且每层的总规模都是$n$。

\subsection{动态点分治}
动态点分治就是在分治时连接当前分治点与子连通块的分治点,
这些点构成了一棵点分树,多次查询时使用点分树来计算。

\subsubsection{例题}

Luogu P4115 Qtree4\footnote{【P4115】Qtree4 - 洛谷
\url{https://www.luogu.org/problemnew/show/P4115}}

首先点分治建出点分树。
对于每个节点维护两个可修改堆,堆A维护子树白节点到该节点父亲的距离,堆B维护
该节点儿子们的堆A最大值,那么经过该节点的最长路径为堆B的最大值+堆B的次大值。
注意自身就是白点的情况。

对于经过每个节点的最长路径,再用一个可修改堆维护即可。

事实上双$std::priority\_queue$维护可修改堆比$std::multiset$跑得更快。

代码如下:
\lstinputlisting[title=Luogu P4115]
{Source/Source/'Point-Based Partition'/4115queue.cpp}
