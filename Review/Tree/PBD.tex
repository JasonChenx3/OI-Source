\section{点分治}
点分治就是每次选择树的重心(儿子的子树大小的最大值最小的节点)作为分治点,
以分治点为根将整棵树分为多棵子树,统计经过当前节点的路径贡献,
然后递归每棵子树分治。这种方法一般用来解决{\bfseries 路径统计问题}。

{\bfseries 能够使用树形DP解决(路径信息可合并)的不要使用点分治。}
\subsection{静态点分治}
\subsubsection{重心性质}
\begin{property}\label{WPPA}
    重心的儿子子树大小不超过整棵树大小的一半。
\end{property}
证明:考虑重心为$R$,设$R$的某个儿子$u$满足其子树大小超过
$\frac{N}{2}$,那么点$u$比点$R$更优,与重心为$R$的条件矛盾。
\begin{property}\label{WPPB}
    所有点到重心的距离和最小,到两个重心的距离和相等(若树大小为奇数则只有一个重心)。
\end{property}
\begin{property}\label{WPPC}
    两棵树合并后的重心在这两棵树的重心的路径上。
\end{property}
\begin{property}\label{WPPD}
    添加或减少一个叶子节点,重心最多偏移一条边。
\end{property}

这个性质搭配启发式合并可以用来快速处理新连通块的重心。

\paragraph{例题} LuoguP4299 首都:

根据性质~\ref{WPPB}可知该题要维护的是动态连通块的重心,且重心一定在这两棵树的
重心的路径上。那么可以在其中一棵树上从连接点开始逐步长叶子,利用性质~\ref{WPPD},每次
试探是否要往另一连通块的重心移动。使用启发式合并可以保证合并复杂度,至于快速查询距离和/
子树大小,可以使用LCT维护。时间复杂度$O(n\lg^2n)$。

查询首都时应该使用常数更小的并查集维护,向某个节点的移动可以使用树链剖分支持查询
(按照是否在到根的链上分类,结合~\ref{CSTA}中提到的jump函数询问)。移动目标仅需
指定为较小子树的连接点,因为再移动就会违反性质~\ref{WPPA}。维护最大子树大小比维护
距离和更方便,不过由于要维护可删堆,性能稍差一点。

参考代码($O(n\lg^2n)$):
\lstinputlisting{Source/Source/LCT/4299A.cpp}

当然根据性质~\ref{WPPC},可以直接link两棵树,然后split出两个重心的链,根据性质~\ref{WPPA}
判重心,在链上DFS剪枝查询(每次都往子树较大的儿子走),复杂度由splay保证。由于去掉了离线
建图支持定向移动的部分且只需要维护子树大小,代码量小很多。由于链两端都是重心,查询时不必考虑
最大虚子树,因为它们不可能超过整棵树大小的一半。

该方法参考了FlashHu的题解\footnote{
    题解 P4299 【首都】\\
    \url{https://www.luogu.org/blog/flashblog/solution-p4299}
}。

参考代码($O(n\lg n)$):
\lstinputlisting{Source/Source/LCT/4299B.cpp}

\paragraph{树的同构判定}
对于一棵有根树可以使用一些Hash规则计算子树的Hash值(合并子树Hash值的运算需要满足交换律或
排序后合并,我的做法是计算多个满足交换律的运算合并值,然后把这些值胡乱混合)。以无根树
的重心(1-2个)为根,就可以计算整棵树的Hash值,用于树的同构判定。

{\bfseries 血泪史:存储当前最优点的数组大小要开$n$而不是2。尽管重心不超过2个,
但在处理过程中仍然存在超过2个的现行最优点。我在下面的板子中由于kp数组开小了而导致
last数组被覆写产生环,因此我调了一晚上。。。望引以为戒。}

\lstinputlisting{Source/Templates/TreeHash.cpp}

\subsubsection{重心选择}
以当前连通块内任意一点为根(当然是上一个分治点的儿子)DFS,计算每个节点儿子的子树大小的
最大值,然后与除自己子树外的节点数求最大值(儿子为有根树意义下的父亲时的子树),就得到
当前节点的权重。

使用$vis$数组来标记节点是否已成为分治点,阻止DFS过程跨出连通块。

\begin{lstlisting}[title=getRoot]
bool vis[size];
int root,tsiz,msiz,siz[size];
void getRootImpl(int u,int p) {
    int maxs=0;
    siz[u]=1;
    for(int i=last[u];i;i=E[i].nxt) {
        int v=E[i].to;
        if(!vis[v] && v!=p) {
            getRootImpl(v,u);
            siz[u]+=siz[v];
            maxs=std::max(maxs,siz[v]);
        }
    }
    maxs=std::max(maxs,tsiz-siz[u]);
    if(maxs<msiz) {
        msiz=maxs;
        root=u;
    }
}
int getRoot(int u,int csiz) {
    msiz=1<<30;
    tsiz=csiz;
    getRootImpl(u,0);
    return root;
}
\end{lstlisting}

\subsubsection{分治与全局统计}
每次把重心作为分治点,DFS统计从分治点出发的路径贡献,将分治点到子树各点的路径信息平铺成序列,
用$O(n)$(单调队列、双指针)或$O(n\lg n)$(线段树)算法两两合并,再对于每棵子树去除来自
同一棵子树的贡献(因为这些路径经过了两次从分治点到子树的边,不是简单路径)。

\begin{lstlisting}[title=divide]
void divide(int u) {
    //count u->child
    vis[u]=true;
    for(int i=last[u];i;i=E[i].nxt) {
        int v=E[i].to;
        if(!vis[v]) {
            //minus v->u->v
            divide(getRoot(v,siz[v]));
        }
    }
}
\end{lstlisting}

{\bfseries Warning:理论上调用$getRoot$时需要重新DFS一遍算出所有子连通块的$siz$,
但是这里直接使用上一次DFS的$siz[v]$代替。事实上对于点分树上的儿子在实际树中夹在父亲和
爷爷中间的情况,$siz[v]$并不是真正的子树大小,但是实践表明这个简化方法的时间复杂度并不会
退化,且层数保证严格$O(\lg n)$,可以放心使用内存池分配空间。不过要用到$siz$时仍需要重新
DFS一遍。时间复杂度证明详见LCA的博客\footnote{
    一种基于错误的寻找重心方法的点分治的复杂度分析
    \url{http://liu-cheng-ao.blog.uoj.ac/blog/2969}}}。

统计过程也可以使用树形dp的惯用技巧:在统计从分治点出发的贡献时将儿子子树分开处理,每棵子树
DFS一遍查询与已处理子树节点(或者与分治点本身)的贡献,再DFS一遍维护数据结构(线段树常数
好大)便于下一次查询。如此就不会统计来自相同子树的贡献了,注意数据结构的清零要控制在子树规模内。
{\bfseries 关键:利用时间来存储数据。}两趟DFS太慢,可以一趟DFS将信息平铺到序列上,然后
查询+加入数据结构。
\subsubsection{局部统计}
有时需要计算从每个点出发的所有路径信息(最好具有无向性与可合并性),同样可以使用点分治处理。
从点$u$出发的路径,它要么在将点$u$作为分治点时被DFS,要么在将点$u$在点分树上的祖先作为
分治点时DFS到$u$,并将这一段路径与其它子树的路径合并时被统计到。

我原先统计合并路径的方法:使用上文类似树形DP的方法,不过由于是局部统计,统计到点$u$时能够
合并的路径只有合并之前被DFS的子树的路径。那么需要使用vector存边,然后正反各做一遍。

从租酥雨的博客\footnote{
    [Luogu2664]树上游戏\\
    \url{https://www.cnblogs.com/zhoushuyu/p/8311249.html}
}学到了更简单的统计方法:先DFS分治点的整个子树,每次统计某个儿子的路径时,先删掉该儿子的
子树路径贡献,统计,再恢复儿子的子树路径贡献。这么做还可以顺便计算单点的贡献和从$u$出发的
路径贡献。
\subsubsection{时间复杂度}
点分治会带来$O(\lg n)$的复杂度,证明:

根据性质~\ref{WPPA},点分治的层数为$O(\lg n)$,而且每层的总规模都是$n$。

\subsection{动态点分治}
动态点分治就是在分治时连接当前分治点与子连通块的分治点,
这些点构成了一棵点分树,多次查询时使用点分树来计算。

注意一般节点$u$需要维护{\bfseries 自己的子树节点到自己在点分树上父亲的信息},记为A。
那么节点$u$子树节点经过自身的路径信息就由儿子的信息A决定。修改信息时自底向上更新祖先的信息。
由于层数控制在$O(\lg n)$内,更新祖先的复杂度有了保证。

\subsubsection{例题}

Luogu P4115 Qtree4\footnote{【P4115】Qtree4 - 洛谷
\url{https://www.luogu.org/problemnew/show/P4115}}

首先点分治建出点分树。
对于每个节点维护两个可修改堆,堆A维护子树白节点到该节点父亲的距离,堆B维护
该节点儿子们的堆A最大值,那么经过该节点的最长路径为堆B的最大值+堆B的次大值。
注意自身就是白点的情况。再用一个全局可修改堆维护经过每个节点的最长路径。

实践中双$std::priority\_queue$维护可修改堆比$std::multiset$跑得更快,
参见第~\ref{MultiSet}节。

代码如下:
\lstinputlisting[title=Luogu P4115]
{Source/Source/'Point-Based Partition'/4115queue.cpp}
\subsubsection{与动态DP的区别}
动态点分治与动态DP都可以在$O(\lg n)$的复杂度内单次更新,但是它们的应用场合不同。

\begin{itemize}
    \item 动态点分治擅长与路径统计有关的问题,询问一般带有``路径''``距离''等关键字。
    \item 动态DP擅长与子树统计有关的问题,单点DP比较简单。
\end{itemize}

\subsubsection{与距离有关的动态点分治}
例题\CJKsout{血泪史}:bzoj3730:震波

计算每个分治点子树内的路径,每个分治点的父亲到自己子树的路径(用于去重)。使用
这两类信息支持询问。

有一些要注意的地方:
\begin{itemize}
    \item 关键在于选用什么数据结构维护路径(要求单点加减,前缀查询)。使用平衡树当然
    可以实现,但是常数大,编码与Debug难度大。如果使用树状数组,注意到点分治对于深度并
    没有保证,是否会MLE呢?

    仔细分析后发现这么做并不会MLE:注意每个分治点子树内的路径长度是连续的,所以BIT不
    可能出现空位。在最极端情况下(BIT上的位置不会被复用),由于每个节点最多在$O(\lg n)$
    棵分治点子树内,因此空间复杂度为$O(n\lg n)$。
    \item 修改点权时跳点分树容易遗漏,干脆给每个节点都记录其被引用的数据结构的指针+位置。
    \item 线性预处理BIT时记得判尾部,单独判$tot==n$或者强制令$A[tot+1].k=-1$。
    \item 可能存在儿子被夹在父亲和爷爷之间的情况,不能使用相对距离逐级减去。查询时使用的是
    {\bfseries 查询点到当前点的距离},不过去重用的BIT仍然是当前点的(表达意义相同)。详见
    代码中的query部分。这个问题耽误了我2个小时。。。

    Update:查询自己到某个祖先的距离可以不使用额外的树剖,在祖先DFS其子树时就可以支持距离
    计算,按顺序放入vector。
\end{itemize}

参考代码:
\lstinputlisting{Source/Source/'Point-Based Partition'/bzoj3730.cpp}

\subsubsection{自底向上统计的点分治Cache}
例题:「SDOI2016」模式字符串

此题我使用了自底向上的DFS统计方法,从模式串左右边界开始匹配。\CJKsout{事实上也有自顶向下的Hash
统计方法。}交上去后发现运行时间是时间的1.5倍,瓶颈在于DFS匹配时的std::vector的频繁操作。

注意到自底向上统计有一个自顶向下统计所没有的优势:不变的子树的DFS结果不变。那么可以考虑cache
计算结果。但是由于点分治过程可能会切断子树的某一部分而导致子树变化。注意到DFS的子树大小是不断
减小的,可以在solve过程中预先处理以分治点为根的子树中每个点的子树大小,同时记录cache的子树
大小,根据这两个数据判断是否要刷新cache。
