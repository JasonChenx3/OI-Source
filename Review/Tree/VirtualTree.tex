\section{虚树}
在多次询问的树型dp问题中,若遇到询问总点数与树的大小同数量级的情况,
可以在每次询问中将询问节点建成一棵``虚树'',然后对虚树做树型dp。
这样做可以有效地降低dp规模($\displaystyle qT(n)->
\sum_{i=1}^q{\left(O(m_i\lg m_i)+T(O(2m_i))\right)}$)。

\subsubsection{构造过程}

首先预处理从根节点DFS遍历整棵树,记录DFS序与深度,预处理计算LCA需要的信息,
同时维护其他信息。

对于每一次询问:
\begin{enumerate}
    \item 将询问节点按DFS序排序;
    \item 将第一个节点加入栈中,栈上维护的是当前节点到根的链;
    \item 对于剩下每一个节点$u$:
    \begin{enumerate}
        \item 计算自己与栈顶节点$v$的$lca$;
        \item 若栈中第二个节点$p$的深度比$lca$大,则连接$p->v$,弹出$v$。
        重复该步骤直至不满足条件;
        \item 若$lca$比$v$浅,连接$lca->v$,弹出$v$。
        \item 若栈为空或$v$比$lca$浅,加入节点$lca$。
        \item 加入节点$u$。
    \end{enumerate}
    \item 此时栈上还有一条链,将链加入树中,栈底就是根节点。
\end{enumerate}
\subsubsection{算法解释}

$lca$有两种可能:
\begin{itemize}
    \item $lca$为$v$:此时的操作只有加入节点$u$,
    就是简单地将自己挂在虚树中的父亲下。
    \item $u$和$v$在$lca$的两棵子树下:
    首先不断地跳$v$直至$u$和$v$在虚树上的直接祖先为$lca$,
    然后记此时栈顶为$p$,继续分类:
    \begin{itemize}
        \item $p$为$lca$,连接$lca->v$后把$v$换成$u$。
        \item $p$为$lca$的祖先,连接$lca->v$后把$v$换成$lca$与$u$。
        \item $lca$为原链头的祖先(对应空栈),把链全部折叠后加入节点$lca$与$u$。
    \end{itemize}
\end{itemize}

\subsubsection{算法实现}

\begin{lstlisting}
int buildTree(int k) {
    g2.cnt = 0;
    int top = 1;
    std::sort(id + 1, id + k + 1, cmp);
    st[1] = id[1];
    for (int i = 2; i <= k; ++i) {
        int u = id[i];
        int lca = getLca(u, st[top]);
        while (top > 1 && d[lca] < d[st[top - 1]]) {
            g2.addEdge(st[top - 1], st[top]);
            --top;
        }
        if (d[lca] < d[st[top]]) {
            g2.addEdge(lca, st[top]);
            --top;
        }
        if (top == 0 || d[st[top]] < d[lca])
            st[++top] = lca;
        st[++top] = u;
    }
    while (top > 1) {
        g2.addEdge(st[top - 1], st[top]);
        --top;
    }
    return st[1];
}
\end{lstlisting}

{\bfseries 注意有时需要把根节点强制加入到虚树中去。}
