\section{树的直径}
\subsection{定义与计算}
树的直径就是树的最长链。

\begin{itemize}
    \item 求直径长:一遍DFS对于每个点计算其到子树的最大与次大距离,
    然后相加更新答案。
    \item
    求某条直径:
    \begin{enumerate}
        \item 任选一点$a$,DFS计算与其相距最远的点$b$;
        \item 从$b$点开始DFS计算与其相距最远的点$c$;
        \item $b-c$就是树的直径。
    \end{enumerate}
\end{itemize}

\subsection{性质}

\begin{property}
    距树上任意一点最远的点必定在树的直径上。
\end{property}

\begin{property}
    所有直径的中心(点或边)相同,树上任意一点到最远点的路径必经过这个中心。
\end{property}

\begin{property}
    所有直径的偏心距相等。
\end{property}

下面这个性质可用于计算两棵子树合并后的直径:
\begin{property}
    树的直径端点是子树的直径端点。
\end{property}

与树上点到路径距离有关的dp题一般先考虑求直径,然后双指针法dp。

上述性质参考了里阿奴摩西的博客\footnote{[树的直径] BZOJ 1999 [Noip2007]Core树网的核
\url{https://blog.csdn.net/u014609452/article/details/69351006}}。

\subsection{多子树直径查询}
对原树按DFS序展开到线段树上,对于每个区间维护其直径与两端点,
合并时取四个点中最远点对当做新树的直径端点。由于询问保证区间内的点连通,不必
考虑合并时区间内的点是否连通。

{\bfseries 注意用LCA求距离时要使用欧拉序+ST表法。}

预处理$O(n\lg n)$,查询$O(\lg n)$。

上述内容参考了rzO\_KQP\_Orz的博客\footnote{用线段树维护树的直径
\url{https://blog.csdn.net/rzo\_kqp\_orz/article/details/52280811}
}。
