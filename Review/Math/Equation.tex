\section{常见数学公式与应用}
\subsection{常见公式}
\begin{eqnarray*}
	\textrm{平方和}&~\sum_{k=1}^n{k^2}&=\frac{n(n+1)(2n+1)}{6}\\
	\textrm{立方和}&~\sum_{k=1}^n{k^3}&=\frac{n^2(n+1)^2}{4}\\
	\textrm{无穷递减几何级数}&~\forall |x|<1,\sum_{k=0}^\infty{x^k}
	&=\frac{1}{1-x}\\
	\textrm{调和级数}&~H_n=\sum_{k=1}^n{\frac{1}{k}}&=\ln n+\gamma+\varepsilon_n\\
	\textrm{Leibniz formula for π}&~\sum_{k=0}^\infty{\frac{(-1)^k}{2k+1}}
	&=\arctan 1=\frac{\pi}{4}\\
	\textrm{Wallis' product}&~\prod_{k=1}^\infty{\frac{2k}{2k-1}
		\cdot\frac{2k}{2k+1}}&=\frac{\pi}{2}\\
	\textrm{Basel problem}&~\zeta(2)=\sum_{k=1}^\infty{\frac{1}{k^2}}
	&=\frac{\pi^2}{6}
\end{eqnarray*}
\subsection{调和级数应用}
$\varepsilon_n\approx\frac{1}{2n}$,
欧拉-马歇罗尼常数$\gamma\approx 0.5772156649\ldots$。
\index{*Constant!$\gamma\approx 0.5772156649\ldots$}
\subsubsection{$O(n\ln n)$dp}
\begin{itemize}
	\item 两层循环分别枚举区间长度和左端点。
	\item Eratosthenes筛法:枚举因子和因子的倍数。
\end{itemize}
\subsubsection{书籍堆叠问题}
有$n$本长度为$l$,质量分布均匀的书,将书叠成一摞,放在桌边,求书最多能伸出多长。

首先书籍从下到上肯定是不断伸出的,从上往下编号,设第$i-1$本书比第$i$本书多伸长$x_i$,
记最上面的书为第$0$本书,桌子为第$n$本书。

若要使书不倒下,需满足上面的书的重心在当前书上,
即
\begin{displaymath}
	\frac{\sum_{i=1}^k{\sum_{j=i}^k{x_i}}}{k}=\frac{\sum_{i=1}^k{ix_i}}{k}\leq \frac{l}{2},k=1,2,\cdots,n
\end{displaymath}

当$x_i=\frac{\frac{l}{2}}{i}$时等号恒成立,此时
伸长量$\displaystyle L=\sum_{i=1}^n{x_i}=\frac{l}{2}H_n$。

注意在计算调和级数时,小规模用暴力,大规模用$O(1)$近似。

如果允许在一层上放多本书,则最大伸出量与$\sqrt[3]{n}$成正比。

该问题源自Wikipedia-EN\footnote{Block-stacking problem - Wikipedia
	\url{https://en.wikipedia.org/wiki/Block-stacking\_problem}}。
\subsubsection{吉普车问题}

给定$n$个单位的燃料,吉普车只能携带$1$单位的燃料,$1$单位燃料可行驶$1$单位距离,
吉普车可以在沙漠的任意地方留下燃料,最大化最后一次的行驶距离。

该问题有两种类型:
\begin{itemize}
    \item 探索沙漠:最后一次要返回基地,答案为$H_n$。
    \item 穿越沙漠:最后一次不返回基地,答案为$\displaystyle
    \sum_{k=1}^n{\frac{1}{2k-1}}=H_{2n-1}-\frac{1}{2}H_{n-1}$。
\end{itemize}

该问题源自Wikipedia-EN\footnote{Jeep problem - Wikipedia
	\url{https://en.wikipedia.org/wiki/Jeep\_problem}}。
\subsubsection{蚂蚁在橡胶绳上}

一只蚂蚁在$1km$长的橡胶绳上以$1cm/s$的速度爬行,同时绳子以$1km/s$的速度拉伸,
蚂蚁是否可以到达绳子的另一端?

答案为是,可用离散方法或积分方法解决,时间$T=\frac{c}{v}(e^{\frac{v}{\alpha}}-1)$。
$c$为初始绳长,$v$为伸长速度,$\alpha$为蚂蚁运动速度。

该问题源自Wikipedia-EN\footnote{Ant on a rubber rope - Wikipedia
	\url{https://en.wikipedia.org/wiki/Ant\_on\_a\_rubber\_rope}}。

以上内容参考了Wikipedia-EN\footnote{Harmonic series (mathematics) - Wikipedia
	\url{https://en.wikipedia.org/wiki/Harmonic\_series\_(mathematics)}}。
