\section{概率与期望}
\subsection{全概率公式}
\begin{theorem}\label{FP}
    若事件$A$可分为独立事件$A_1,A_2,\cdots,A_n$,则
    \begin{displaymath}
        P(B)=\sum_{i=1}^n{P(B|A_i)P(A_i)}
    \end{displaymath}
\end{theorem}
\subsection{贝叶斯定理}
由\begin{displaymath}
    P(A\cap B)=P(B)P(A|B)=P(A)P(B|A)
\end{displaymath}
可得贝叶斯定理
\begin{theorem}
    \begin{displaymath}
        P(A|B)=\frac{P(A)P(B|A)}{P(B)}
    \end{displaymath}
\end{theorem}
结合定理~\ref{FP}得
\begin{inference}
    \begin{displaymath}
        P(A_i|B)=\frac{P(A_i)P(B|A_i)}{\displaystyle \sum_{j=1}^n{P(B|A_j)P(A_j)}}
    \end{displaymath}
\end{inference}
\subsection{期望}
\begin{theorem}[期望的线性性质]
    $E[X+Y]=E[X]+E[Y]$
\end{theorem}
\begin{theorem}
    $E[aX]=aE[X]$
\end{theorem}
\begin{theorem}
    若随机变量X,Y独立且期望$E[XY]$有定义时,$E[XY]=E[X]E[Y]$。
\end{theorem}
有一个十分常用的优化:
\begin{displaymath}
    E[X]=\sum_{i=0}^n{i\cdot P(X=i)}
    =\sum_{i=0}^n{i(P(X\geq i)-P(X\geq i+1))}
    =\sum_{i=1}^n{P(X\geq i)}
\end{displaymath}
\index{J!Jensen's Inequality}
\begin{theorem}[Jensen's Inequality]
    若函数$f(x)$为凸函数(即对于任意$x,y$和$\lambda\in [0,1]$,有
    $f(\lambda x+(1-\lambda)y)\geq\lambda f(x)+(1-\lambda)f(y)$)
    ,则$E[f(X)]\geq f(E[X])$。
\end{theorem}
\begin{theorem}
    对于在$[0,1]$上均匀分布的随机变量$X$,$n$个随机变量的第$k$小值
    的期望为$\frac{k}{n+1}$。
\end{theorem}
证明:
随机变量$X$的概率分布函数cdf(Cumulative Distribution Function
\index{C!Cumulative Distribution\\ Function})
为$\displaystyle cdf(x)=P(X\leq x)=\int_0^1{pdf(x)\ud x}$。
对其求导得到概率密度函数pdf(Probability Density Function)
\index{P!Probability Density Function}:
\begin{displaymath}
    pdf(x)=P(X=x)=k\binomial{n}{k}x^{k-1}(1-x)^{n-k}
\end{displaymath}
乘上随机变量后积分即为期望:
\begin{displaymath}
    \int_0^1{xpdf(x)\ud x}=\frac{k}{n+1}
\end{displaymath}
对于求解其它连续区间的期望问题也是这个思路。
\subsubsection{随机变量的方差}
$Var[X]$表示随机变量$X$的方差。
\begin{theorem}
    若随机变量$X$的均值为$E[X]$,则有$E[X^2]=Var[X]+E^2[X]$。
\end{theorem}
\begin{theorem}
    $Var[aX]=a^2Var[X]$
\end{theorem}
\begin{theorem}
    若随机变量$X_1,X_2,\cdots,X_n$两两独立,则有
    \begin{displaymath}
        Var[\sum_{i=1}^n{X_i}]=\sum_{i=1}^n{Var[X_i]}
    \end{displaymath}
\end{theorem}
\section{高斯消元求期望}
对于一般的图,可以列出每个点期望之间的线性关系,构造出
方程组后高斯消元求解。注意要判断矩阵是否为稀疏矩阵,若为稀疏矩阵,继续研究其特殊性质,
一般可以不断地递推求解一元一次方程得到所有解。

求{\bfseries 树}上期望时,一般的套路是将方程表达为$dp[u]=k\cdot dp[p]+b$
的形式,自底向上把方程推到根后再从根往下推数值解。
