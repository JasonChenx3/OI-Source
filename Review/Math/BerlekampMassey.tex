\section{Berlekamp–Massey算法}
\index{B!Berlekamp–Massey\\ Algorithm}
BM算法用来求一个序列的最短线性递推式。

BM算法维护当前的线性递推式,按顺序将元素加入序列,若不满足当前线性递推式则
根据历史信息矫正线性递推式。

具体算法如下(使用生成函数$A(x)$来表示递推式,下标从1开始,常数项表示当前项的系数,
最后要被缩放到-1):
\begin{enumerate}
    \item 维护当前递推式$A(x)$,上次失配时的递推式$B(x)$,以及上次失配时
    $B(x)$的值与失配位置值的差值$ld$,当前位置与上次失配位置的偏移$off$。
    \item 计算$A(x)$在当前位置的值:
    \begin{itemize}
        \item 若与序列的值相等,则迭代检查下一项。
        \item 否则记当前差值为$cd$,矫正多项式为$A'(x)=A(x)-\frac{cd}{ld}x^{off}B(x)$。
    \end{itemize}
\end{enumerate}

接下来证明算法正确性:

\begin{itemize}
    \item 当前位置满足递推式$A'(x)$:注意到$x^{off}$项将$B(x)$的取值移动到了上次
    失配位置,差值为$ld$,而递推式$A(x)$的差值为$cd$,所以$A'(x)$的差值被消为0。
    \item 先前位置满足递推式$A'(x)$:首先所有位置满足$A(x)$;因为
    $[1,\textrm{上次失配位置})$的位置满足递推式$B(x)$,所以
    $[off+1,off+\textrm{上次失配位置})$的位置满足递推式$x^{off}B(x)$;
    而$[1,off]$的位置全部作为递推式$x^{off}B(x)$的递推基础,也满足递推式。
\end{itemize}

注意由于$ld$仅在失配时改变,必定存在乘法逆元(实现时仅存储其乘法逆元,然后从0阶递推式开始,
初始$ld$任选一个非0值)。

代码如下:
\lstinputlisting{Source/Templates/BerlekampMassey.cpp}

算法最坏时间复杂度$O(n^2)$,但该算法似乎只能保证给出可行解,是否能给出最短线性递推式
有待考证。
\index{*TODO!验证BM算法的最优性质}

BM算法的应用还不太多,一般用来求DP值的线性递推式,然后搭配线性递推快速求出第$n$项的值。

上述内容参考了Rayment\_cc的博客\footnote{
    Berlekamp–Massey算法简要介绍\\
    \url{https://blog.csdn.net/As\_A\_Kid/article/details/86286891}
}。

测试数据来自fjzzq2002的博客\footnote{
    Berlekamp-Massey算法简单介绍\\
    \url{https://www.cnblogs.com/zzqsblog/p/6877339.html}
}。
