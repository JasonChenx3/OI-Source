\section{反演}
\subsection{反演定义}
若数列${f_n}$与${g_n}$满足
\begin{displaymath}
	f_n=\sum_{i=0}^n{a_{ni}g_i}
\end{displaymath}
反演就是已知数列${f_n}$(函数较简单),
求数列${g_n}$的过程(关键是要推出$b_{ni}$):
\begin{displaymath}
	g_n=\sum_{i=0}^n{b_{ni}f_i}
\end{displaymath}
这其实是一个求解线性方程组的过程。
\subsection{二项式反演}\label{BI}
\begin{theorem}
	\begin{displaymath}
		f(n)=\sum_{i=0}^n{(-1)^i\binomial{n}{i}g(i)}
		\Leftrightarrow g(n)=\sum_{i=0}^n{(-1)^i\binomial{n}{i}f(i)}
	\end{displaymath}
\end{theorem}
使用容斥证明:

设集合$S$中拥有性质$P_1,P_2,\cdots,P_n$的集合分别为$A_1,A_2,\cdots,A_n$,
根据定理~\ref{ExDML}可得不具有这$n$个性质的对象的集合大小为
\begin{displaymath}
	f(n)=|S|+\sum_{\emptyset \neq J\subseteq{1,2,\cdots,n}}
	{(-1)^{|J|}\left|\bigcap_{j\in J}{A_j}\right|}
\end{displaymath}
若集合$A_1,A_2,\cdots,A_n$满足任意$i$个集合的并集大小相等,记为$g(i)$,
定义$g(0)=|S|$,有
\begin{displaymath}
	f(n)=\sum_{i=0}^n{(-1)^i \binomial{n}{i}g(i)}
\end{displaymath}
同样对$g(i)$使用容斥可以得到右式。
\begin{inference}\label{BII}
	\begin{displaymath}
		f(n)=\sum_{i=0}^n{\binomial{n}{i}g(i)}
		\Leftrightarrow g(n)=\sum_{i=0}^n{(-1)^{n-i}\binomial{n}{i}f(i)}
	\end{displaymath}
\end{inference}
把$g(i)$代入后把外面的$(-1)^i$丢进去即可。
\subsubsection{错位排序问题}
求$n$个人均站错位置的方案数。

记$f(n)$为$n$个人任意站的方案数,$g(n)$为$n$个人都站错的方案数。

显然$f(n)=n!$且$\displaystyle f(n)=\sum_{i=0}^n{\binomial{n}{i} g(i)}$,
由推论~\ref{BII}得
\begin{eqnarray*}
	g(n)&=&\sum{i=0}^n{(-1)^{n-i}\binomial{n}{i}i!}\\
	&=&\sum_{i=0}^n{(-1)^(n-i)\frac{n!}{(n-i)!}}\\
	&=&n!\cdot\sum_{i=0}^n{\frac{(-1)^i}{i!}}
\end{eqnarray*}
\subsubsection{球染色问题}
求用$k$种颜色给$n$个排成一排的球染色,满足相邻球不同色且必须用上所有颜色的方案数。

记$f(k)$为使用$k$种颜色,相邻球不同色,不要求用上所有颜色的染色方案数,
$g(k)$为使用$k$种颜色的方案数。

那么有$f(k)=k(k-1)^{n-1}$且$f(k)=\sum_{i=0}^k{\binomial{k}{i}g(i)}$。

同理可得$g(k)=\sum_{i=2}^k{(-1)^{k-i}\binomial{k}{i}k(k-1)^{n-1}}$。

\subsection{斯特林反演}
\begin{theorem}
    \begin{displaymath}
        f(n)=\sum_{i=1}^n{\stirlingB{n}{i}g(i)}
        \Leftrightarrow
        g(n)=\sum_{i=1}^n{(-1)^{n-i}\stirlingA{n}{i}f(i)}
    \end{displaymath}
\end{theorem}
\subsection{子集反演}
\begin{theorem}
	\begin{displaymath}
		f(S)=\sum_{T\subseteq S}{(-1)^{|T|}g(T)}
		\Leftrightarrow
		g(S)=\sum_{T\subseteq S}{(-1)^{|T|}f(T)}
	\end{displaymath}
\end{theorem}
\begin{inference}
	\begin{displaymath}
		f(S)=\sum_{T\subseteq S}{g(T)}
		\Leftrightarrow
		g(S)=\sum_{T\subseteq S}{(-1)^{|S|-|T|}f(T)}
	\end{displaymath}
\end{inference}
证明同~\ref{BI}节所述。
\subsection{多重子集反演}
一个数的质因数分解可以对应一个多重子集,莫比乌斯反演即可。

以上内容参考了vfleaking的幻灯片
\footnote{炫酷反演魔术 \url{http://vfleaking.blog.uoj.ac/slide/87\#}}
和Miskcoo的博客\footnote{反演魔术:反演原理及二项式反演 – Miskcoo's Space.
	\url{http://blog.miskcoo.com/2015/12/inversion-magic-binomial-inversion}
}。
