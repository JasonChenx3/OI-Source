\section{反演}
\subsection{反演定义}
若数列${f_n}$与${g_n}$满足
\begin{displaymath}
	f_n=\sum_{i=0}^n{a_{ni}g_i}
\end{displaymath}
反演就是已知数列${f_n}$(函数较简单),
求数列${g_n}$的过程(关键是要推出$b_{ni}$):
\begin{displaymath}
	g_n=\sum_{i=0}^n{b_{ni}f_i}
\end{displaymath}
这其实是一个求解线性方程组的过程。
\subsection{二项式反演}\label{BI}
\begin{theorem}
	\begin{displaymath}
		f(n)=\sum_{i=0}^n{(-1)^i\binomial{n}{i}g(i)}
		\Leftrightarrow g(n)=\sum_{i=0}^n{(-1)^i\binomial{n}{i}f(i)}
	\end{displaymath}
\end{theorem}
使用容斥证明:

设集合$S$中拥有性质$P_1,P_2,\cdots,P_n$的集合分别为$A_1,A_2,\cdots,A_n$,
根据定理~\ref{ExDML}可得不具有这$n$个性质的对象的集合大小为
\begin{displaymath}
	f(n)=|S|+\sum_{\emptyset \neq J\subseteq{1,2,\cdots,n}}
	{(-1)^{|J|}\left|\bigcap_{j\in J}{A_j}\right|}
\end{displaymath}
若集合$A_1,A_2,\cdots,A_n$满足任意$i$个集合的并集大小相等,记为$g(i)$,
定义$g(0)=|S|$,有
\begin{displaymath}
	f(n)=\sum_{i=0}^n{(-1)^i \binomial{n}{i}g(i)}
\end{displaymath}
同样对$g(i)$使用容斥可以得到右式。
\begin{inference}\label{BII}
	\begin{displaymath}
		f(n)=\sum_{i=0}^n{\binomial{n}{i}g(i)}
		\Leftrightarrow g(n)=\sum_{i=0}^n{(-1)^{n-i}\binomial{n}{i}f(i)}
	\end{displaymath}
\end{inference}
把$g(i)$代入后把外面的$(-1)^i$丢进去可证。
\subsubsection{错位排序问题}
求$n$个人均站错位置的方案数。

记$f(n)$为$n$个人任意站的方案数,$g(n)$为$n$个人都站错的方案数。

显然$f(n)=n!$且$\displaystyle f(n)=\sum_{i=0}^n{\binomial{n}{i} g(i)}$,
由推论~\ref{BII}得
\begin{eqnarray*}
	g(n)&=&\sum_{i=0}^n{(-1)^{n-i}\binomial{n}{i}i!}\\
	&=&\sum_{i=0}^n{(-1)^{n-i}\frac{n!}{(n-i)!}}\\
	&=&n!\cdot\sum_{i=0}^n{\frac{(-1)^i}{i!}}
\end{eqnarray*}
\subsubsection{球染色问题}
求用$k$种颜色给$n$个排成一排的球染色,满足相邻球不同色且必须用上所有颜色的方案数。

记$f(k)$为使用$k$种颜色,相邻球不同色,不要求用上所有颜色的染色方案数,
$g(k)$为使用$k$种颜色的方案数。

那么有$f(k)=k(k-1)^{n-1}$且$\displaystyle f(k)=\sum_{i=0}^k{\binomial{k}{i}g(i)}$。

同理可得$\displaystyle g(k)=\sum_{i=2}^k{(-1)^{k-i}\binomial{k}{i}k(k-1)^{n-1}}$。

\subsection{斯特林反演}
\begin{theorem}
    \begin{displaymath}
        f(n)=\sum_{i=1}^n{\stirlingB{n}{i}g(i)}
        \Leftrightarrow
        g(n)=\sum_{i=1}^n{(-1)^{n-i}\stirlingA{n}{i}f(i)}
    \end{displaymath}
\end{theorem}
\subsection{子集反演}
\begin{theorem}
	\begin{displaymath}
		f(S)=\sum_{T\subseteq S}{(-1)^{|T|}g(T)}
		\Leftrightarrow
		g(S)=\sum_{T\subseteq S}{(-1)^{|T|}f(T)}
	\end{displaymath}
\end{theorem}
\begin{inference}
	\begin{displaymath}
		f(S)=\sum_{T\subseteq S}{g(T)}
		\Leftrightarrow
		g(S)=\sum_{T\subseteq S}{(-1)^{|S|-|T|}f(T)}
	\end{displaymath}
\end{inference}
证明同~\ref{BI}节所述。
\subsection{多重子集反演}
一个数的质因数分解可以对应一个多重子集,考虑莫比乌斯反演。

以上内容参考了vfleaking的幻灯片
\footnote{炫酷反演魔术 \url{http://vfleaking.blog.uoj.ac/slide/87}}
和Miskcoo的博客\footnote{反演魔术:反演原理及二项式反演 – Miskcoo's Space.\\
	\url{http://blog.miskcoo.com/2015/12/inversion-magic-binomial-inversion}
}。
\subsection{最值反演(minmax容斥)}
\subsubsection{一般形式}
记集合为$S$,$max(S)$为集合$S$的最大元素,$min(S)$为集合$S$的最小元素。
实际上这两个函数可推广为关于最值的函数。

最值反演的公式如下:
\begin{eqnarray*}
	max(S)&=&\sum_{T\subseteq S}{(-1)^{|T|+1}min(T)}\\
	min(S)&=&\sum_{T\subseteq S}{(-1)^{|T|+1}max(T)}
\end{eqnarray*}

证明(以第一个式子为例):记$x=max(S)$(如果集合可重,就给相同元素编号使其不相等,
下面仅考虑不可重集合的情况)。当且仅当$T=\{x\}$时贡献才含有$x$项。对于$T\neq \{x\}$,
设$y=min(T)$,设$S$中$\geq y$的元素有$k$个,有$k>1$,由于在这$k$个中选择奇数个与
选择偶数个的方案数相同(使用二项式定理可证明),最终$y$的贡献会被抵消。

\subsubsection{期望形式}
由期望的线性性质可以推出在$max(S/T)$与$min(S/T)$外再套上一层
期望后等式仍然成立。

一般的套路是每位都有一定的概率从0变为1,求全部变为1的期望步数。
最值反演后转换为至少1位变为1的期望步数。

\paragraph{例题~「HAOI2015」按位或}
记$max(S)$为状态S中最晚出现的1的出现的期望时间,$min(S)$为状态S中最早出现
的1的出现期望时间。$min(S)$很容易求得其表达式,考虑与$S$的交不为空集的$T$,表达式为
$min(S)=\frac{1}{\displaystyle \sum_{T\cap S \neq \emptyset}{P_T}}$(根据
伯努利试验中几何分布的期望求得,参见第~\ref{Bernoulli}节)。
这个式子仍然不好求,可以考虑补集转换,即考虑$T$与$S$的交为空集的情况。这个条件蕴含了
$T\subseteq C_US$,使用第~\ref{FMT}节的FMT可快速求出。

代码:
\lstinputlisting{Source/Source/MinMax/LOJ2127.cpp}

\subsubsection{推广}
$S$的第$k$大$max_k(S)$满足如下等式:

\begin{displaymath}
	max_k(S)=\sum_{T\subseteq S}{(-1)^{|T|+k}\binomial{|T|-1}{k-1}min(T)}
\end{displaymath}

证明留坑待补。
\index{*TODO!最值反演推广证明}

上述内容参考了Mr\_Spade的博客\footnote{
	min-max容斥/最值反演及其推广
	\url{http://www.cnblogs.com/Mr-Spade/p/9636968.html}
}。
\subsection{单位根反演}
单位根反演用来解决已知生成函数$F(x)=\displaystyle \sum_{i=0}^n{f(i)x^i}$,
求$\displaystyle \sum_{i=0}^n{f(i)[k|(i+b)]}$,$k,b$为定值。

考虑$b$为0的情况,有下列定理:
\begin{theorem}
	\begin{displaymath}
		[k|i]=\frac{1}{k}\sum_{j=0}^{k-1}{\omega_k^{ij}}
	\end{displaymath}
\end{theorem}

证明:与IDFT的证明类似,若$[k\nmid i]$,则根据求和引理~\ref{FFTL4},该式的值为0。
否则$\omega_k^{ij}$恒为1,该式的值为1。

该定理在模意义下仍然成立。\CJKsout{一般来说模意义下的给定的k都是2的幂,因为给定模数欧拉
函数值要求2的幂次足够大。}

接下来考虑将要求值的式子展开,有$\displaystyle \sum_{i=0}^n{f(i)\cdot \frac{1}{k}
\sum_{j=0}^{k-1}{\omega_k^{ij}}}$,交换求值顺序化简为$\frac{1}{k}\displaystyle
\sum_{i=0}^{k-1}{F(\omega_k^i)}$。需要用到单位根反演的场合肯定无法快速计算$F$的每一项
系数,但是容易快速求出$F(x)$,目前遇到的题目基本上结合二项式展开快速求出$F(x)$的值。对于
$b\neq 0$的情况,可以将$F(x)$乘以$x^{k-b}$平移。

上述内容参考了czyhe的博客\footnote{
	单位根反演
\url{https://czyhe.me/algorithm/unit-root-inv/unit-root-inv/}}。
\subsection{拉格朗日反演}\label{LIT}
\index{L!Lagrange Inversion Theorem}
如果要求某个函数的反函数的某一项,可以使用拉格朗日反演:
\begin{theorem}
	若多项式$F(x),G(x)$都没有常数项且一次项系数互为逆元,满足$F(G(x))=x$,
	则\begin{displaymath}
		[x_n]G(x)=\frac{1}{n}[x^{n-1}](\frac{x}{F(x)})^n
	\end{displaymath}
\end{theorem}
\subsubsection{扩展拉格朗日反演}
\begin{theorem}
	\begin{displaymath}
		[x_n]H(G(x))=\frac{1}{n}[x^{n-1}]H'(x)(\frac{x}{F(x)})^n
	\end{displaymath}
\end{theorem}
证明留坑待补。

这两个定理一般用在生成函数解决图的计数问题中。

上述内容参考了ZJT的博客\footnote{
	拉格朗日反演
	\url{http://zjt-blog.cc/articles/63}
}。
