\section{生成图计数}
鉴于生成函数在生成图计数中的应用广泛且逐渐普及,另开一节记录生成图计数
的要点。

下面的内容参考了WC2019汪乐平的讲义《生成函数,多项式算法与图的计数》、
WC2017营员交流汪乐平的课件《仙人掌计数》以及2015年国家集训队论文集中金策的
《生成函数的运算与组合计数问题》。
\CJKsout{当时我在现场冬眠}
\subsection{基本要点}
\subsubsection{各类生成函数的用途}
\begin{itemize}
    \item 普通生成函数通常用在无标号的计数中。
    \item 指数生成函数通常用在有标号的计数中。
    \item 特殊生成函数(SGF)的表达式如下:
    \begin{displaymath}
        G(x)=\sum_{n=0}^\infty{\frac{a_n}{2^{\binomial{n}{2}}n!}x^n}
    \end{displaymath}

    一般用于有标号有向图的计数。$2^{\binomial{n}{2}}$用于确定任意两点之间是否连边。
\end{itemize}
\subsubsection{是否连通?}
\paragraph{无标号集合计数}
假设我们知道了元素集合组成的OGF为$A(x)$,那么$n$个元素组合对应的OGF为$A(x)^n$,
答案即为$\displaystyle \sum_{n=0}^\infty{A(x)^n}=\frac{1}{1-A(x)}$。
使用多项式求逆解决。
\paragraph{有标号集合计数}
假设我们知道了元素集合组成的EGF为$A(x)$,那么$n$个元素排列对应的EGF为$A(x)^n$,
由于集合内无顺序关系,要再除以$n!$。答案即为
$\displaystyle \sum_{n=0}^\infty{\frac{A(x)^n}{n!}}=e^{A(x)}$。
使用多项式exp/ln解决。
\paragraph{实际问题转化}
通常可以将元素视为单个连通图,而将集合视为多个连通图的组合。
\subsubsection{有无标号?}
有标号与无标号的方案数只差$n!$。
如果仅仅是有根与无根的区别,系数差$n$。
一般先计算有标号/根的方案数,再推回无标号/根。
\subsubsection{复合逆}
有些题目仅仅要求生成函数的某一项,有时会推出符合逆的式子,使用~\ref{LIT}所述的
拉格朗日反演解决。
\subsection{实例}
\subsubsection{有标号森林}
因为森林是由多棵树组成的,因此先考虑有标号树的EGF。

这是一个完全图生成树问题,有两种解法:
\begin{itemize}
    \item 根据~\ref{MatrixTree}所述的矩阵树定理计算行列式。
    \item 根据~\ref{PurferSeq}所述的Purfer序列可知,一个Purfer序列
    唯一对应一棵有标号树,序列长度为$n-2$,因此方案数为$n^{n-2}$。
\end{itemize}

那么有标号树的EGF为$G(x)=\displaystyle \sum_{n=0}^\infty{\frac{n^{n-2}}{n!}x^n}$。

将森林视为树的集合,有标号森林的EGF为$F(x)=e^{G(x)}$。
\subsubsection{有标号无向连通图}
多个有标号无向连通图可组合为有标号无向简单图,考虑有标号无向简单图EGF的计算。

枚举完全图的每条边是否选中,大小为$n$的无向简单图个数为$2^{\binomial{n}{2}}$。
那么EGF为$G(x)=\displaystyle \sum_{n=0}^\infty{\frac{2^{\binomial{n}{2}}}{n!}x^n}$。

将简单图视为连通图的集合,记有标号无向连通图的EGF为$F(x)$,有$G(x)=e^{F(x)}$,
解得$F(x)=\ln G(x)$。
\subsubsection{有标号仙人掌}
无根仙人掌不太好去重,考虑计算有根仙人掌的个数,根或者仙人掌不同都视为不同的有根仙人掌。
记有根仙人掌的EGF为$G(x)$,无根仙人掌的EGF为$F(x)=\int \frac{G(x)}{x} \ud x$
\CJKsout{(说白了就是对应系数除以n)}。

接下来考虑有根仙人掌的计数。删掉从根出发的边,讨论每条边是否在环上,以及对应子连通块的EGF。
\begin{itemize}
    \item 边不在环上:删掉这条边后,分离出的连通块也是一棵仙人掌,以边的另一个
    端点为子仙人掌的根,对应了一棵新的仙人掌,EGF为$G(x)$。
    \item 边在环上:显然要删去2条边才能分离出子连通块,将其一起考虑。设环上删去根节点后
    有$i$个点($i\geq 2$),以每个点为根都有一棵仙人掌(不往环上的点走),这些仙人掌串成
    了一条链,由于链可以翻转,EGF为$\frac{G(x)^i}{2}$。
\end{itemize}

考虑删去这条边后子连通块的EGF,即上面分析出的所有情况相加,有
\begin{displaymath}
    G(x)+\sum_{i=2}^\infty{\frac{G(x)^i}{2}}=
    \frac{2G(x)-G^2(x)}{2-2G(x)}
\end{displaymath}

将子连通块组合并加回根节点,有$G(x)=xe^{\frac{2G(x)-G^2(x)}{2-2G(x)}}$。
$G(x)$可以使用牛顿迭代法计算。
\subsubsection{有标号有向无环图}
\subsubsection{有标号环}
\subsubsection{无标号环}
\subsubsection{带边数限制的有标号无向连通图}
\subsubsection{有标号点双连通分量}
