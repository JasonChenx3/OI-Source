\section{拉格朗日插值}
\subsection{原理}
拉格朗日插值法的思路是将每个点对应一个基函数,
使得点$(x_i,y_i)$对应的基函数$F_i(x)$满足
\begin{displaymath}
F_i(x)=\left\{\begin{array}{ll}
    y_i& \textrm{~if~$x=x_i$}\\
    0&\textrm{otherwise}
\end{array}\right.
\end{displaymath}
如何构造$F_i(x)$呢?首先可以提出$y_i$令$F_i(x)=y_iG_i(x)$,
然后显然$G_i(x)$有$\displaystyle \prod_{j\neq i}{x-x_j}$项,
为了让$G_i(x_i)=1$,再除以$\displaystyle \prod_{j\neq i}{x_i-x_j}$。

综上所述,插值函数为
\begin{displaymath}
    F(x)=\sum_{i=1}^n{y_i\prod_{j\neq i}{\frac{x-x_j}{x_i-x_j}}}
\end{displaymath}
\subsection{插值多项式计算}
时间复杂度$\Theta(n^2)$,步骤如下:
\begin{enumerate}
    \item $O(n^2)$计算出多项式$\displaystyle F(x)=\prod_{i=1}^i{x-x_i}$;
    \item $\Theta(n^2)$计算每个$i$的多项式$\frac{F(x)}{x-x_i}$;
    \item $\Theta(n^2)$将每个多项式乘以系数$\frac{y_i}{\prod_{i\neq j}{x_i-x_j}}$
    (分母部分直接将$x_i$带入$i$对应的分子的多项式求出);
    \item $\Theta(n^2)$将多项式相加合并。
\end{enumerate}
单点插值也可以使用这种优化。
\subsection{多点插值}
\begin{itemize}
    \item 预处理出多项式,然后每次使用霍纳法则$O(n)$求值;
    \item 预处理出每个$i$的$\displaystyle y_i
    \prod_{i\neq j}\frac{1}{x_i-x_j}$,每次$O(n^2)$求分子后相加。
\end{itemize}
\subsection{缺点}
\begin{itemize}
    \item 拉格朗日插值法的给出的曲线的确经过了样本点,但是这个曲线有可能十分曲折,而且
    受单个点的影响大。可以考虑使用最小二乘逼近来消除噪声,或者使用样条。
    \item $\Theta(n^2)$拉格朗日插值法的时间复杂度并没有比FFT插值方法优。
\end{itemize}
\subsection{求解自然数幂和}\label{psum}
易知$F_k(n)=\displaystyle \sum_{i=0}^n{i^k}$是一个$k+1$次多项式。
预处理出$F_k$在$1,\cdots,k+2$上的值,插值出多项式。
