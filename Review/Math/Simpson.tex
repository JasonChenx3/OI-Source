\section{Simpson积分}
\subsection{形式与推导}
Simpson积分是用二次函数来拟合等距三点$(l,f(l)),(m,f(m)),(r,f(r))\\
(m=\frac{l+r}{2})$来积分的。推导如下:
\begin{eqnarray*}
    \int_l^r{f(x)\ud x}&\approx&\int_l^r{(ax^2+bx+c)\ud x}\\
    &=&\frac{a}{3}(r^3-l^3)+\frac{b}{2}(r^2-l^2)+c(r-l)\\
    &=&\frac{r-l}{6}(2a(r^2+rl+l^2)+3b(r+l)+6c)\\
    &=&\frac{r-l}{6}(f(l)+4f(m)+f(r))
\end{eqnarray*}

\subsection{自适应Simpson}
当Simpson未能精确拟合函数值时,将其区间二分拟合。判断拟合程度
可以将当前积分与子区间积分之和相比较。

\begin{lstlisting}
typedef double FT;
const FT eps=1e-8;
FT f(FT x);
FT simpson(FT l,FT r,FT fl,FT fm,FT fr) {
    return (r-l)*(fl+4.0*fm+fr)/6.0;
}
FT SAAImpl(FT l,FT m,FT r,FT fl,FT fm,FT fr,FT sm) {
    FT lm=(l+m)*0.5,flm=f(lm),sl=simpson(l,m,fl,flm,fm);
    FT rm=(m+r)*0.5,frm=f(rm),sr=simpson(m,r,fm,frm,fr);
    FT esm=sl+sr;
    if(fabs(sm-esm)<eps)return esm;
    return SAAImpl(l,lm,m,fl,flm,fm,sl)+
    SAAImpl(m,rm,r,fm,frm,fr,sr);
}
FT SAA(FT l,FT r) {
    FT m=(l+r)*0.5;
    FT fl=f(l),fm=f(m),fr=f(r);
    return SAAImpl(l,m,r,fl,fm,fr,simpson(l,r,fl,fm,fr));
}
\end{lstlisting}

\paragraph{警告}
\begin{itemize}
    \item {\bfseries 最好不要在圆上做Simpson积分。}
    \item {\bfseries 注意初始定义域端点的位置(比如函数只在$[0,50]$有较大贡献,
其余区域贡献几乎为0,若对$[0,10000]$Simpson积分则误差较大,考虑分析函数图像后
倍增区间大小)。}
\end{itemize}
