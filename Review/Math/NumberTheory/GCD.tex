\subsection{辗转相除法GCD}
\subsubsection{裴蜀定理}
\index{B!Bézout's Theorem}
\begin{theorem}[Bézout's Theorem]\label{BT}
    对于任意$a,b\in \mathbb{Z}$,关于$x,y$的线性不定方程(裴蜀方程)
    $ax+by=c$有无穷多整数解$(x,y)$当且仅当$(a,b)|c$。特别地,
    一定存在$(x,y)$使得$ax+by=(a,b)$成立。
\end{theorem}

由此可得推论:

\begin{inference}
    $a,b$互质的充要条件是存在整数$(x,y)$使得$ax+by=1$。
\end{inference}

接下来证明一定存在$(x,y)$使得$ax+by=(a,b)$成立:

设$s$是$a$和$b$线性组合集中的最小正元素,对于某个整数组$(x,y)$有$ax+by=s$,
令$q=[a/s],r=a mod s=a-q(ax+by)=a(1-qx)+b(-qy)$,所以$r$也是一个线性组合。
因为$s$是线性组合集中的最小正元素,且$0\leq r \le s$,所以$r=0$,可得$s|a$。
同理$s|b$,因此$s$是$a,b$的公约数,可得$(a,b) \geq s$。因为$(a,b)|ax+by$
且$s>0$,所以$(a,b) \leq s$。结合$(a,b) \geq s$与$(a,b) \leq s$可得
$s=(a,b)$。

然后证明对于任意$a,b\in \mathbb{Z}$,$ax+by=c$有整数解
$(x,y) \Leftrightarrow (a,b)|c$:

充分性:

必要性:

至于无穷多整数解嘛。。。拿最小公倍数调一调初始解$(x,y)$即可。

证明参考了霜刃未曾试的博客\footnote{关于裴蜀定理的一些证明\\
\url{https://blog.csdn.net/discreeter/article/details/69833579}}与
算法导论\cite{ITA3}第31.1节定理31.2的证明。
\subsubsection{exgcd}
由定理~\ref{BT}可知一定存在整数解$(x,y)$满足$ax+by=(a,b)$,如何构造
出一组解呢?

$exgcd$(扩展欧几里得算法)可求出一组特殊的整数解。

$exgcd$构造出的解特殊性在于

\subsubsection{位运算gcd}
原理

位扫描优化

如果某个数末尾有多个0,则可以直接使用右移k位代替不断右移1位。
下面是统计末尾0的个数k的方法:

\begin{itemize}
    \item GCC自带了对位扫描指令的封装,即$\_\_builtin\_$系列函数,
    直接使用$\_\_builtin\_clz$函数即可。
    \item
\end{itemize}

实现
