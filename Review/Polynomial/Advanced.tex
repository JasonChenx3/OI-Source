\section{多项式高级算法}
{\bfseries 警告:慎卡常导致编码/调试困难。}

Update:多项式算法模块化封装已完成,参见~\ref{Module}节与~\ref{Optmize}节的内容。

\subsection{牛顿迭代法}
已知函数$G(z)$,求函数$F(z) \bmod{z^n}$满足$G(F(z))\equiv 0 \pmod{z^n}$。

当$n=1$时,多项式退化为常数,直接求解。

记$h=\left\lfloor\frac{n+1}{2}\right\rfloor$,
若已知$G(F_0(z)) \equiv 0\pmod{z^h}$,
尝试计算$G(F(z))$在$F_0(z)$处的泰勒展开:
\begin{displaymath}
    G(F(z))=\sum_{i=0}^\infty{\frac{G^{(i)}(F_0(z))}{i!}\cdot (F(z)-F_0(z))^i}
\end{displaymath}
可以发现$F(z)$与$F_0(z)$的后$h$项均相等,所以两多项式之差的$n\geq 2$次方多项式的
最小非0项次数$\geq n$,模$z^n$意义下无贡献,因此仅前两项展开有效,即
\begin{displaymath}
    G(F(z))\equiv G(F_0(z))+G'(F_0(z))(F(z)-F_0(z)) \pmod{z^n}
\end{displaymath}
结合$G(F(z))\equiv 0 \pmod{z^n}$可得到新的$F(z)$:
\begin{displaymath}
    F(z)\equiv F_0(z)-\frac{G(F_0(z))}{G'(F_0(z))} \pmod{z^n}
\end{displaymath}
这就是牛顿迭代法。

\subsubsection{注意事项}
\begin{itemize}
    \item 使用FFT加速卷积后及时取模,即把不需要的位置0。否则循环卷积性质会干扰低次项。
    当然也可以直接将规模对齐到2的幂次,甚至可以利用循环卷积优化,最后一次再取模。
    \item 卷积时使用$>2$倍模次数的2的幂作为卷积规模,因为乘法操作至少需要这么多项
    才足够确定多项式系数。
    \item 若要求$\bmod{z^n}$意义下的结果,求导/积分这类导致多项式次数变化的操作,
    显然仅把次数设为$n$会导致信息丢失,而且讨论每种操作的最高次数也很麻烦。不妨全部求
    $\bmod{z^{n+c}}$意义下的结果,$c$为足够大的小常数,最后直接截断输出。
\end{itemize}

\subsection{多项式开方}
对于给定的$A(z)$,求$F(z) \pmod{z^n}$,使得$F^2(z)\equiv A(z)\pmod{z^n}$。

构造方程$F^2(z)-A(z)\equiv 0\pmod{z^n}$,
同理可得
\begin{eqnarray*}
    F(z)&\equiv& F_0(z)-\frac{F_0(z)^2-A(z)}{2F_0(z)} \pmod{z^n}\\
    &\equiv& \frac{F_0(z)^2+A(z)}{2F_0(z)} \pmod{z^n}
\end{eqnarray*}

注意当$t=0$时可能需要用二次剩余在模意义下开根。

\subsection{多项式求逆}
{\bfseries 多项式求逆是基于牛顿迭代法的多项式算法的基本工具。}

对于给定的$A(z)$,求$F(z) \pmod{z^n}$,使得$F(z)\cdot A(z)\equiv 1\pmod{z^n}$。

构造方程$F(z)\cdot A(z)-1\equiv 0\pmod{z^n}$,
同理可得
\begin{eqnarray*}
    F(z)&\equiv& F_0(z)-\frac{F_0(z)A(z)-1}{A(z)} \pmod{z^n}\\
    &\equiv& F_0(z)-(F_0(z)A(z)-1){F(z)} \pmod{z^n}\\
    &\equiv& F_0(z)-(F_0(z)A(z)-1){F_0(z)} \pmod{z^n}~(F_0(z)A(z)-1\equiv 0\pmod{z^h})\\
    &\equiv& 2F_0(z)-F_0^2(z)A(z) \pmod{z^n}
\end{eqnarray*}
\subsection{多项式取模}
给定一个$n$次多项式$A(z)$与$m$次多项式$B(z)$($m\leq n$),
求多项式$D(z),R(z)$满足
\begin{displaymath}
    A(z)=D(z)B(z)+R(z),deg(D)\leq n-m,deg(R)<m
\end{displaymath}
记$n$次多项式$A(z)$的系数翻转$A^R(z)=z^nA(\frac{1}{z})$。

令$deg(D)=n-m,deg(R)=m-1$,不足高位补0。
将原方程的$z$全部换成$\frac{1}{z}$,然后乘上$z^n$,有
\begin{eqnarray*}
    z^nA(\frac{1}{z})&=&z^nD(\frac{1}{z})B(\frac{1}{z})+z^nR(\frac{1}{z})\\
    A^R(z)&=&D^R(z)B^R(z)+z^{n-m+1}R^R(z)
\end{eqnarray*}
由于$deg(D^R)\leq n-m$,所以可以在模$z^{n-m+1}$的情况下求解$D^R(z)$,
翻转后带入原方程求出$R(z)$。

{\bfseries 一定要想清楚每个多项式的次数,以及在模$x$的几次方下计算。}
\subsection{多项式求导与积分}
根据$(x^n)'=nx^{n-1}$可得
\begin{eqnarray*}
    F'(z)&=&\sum_{i=1}^{n-1}{ic_iz^{i-1}}\\
    \int F(z)\ud z&=&\sum_{i=1}^{n-1}{\frac{c_{i-1}}{i}\cdot z^i}
\end{eqnarray*}
时间复杂度$O(n)$。
\subsection{多项式ln}
考虑对$\ln A(z)$求导:
\begin{displaymath}
    (\ln A(z))'=\frac{A'(z)}{A(z)}
\end{displaymath}
所以有
\begin{displaymath}
    \ln A(z)=\int \frac{A'(z)}{A(z)}\ud z
\end{displaymath}
由于要求逆元,时间复杂度$O(n \lg n)$。
\subsection{多项式exp}
先进行如下变换:
\begin{displaymath}
    F(z)-e^{A(z)}\equiv 0 \pmod{z^n}
    \Rightarrow \ln F(z)-A(z)\equiv 0 \pmod{z^n}
\end{displaymath}
直接牛顿迭代法得
\begin{eqnarray*}
    F(z)&=&F_0(z)-(\ln F_0(z)-A(z))F_0(z)\\
    &=&F_0(z)(1-\ln F_0(z)+A(z))
\end{eqnarray*}

一般其输入的常数项为0,由麦克劳林级数展开式得常数项恒为1。
\subsection{多项式快速幂}
使用常规快速幂可以得到$O(n\lg n\lg k)$的复杂度。
但是通过如下变形:
\begin{displaymath}
    F^k(z)=e^{k \ln F(z)}
\end{displaymath}
使用多项式ln/exp可以得到$O(n\lg n)$的复杂度。

{\bfseries 注意此种情况只适用于常数项为1的情况,其它情况需要缩放多项式。}
\subsection{多项式三角函数}
由欧拉公式可得
\begin{displaymath}
    e^{F(z)i}=\cos F(z)+\sin F(z) i
\end{displaymath}
在复数域上做多项式exp。

上述内容基本上使用牛顿迭代法递归求解,下面的模板(LOJ150 挑战多项式)涵盖了大部分
操作:
\lstinputlisting{Source/Source/'FFT NTT'/LOJ150.cpp}

\subsection{进制转换}
将$A$进制数转换为$B$进制数。

$A$进制数可表示为$\displaystyle \sum_{i=0}^n{a_iA^i}$,求出其在
$B$进制下的值。

对其分治,即
\begin{displaymath}
    \sum_{i=0}^n{a_iA^i}=\sum_{i=0}^{\frac{n}{2}-1}{a_iA^i}
    +A^{\frac{n}{2}}\sum_{i=0}^{\frac{n}{2}}{a_{i+\frac{n}{2}}A^i}
\end{displaymath}

预处理出$A^i$对应的$B$进制数,然后分治,同时使用FFT优化右边的卷积。

时间复杂度$O(n \lg^2 n)$。

\subsection{多项式多点求值}
已知$A(z)$与$n$个点$z_0,z_1,\cdots,z_{n-1}$,
求$A(z_0),A(z_1),\cdots,A(z_{n-1})$。

将要求的$x$分为两半,那么左右两半对应的插值多项式的次数为$[\frac{n}{2}]$,
求出这两个插值多项式后递归计算。

对于左半部分,考虑多项式$\displaystyle B(z)=\prod_{i=0}^
{[\frac{n}{2}]}(z-z_i)$,满足$deg(B)=[\frac{n}{2}]$。
令$A(z)=D(z)B(z)+R(z)$,当$z$为左半部分中的点时,$D(z)B(z)$为0,即
$A(z)=R(z)$。那么可以两边同时模$B(z)$,达到次数减半的效果。
右边部分类似。

{\bfseries 注意$B(z)$要$O(n\lg^2 n)$分治预处理,空间$O(n\lg n)$,递归树根的
多项式不必计算。}。

时间复杂度$O(n\lg^2n)$。

代码:
\lstinputlisting{Source/Templates/Evaluator.cpp}
\subsection{多项式多点插值}

对待插值点分治,假设求出了左半部分插值多项式$A_{left}(z)$,构造出左半部分的$B(z)$
使其满足$A(z)=A_{left}(z)+A_{right}(z)B(z)$,那么左半部分的点都将在$A(z)$上。
仅考虑右半部分:
\begin{displaymath}
    \forall(x_i,y_i)\in P_{right},y_i=A_{left}(x_i)+A_{right}(x_i)B(x_i)
\end{displaymath}
化简得
\begin{displaymath}
    A_{right}(x_i)=\frac{y_i-A_{left}(x_i)}{B(x_i)}
\end{displaymath}
那么可利用多点求值得到一组新的插值点,递归求解。时间复杂度$O(n \lg^3 n)$。

\subsection{组合数取模}

求$\binomial{n}{m} \bmod{p}$的值,其中$p$为素数,$n,m\leq 1e9$。

\begin{itemize}
    \item $p\leq 1e6$:线性预处理逆元+lucas;
    \item $n<p$:可知$n!$与$p$互质,因此只要求$n!$。
    可构造多项式:
    \begin{eqnarray*}
        Q(x)&=&\prod_{i=1}^{\sqrt{n}}{(x+i)}\\
        n!&=&\prod_{i=0}^{\sqrt{n}-1}{Q(i\sqrt{n})}
    \end{eqnarray*}
    卷积出$Q(x)$后在$i\sqrt{n}=0,\sqrt{n},\cdots,(\sqrt{n}-1)\sqrt{n}$上
    多点求值,时间复杂度\\$O(\sqrt{n}\lg^2 n)$。(在考场上不太好写,可以用一些预处理时间
    每隔一段打表,查询时定位到对应位置暴力计算,适用于$n$较大而查询很少的情况。)
    \item $n\geq p$:结合上述两种做法,易知第二种做法最多只执行一次。
\end{itemize}

\subsection{CDQ分治FFT}
已知函数$g$在$[1,n)$上的值,求函数$f(x)=\displaystyle \sum_{y=1}^x{f(x-y)g(y)}$
在$[1,n)$上的值,其中$f(0)=1$。

对于这种自我依赖的卷积,可以考虑将待求值一分为二,分治求解。{\bfseries 分治到小规模时
直接$O(n^2)$暴力求解。}

记卷积过程为$conv(l,r)$,$m$为$l,r$中点,首先递归调用$conv(l,m)$求出区间的前一半,
然后计算使用卷积累加前一半对后一半的贡献,然后递归调用$conv(m+1,r)$补足区间后一半的剩余项。
时间复杂度$O(n\lg^2 n)$。

事实上这就是CDQ分治的过程,思路参见第\ref{CDQ}节。

CDQ分治FFT时,需要对整个多项式进行平移,可能导致不知道该从哪个位置,哪个方向取
卷积结果。这时使用端点来计算对应的位置,找出对应规律。
\subsubsection{优化}
将$F$与$G$看做序列$f,g$的OGF,令$g(0)=0$,计算$F(x)\otimes G(x)=F(x)-f(0)x^0$,
变形为$F(x)\equiv \frac{1}{1-G(x)} \pmod{x^n}$,使用多项式求逆$O(n\lg n)$解决。

\subsection{多项式乘积(普通分治FFT)}
求多个多项式之积(积的次数上界为$W$)可以使用分治乘法解决,时间复杂度
$O(W\lg W\lg n)$。

\subsubsection{优化A} 由于求的是多项式乘积,因此在IDFT时不用做除法,而是记录除法因子,
在最终答案中做1次除法。

\subsubsection{优化B} 使用优先队列维护多项式次数,然后按照类似赫夫曼编码的策略
贪心合并多项式。参见kczno1的代码\footnote{\url{https://loj.ac/submission/110667}}。

\subsubsection{优化C} 若它为多个一次多项式相乘,则会出现大量小规模卷积,使用暴力解决。

以上内容参考了picks
\footnote{
Newton's Method of Polynomial $\ll$ Picks's Blog
\\\url{http://picks.logdown.com/posts/209226-newtons-method-of-polynomial}

Positional Notation Conversion $\ll$ Picks's Blog
\\\url{http://picks.logdown.com/posts/208342-positional-notation-conversion}

Binomial Coefficient Modulo Prime $\ll$ Picks's Blog
\\\url{http://picks.logdown.com/posts/245545-binomial-coefficient-modulo-prime}

}
、Miskcoo\footnote{
牛顿迭代法在多项式运算的应用 – Miskcoo's Space
\\\url{http://blog.miskcoo.com/2015/06/polynomial-with-newton-method}

多项式除法及求模 – Miskcoo's Space
\\\url{http://blog.miskcoo.com/2015/05/polynomial-division}

多项式的多点求值与快速插值 – Miskcoo's Space
\\\url{http://blog.miskcoo.com/2015/05/polynomial-multipoint-eval-and-interpolation}
}和VictoryCzt\footnote{
    分治FFT学习笔记
    \url{https://blog.csdn.net/VictoryCzt/article/details/82939586}
}的博客。

\subsection{二元多项式卷积}
该需求源自2015年国家集训队论文集中金策的《生成函数的运算与组合计数问题》,
用于解决二元生成函数的卷积。

将二元函数的系数排为矩阵,对每一行做DFT,再对每一列做DFT,将点值相乘,
对每一列做IDFT,每一行做IDFT。整个过程与DFT计算卷积的过程类似。记多项式次数
为$n,m$,时间复杂度为$O(nm\lg nm)$。
\subsection{循环卷积}
该需求来自CTSC2010 性能优化,题意为求$ab^C$在模$x^n$意义下的循环卷积模$n+1$的值。

使用常用的基2NTT,必须把卷积规模开到$2n$,然后每次把溢出部分加回去。即使使用快速幂也
带来$O(\lg n)$的复杂度。

不过NTT可以使用混合基,例题有一个性质,即$n$是smooth number,可以分解为
$2^a3^b5^c7^d$的形式。那么可以按照当时FFT的推导写出基2,基3,基5,基7的蝴蝶操作,
然后使用递归形式求解。

\index{*TODO!混合基FFT的非递归形式}

由于是循环卷积,求出点值后可以不考虑溢出,即无需限制点值乘法的次数,允许直接在点值上
做快速幂。

参考代码:
\lstinputlisting{Source/Source/'FFT NTT'/P4191.cpp}

该内容参考了skywalkert的博客\footnote{
    BZOJ 1919 [Ctsc2010]性能优化\\
    \url{https://blog.csdn.net/skywalkert/article/details/51737272}
}。

对于没有特殊性质的$n$,若要求模$x^n$意义下的循环卷积,可以使用Bluestein’s Algorithm。
参见国家集训队2016论文集毛啸的《再探快速傅里叶变换》。

\subsection{CZT}
记$\omega$为主$n$次单位根,DFT求的是:
\begin{displaymath}
    B_i=\sum_{k=0}^{n-1}{\omega^{ki}A_k}
\end{displaymath}

尝试将其变形:
\begin{eqnarray*}
    B_i&=&\sum_{k=0}^{n-1}{\omega^{ki}A_k}\\
    &=&\sum_{k=0}^{n-1}{\omega^{-\frac{(i-k)^2-i^2-k^2}{2}}A_k}\\
    &=&\omega^{-i^2/2}\sum_{k=0}^{n-1}{\omega^{-(i-k)^2/2}\omega^{-k^2/2}A_k}
\end{eqnarray*}

如此便将DFT表示为类似卷积的形式,唯一不足在于求和上下界与卷积不同。

继续变形,记$C_i=\omega^{-(i-n)^2/2}$:
\begin{eqnarray*}
    \omega^{i^2/2}a_i=D_{i+n}&=&\sum_{k=0}^{n-1}{\omega^{-(i-k)^2/2}\omega^{-k^2/2}A_k}\\
    &=&\sum_{k=0}^{n-1}{C_{i+n-k}\omega^{-k^2/2}A_k}\\
    &=&\sum_{k=0}^{i+n}{C_{i+n-k}\omega^{-k^2/2}A_k}\\
\end{eqnarray*}

该变形利用了$A_k$只在$[0,n-1]$处有贡献的性质。

如此便可以使用FFT计算循环卷积。

参考代码:
\lstinputlisting{Source/Templates/bzoj1919-CZT.cpp}

事实上CZT是DFT的广义形式,注意到$\omega$的单位根性质并没有被用到,可以
使用其它数代替。

该算法参考了国家集训队2016论文集毛啸的《再探快速傅里叶变换》。
