\section{多项式高级操作}
{\bfseries 警告:慎卡常导致编码/调试困难。}

{\bfseries 优化:分治到小规模时直接暴力。}
\subsection{牛顿迭代法}
已知函数$G(z)$,求函数$F(z) \bmod{z^n}$满足$G(F(z))\equiv 0 \pmod{z^n}$。

令$n=2^t$,若$n$不为2的幂,补齐后截断即可。

当$t=0$时,简单地令$F(z)$的常数项为0。

若已知$G(F_0(z)) \equiv 0\pmod{z^{2^t}}$,
可以计算$G(F(z))$在$F_0(z)$上的泰勒展开:
\begin{displaymath}
    G(F(z))=\sum_{i=0}^\infty{\frac{G^{(i)}(F_0(z))}{i!}\cdot (F(z)-F_0(z))^i}
\end{displaymath}
由于$F(z)$与$F_0(z)$后$2^t$项均相等,所以它们之差的平方的最小非0项次数$\geq 2^{t+1}$,
因此仅前两项有效,即
\begin{displaymath}
    G(F(z))\equiv G(F_0(z))+G'(F_0(z))(F(z)-F_0(z)) \pmod{z^{2^{t+1}}}
\end{displaymath}
结合$G(F(z))\equiv 0 \pmod{z^{2^{t+1}}}$可得到新的$F_0(z)$:
\begin{displaymath}
    F(z)\equiv F_0(z)-\frac{G(F_0(z))}{G'(F_0(z))} \pmod{z^{2^{t+1}}}
\end{displaymath}
这就是牛顿迭代法。
\subsection{多项式开方}
对于给定的$A(z)$,求$F(z) \pmod z^n$,使得$F^2(z)\equiv A(z)\pmod{z^n}$。

构造方程$F^2(z)-A(z)\equiv 0\pmod{z^n}$,
同理可得
\begin{eqnarray*}
    F(z)&\equiv& F_0(z)-\frac{F_0(z)^2-A(z)}{2F_0(z)} \pmod{z^{2^{t+1}}}\\
    &\equiv& \frac{F_0(z)^2+A(z)}{2F_0(z)} \pmod{z^{2^{t+1}}}
\end{eqnarray*}

注意当$t=0$时可能需要用二次剩余在模意义下开根。

\subsection{多项式求逆}
对于给定的$A(z)$,求$F(z) \pmod z^n$,使得$F(z)\cdot A(z)\equiv 1\pmod{z^n}$。

构造方程$F(z)\cdot A(z)-1\equiv 0\pmod{z^n}$,
同理可得
\begin{eqnarray*}
    F(z)&\equiv& F_0(z)-\frac{F_0(z)\cdot A(z)-1}{A(z)} \pmod{z^{2^{t+1}}}\\
    &\equiv& F_0(z)-(F_0(z)\cdot A(z)-1)\cdot{F_0(z)} \pmod{z^{2^{t+1}}}
    \textrm{~(考虑最小非0项可知可乘$F_0(z)$代替$F(z)$)}\\
    &\equiv& 2F_0(z)-F_0^2(z)\cdot A(z) \pmod{z^{2^{t+1}}}
\end{eqnarray*}
\subsection{多项式取模}
给定一个$n$次多项式$A(z)$与$m$次多项式$B(z)$($m\leq n$),
求多项式$D(z),R(z)$满足
\begin{displaymath}
    A(z)=D(z)B(z)+R(z),deg(D)\leq n-m,deg(R)<m
\end{displaymath}
记$n$次多项式$A(z)$的系数翻转$A^R(z)=z^nA(\frac{1}{z})$。

令$deg(D)=n-m,deg(R)=m-1$,不足高位补0。
将原方程的$z$全部换成$\frac{1}{z}$,然后乘上$z^n$,有
\begin{eqnarray*}
    z^nA(\frac{1}{z})&=&z^nD(\frac{1}{z})B(\frac{1}{z})+R(\frac{1}{z})\\
    A^R(z)&=&D^R(z)B^R(z)+z^{n-m+1}R^R(z)
\end{eqnarray*}
由于$deg(D^R)\leq n-m$,所以可以在模$z^(n-m+1)$的情况下求解$D^R(z)$,
翻转后带入原方程求出$R(z)$。

\subsection{多项式求导与积分}
根据$(x^n)'=nx^{n-1}$可得
\begin{eqnarray*}
    F'(z)&=&\sum_{i=1}^{n-1}{ic_iz^{i-1}}\\
    \int F(z)&=&\sum_{i=1}^{n-1}{\frac{c_{i-1}}{i}\cdot z^i}
\end{eqnarray*}
时间复杂度$O(n)$。
\subsection{多项式ln}
考虑对$\ln A(z)$求导:
\begin{displaymath}
    (\ln A(z))'=\frac{A'(z)}{A(z)}
\end{displaymath}
所以有
\begin{displaymath}
    \ln A(z)=\int \frac{A'(z)}{A(z)}
\end{displaymath}
由于要求逆元,时间复杂度$O(n \lg n)$。
\subsection{多项式exp}
先进行如下变换:
\begin{displaymath}
    F(z)-e^{A(z)}\equiv 0 \pmod{z^n}
    \Rightarrow \ln F(z)-A(z)\equiv 0 \pmod{z^n}
\end{displaymath}
直接牛顿迭代法得
\begin{eqnarray*}
    F(z)&=&F_0(z)-(\ln F_0(z)-A(z))F_0(z)\\
    &=&F_0(z)(1-\ln F_0(z)+A(z))
\end{eqnarray*}
\subsection{多项式快速幂}
使用常规快速幂可以得到$O(n\lg n\lg k)$的复杂度。
但是通过如下变形:
\begin{displaymath}
    F^k(z)=e^{k \ln F(z)}
\end{displaymath}
使用多项式ln/exp可以得到$O(n\lg n)$的复杂度。
\subsection{多项式三角函数}
由欧拉公式可得
\begin{displaymath}
    e^{F(z)i}=\cos F(z)+\sin F(z) i
\end{displaymath}
在复数域上做多项式exp即可。
\subsection{进制转换}
将$A$进制数转换为$B$进制数。

$A$进制数可表示为$\displaystyle \sum_{i=0}^n{a_iA^i}$,求出其在
$B$进制下的值。

对其分治,即
\begin{displaymath}
    \sum_{i=0}^n{a_iA^i}=\sum_{i=0}^{\frac{n}{2}-1}{a_iA^i}
    +A^{\frac{n}{2}}\sum_{i=0}^{\frac{n}{2}}{a_{i+\frac{n}{2}}A^i}
\end{displaymath}

预处理出$A^i$对应的$B$进制数,然后分治即可,FFT优化右边的卷积。

时间复杂度$O(n \lg n)$。

\subsection{多项式多点求值}
已知$A(z)$与$n$个点$z_0,z_1,\cdots,z_{n-1}$,
求$A(z_0),A(z_1),\cdots,A(z_{n-1})$。

将要求的$x$分为两半,那么左右两半对应的插值多项式的次数为$[\frac{n}{2}]$,
只要求出这两个插值多项式后递归计算即可。

对于左半部分,考虑多项式$\displaystyle B(z)=\prod_{i=0}^
{[\frac{n}{2}]}(z-z_i)$,满足$deg(B)=[\frac{n}{2}]$。
令$A(z)=D(z)B(z)+R(z)$,当$z$为左半部分中的点时,$D(z)B(z)$为0,即
$A(z)=R(z)$。那么可以两边同时模$B(z)$,达到次数减半的效果。
右边部分类似。

{\bfseries 注意$B(z)$要$O(n\lg^2 n)$分治预处理,空间$O(n\lg n)$}。

时间复杂度$O(n\lg^2n)$。
\subsection{多项式多点插值}

对待插值点分治,假设求出了左半部分插值多项式$A_{left}(z)$,构造出左半部分的$B(z)$
使其满足$A(z)=A_{left}(z)+A_{right}(z)B(z)$,那么左半部分的点都将在$A(z)$上。
仅考虑右半部分即可:
\begin{displaymath}
    \forall(x_i,y_i)\in P_{right},y_i=A_{left}(x_i)+A_{right}(x_i)B(x_i)
\end{displaymath}
化简得
\begin{displaymath}
    A_{right}(x_i)=\frac{y_i-A_{left}(x_i)}{B(x_i)}
\end{displaymath}
那么可利用多点求值得到一组新的插值点,递归求解。

求出$A_{right}(z)$后合并即可,时间复杂度$O(n \lg^3 n)$。
\subsection{组合数取模}

以上内容参考了picks
\footnote{
Newton's Method of Polynomial « Picks's Blog
\\\url{http://picks.logdown.com/posts/209226-newtons-method-of-polynomial}

Positional Notation Conversion « Picks's Blog
\\\url{http://picks.logdown.com/posts/208342-positional-notation-conversion}

Binomial Coefficient Modulo Prime « Picks's Blog
\\\url{http://picks.logdown.com/posts/245545-binomial-coefficient-modulo-prime}

}
和Miskcoo\footnote{
牛顿迭代法在多项式运算的应用 – Miskcoo's Space
\\\url{http://blog.miskcoo.com/2015/06/polynomial-with-newton-method}

多项式除法及求模 – Miskcoo's Space
\\\url{http://blog.miskcoo.com/2015/05/polynomial-division}

多项式的多点求值与快速插值 – Miskcoo's Space
\\\url{http://blog.miskcoo.com/2015/05/polynomial-multipoint-eval-and-interpolation}
}的博客。
