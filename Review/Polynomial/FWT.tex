\section{快速沃尔什变换FWT}
\index{F!Fast Walsh-Hadamard Transform}
\subsection{FWT原理}
FWT主要用来求下列卷积:
\begin{displaymath}
    z_n=\sum_{i\oplus j=n}{a_ib_j}
\end{displaymath}
其中$\oplus$为二元位运算符。

其原理与FFT相同,关键在于蝴蝶操作。

序列可表示为$A=(A_0,A_1)$,01表示当前处理的位,只需找到对于$C=A\otimes B$,
有$FWT(C)=FWT(A)*FWT(B)$,由$UFWT(FWT(A))=A$可推出其逆变换。

\index{*TODO!FWT公式推导与记忆}

现场推导时可以仅考虑由倒数第二层合并至倒数第三层的情况,其余自底向上可以归纳证明。

\subsubsection{and}
由$C=(A_0*B_0+A_0*B_1+A_1*B_0,A_1*B_1)$
可构造出
\begin{eqnarray*}
    FWT_{and}(A)&=&(FWT_{and}(A_0)+FWT_{and}(A_1),FWT_{and}(A_1))\\
    UFWT_{and}(A)&=&(UFWT_{and}(A_0)-UFWT_{and}(A_1),UFWT_{and}(A_1))
\end{eqnarray*}
\subsubsection{or}
\begin{eqnarray*}
    FWT_{or}(A)&=&(FWT_{or}(A_0),FWT_{or}(A_1)+FWT_{or}(A_0))\\
    FWT_{or}(A)&=&(UFWT_{or}(A_0),UFWT_{or}(A_1)-UFWT_{or}(A_0))
\end{eqnarray*}
\subsubsection{xor!!!}
\begin{eqnarray*}
    FWT_{xor}(A)&=&(FWT_{xor}(A_0)+FWT_{xor}(A_1),FWT_{xor}(A_0)-FWT_{xor}(A_1))\\
    FWT_{xor}(A)&=&\left(\frac{UFWT_{xor}(A_0)+UFWT_{xor}(A_1)}{2},
    \frac{UFWT_{xor}(A_0)-UFWT_{xor}(A_1)}{2}\right)
\end{eqnarray*}
\subsection{nand,nor,nxor}
求出$\oplus$为and,or,xor时的FWT后按取反交换。

以上内容参考了picks的博客
\footnote{Fast Walsh-Hadamard Transform « Picks's Blog
    \url{http://picks.logdown.com/posts/179290-fast-walsh-hadamard-transform}}。
\subsection{FWT实现}
\lstinputlisting[title=FWT]{Source/Templates/FWT.cpp}
