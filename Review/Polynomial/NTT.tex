\section{快速数论变换NTT}
\subsection{NTT原理}
NTT的原理与FFT类似,即找到单位根$x$满足$x^n\equiv 1 \pmod{p}$。
NTT模数$p$需满足$p$为素数且$p=r\cdot 2^k+1$。

根据定理~\ref{FLT}可知若模数$p$为素数则有$x^{p-1}\equiv 1 \pmod{p}$,
所以当$n|(p-1)$时才能进行NTT。

根据~\ref{PrimitiveRoot}节所述,$p$必有原根,设$p$的原根为$g$,则
$g^\frac{p-1}{n}$就是{\bfseries 主$n$次单位根},$n$个单位根即为
$w_n^k=g^{k\cdot \frac{p-1}{n}}$。

其余部分与FFT相同。

\subsection{NTT实现}
NTT仅预处理单位复数根部分不同,以模998244353为例:
\begin{lstlisting}
int tot, root[size], invR[size];
void init(int n) {
    const int g = 3;
    tot = n;
    Int64 base = powm(g, (mod - 1) / n);
    Int64 invBase = powm(base, mod - 2);
    root[0] = invR[0] = 1;
    for (int i = 1; i < n; ++i)
        root[i] = root[i - 1] * base % mod;
    for (int i = 1; i < n; ++i)
        invR[i] = invR[i - 1] * invBase % mod;
}
\end{lstlisting}
\subsection{NTT常见模数}
\begin{itemize}
    \item $469762049=7*2^{26}+1$。
    \item $998244353=119*2^{23}+1$。
    \item $1004535809=479*2^{21}+1$,加起来不爆int。
    \item $2281701377=17*2^{27}+1$,平方恰好不爆long long。
\end{itemize}
\index{*Constant!NTT模数P=\{469762049,\\998244353,1004535809,\\2281701377\},g=3}
它们的原根均为3。
\subsection{MTT之三模数NTT}
选取3个模数,比如\{469762049,998244353,1004535809\},要求它们的乘积大于卷积
过程中最大的数,分别以这三个数为模数求NTT,最后使用CRT求解同余方程组。

但是使用CRT求解会爆long long,因此先合并前两项,得到
\begin{eqnarray*}
    x&\equiv&n_1 \pmod{p_1}\\
    x&\equiv&n_2 \pmod{p_2}
\end{eqnarray*}
设$x=k_1p_1+n_1=k_2p_2+n_2$,由于$k_1<p_2$,我们可以求解
$k_1p_1\equiv n_2-n_1 \pmod{p_2}$得到$k_1$,带入原式求出$x \bmod{p}$的值。

该方法来自AntiLeaf\footnote{COGS2294 释迦 - AntiLeaf
\url{http://www.cnblogs.com/hzoier/p/6441967.html}}
的博客。
