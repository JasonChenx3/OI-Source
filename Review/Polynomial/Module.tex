\section{FFT封装}
牛顿迭代法/分治FFT时使用。

使用vector存储多项式。
\begin{lstlisting}
typedef std::vector<int> Poly;
\end{lstlisting}

封装一个工具函数计算做乘法所需最小规模。
\begin{lstlisting}
int getSize(int x) {
    x <<= 1;
    int p = 1;
    while(p < x)
        p <<= 1;
    return p;
}
\end{lstlisting}

封装一个工具函数用于拷贝多项式,参数顺序遵循memcpy。
\begin{lstlisting}
void copy(Poly& dst, const Poly& src, int siz) {
    memcpy(dst.data(), src.data(), sizeof(int) * siz);
}
\end{lstlisting}

封装DFT、IDFT,注意再给IDFT一个参数指示模数,只对需要的系数做除法。
\begin{lstlisting}
void DFT(int n, Poly& A) {
    NTT(n, A.data(), root);
}
void IDFT(int n, Poly& A, int rn) {
    NTT(n, A.data(), invR);
    Int64 div = powm(n, mod - 2);
    for(int i = 0; i < rn; ++i)
        A[i] = A[i] * div % mod;
    memset(A.data() + rn, 0, sizeof(int) * (n - rn));
}
\end{lstlisting}

牛顿迭代法系列函数一般使用如下签名:
\begin{lstlisting}
void func(int n, const Poly& sf, Poly& g)
\end{lstlisting}
其中$sf$为原多项式,使用时拷贝,$g$尽量重复使用,空间由调用端分配。

以多项式求逆为例:
\begin{lstlisting}
void inv(int n, const Poly& sf, Poly& g) {
    if(n == 1)
        g[0] = powm(sf[0], mod - 2);
    else {
        int h = (n + 1) >> 1;
        inv(h, sf, g);
        int p = getSize(n);
        DFT(p, g);

        Poly f(p);
        copy(f, sf, n);
        DFT(p, f);

        for(int i = 0; i < p; ++i) {
            g[i] = (2 - asInt64(g[i]) * f[i]) % mod *
                g[i] % mod;
            if(g[i] < 0)
                g[i] += mod;
        }
        IDFT(p, g, n);
#ifdef CHECK
        for(int i = 0; i < n; ++i) {
            int sum = 0;
            for(int j = 0; j <= i; ++j)
                sum =
                    (sum + asInt64(sf[j]) * g[i - j]) %
                    mod;
            if(i == 0 && sum != 1)
                throw;
            if(i != 0 && sum != 0)
                throw;
        }
#endif
    }
}
\end{lstlisting}

求逆/开方这些操作最好写个$O(n^2)$验证其正确性。

多项式乘法根据规模使用暴力/FFT解决。
