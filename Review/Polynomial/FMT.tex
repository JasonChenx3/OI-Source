\section{快速莫比乌斯变换/反演}
\index{F!Fast Mobius Transformation}
快速莫比乌斯变换用于计算集合并卷积。
快速莫比乌斯反演其实就是其逆变换。
实际上这就是FWT的非递归形式。

给定数组$f,g$,求$f*g$的集合并卷积$\displaystyle h[k]=\sum_{i|j==k}{f[i]g[j]}$,
数组规模为$2^n$。

记数组$F$为$f$的莫比乌斯变换,满足$\displaystyle F[i]=\sum_{i\&j==j}{f[j]}$
(实际上就是子集和)。同理计算数组$g$的莫比乌斯变换$G$。令$H[i]=F[i]*G[i]$,会发现
$\displaystyle H[i]=\sum_{i\&j==j}{h[j]}$。

接下来考虑如何从$f$快速推出$F$:枚举每一个比特位,对每一比特位做前缀和,这样就可以保证
得到正确结果。时间复杂度$O(2^nn)$。
\begin{lstlisting}
int end = 1 << n;
for(int i = 1; i < end; i <<= 1)
    for(int j = 0; j < end; ++j)
        if(j & i)
            A[j] += A[j ^ i];
\end{lstlisting}

逆变换只要将其倒着做就可以了(从高到低枚举与从低到高枚举等价,$A$值改变
的影响不会传递,所以不必控制枚举顺序)。
\begin{lstlisting}
int end = 1 << n;
for(int i = 1; i < end; i <<= 1)
    for(int j = 0; j < end; ++j)
        if(j & i)
            A[j] -= A[j ^ i];
\end{lstlisting}

用容斥可得$\displaystyle h[i]=\sum_{i\&j==j}{(-1)^{|I|-||J|}H[j]}$,其中$|I|$表示
$i$对应的集合$I$的大小。

\paragraph{例题}
Luogu P3175 [HAOI2015]按位或\footnote{
    【P3175】[HAOI2015]按位或 - 洛谷
    \url{https://www.luogu.org/problemnew/show/P3175}
}

记$a_i$为走$i$步的概率,答案为$\displaystyle \sum_{k=1}^\infty{k(a_i-a_{i-1})}=
-\sum_{k=1}^\infty a_i$。令概率数组为集合幂级数$f$,定义乘法运算为集合幂卷积,可知答案
为$f$的等比数列求和,若其收敛则其答案为$\left(\frac{1}{f-1}\right)_{2^n-1}$。计算
$f$的莫比乌斯变换$F$,那么$F$的幂就是$F$每一项的幂组合起来的序列。那么$F$的等比数列求和
就是${\frac{1}{F_i-1}}$,最后使用容斥计算出$f_n$,不必做反演。

代码:
\lstinputlisting{Source/Templates/FMT.cpp}

上述内容参考了liu\_runda的博客\footnote{
    bzoj4036[HAOI2015]set 按位或
    \url{https://www.cnblogs.com/liu-runda/p/6443577.html}
    不会FWT的选手计算集合并卷积的方法
    \url{http://liu-runda.blog.uoj.ac/blog/2360}
}。
