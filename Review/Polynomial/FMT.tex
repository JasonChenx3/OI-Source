\section{快速莫比乌斯变换/反演}\label{FMT}
\index{F!Fast Mobius Transformation}
快速莫比乌斯变换用于计算集合并卷积。
快速莫比乌斯反演其实就是其逆变换。
实际上这就是FWT的非递归形式。

给定数组$f,g$,求$f*g$的集合并卷积$\displaystyle h[k]=\sum_{i|j==k}{f[i]g[j]}$,
数组规模为$2^n$。

记数组$F$为$f$的莫比乌斯变换,满足$\displaystyle F[i]=\sum_{i\&j==j}{f[j]}$
(实际上就是子集和)。同理计算数组$g$的莫比乌斯变换$G$。令$H[i]=F[i]*G[i]$,会发现
$\displaystyle H[i]=\sum_{i\&j==j}{h[j]}$。

接下来考虑如何从$f$快速推出$F$:枚举每一个比特位,对每一比特位做前缀和,这样就可以保证
得到正确结果。时间复杂度$O(2^nn)$。
\begin{lstlisting}
int end = 1 << n;
for(int i = 1; i < end; i <<= 1)
    for(int j = 0; j < end; ++j)
        if(j & i)
            A[j] += A[j ^ i];
\end{lstlisting}

注意到所有的$j$都是$i$的父集,且$A$值改变不传递,因此可以枚举$i$的父集去掉
判断。不过实际优化效果并达不到2倍,可能性能瓶颈在于存取。

\begin{lstlisting}
int end = 1 << n;
for(int i = 1; i < end; i <<= 1)
    for(int j = i; j < end; j = (j + 1) | i)
        A[j] += A[j ^ i];
\end{lstlisting}

Update:似乎FWT比FMT更快些,可能它更Cache-Friendly吧。

逆变换只要将其倒着做就可以了(从高到低枚举与从低到高枚举等价,$A$值改变
的影响不会传递,所以不必控制枚举顺序)。
\begin{lstlisting}
int end = 1 << n;
for(int i = 1; i < end; i <<= 1)
    for(int j = 0; j < end; ++j)
        if(j & i)
            A[j] -= A[j ^ i];
\end{lstlisting}

用容斥可得$\displaystyle h[i]=\sum_{i\&j==j}{(-1)^{|I|-|J|}H[j]}$,其中$|I|$
表示$i$对应的集合$I$的大小。

证明:该式的$H[j]$子集中含有$h[i]$当且仅当$i==j$,贡献为$h[i]$。考虑其余$h[k]$对
$H[j]$的贡献,以及它们在式中最终对$h[i]$的贡献,设$d=|I|-|K|$,$h[k]$对$h[i]$的贡献
为$\displaystyle \sum_{i=0}^{2^d-1}{(-1)^{d-|i|}}$。由于选择奇数位与选择偶数位的
方案数相等,贡献为0。

\paragraph{注意事项} 单次FMT的实际意义很重要,经常用于辅助容斥预处理,
即使对集合幂级数的操作无定义。

\paragraph{例题}
Luogu P3175 [HAOI2015]按位或\footnote{
    【P3175】[HAOI2015]按位或 - 洛谷
    \url{https://www.luogu.org/problemnew/show/P3175}
}

记$a_i$为走$i$步到达目标状态的概率,答案为$\displaystyle \sum_{k=1}^
\infty{k(a_k-a_{k-1})}=-\sum_{k=0}^\infty a_k$。令概率数组为集合幂级数$f$,
定义乘法运算为集合幂卷积,可知答案为$f$的无限递减几何级数第$2^n-1$项的倒数,
若其收敛则其答案为$\left(\frac{1}{f-1}\right)_{2^n-1}$。计算$f$的莫比乌斯变换$F$,
那么函数$T(f)$对应序列${T(F_i)}$。记$G_i={\frac{1}{F_i-1}}$,
最后使用容斥计算出$g_{2^n-1}$,不必做反演。

若有解则$F_{2^n-1}=1$,此时对应的$G_{2^n-1}$应该为0,所以容斥时不考虑目标状态项。

代码:
\lstinputlisting{Source/Templates/FMT.cpp}

上述内容参考了liu\_runda的博客\footnote{
    bzoj4036[HAOI2015]set 按位或
    \url{https://www.cnblogs.com/liu-runda/p/6443577.html}
    不会FWT的选手计算集合并卷积的方法
    \url{http://liu-runda.blog.uoj.ac/blog/2360}
}。

有时按照1的个数将原数组分解也是一个比较好的思路。

例题:LOJ161 子集卷积

分析后可得到要求的是$h[k]=\displaystyle \sum_{i\&j=0 \land i|j=k}{f[i]g[j]}$。
考虑除去约束$i\&j=0$,由于我们能够做约束$i|j=k$的卷积,在满足这个约束的情况下,约束
$i\&j=0$可以等价为$bitcount(i)+bitcount(j)=bitcount(k)$。那么将原数组按照1的个数
分解,然后对每类$bitcount$卷积,最后只要输出$bitcount(k)$的$k$的值。
