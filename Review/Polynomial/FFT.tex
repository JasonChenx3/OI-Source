\section{快速傅里叶变换FFT}
\index{F!Fast Fourier Transformation}
\subsection{FFT原理}
FFT求多项式卷积的过程为:$\Theta(n\lg n)$求值->$\Theta(n)$点值乘法->
$\Theta(n\lg n)$插值。

$\Theta(n\lg n)$求值/插值的复杂度是在单位复数根上计算得到的。

\subsubsection{单位复数根}

定义{\bfseries $n$次单位复数根}是满足$\omega^n=1$的复数$\omega$,恰好有$n$个,即
$\omega_n^k=e^{2\pi ik/n},k=0,1,\cdots,n-1$。

定义{\bfseries 主$n$次单位根}$\omega_n=e^{2\pi i/n}$。

下面是关于$n$次单位复数根的性质:

\begin{lemma}[消去引理]\label{FFTL1}
	对于任意整数$n\geq 0,k \geq 0,d>0$,
	\begin{displaymath}
		\omega_{dn}^{dk}=\omega_n^k
	\end{displaymath}
\end{lemma}
证明:
\begin{displaymath}
	\omega_{dn}^{dk}=e^{2\pi i dk/dn}=e^{2\pi i k/n}=\omega_n^k
\end{displaymath}

\begin{inference}\label{FFTI2}
	对于任意偶数$n>0$,有
	\begin{displaymath}
		\omega_n^{n/2}=\omega_2=-1
	\end{displaymath}
\end{inference}

\begin{lemma}[折半引理]
	对于偶数$n>0$,$n$个$n$次单位复数根的平方组成的集合为$n/2$个$n/2$
	次单位复数根的集合。
\end{lemma}
证明:根据引理~\ref{FFTL1}可得$(\omega_n^k)^2=(\omega_n^{k+n/2})^2=
	\omega_{n/2}^k$,每个$n/2$次单位复数根恰好被得到2次。

\begin{lemma}[求和引理]\label{FFTL4}
	对于任意整数$n\geq 1$与不能被$n$整除的非负整数$k$,有
	\begin{displaymath}
		\sum_{i=0}^{n-1}{(w_n^k)^i}=0
	\end{displaymath}
\end{lemma}
证明:
\begin{displaymath}
	\sum_{i=0}^{n-1}{(w_n^k)^i}=\frac{(w_n^k)^n-1}{w_n^k-1}=0
\end{displaymath}
$n$不整除$k$保证了分母不为0。

\subsubsection{DFT}
\index{D!Discrete Fourier Transform}
对于次数界为$n$的多项式
\begin{displaymath}
	A(x)=\sum_{i=0}^{n-1}{a_ix^i}
\end{displaymath}
其DFT为
\begin{displaymath}
	DFT_n(a)=(y_0,y_1,\cdots,y_{n-1})=
	(A(\omega_n^0),A(\omega_n^1),\cdots,A(\omega_n^{n-1}))
\end{displaymath}

\subsubsection{FFT}
FFT采用分治策略,假设$n$是2的幂(不足补0),其步骤如下:
\begin{enumerate}
	\item 若次数界为1,则返回$a_0$。
	\item 定义新的次数界为$n/2$多项式
	      \begin{eqnarray*}
		      A^{[0]}(x)&=&a_0+a_2x+\cdots+a_{n-2}x^{n/2-1}\\
		      A^{[1]}(x)&=&a_1+a_3x+\cdots+a_{n-1}x^{n/2-1}
	      \end{eqnarray*}
	      递归计算其在点$(\omega_n^0)^2,(\omega_n^1)^2,\cdots,(\omega_n^{n-1})^2$
	      的值(实际上递归只求了前一半)。
	\item 该多项式满足等式\begin{equation}\label{RFFTE}
		      A(x)=A^{[0]}(x^2)+xA^{[1]}(x^2)
	      \end{equation}
	      可利用递归计算的值合并。
	      对于$k=0,1,\cdots,n/2-1$,
	      \begin{eqnarray*}
		      y_k&=&y_k^{[0]}+\omega_n^ky_k^{[1]}\\
		      y_{k+n/2}&=&y_k^{[0]}-\omega_n^ky_k^{[1]}
	      \end{eqnarray*}
	      正确性证明:
	      \begin{eqnarray*}
		      y_k&=&y_k^{[0]}+\omega_n^ky_k^{[1]}\\
		      &=&A^{[0]}(\omega_{n/2}^k)+\omega_n^kA^{[1]}(\omega_{n/2}^k)\\
		      &=&A^{[0]}(\omega_n^{2k})+\omega_n^kA^{[1]}(\omega_n^{2k})
		      \textrm{~(根据引理~\ref{FFTL1})}\\
		      &=&A(\omega_n^k) \textrm{~(根据式~\ref{RFFTE})}\\
		      y_{k+n/2}&=&y_k^{[0]}-\omega_n^ky_k^{[1]}\\
		      &=&A^{[0]}(\omega_{n/2}^k)+\omega_n^{k+n/2}A^{[1]}(\omega_{n/2}^k)
		      \textrm{~(根据推论~\ref{FFTI2})}\\
		      &=&A^{[0]}(\omega_n^{2k+n})+\omega_n^{k+n/2}A^{[1]}(\omega_n^{2k+n})
		      \textrm{~(根据引理~\ref{FFTL1}与$\omega_n^n=1$)}\\
		      &=&A(\omega_n^{k+n/2}) \textrm{~(根据式~\ref{RFFTE})}\\
	      \end{eqnarray*}
\end{enumerate}
\subsubsection{逆DFT}
\begin{theorem}
	对于范德蒙德矩阵$V_n$满足$v_{ij}=\omega_n^{ij}$,
	其逆矩阵$V_n^{-1}$满足$v^{-1}_{ij}=\omega_n^{-ij}/n$(矩阵下标从0开始)。
\end{theorem}
证明:
\begin{eqnarray*}
	[V_nV_n^{-1}]_{ij}&=&\sum_{k=0}^{n-1}{\omega_n^{ik}\omega_n^{-kj}/n}\\
	&=&\sum_{k=0}^{n-1}{\omega_n^{k(i-j)}/n}\\
	&=&[i=j]\textrm{~(根据引理~\ref{FFTL4})}\\
	&\Rightarrow& V_nV_n^{-1}=I_n
\end{eqnarray*}
所以逆DFT的矩阵就是DFT矩阵的$\frac{1}{n}$,用$\omega_n^{-1}$代替$\omega_n$,
对多项式的DFT再做一次DFT后除以$n$就能得到原多项式。

以上内容来自算法导论\cite{ITA3}第30章 多项式与快速傅里叶变换。
\subsection{迭代FFT实现}
\subsubsection{单位复数根预处理}
一般可以确定FFT所需最大规模,因此可以在FFT前预处理。
\begin{lstlisting}
typedef double FT;
typedef std::complex<FT> Complex;
int tot;
Complex root[size],invR[size];
void init(int n) {
	tot=n;
	FT base=2.0*acos(-1.0)/n;
	for(int i=0;i<n;++i) {
		root[i]=Complex(cos(base*i),sin(base*i));
		invR[i]=conj(root[i]);
	}
}
\end{lstlisting}
根据引理~\ref{FFTL1},$root[tot/n*k]=w_n^k$。
\subsubsection{离线位逆序置换}
为了做迭代FFT,我们需要置换原多项式的系数,第$i$个系数被换到基于$2^k$的$i$的位逆序上。
若FFT的规模不变,比如求多项式乘法之类的简单问题,可预处理位逆序置换数组后,按照数组下标
置换(规定$i<rev[i]$保证只置换1次)。
\begin{lstlisting}
int rev[size];
void initRev(int k) {
	rev[0]=0;
	int end=1<<k;
	for(int i=1;i<end;++i)
		rev[i]=rev[i>>1]>>1|(i&1)<<(k-1);
}
\end{lstlisting}
\subsubsection{在线位逆序置换}
在分治FFT时,需要不同规模的位逆序置换,预处理置换数组不太适用。这里介绍的是时间复杂度
$O(n)$,空间复杂度$O(1)$的在线算法。
\begin{lstlisting}
void rev(int* A,int n) {
	for(int i=0,j=0;i<n;++i) {
		if(i>j)std::swap(A[i],A[j]);
		int l=n>>1;
		while((j^=l)<l)l>>=1;//key
	}
}
\end{lstlisting}
在每轮循环开始时,$i$与$j$互为位逆序。
代码中key部分是整个算法的关键,$i$的递增就是正向二进制加法,$j$就要
进行反向二进制加法,即从高到低找到第1个0,从高到低将这一段取反。

时间复杂度证明:设$n=2^k$,按$while$迭代次数分类,有
\begin{eqnarray*}
	T(n)&=&\sum_{i=1}^k{i\cdot 2^{k-i}}\\
	&=&n\sum_{i=1}^k{i\cdot 2^{-i}}\\
	&=&n\sum_{i=1}^k{\sum_{j=i}^k{2^{-j}}}\\
	&<&n\sum_{i=1}^k{2^{1-i}}\\
	&<&2n\Rightarrow T(n)=O(n)
\end{eqnarray*}
\subsubsection{迭代FFT}
考虑每一层递归树,发现最底层的递归树是按位逆序顺序排的(每一次递归相当于一次01划分)。
若系数数组按照这种顺序存储,则可以很快地找到每一层对应的位置。

迭代FFT的计算顺序是自底向上的,代码如下:

\begin{lstlisting}
void FFT(int n, Complex *A, Complex *w) {
    for (int i = 0, j = 0; i < n; ++i) {
        if (i > j) std::swap(A[i], A[j]);
        int l = n >> 1;
        while ((j ^= l) < l) l >>= 1;
    }
    for (int i = 2; i <= n; i <<= 1) {
        int m = i >> 1, fac = tot / i;
        for (int j = 0; j < n; j += i)
            for (int k = 0; k < m; ++k) {
                Complex t = A[j + k + m] * w[k * fac];
                A[j + k + m] = A[j + k] - t;
                A[j + k] += t;
            }
    }
}
\end{lstlisting}
首先做位逆序置换,然后从倒数第二层开始向上递推,$i$表示当前层每一块的长度,$m$是下一层
每一块的长度,$fac$是单位复数根缩放系数,$j$是块编号,内层循环与递归FFT合并答案的方法
相同,注意蝴蝶操作的求值顺序。

以上内容参考了Miskcoo的博客\footnote{从多项式乘法到快速傅里叶变换 – Miskcoo's Space
\url{http://blog.miskcoo.com/2015/04/polynomial-multiplication-and-fast-fourier-transform}}。
\subsection{实序列DFT}\label{RDFT}
对于多项式乘法这类问题,常规方法需要对每个序列做1次DFT。但由于我们
只在实数域上做FFT,可以考虑将两个实序列合并为一个复序列,
直接使用1次DFT以及一点后处理得到两个实序列的DFT。

记大小为$N$的序列$a$的DFT为$A$,考虑序列$A$的共轭为$\overline{A}$:
\begin{displaymath}
	\overline{A(n)}=\sum_{k=0}^{N-1}{\overline{a(k)}\cdot\omega_N^{-nk}}=
	\sum_{k=0}^{N-1}{\overline{a(k)}\cdot\omega_N^{K(N-n)}}
	n=0,1,\cdots,N-1
\end{displaymath}

对于实序列$a$,有$a(n)=\overline{a(n)}$,由此可得$\overline{A(n)}=A(N-n)$,
即实序列的DFT共轭对称。

令$z(i)=x(i)+i\cdot y(i)$,则$Z(n)=X(n)+i\cdot Y(n)$,
记$Z(n)$的实部与虚部分别为$X(n)$与$Y(n)$,有
\begin{eqnarray*}
	\overline{Z(N-n)}&=&\overline{X(N-n)+i\cdot Y(N-n)}\\
	&=&\overline{A(N-n)+i\cdot B(N-n)}\\
	\Rightarrow X(n)-i\cdot Y(n)&=&A(N-n)-i\cdot B(N-n)
\end{eqnarray*}

联立解得
\begin{eqnarray*}
	2X(n)&=&(A(n)+A(N-n))+i\cdot (B(n)-B(N-n))\\
	2Y(n)&=&(B(n)+B(N-n))-i\cdot (A(n)-A(N-n))
\end{eqnarray*}

对复序列$z(n)$的DFT后处理一下就可以得到$X(n)$与$Y(n)$(注意要对$n=0$
进行特别处理)。

代码如下:
\begin{lstlisting}
FFT(p,A,root);
X[0] = A[0].real();
Y[0] = A[0].imag();
for (int i = 1; i < p; ++i) {
	FT x1 = A[i].real(), y1 = A[i].imag();
	FT x2 = A[p - i].real(), y2 = A[p - i].imag();
	X[i] = Complex((x1 + x2) * 0.5, (y1 - y2) * 0.5);
	Y[i] = Complex((y1 + y2) * 0.5, (x2 - x1) * 0.5);
}
\end{lstlisting}

以上内容参考了Miskcoo的博客\footnote{实序列离散傅里叶变换的快速算法 – Miskcoo's Space
\url{http://blog.miskcoo.com/2018/01/real-dft}}。
\subsection{MTT之拆系数FFT}
MTT主要用来解决模数不满足NTT条件的卷积问题,可以使用拆系数FFT的实数运算来
绕开限制。

设模数为$M$,解决方法是找一个$k\geq\sqrt{M}$,将序列$a(n)$拆为2个新序列
$a_0(n)=a(n)/p,a_1(n)=a(n)\% p$,两两相乘后按照系数合并。共4次DFT,
4次IDFT。

优化:
\begin{itemize}
	\item 注意到这4个多项式均为实序列,所以可按照~\ref{RDFT}节中所述方法
	仅使用2次DFT与2次IDFT。
	\item 合并时有两个多项式的系数均为$k$,可以加在一起IDFT,
	使用4次DFT与3次IDFT。
\end{itemize}
