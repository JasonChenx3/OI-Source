\section{计算形式幂级数的牛顿迭代法的常数优化}
该内容基于negiizhao的博客\footnote{
    \sout{noip退役选手的一些扯淡}关于优化形式幂级数计算的牛顿法的常数\\
    \url{http://negiizhao.blog.uoj.ac/blog/4671}
}。

一般在多项式开根及exp等常数较大的计算中使用,简单的计算仍然使用原算法。

记长度为$2n$的DFT/IDFT的计算时间为$T(n)$,$n$次形式幂级数乘法的时间为$(3+o(1))T(n)$,
下面算法的运行时间均以$T(n)$为基准衡量。运行时间有理论上证明与实际测试结果支持。测试用
模数为998244353,测试规模为$2^{22}$,每个算法重复运行100次。以下时间复杂度证明假装
$T(n)=2T(n/2)$,即不考虑lg的增长。

\subsection{卷积}
牛顿迭代法需要遇到不同规模的卷积,原先的卷积实现使用动态计算位逆序和利用消去引理定位对应的
单位根。既然我们已经知道了卷积的最大规模,可以使用最大规模2倍的空间预处理出每种规模的单位根
和位逆序数组。这样在访问单位根时对缓存友好。实验表明此法的速度是原先的2倍,且代码几乎不需要
改动(仅修改init与NTT部分,封装的魅力!!!)。由于本节主要注重牛顿迭代法本身的优化,下面
的算法无论是否优化,都使用优化后的NTT作为基础设施。不过为了体现性能提升的重要性,未优化的
算法使用的子过程也是未优化的。

\subsection{求逆}
\subsubsection{原算法}
原算法在每次迭代中执行2次DFT,1次IDFT,长度均为$2n$级别。记规模为$n$的多项式求逆时间复杂度
为$A(n)$,有$A(n)=A(n/2)+(3+o(1))T(n)$,解得$A(n)=(6+o(1))T(n)$。事实上,从模$x^n$
推到$x^{2n}$的过程的复杂度等于求模$x^n$意义的复杂度。
\subsubsection{优化}
注意到我们已经有了$g$的前$h$项,没有必要再计算一遍。记子过程求出的结果为$g_0$,有
$fg_0\equiv 1\pmod{x^h}$。考虑最初牛顿迭代法的计算过程,迭代式为
$g'=g_0-(fg_0-1)g_0$,尝试按照表达式计算而不是化简计算。那么一共需要2次卷积,一次为
$fg_0$,另一次为$(fg_0-1)g_0$。

注意这两个卷积有特殊的性质,假若$n$对齐到2的幂,且卷积规模为$n$而不是$2n$,卷积时会导致
高次项溢出。根据FFT的性质,溢出部分将右移$2n$次加到原来的结果中,即FFT乘法做的是循环卷积。
如果是一般多项式的乘法,肯定无法将这两部分拆开。记函数$A$的低h次项为$A_L$,高$h$次项右移
h次为$A_H$。那么子过程返回的答案满足$f_Lg_0\equiv 1\pmod{x^h},g'_L=g_0$。那么有
$(f_L+f_Hx^h)g_0=f_Lg_0+f_Hg_0x^h$,在规模为$n$的卷积下,低h项由$f_Lg_0$的低h项
与$f_Hg_0$的高h项组成,高h项由$f_Lg_0$的高h项与$f_Hg_0$的低h项组成,由于已知
$f_Lg_0$的低h项为1,很容易分离出溢出的部分,不过溢出部分在模$x^n$意义下是不需要的。
分离出溢出部分后,注意到接下来要用的是$fg_0-1 \bmod{x^n}$,其低h项全为0。综上所述,
我们只需要做:求$f$与$g_0$在模$x^n$意义下的循环卷积,然后将低h项值0。

接下来需要计算$(fg_0-1)g_0\bmod{x^n}$,而$fg_0-1$的低h项全为0,故最后不需要按照
式子做减法,而是求出该式的高h项,取反后拷贝到$g'$的高h项。注意到$(fg_0-1)_H$与
$g_0$卷积时,高h项全部被放到卷积结果的低h项,而我们需要的部分对应于其低h项,对应于
卷积结果的高h项。综上所述,我们只需要做:求$fg_0-1$与$g_0$在模$x^n$意义下的循环卷积,
然后把高h项的值取反拷贝到$g'$的高h项,$g'$的低h项就是$g_0$。

注意到两次卷积都有$g_0$,可以重复使用$DFT(g_0)$。最终这个算法需要3次DFT与2次IDFT,
且卷积规模都是$n$,时间复杂度$(5+o(1))T(n)$。

{\bfseries 注意输入数组也要对齐到2的幂,否则会导致RE或者漫长的Debug。}
\subsection{平方根}
\subsubsection{原算法}
原算法在每次迭代中执行3次DFT,1次IDFT,1次求逆。从模$x^n$推到模$x^{2n}$的时间
复杂度为$(20+o(1))T(n)$,求模$x^n$意义下的答案也是如此。
\subsubsection{优化}
注意到每次迭代都要求逆,但是每次求逆的规模也是倍增的。考虑在维护$f^{1/2}$时同时维护
$f^{-1/2}$。每次计算出$f^{1/2}$后使用多项式求逆的迭代方法计算$f^{-1/2}$。

当然循环卷积同样适用于求平方根的迭代。注意到$g_0^2-f\equiv 0 \pmod{x^h}$,
在模$x^n$意义下只需取$g_0^2$卷积结果的高h项与f做减法,其余项置零,然后
与$g^{-1}_0$卷积,取高h项乘以$-1/2$赋值到$g'$的高h项。开方与求逆分别2次DFT与2次IDFT,
同时共享一次对$g_0^{-1}$的DFT。时间复杂度为$(9+o(1))T(n)$。
若无必要,最后一次迭代不需要求逆,时间复杂度为$(7+o(1))T(n)$。
\subsection{指数}
\subsubsection{原算法}
原算法的ln部分有一次求逆,一次乘法。迭代其余部分有3次DFT,1次IDFT。时间复杂度
$(26+o(1))T(n)$。
\subsubsection{优化}
注意到ln部分需要用到求逆,考虑迭代的同时维护逆,再加上循环卷积优化。

但是ln部分的求逆精度要达到模$x^n$,而子过程给的精度是模$x^h$。因此需要对原迭代式进行
变形,降低精度要求。记$h_0=g_0^{-1}\bmod{x^h},f_0=f\bmod{x^h}$。首先将$f$丢入
积分中,然后注意到$g_0'/g_0-f'$的低h-1项为0,我们需要的是它的高$h~n-1$项,考虑引入
$g_0h_0$以消去$g_0^{-1}$。因为$g_0h_0$的低h项为1,所以乘上$g_0h_0$后需要的部分
仍被保留,且$g_0h_0$的高h项无影响。再利用$g_0h_0-1\equiv 0 \pmod{x^h}$减小卷积规模。

\begin{eqnarray*}
    g'&\equiv&g_0-(\int(g_0'/g_0)-f)g_0\\
    &\equiv&g_0-g_0\int(g_0'/g_0-f')\\
    &\equiv&g_0-g_0\int((g_0h_0)(g_0'/g_0-f'))\\
    &\equiv&g_0-g_0\int(h_0g_0'-f'(g_0h_0))\\
    &\equiv&g_0-g_0\int(h_0g_0'-f'(g_0h_0-1)-f')\\
    &\equiv&g_0-g_0(\int(h_0g_0'-f_0'(g_0h_0-1))-f) \pmod{x^n}
\end{eqnarray*}

时间复杂度为$(12+o(1))T(n)$。
\subsection{优化方法总结与实现细节}
这两个方法通用且优化效果显著。
\begin{itemize}
    \item 对于子过程需要用到求逆的迭代,可以考虑同步迭代计算逆。
    \item 牛顿迭代法中分子项模$x^h$恒为0,考虑循环卷积减小卷积规模。
\end{itemize}

{\bfseries 鉴于自己经常忘记初始化与初始化规模不够的情况,今后全面使用惰性初始化。}

为了适应循环卷积惰性除法与清零的需要,更新IDFT接口:
\begin{lstlisting}
void IDFT(int n, Poly& A, int b, int e,
    bool clear = true) {
    NTT(n, A, getInvRoot);
    Int64 div = powm(n, mod - 2);
    for(int i = b; i < e; ++i)
        A[i] = A[i] * div % mod;
    if(clear) {
        memset(data(A), 0, sizeof(int) * b);
        memset(data(A) + e, 0, sizeof(int) * (n - e));
    }
}
\end{lstlisting}

clear标志指示是否需要清零低位/高位,若需要继续循环卷积则清零,若仅做
加减法则不清零。\sout{原先使用lc/hc标志独立控制高低位清零,后来发现
一般只保留一半,规模较小的空位清零不影响性能。}
\subsection{实验结果}
测试代码参见仓库中的/Review/Polynomial/Perf文件夹。

\begin{tabular}{c|c}
\hline
算法 & 测试比值(优化前/优化后,以$T(n)$为基准)\\
\hline
多项式乘法 & -/3.12\\
多项式求逆&5.89/4.44\\
多项式平方根&18.87/6.10\\
多项式指数&24.69/10.66\\
\hline
\end{tabular}

使用上述优化的LOJ150挑战多项式速度提升5倍,从倒数第4一跃成为rank3\\(2019.2.20)。
