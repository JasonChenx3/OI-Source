\section{圆}
处理圆的交并问题一般考虑圆之间的覆盖区间。
\subsection{圆的并}
求圆的并的面积。

首先去除被其他圆覆盖的圆。然后对于每个圆与其它圆求交点,得到每个圆被覆盖的弧度区间
(逆时针为正方向)。注意跨越弧度$\pi$的区间要分为两个区间。对每个圆的覆盖区间排序
得到不覆盖的区间。

画图可以发现圆并的面积等于圆弧面积+多边形的有向面积。由于多边形需要计算有向面积(直接
以原点作为基准点),不能直接计算扇形面积和。

对于半径为$R$的圆,弧度为$x$的圆弧的面积为$\frac{1}{2}(x-\sin x)R^2$。
当$x> \pi$时该等式也满足。求交时先计算交点弦的中点,再根据勾股定理计算两交点。

时间复杂度$O(n^2 \lg n)$。

代码如下:
\lstinputlisting{Source/Templates/CIRU.cpp}

\subsubsection{扩展}
求被$i$个圆覆盖的区域面积。

同样对于每个圆考虑其覆盖区间,发现覆盖$i$次的区间贡献给覆盖$i+1$次及以上的答案,
排序+差分统计区间覆盖次数后计算贡献,输出时差分即可。注意覆盖次数要加上整个圆被覆盖的次数。

\lstinputlisting{Source/Source/CG/CIRUT.cpp}

上述内容参考了Oyking的博客\footnote{
	SPOJ 8073 The area of the union of circles(计算几何の圆并)(CIRU)
	\url{https://www.cnblogs.com/oyking/p/3424999.html}
}。
\subsection{圆的交}
与求圆并同理,对于每个圆求出覆盖区间的并,答案贡献为圆弧面积+$\frac{1}{2}$区间并
两端点的叉积。
\subsection{最小圆覆盖}
\subsection{圆的反演}
