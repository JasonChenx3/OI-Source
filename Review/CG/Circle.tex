\section{圆}
处理圆的交并问题一般考虑圆之间的覆盖区间。
\subsection{圆的并}
求圆的并的面积。

首先去除被其他圆覆盖的圆。然后对于每个圆与其它圆求交点,得到每个圆被覆盖的弧度区间
(逆时针为正方向)。注意跨越弧度$\pi$的区间要分为两个区间。对每个圆的覆盖区间排序
得到不覆盖的区间。

画图可以发现圆并的面积等于圆弧面积+多边形的有向面积。由于多边形需要计算有向面积(直接
以原点作为基准点),不能直接计算扇形面积和。

对于半径为$R$的圆,弧度为$x$的圆弧的面积为$\frac{1}{2}(x-\sin x)R^2$。
当$x> \pi$时该等式也满足。

两圆求交时用圆心、交点弦中点、某一交点构造两个直角三角形,然后根据勾股定理列出方程组,
解出一个圆心到交点弦中点的距离,进而解出该三角形各边长,使用向量偏移计算交点弦中点和交点。

时间复杂度$O(n^2 \lg n)$。

代码如下:
\lstinputlisting{Source/Templates/CIRU.cpp}

\subsubsection{扩展}
求被$i$个圆覆盖的区域面积。

同样对于每个圆考虑其覆盖区间,发现覆盖$i$次的区间对覆盖$>i$次的答案有贡献,
排序+差分统计区间覆盖次数后计算贡献,输出时差分。注意覆盖次数要加上整个圆被覆盖的次数。

\lstinputlisting{Source/Source/CG/CIRUT.cpp}

上述内容参考了Oyking的博客\footnote{
	SPOJ 8073 The area of the union of circles(计算几何の圆并)(CIRU)
	\url{https://www.cnblogs.com/oyking/p/3424999.html}
}。
\subsection{圆的交}
与求圆并同理,对于每个圆求出覆盖区间的并,答案贡献为圆弧面积+$\frac{1}{2}$区间并
两端点的叉积。
\subsection{最小圆覆盖}
\index{M!Minimum Covering Circle}
一般使用随机增量法:
\begin{enumerate}
	\item 将输入的圆随机化重排列;
	\item 以前两个点为直径端点构造初始圆;
	\item 不断加入点更新最小覆盖圆,设当前点为$P_i$:
	\begin{enumerate}
		\item 若该点已经在该圆内,跳出;
		\item 否则该点必在新圆上,与$P_1$构造初始圆,按顺序加入点$P_j,j<i$
		,若$P_j$不在当前圆内,则$P_j$也在新圆上。同理固定$P_i,P_j$,找出不在
		当前圆上的点$P_k,k<j$,则$P_i,P_j,P_k$三点可确定一个圆,
		保证该圆是$P_i,P_j,P_x,x\leq k$的最小覆盖圆。
	\end{enumerate}
\end{enumerate}

\paragraph{实现细节}
在求三角形外接圆时注意对三点共线的点取最远点对作为直径,三点不共线则使用中垂线求交
求外接圆。

代码如下:
\lstinputlisting{Source/Templates/Cover.cpp}

上述内容参考了Wikipedia-EN\footnote{
	Smallest-circle problem - Wikipedia\\
	\url{https://en.wikipedia.org/wiki/Smallest-circle\_problem}
}。
\subsection{圆的反演}
\paragraph{定义} 已知$\odot O$的半径为$r$,若点$O,P_1,P_2$在同一直线上,
且$OP_1\cdot OP_2=r^2$,则称$P_1,P_2$关于$\odot O$互为反演,称点$O$为反演中心。

\begin{property}
	一条不经过反演中心的直线的反演图形是一个经过反演中心的圆。
\end{property}
可以根据直线到反演中心的距离与圆的直径的关系来互推。

\begin{property}
	一个不经过反演中心的圆的反演图形还是一个不经过反演中心的圆。
\end{property}
可以根据圆心与反演中心的直线上与两圆的四个交点(两两对应)解出反演圆的
半径和到反演中心的距离。

\subsubsection{求过定点且与两个圆相切的圆}
以定点为反演中心,计算两个圆的反演圆,求反演圆的公切线,再做一次反演就可以得到
经过反演中心的圆了。

以上内容参考了ACdreamer的博客\footnote{
	圆的反演变换 - ACdreamer
	\url{https://blog.csdn.net/acdreamers/article/details/16966369}
}。
