\section{凸包}
\index{C!Convex Hull}
\subsection{极角序凸包}
经典算法是Graham扫描法\index{G!Graham Scan}。
算法步骤如下:
\begin{itemize}
    \item 选择一个纵坐标最低的点加入凸包,以此为原点按极角对其他点排序;
    \item 按照极角序加入每一个节点,保持凸包相邻3个节点的凸性质。
\end{itemize}
\subsection{水平序凸包}
极角序计算凸包容易由于$atan2$的精度问题而造成错误。
考虑对横坐标进行排序,分别计算其凸包的上凸壳和下凸壳,最后合并即可。

代码如下(逆时针顺序):
\begin{lstlisting}
Vec P[size],C[size];
void convexHull(int n) {
    std::sort(P+1,P+n+1,[](const Vec& a,const Vec& b) {
        return a.x<b.x;
    });
    int top=1;
    C[1]=P[1];
    for(int i=2;i<=n;++i) {
        while(top>=2 && cross(C[top]-C[top-1],
            P[i]-C[top-1])<eps)
            --top;
        C[++top]=P[i];
    }
    for(int i=n-1;i>=1;--i) {
        while(top>=2 && cross(C[top]-C[top-1],
            P[i]-C[top-1])<eps)
            --top;
        C[++top]=P[i];
    }
}
\end{lstlisting}
\subsection{在线凸包}
在线凸包即每次向点集中加入新点,求当前凸包的某些信息。
经典思路是按照极角序(选择一个范围外的定点,否则将产生循环,不易维护)将
凸包上的点存储在set上。
\subsubsection{水平序在线凸包}
\subsection{凸包加法}
\subsection{凸包合并}
\subsection{稀疏包分布}
\begin{itemize}
    \item 若点在圆面上均匀分布,则凸包期望规模为$\Theta(n^{1/3})$。
    \item 若点在凸多边形内部取得(平面,半平面也是凸多边形),凸包期望
    规模为$\Theta(\lg n)$。
    \item 若点根据二维正态分布取得,凸包期望规模为$\Theta(\sqrt{\lg n})$。
\end{itemize}
该内容来自算法导论\cite{ITA3}思考题33-5。
\subsection{二维最小乘积生成树}
\subsection{三维凸包}
