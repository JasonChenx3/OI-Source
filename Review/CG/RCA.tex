\section{旋转卡壳}
\index{R!Rotating Calipers}
其主要思想是维护一对与多边形顶点相切的平行线(一个或两个凸多边形),切点
称为对踵点对(两个多边形时称为并踵点对)。一般当某条平行线与凸多边形的某条边
相切时,所求答案可能取得最值。维护平行线时计算其旋转到与下一条边相切的角度大小,
然后取最小角度旋转。

下面给出求对踵点对的算法:

我的方法(计算角度):算出每个向量到目标向量的角度后比较。
\begin{lstlisting}
bool cmpAngle(const Vec& a,const Vec& b) {
    return cross(a,b)>eps;
}
Vec calcAngle(const Vec& src,const Vec& dst) {
    return Vec(dot(src,dst),cross(src,dst));
}
\end{lstlisting}

为了避免$atan2$精度损失,使用$Vec$维护角,其坐标值表示$(k\cos \theta,k\sin \theta)$。

标准方法(枚举每个对踵点对):
\index{S!Shamos's Algorithm}
\begin{lstlisting}
void process(int i,int j);
FT area(int a,int b,int c) {
    return cross(P[b]-P[a],P[c]-P[a]);
}
void solve(int n) {
    Pos i=(1,n),j(2,n);
    while(ge(area(i,i+1,j+1),area(i,i+1,j))) {
        int j0=++j;
        while(j!=n) {
            ++i;
            process(i,j);
            while(ge(area(i,i+1,j+1),area(i,i+1,j))) {
                ++j;
                if(i==j0 && j==n)
                    return;
                process(i,j);
            }
            if(eq(area(i,i+1,j+1),area(i,i+1,j))) {
                if(i==j0 && j==n)
                    process(i+1,j);
                else
                    process(i,j+1);
            }
        }
    }
}
\end{lstlisting}
注意下标在$[1,n]$内循环右移。

该算法正确性待验证,解释留坑待补。
\index{*TODO!验证解释Shamos算法}

上述内容参考了ACMaker的专栏\footnote{
    \url{https://blog.csdn.net/ACMaker/article/details/3176910}

    原文:\url{http://cgm.cs.mcgill.ca/\~orm/rotcal.frame.html}
}。
