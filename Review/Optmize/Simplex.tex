\section{单纯形算法}
\index{L!Linear Programming}
\index{S!Simplex Algorithm}
给定一组变量$x_1,\cdots,x_k$,这些变量满足一些线性(不)等式的约束。
求在满足约束的前提下目标线性函数的最大值。

\subsection{标准形}
单纯形算法仅解决标准形式的线性规划问题。

标准形式有如下要求:
\begin{itemize}
    \item 最大化目标函数
    \item 约束条件均为等式
    \item 变量均非负
\end{itemize}

不满足标准形式的线性规划问题可以通过一些修改来转化:
\begin{itemize}
    \item 最小化目标函数:系数取反转化为最大化。
    \item 若不等式为小于等于不等式或大于等于不等式,可以引入一个新的
    非负变量,对不等式进行调整使不等号变为等号。
    \item 若变量没有约束,则将该变量表示为两个非负变量之差。
\end{itemize}

\subsection{单纯形法}
上述标准形可以描述为矩阵形式,其中$A$为$m*n$的矩阵,满足$x\geq 0,Ax=b$,
最大化$cx$。

单纯形法的步骤如下:
\begin{enumerate}
    \item 选取一组基变量
    \item 得到基可行解
    \item 判断该解是否为最优解,若是则退出,否则返回步骤1。
\end{enumerate}

\subsubsection{基变量}
选取矩阵$A$中$m$列线性无关的向量,它们对应的变量称为基变量。对整个矩阵进行
高斯消元,使得这$m$列向量可以组成$I_m$。那么此时基变量可以表示为常数+非基
变量的线性组合。

若常数均非负,简单地令非基变量为0,则得到一组可行解。它对应了所有约束构成的可行域内
的某一顶点。单纯形法的思路就是选取不同的基向量得到不同的可行域顶点。当然当前顶点的
计算结果可以指导下一次迭代的基向量选择。
\subsubsection{判断最优解}
用消元后的矩阵$A'$消去目标函数向量$c$中基向量的系数,可以将目标函数$cx$表示为常数+
非基向量的线性组合。

\begin{theorem}
    某基向量得出的解$x$为最优解当且仅当它对应的目标函数$cx$的常数+非基向量线性组合
    形式中,非基向量的系数均不大于0。
\end{theorem}

\paragraph{证明} 若存在一个系数$>0$,则

\subsubsection{初始化基变量}
\subsubsection{转动}
\subsection{对偶线性规划}
\subsection{全幺模矩阵}
上述内容参考了Angel\_Kitty的博客\footnote{
    线性规划之单纯形法【超详解+图解】
    \url{http://www.cnblogs.com/ECJTUACM-873284962/p/7097864.html}
}。
