\section{线性规划}\label{LP}
\index{L!Linear Programming}
\subsection{定义与规范描述形式}
\subsubsection{定义}
\paragraph{线性函数} 给定$n$个实数$a_1,a_2,\cdots,a_n$与$n$个变量
$x_1,x_2,\cdots,x_n$,线性函数$f$是这些变量的线性加权和,即
\begin{displaymath}
    f(x_1,x_2,\cdots,x_n)=\sum_{i=1}^n{a_ix_i}
\end{displaymath}
\paragraph{线性约束}
给定一个实数$b$,线性约束是满足$f(x_1,x_2,\cdots,x_n)=b,\leq b$或$\geq b$的
线性等式/不等式。
\paragraph{线性规划}
一个线性规划问题是在使一组变量满足一组有限个线性约束的前提下,
最大(小)化某个线性函数值。
\paragraph{可行解}
满足所有线性约束的解。
\paragraph{目标函数与目标值}
目标函数是我们希望最大(小)化其值的线性函数,目标值是特定变量组合对应的目标函数值。
\paragraph{线性规划的解}
一个线性规划称为不可行的当且仅当它没有可行解。一个可行线性规划称为
无界的当且仅当它没有最优值。
\paragraph{矩阵表示}
下面的内容中,大小为$n$的向量$x$表示由变量$x_1,x_2,\cdots,x_n$组成的向量,
大小为$n$的向量$c$表示目标函数$f:x\rightarrow c^Tx$的系数,大小为$m*n$的矩阵$A$
表示$m$个约束的系数,大小为$n$的向量$b$表示由这$m$个约束的常数项组成的向量。
\subsubsection{标准型}
标准型的描述如下:
\begin{itemize}
    \item 最大化 $c^Tx$
    \item 满足约束$Ax\leq b$
    \item 满足非负约束$x\geq 0$
\end{itemize}

任意线性规划都可以按照如下方法将其转换为等价的标准型线性规划:
\begin{itemize}
    \item 要求最小化目标函数值:将目标函数的系数取反。
    \item 变量不具有非负约束:若变量$x_i$没有非负约束,则引入两个非负变量
    $x_{i1},x_{i2}$,满足$x_i=x_{i1}-x_{i2}$。然后把线性规划中的$x_i$
    替换为$x_{i1}-x_{i2}$。
    \item 约束中有等式约束:将$f(x)=b$拆为$f(x)\geq b$和$f(x)\leq b$。
    \item 约束中有$f(x)\geq b$形式的约束:约束的系数与常数均取反。
\end{itemize}
\subsubsection{松弛型}
单纯形算法需要把不等式约束转换为等式约束。
对于每一个约束$f_i(x)\leq b_i$,引入一个非负{\bfseries 松弛变量}$x_{n+i}$使得
$x_{n+i}=b_i-f_i(x)$。记目标函数值为$z$,则也可以引入等式$z=c^Tx$。在此称等式
左边的变量为{\bfseries 基本变量},等式右边的变量称为{\bfseries 非基本变量}。实际上
基本变量与目标函数值都被表示为常数+非基本变量线性加权和的形式。等式组的变量不同时出现
于等号两边,基本变量也不会出现两次。{\bfseries 注意等式左边的变量不等同于松弛变量,
因为单纯形算法的转动操作会使变量的位置改变。}
\subsection{单纯形算法}
\index{S!Simplex Algorithm}
\subsubsection{原理}
\paragraph{基本解}
将线性规划问题转换为松弛型,令非基本变量的值为0,就可以确定基本变量的值与目标函数值。
这是该线性规划的一个{\bfseries 基本解}。若它对应的基本变量的值均非负,则说明它是
一个{\bfseries 基本可行解}。

单纯形算法的步骤就是:
\begin{itemize}
    \item 找到初始基本可行解,若没有则说明该线性规划不可行。
    \item 不断迭代:计算当前基本解对应的解和目标函数值,判断是否最优。如果不是,
    根据计算结果执行``转动''操作更换基本变量得到更优解。
\end{itemize}
\subsubsection{转动pivot}
转动过程每次选取一个非基本变量$x_e$和一个基本变量$x_l$,交换它们在约束中的
位置。因此$x_e$称为替入变量,$x_l$称为替出变量。

在实现中用$A[m][n]$表示约束和目标函数,$id[1\cdots n]$表示等式右边每列的非基本
变量编号,$id[n+1\cdots n+m]$表示等式左边每列的基本变量编号。行$A[0]$用于存储
目标函数信息。$A[i][0]$满足
\begin{displaymath}
    A[i][0]+\textrm{目标函数值}z=\sum_{j=1}^n{A[i][j]x_{id[j]}}\textrm{~if~}i=0
\end{displaymath}
该式可由$z=c^Tx$推导。

或
\begin{displaymath}
    A[i][0]-x_{id[n+i]}=\sum_{j=1}^n{A[i][j]x_{id[j]}}\textrm{~otherwise~}
\end{displaymath}
该式可由$x'=b-f(x)$推导。

令非基本变量值为0后,$-A[0][0]$就是目标函数值$z$,$A[i][0](i=1\cdots m)$
就是$x_{id[n+i]}$的值。初始化矩阵时,令$A[0][0]$为0,其余参数直接填入对应位置,
然后让$id[i]=i(i=1\cdots n)$。由于最终我们只要$x_i(i=1\cdots n)$的值,不必
初始化松弛变量的编号。

执行转动时,首先交换对应位置的变量编号,然后处理替出变量所在约束的矩阵行,
最后处理其余行。

\paragraph{处理替出变量所在行}
记替入变量为$x_{id[e]}$,替出变量为$x_{id[n+l]}$,该行$L=A[l]$除
$x_{id[e]},x_{id[n+l]}$外的其余非基本变量组成的向量为$X$,系数向量为$c$,有
\begin{eqnarray*}
    L[0]-x_{id[n+l]}&=&c^TX+L[e]x_{id[e]}\\
    \Rightarrow \frac{L[0]}{L[e]}-\frac{1}{L[e]}x_{id[n+l]}&=&
    \frac{c^T}{L[e]}X+x_{id[e]}\\
    \Rightarrow \frac{L[0]}{L[e]}-x_{id[e]}&=&
    \frac{c^T}{L[e]}X+\frac{1}{L[e]}x_{id[n+l]}
\end{eqnarray*}

\paragraph{处理其余行}
类似于高斯消元法,用行$A[l]$消去其它行中的$A[i][e]$项。注意还要计算替出变量的系数,
所以$A[i][e]$要先置0。
\paragraph{性质}
如果当前基本解是可行解且pivot操作的参数$l,e$满足$A[l][0]/A[l][e]$是所有满足
$A[l][e]>0$的行中的最小值,那么pivot后得到的基本解仍然是一个可行解。

证明:
\begin{itemize}
    \item 替出变量所在行:由于$L[0]$非负且$L[e]$为正,pivot后
    $x_{id[e]}$的值$\frac{L[0]}{L[e]}$非负。
    \item 其余行:由于$A[l][0]/A[l][e]$最小,转动替出变量所在行后
    该行的常数项$A[l][0]$最小。设当前要消第$j$行:
    \begin{itemize}
        \item 若$A[j][e]>0$,则满足$A[l][0]/A[l][e]\leq A[j][0]/A[j][e]$。
        变换得$A[j][0]'=A[j][0]-A[l][0]/A[l][e]*A[j][e]\geq 0$,仍然保持非负。
        \item 若$A[j][e]\leq 0$,则$A[j][0]$消元后不减,仍然保持非负。
    \end{itemize}
\end{itemize}

\subsubsection{主过程simplex}
\paragraph{判断是否为最优解}
当对固定的基本变量组进行矩阵行变换后目标函数中所有的系数都非正,
则说明当前基本解是最优解。

证明:由于目标函数值只取决于非基本变量的值,且基本解中非基本变量均为0,只能提升
非基本变量的值来改变目标函数值。由于系数均非正,即使提升非基本变量的值仍然可行,
目标函数值也不增。
\paragraph{选取替入/替出变量}
选取目标函数值中系数非正的非基本变量进行提升是无用的,因此要选取一个满足$A[0][e]>0$
的变量当做替入变量。由pivot的性质可得只有$A[l][e]>0$且$A[l][0]/A[l][e]>0$才能在
pivot操作后保证其仍然为基本可行解。也就是说,一旦确立了替入变量,替出变量的选择就被
限制了。在使用pivot操作提升非基本变量后,不仅保证了当前基本解仍然是可行解,还能提升
目标函数值(除非pivot后替入变量的值为0)。目标函数值不提升的现象被称为{\bfseries 退化},
可能导致算法出现循环无法终止,因此需要使用Bland规则来避免循环。

\index{B!Bland's Rule}
根据Bland规则,总是选取下标最小的替入变量以及对应最优且下标最小的替出变量,可以避免
单纯形算法的循环。使用Bland规则后,可以保证算法在$\binomial{n+m}{m}$次迭代内终止,当然
实践中单纯形算法的表现很好,很少遇到如此刁钻的数据(或许随机化有助于改善算法运行时间)。

算法每次迭代后的目标函数值不降,且算法会在有限次迭代内结束(枚举完所有可能的基本变量
组后),因此算法会输出最优解。{\bfseries 这里还需要证明基本可行解集合中含有最优解。
不过基本解对应了凸集中的边界,根据经验最优解一定在边界上取得。更标准的证明留坑待补。}
\index{*TODO!单纯形算法正确性证明}
\paragraph{判断无界}
若存在系数非负的候选替入变量但不存在对应的替出变量,则该线性规划是无界的。

不存在对应的替出变量意味着所有$A[i][e]\leq 0$,那么任意提高该替入变量的值,
仍然保持其它非基本变量为0,既能保持当前基本解仍然为可行解,还能提高目标函数值。
因此该线性规划是无界的。
\subsubsection{初始化init}
找到一个初始基本可行解意味着要让松弛型中约束的常数项非负。
\paragraph{辅助线性规划}
首先检查初始松弛型是否已经对应了基本可行解。记$k$为满足$A[k][0]$最小的下标,
若基本解不可行则有$A[k][0]<0$。

然后引入一个新变量$x_0$,和原有的线性规划组成一个新的标准型线性规划:
\begin{itemize}
    \item 最大化$-x_0$
    \item 满足约束$Ax-x_0\leq b$
    \item 满足约束$x,x_0\geq 0$
\end{itemize}
原线性规划可行当且仅当该线性规划的最优值是0。

首先构造出该线性规划$B$,其中$x_0$放置于非基本变量组的末尾。然后执行\\
$pivot(k,n_B)$,这样就得到了线性规划$B$的一个基本可行解。

证明:
\begin{itemize}
    \item 替出变量所在行:$L[0]$为负且$L[e]=-1$,转动后$L[0]'=-L[0]>0$。
    \item 其余行:由于$A[i][e]=-1$,每行都要加上一倍$L$,由于$L[0]$是
    所有$A[i][0]$中的最小项,所以$A[i][0]'\geq 0$。
\end{itemize}

然后对$B$运行simplex找到最优值,最优值不为0则返回不可行。

若$x_0$此时为基本变量,在其对应行中选取一个系数非0的非基本变量作为替入变量,
把$x_0$换出去。

注意:
\begin{itemize}
    \item pivot操作并不会使最优解变小,因为转动后由基本解的定义
    可知$x_0=0$。
    \item pivot操作不会让基本解不可行,因为$L[0]=0$,其它的$A[i][0]$仍然
    保持非负。
\end{itemize}

移除$x_0$,把$B$的结果写回原线性规划中,完成初始化。
\paragraph{随机初始化}
不断迭代,每次随机选一个$l$满足$A[l][0]<0$,再随机选择一个$e$满足$A[l][e]<0$,
执行pivot后可以使$A[l][0]$为正。
\paragraph{避免初始化}
可以根据题目性质,松弛一些约束(比如将$=$松弛为$\leq$),使得初始线性规划就是合法
的松弛形。
\subsubsection{算法实现(UOJ179)}
为了过掉UOJ的Extra Test,这里使用了两种初始化方法的混合(然而还是过不去,
只能指望Mehrotra predictor–corrector method了)。
\lstinputlisting{Source/Templates/SimplexSLP.cpp}

\subsubsection{稀疏矩阵优化}
单纯形算法性能的关键在于pivot操作。如果矩阵$A$为稀疏矩阵,则首先考虑使用其它算法
解决。一定要用单纯形算法解决的,可以扫描一遍行$L$记录非零元素到队列,消元时也判断
一下系数是否为0。
\subsection{对偶线性规划}
标准型的对偶线性规划为:
\begin{itemize}
    \item 最小化$b^Ty$
    \item 满足约束$A^Ty\geq c$
    \item $y\geq 0$
\end{itemize}

标准型和它的对偶线性规划最优值相等,即$c^Tx=b^Ty$。

对偶线性规划用来转化问题,但要慎用单纯形算法计算。

要注意转化问题时要不要加入选择个数限制,有时由于贪心的缘故可以去掉这个限制以
简化问题。
\subsection{全幺模矩阵}
\index{T!Totally Unimodular Matrix}

一个矩阵$A$是{\bfseries 全幺模矩阵}的充分条件如下:
\begin{itemize}
    \item $A$的元素仅有-1,0,1。
    \item 每列最多有2个非0数。
    \item 行可以分为2个集合,根据列来划分集合:
    \begin{itemize}
        \item 若列中有2个同号非0数,两行不在同一集合
        \item 若列中有2个异号非0数,两行在同一集合
    \end{itemize}
\end{itemize}

若线性规划的系数矩阵$A$为全幺模矩阵,或许可以用int代替double存储矩阵。

任何最大流以及最小费用最大流问题的线性规划矩阵都是全幺模矩阵。\CJKsout{如果发现
线性规划矩阵是个全幺模矩阵,可以考虑将其转化为简单的网络流模型来做。}

上述内容参考了算法导论\cite{ITA3}第29章与Angel\_Kitty的博客\footnote{
    线性规划之单纯形法【超详解+图解】
    \url{http://www.cnblogs.com/ECJTUACM-873284962/p/7097864.html}
},该篇文章末尾引用了Candy?的博客\footnote{
    [单纯形法与线性规划]【学习笔记】
    \url{https://www.cnblogs.com/candy99/p/lp.html}
}。网上博客的术语定义杂乱,这里使用算法导论中的定义。
