\section{线性规划}
\index{L!Linear Programming}
\subsection{定义与规范描述形式}
\subsubsection{定义}
\paragraph{线性函数} 给定$n$个实数$a_1,a_2,\cdots,a_n$与$n$个变量
$x_1,x_2,\cdots,x_n$,线性函数$f$是这些变量的线性加权和,即
\begin{displaymath}
    f(x_1,x_2,\cdots,x_n)=\sum_{i=1}^n{a_ix_i}
\end{displaymath}
\paragraph{线性约束}
给定一个实数$b$,线性约束是满足$f(x_1,x_2,\cdots,x_n)=b,\leq b$或$\geq b$的
线性等式/不等式。
\paragraph{线性规划}
一个线性规划问题是在使一组变量满足一组有限个线性约束的前提下,
最大(小)化某个线性函数值。
\paragraph{可行解}
满足所有线性约束的解。
\paragraph{目标函数与目标值}
目标函数是我们希望最大(小)化其值的线性函数,目标值是目标函数在某个特定变量
输入的值。
\paragraph{线性规划的解}
一个线性规划称为不可行的当且仅当它没有可行解。一个可行线性规划称为
无界的当且仅当它没有最优值。
\paragraph{矩阵表示}
下面的内容中,大小为$n$的向量$x$表示由变量$x_1,x_2,\cdots,x_n$组成的向量,
大小为$n$的向量$c$表示目标函数$f:x->c^Tx$的系数,大小为$m*n$的矩阵$A$表示
$m$个约束的系数,大小为$n$的向量$b$表示由这$m$个约束的常数$b$组成的向量。
\subsubsection{标准型}
标准型的描述如下:
\begin{itemize}
    \item 最大化 $c^Tx$
    \item 满足约束$Ax\leq b$
    \item 满足非负约束$x\geq 0$
\end{itemize}

任意线性规划都可以按照如下方法将其转换为等价的标准型线性规划:
\begin{itemize}
    \item 要求最小化目标函数值:将目标函数的系数取反。
    \item 变量不具有非负约束:若变量$x_i$没有非负约束,则引入两个非负变量
    $x_{i1},x_{i2}$,满足$x_i=x_{i1}-x_{i2}$。然后把线性规划中的$x_i$
    替换为$x_{i1}-x_{i2}$。
    \item 约束中有等式约束:将$f(x)=b$拆为$f(x)\geq b$和$f(x)\leq b$。
    \item 约束中有$f(x)\geq b$形式的约束:约束的系数与常数均取反。
\end{itemize}
\subsubsection{松弛型}
单纯形算法需要把不等式约束转换为等式约束。
对于每一个约束$f_i(x)\leq b_i$,引入一个非负松弛变量$x_{n+i}$使得
$x_{n+i}=b_i-f_i(x)$。记目标函数值为$z$,则也可以引入等式$z=cx$。
在此称等式左边的变量为{\bfseries 基本变量},等式右边变量称为
{\bfseries 非基本变量}。等式组的变量不同时出现于等号两边,基本变量也不会出现两次。
事实上,这些约束把基本变量与目标函数值表示为非基本变量的线性加权和。{\bfseries
注意等式左边的变量不等同于松弛变量,因为单纯形算法的转动操作会使变量的位置改变。}
\subsection{单纯形算法}
\index{S!Simplex Algorithm}
\subsubsection{原理}
\paragraph{基本解}
将线性规划问题转换为松弛型,令非基本变量的值为0,就可以确定基本变量的值与目标函数值。
这是该线性规划的一个{\bfseries 基本解}。若它对应的基本变量的值均非负,则说明它是
一个{\bfseries 基本可行解}。

单纯形算法的步骤就是:
\begin{itemize}
    \item 找到初始基本可行解,若没有则说明该线性规划不可行。
    \item 不断迭代:计算当前基本解对应的解和目标函数值,判断是否最优。如果不是,
    根据计算结果执行``转动''操作更换基本变量得到更优解。
\end{itemize}
\subsubsection{转动pivot}
转动过程每次选取一个非基本变量$x_e$和一个基本变量$x_l$,交换它们在约束中的
位置。因此$x_e$称为替入变量,$x_l$称为替出变量。转动过程实质上就是矩阵的初等行变换。

在实现中用$A[m][n]$表示约束和目标函数,$id[1\cdots n]$表示等式右边每列的非基本
变量编号,$id[n+1\cdots n+m]$表示等式左边每列的基本变量编号。$A[i][0]$满足
\begin{displaymath}
    A[i][0]+\mathrm{目标函数值}z=\sum_{j=1}^n{A[i][j]x_{id[j]}}
\end{displaymath}
该式可由$z=cx$推导。

或
\begin{displaymath}
    A[i][0]-x_{id[n+i]}=\sum_{j=1}^n{A[i][j]x_{id[j]}}
\end{displaymath}
该式可由$x'=b-f(x)$推导。

行$A[0]$用于存储目标函数信息。令非基本变量值为0后,$-A[0][0]$就是目标函数值$z$,
$A[i][0](i=1\cdots m)$就是$x_{id[n+i]}$的值。初始化矩阵时,令$A[0][0]$为0,
其余参数直接填入对应位置,然后让$id[i]=i(i=1\cdots n)$。由于最终我们只要
$x_i(i=1\cdots n)$的值,不必初始化其它变量的编号。

执行转动时,首先交换对应位置的变量编号,然后处理替出变量所在约束的矩阵行,
最后处理其余行。

\paragraph{处理替出变量所在行}
记替入变量为$x_{id[e]}$,替出变量为$x_{id[n+l]}$,该行$L=A[l]$除
$x_{id[e]},x_{id[l]}$外的其余非基本变量组成的向量为$X$,系数向量为$c$,有
\begin{eqnarray*}
    L[0]-x_{id[l]}&=&c^TX+L[e]x_{id[e]}\\
    \Rightarrow \frac{L[0]}{L[e]}-\frac{1}{L[e]}x_{id[l]}&=&
    \frac{c^T}{L[e]}X+x_{id[e]}\\
    \Rightarrow \frac{L[0]}{L[e]}-x_{id[e]}&=&
    \frac{c^T}{L[e]}X+\frac{1}{L[e]}x_{id[l]}
\end{eqnarray*}

\paragraph{处理其余行}
类似于高斯消元法,用行$A[l]$消去其它行中的$A[i][e]$项。注意要算上替出变量的系数,
所以$A[i][e]$要先置0。
\subsubsection{主过程simplex}
\paragraph{判断是否为最优解}
当对固定的基本变量组进行矩阵行变换后目标函数中所有的系数都非正,
则说明当前基本解是最优解。

证明:若存在某个系数$c_i$为正,则说明把对应的非基本变量从0稍微向上
调整就可以得到更优解。
\paragraph{选取替入/替出变量}
\paragraph{判断无界}
\paragraph{Bland规则}
总是选取下标最小的替入变量以及对应最优且下标最小的替出变量,可以避免
单纯形算法的循环。
\subsubsection{初始化init}
\paragraph{辅助线性规划}
\paragraph{随机初始化}

模板(UOJ179):
\lstinputlisting{Source/Templates/Simplex.cpp}
\subsection{对偶线性规划}
\subsection{全幺模矩阵}

上述内容参考了算法导论\cite{ITA3}第29章与Angel\_Kitty的博客\footnote{
    线性规划之单纯形法【超详解+图解】
    \url{http://www.cnblogs.com/ECJTUACM-873284962/p/7097864.html}
},该篇文章末尾引用了Candy?的博客\footnote{
    [单纯形法与线性规划]【学习笔记】
    \url{https://www.cnblogs.com/candy99/p/lp.html}
}。网上博客的术语定义杂乱,实践证明算法导论中的定义比较清楚。
