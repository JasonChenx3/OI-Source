\section{随机化算法}
随机化算法一般用来解决判定性问题与最优化问题。
\subsection{Monte Carlo算法}
\subsubsection{有关枚举元素的问题}
该问题需要枚举每个子集,然后对单个枚举进行计算。

对于这种问题我们有复杂度无法接受的穷举法,将穷举改为随机选取固定次数
的子集,可以在规定时间内完成计算。

使用一些贪心技巧或利用题目性质可以提高单次枚举正确率与采样数。
\subsubsection{线性加权和问题}
每次随机生成一个权重向量与向量做内积/与矩阵做乘法。
\subsubsection{复杂度与正确率}
OI中一般只需设计产生单侧错误算法。

若单次正确率为$p$,复杂度为$O(f(n))$,则测量$k$次的正确率为$1-(1-p)^k$,
复杂度为$O(kf(n))$。
\subsection{Las Vegas算法}
留坑待补。

上述内容参考了国家集训队2014论文集胡泽聪的《随机化算法在信息学竞赛中的应用》。
