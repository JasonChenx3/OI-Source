\section{01分数规划}
\index{F!Fractional Programming}
该问题可表述为有一堆物品,每个物品有价值$a_i$与代价$b_i$,
询问选择的物品集合所能得到的$\frac{\displaystyle \sum_{i\in S}{a_i}}
	{\displaystyle \sum_{i\in S}{b_i}}$的最大值。

{\bfseries 注意当$\displaystyle \sum_{i\in S}{a_i}$或
$\displaystyle \sum_{i\in S}{b_i}$表示为$|S|$时,这是一个隐式的分数规划问题。
血泪史:[SCOI2014]方伯伯运椰子}

\subsection{二分答案法}

设答案为$x$,存在点集$S$满足$\frac{\displaystyle \sum_{i\in S}{a_i}}
	{\displaystyle \sum_{i\in S}{b_i}}\geq x$,将该式转化为
$\displaystyle \sum_{i\in S}{a_i}-x\cdot \sum_{i\in S}{b_i}\geq 0$。
因此可二分答案$x$,对原数据进行一些修改,便可将原问题转化为解决一个判定问题
(或是新的最优化问题,即检查
$\displaystyle f(x)=\sum_{i\in S}{a_i}-x\cdot \sum_{i\in S}{b_i}$
的最大值是否不小于0)。

一般使用网络流/SPFA判负环/MST辅助判定。

例如对于最小平均值环问题,二分环平均长度$x$,将每条边的长度减去$x$,然后
判断负环,得到下一次迭代的二分范围。

\subsection{Dinkelbach法}
\index{D!Dinkelbach Method}
记选择方案$S$对应直线$f_S(x)=\sum_{i\in S}{a_i}-x\cdot \sum_{i\in S}{b_i}$,
该直线的横截距为该方案的目标函数值。若$f_S(x)\geq 0$则说明该直线所对应的横截距$\geq x$。

在二分过程中检查二分点$x$的合法性时,有时通过求$max\{f(x)\}$来判定。
满足检查条件时仅仅令$L=x$就显得浪费了,因为检查下一二分点时得到的最优解$x$可能不变。
Dinkelbach法的思路是维护答案的下界,使用当前最优解对应的直线横截距来当做下一次
迭代的起点,这种做法比二分法更快。不过需要保存解这一要求限制了它的应用范围(编码更麻烦)。

事实上解不必逐步递增,也可以使用类似牛顿迭代法的思想,解可以左右移动,但是每次迭代要选取
最大的移动截距(例如取最大值时尽可能向右移,尽可能少后退)。选取一个好的初始解可以大幅加快
迭代速度。

此法参考了tianxiang971016的博客\footnote{
	01分数规划问题相关算法与题目讲解(二分法与Dinkelbach算法) - ztx
	\url{https://blog.csdn.net/hzoi\_ztx/article/details/54898323}
}。

上述方法的性能比较(以LuoguP4292 [WC2010]重建计划的评测结果为准):
\begin{itemize}
	\item 二分法:9622ms
	\item Dinkelbach法:1864ms
	\item Dinkelbach法优化:1085ms
\end{itemize}

\CJKsout{2019.3.3:为什么又是rank2。。。}
