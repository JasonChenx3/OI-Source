\section{Catalan数}
\subsection{性质}
\index{C!Catalan Numbers}
Catalan数是组合数学中的常见数列
\footnote{A000108 - OEIS \url{http://oeis.org/A000108}},其前几项为
\begin{displaymath}
	1, 1, 2, 5, 14, 42, 132, 429, 1430, \cdots
\end{displaymath}

Catalan数(记为$C_n$)满足如下关系:
\begin{eqnarray}
	C_0&=&C_1=1\\
	C_{n+1}&=&\sum_{i=0}^n{C_iC_{n-i}}\label{CT2}\\
	&=&\sum_{i=1}^n{C_iC_{n+1-i}}\label{CT3}\\
	C_n&=&\frac{4n-2}{n+1}C_{n-1}\\
	C_n&=&\binomial{2n}{n}-\binomial{2n}{n+1}=\frac{1}{n+1}\binomial{2n}{n}\\
	C_n&=&\prod_{k=2}^n\frac{n+k}{k}
\end{eqnarray}
根据Striling近似公式
\begin{displaymath}
	n!\sim\sqrt{2\pi n}\left(\frac{n}{e}\right)^n
\end{displaymath}
可得
\begin{displaymath}
	C_n\sim\frac{4^n}{\sqrt{\pi} n^\frac{3}{2}}
\end{displaymath}
\subsection{常见应用}
\subsubsection{括号序列,出栈序列,网格行走}
\paragraph{括号序列} 给定$2n$个位置填上左右括号使括号匹配(对于每一个位置之前的
左括号必须不少于右括号)。
\paragraph{出栈序列} 求将$n$个元素入栈一次(限制顺序)并出栈一次(不限制顺序)
的方案数(对于每一次操作都要保证栈不出现下溢,即入栈元素不少于出栈元素)。
\paragraph{网格行走} 在一个$n*n$的网格内从左下角移动到右上角,纵坐标必须不少于
横坐标,求方案数。
\paragraph{分析}
这三个问题是同构的,都满足操作数为$2n$且限制任意时刻操作A的数目不少于操作B的数目。
它们的答案都是$C_n$,以括号序列问题为例,通过等式~\ref{CT2}理解:
将括号序列看做由一个可分割的序列加上一个不可分割的序列(即最外层有一对配对括号)得来,
左边为$n_1+1$对,右边为$n_2$对,满足$n_1+n_2=n-1$,这种方案的贡献为
$C_{n_1}C_{n_2}$。
\subsubsection{二叉树构型计数}
\paragraph{有$n$个节点的二叉树}
通过等式~\ref{CT2}理解:枚举左右子树大小,满足左右子树节点数为$n-1$。
\paragraph{有$n+1$个叶子节点的满二叉树}
通过等式~\ref{CT3}理解:枚举左右子树叶子节点数,满足其和为$n+1$。
\subsubsection{阶梯填充}
用$n$个长方形填充$n*n$的阶梯的方案数为$C_n$。

不严格证明:填充一个以直角顶点与阶梯顶点为对顶点的长方形,使其分为大小为
$n_1*n_1,n_2*n_2$的两个小阶梯,满足$n_1+n_2=n-1$,分别分配$n_1,n_2$
个长方形的份额,就成为子问题了。该分析满足等式\ref{CT2}。
\subsubsection{凸包分割}
将$n+2$个顶点的凸包分为三角形的方案数为$C_n$。

猜想:最终将分为$n$个三角形。

证明留坑待补。
\subsubsection{圆上点连线}
将圆上的$2n$个点两两配对连线,所连$n$条线段不相交的方案数为$C_n$。

证明留坑待补。

\index{*TODO!Catalan应用证明}
上述内容参考了Wikipedia-EN\footnote{Catalan number - Wikipedia
	\url{https://en.wikipedia.org/wiki/Catalan\_number}}。
