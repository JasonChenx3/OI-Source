\section{Stirling数}
\index{S!Stirling Number}
\subsection{第一类Stirling数}
令$\stirlingA{n}{k}$为把$n$个点放到$k$个环内的方案数,有
\begin{eqnarray*}
    \stirlingA{n}{0}&=&0 \textrm{~for all $n\geq 1$}\\
    \stirlingA{n}{n}&=&1 \textrm{~for all $n\geq 0$}\\
    \stirlingA{n}{k}&=&(n-1)\stirlingA{n-1}{k}+\stirlingA{n-1}{k-1}\\
    \sum_{i=0}^n{\stirlingA{n}{i}}&=&n!
\end{eqnarray*}

证明递推式:
\begin{itemize}
    \item 若将当前点丟给之前的环,则可以选择$n-1$个点在其右边,因此贡献
    $(n-1)\stirlingA{n-1}{k}$。
    \item 当前点自成一环,贡献$\stirlingA{n-1}{k-1}$。
\end{itemize}
\subsection{第二类Stirling数}
令$\stirlingB{n}{k}$为把$n$个点放到$k$个集合内的方案数,有
\begin{eqnarray}
    \stirlingB{n}{0}&=&0 \textrm{~for all $n\geq 1$}\\
    \stirlingB{n}{n}&=&1 \textrm{~for all $n\geq 0$}\\
    \stirlingB{n}{k}&=&k\stirlingB{n-1}{k}+\stirlingB{n-1}{k-1}\label{SB3}\\
    \stirlingB{n}{k}&=&\frac{1}{k!}\sum_{i=0}^k{(-1)^{k-i}\binomial{k}{i}i^n}
    \label{SB4}
\end{eqnarray}

证明等式~\ref{SB3}:
\begin{itemize}
    \item 若将当前点丟给之前的集合,则可以选择$k$个集合,因此贡献
    $k\stirlingA{n-1}{k}$。
    \item 当前点自成一集合,贡献$\stirlingB{n-1}{k-1}$。
\end{itemize}

证明~\ref{SB4}:
考虑计算将$n$个点放入$k$个带编号集合且无空集的方案数:
\begin{eqnarray*}
    N&=&k^n+\sum_{i=1}^k{(-1)^i\binomial{k}{i}(k-i)^n}
    \textrm{~(根据定理~\ref{ExDML})}\\
    &=&k^n+\sum_{i=0}^{k-1}{(-1)^{k-i}\binomial{k}{i}i^n}\\
    &=&\sum_{i=0}^k{(-1)^{k-i}\binomial{k}{i}i^n}
\end{eqnarray*}
由于要求的是无标号的方案数,因此最终答案需要除以$k!$。

一个常用的转化:
\begin{theorem}
    \begin{displaymath}
        n^k=\sum_{i=0}^n{\stirlingB{k}{i}\binomial{n}{i}i!}
    \end{displaymath}
\end{theorem}

意义:左边是将$k$个球放入$n$个盒子里的方案数,右边枚举非空盒子数,那么$i$个非空盒子的
方案数即为划分非空子集的方案数*盒子排列数(盒子选取方案数*盒子标号方案数)。
