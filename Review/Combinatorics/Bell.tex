\section{贝尔数}
贝尔数$B_n$是大小为$n$的集合的划分方案数。

贝尔数的前几项为:$1, 1, 2, 5, 15, 52, 203, 877,\cdots$(参见OEIS-A000110\\
\url{https://oeis.org/A000110})

$B_n$满足等式:
\begin{eqnarray*}
	B_0=B_1&=&1\\
	B_{n+1}&=&\sum_{i=0}^n{\binomial{n}{i}B_i}\\
	B_n&=&\sum_{i=1}^n{\stirlingB{n}{i}}
\end{eqnarray*}

\subsection{因子分解应用}
若无平方因子的整数$\displaystyle n=\sum_{i=1}^k{p_i}$,
则整数$n$可以分解为$B_k$种因子的乘积表达式。
\subsection{贝尔数计算}
\subsubsection{DFS}若要得到不同方案的状态表示,限制下阶段DFS的集合大小以避免重复计算。
\subsubsection{贝尔三角形}
与推杨辉三角形类似,递推方法如下:
\begin{eqnarray*}
	A[0][1]&=&1\\
	A[n][1]&=&A[n-1][n]\\
	A[n][m]&=&A[n][m-1]+A[n-1][m-1]\\
	B_n&=&A[n][1]
\end{eqnarray*}
使用滚动数组优化空间。

\index{*TODO!贝尔三角形的意义}
\subsubsection{生成函数法}
利用Bell数的生成函数$e^{e^x-1}$,计算多项式exp。
\subsubsection{贝尔数模质数}
\index{T!Touchard's Congruence}
\begin{theorem}{Touchard's Congruence}
    \begin{eqnarray*}
        B_{p+n}&\equiv& B_n+B_{n+1}\pmod{p}\\
        B_{p^m+n}&\equiv& mB_n+B_{n+1}\pmod{p}\\
    \end{eqnarray*}
\end{theorem}

首先$O(p^2)$使用贝尔三角形预处理出$p$以内的贝尔数模$p$的值。

若要求$B_n$,则对$n$进行$p$进制分解,记$n_i$为$n$在权为$p^i$时的位。

以最低位作为下标,从$B_{0\ldots p}$开始递推,逐步加$p^i$递推直至次数等于$n_i$。
最后取数组第$n_0$项。

参考代码(PA2008 Cliquers Strike Back):
\lstinputlisting{Source/Source/Count/bzoj3501.cpp}

时间复杂度$O(p^2lg_pn)$。

该内容参考了Claris的博客\footnote{
    BZOJ3501 : PA2008 Cliquers Strike Back
    \url{https://www.cnblogs.com/clrs97/p/4714467.html}
}和Wikipedia-EN\footnote{
    Bell number - Wikipedia:Modular arithmetic\\
    \url{https://en.wikipedia.org/wiki/Bell\_number\#Modular\_arithmetic}
}。
{\bfseries 令人失望的是维基百科已全面被封。}
