\chapter{组合数学}
\minitoc
\section{Catalan数}
\subsection{性质}
\index{C!Catalan Numbers}
Catalan数是组合数学中的常见数列
\footnote{A000108 - OEIS \url{http://oeis.org/A000108}},其前几项为
\begin{displaymath}
	1, 1, 2, 5, 14, 42, 132, 429, 1430, \ldots
\end{displaymath}

Catalan数(记为$C_n$)满足如下关系:
\begin{eqnarray}
	C_0&=&C_1=1\\
	C_{n+1}&=&\sum_{i=0}^n{C_iC_{n-i}}\label{CT2}\\
	&=&\sum_{i=1}^n{C_iC_{n+1-i}}\label{CT3}\\
	C_n&=&\frac{4n-2}{n+1}C_{n-1}\\
	C_n&=&{2n \choose n}-{2n \choose n+1}=\frac{1}{n+1}{2n \choose n}\\
	C_n&=&\prod_{k=2}^n\frac{n+k}{k}
\end{eqnarray}
根据Striling近似公式
\begin{displaymath}
	n!\sim\sqrt{2\pi n}\left(\frac{n}{e}\right)^n
\end{displaymath}
可得
\begin{displaymath}
	C_n\sim\frac{4^n}{\sqrt{\pi} n^\frac{3}{2}}
\end{displaymath}
\subsection{常见应用}
\subsubsection{括号序列,出栈序列,网格行走}
\paragraph{括号序列} 给定$2n$个位置填上左右括号使括号匹配(对于每一个位置之前的
左括号必须不少于右括号)。
\paragraph{出栈序列} 求将$n$个元素入栈一次(限制顺序)并出栈一次(不限制顺序)
的方案数(对于每一次操作都要保证栈不出现下溢,即入栈元素不少于出栈元素)。
\paragraph{网格行走} 在一个$n*n$的网格内从左下角移动到右上角,纵坐标必须不少于
横坐标,求方案数。
\paragraph{分析}
这三个问题是同构的,都满足操作数为$2n$且限制任意时刻操作A的数目不少于操作B的数目。
它们的答案都是$C_n$,以括号序列问题为例,通过等式~\ref{CT2}理解:
将括号序列看做由一个可分割的序列加上一个不可分割的序列(即最外层有一对配对括号)得来,
左边为$n_1+1$对,右边为$n_2$对,满足$n_1+n_2=n-1$,这种方案的贡献为
$C_{n_1}C_{n_2}$。
\subsubsection{二叉树构型计数}
\paragraph{有$n$个节点的二叉树}
通过等式~\ref{CT2}理解:枚举左右子树大小,满足左右子树节点数为$n-1$。
\paragraph{有$n+1$个叶子节点的满二叉树}
通过等式~\ref{CT3}理解:枚举左右子树叶子节点数,满足其和为$n+1$。
\subsubsection{阶梯填充}
用$n$个长方形填充$n*n$的阶梯的方案数为$C_n$。

不严格证明:填充一个以直角顶点与阶梯顶点为对顶点的长方形,使其分为大小为
$n_1*n_1,n_2*n_2$的两个小阶梯,满足$n_1+n_2=n-1$,分别分配$n_1,n_2$
个长方形的份额,就成为子问题了。该分析满足等式\ref{CT2}。
\subsubsection{凸包分割}
将$n+2$个顶点的凸包分为三角形的方案数为$C_n$。

猜想:最终将分为$n$个三角形。

证明留坑待补。
\subsubsection{圆上点连线}
将圆上的$2n$个点两两配对连线,所连$n$条线段不相交的方案数为$C_n$。

证明留坑待补。

\index{*TODO:Catalan应用证明}
上述内容参考了Wikipedia-EN\footnote{Catalan number - Wikipedia
	\url{https://en.wikipedia.org/wiki/Catalan\_number}}。

\section{Stirling数}

\section{Lucas/ExLucas}
\index{L!Lucas's Theorem}
\subsection{Lucas定理}
\begin{theorem}[Lucas's Theorem]
	对于非负整数$n,m$以及质数$p$,若
	\begin{eqnarray*}
		n&=&\sum_{i=0}^k{n_ip^i}\\
		m&=&\sum_{i=0}^k{m_ip^i}
	\end{eqnarray*}
	则
	\begin{displaymath}
		\binomial{n}{m}\equiv\prod_{i=0}^k\binomial{n_i}{m_i} \pmod{p}
	\end{displaymath}
\end{theorem}
Nathan Fine的证明:
对于质数$p$与整数$n$满足$1\leq n <p$,有
\begin{lemma}
	\begin{displaymath}
		p|\binomial{p}{n}=\frac{p\cdot(p-1)\cdots(p-n+1)}{n\cdot(n-1)\cdots 1}
	\end{displaymath}
\end{lemma}
证明:注意到$p$是质数且与分母的每一个数互质,不可被分母的因子约去,所以最终值必有因子$p$。
那么可用普通生成函数表达为:
\begin{displaymath}
	(1+x)^p\equiv 1+x^p \pmod{p}
\end{displaymath}
可归纳推广为
\begin{inference}\label{LucasI}
	\begin{displaymath}
		(1+x)^{p^i}\equiv 1+x^{p^i} \pmod{p},i\in \mathbb{N}
	\end{displaymath}
\end{inference}
利用生成函数证明:
\begin{eqnarray*}
	\sum_{m=0}^n{\binomial{n}{m}x^m}&=&(1+x)^n\\
	&=&\prod_{i=0}^k{((1+x)^{p^i})^{n_i}}\\
	&\equiv&\prod_{i=0}^k{(1+x^{p^i})^{n_i}} \textrm{~(根据推论~\ref{LucasI})}\\
	&=&\prod_{i=0}^k{\left(\sum_{m_i=0}^{n_i}
	\binomial{n_i}{m_i}x^{p_im_i}\right)}\\
	&=&\prod_{i=0}^k{\left(\sum_{m_i=0}^{p-1}
	\binomial{n_i}{m_i}x^{p_im_i}\right)}\\
	&=&\sum_{m=0}^n{\left(\prod_{i=0}^k\binomial{n_i}{m_i}\right)x^m} \pmod{p}
\end{eqnarray*}
以上内容参考了Wikipedia-EN\footnote{Lucas's theorem - Wikipedia
	\url{https://en.wikipedia.org/wiki/Lucas\%27s\_theorem}}
\subsection{ExLucas}
对于模数为合数的情况,将$p$质因数分解,即
$\displaystyle p=\sum_{i=1}^k{p_i^{c_i}}$。
然后用求出$\binomial{n}{m} \bmod{p_i^{c_i}}$的值,最后使用CRT合并。

首先将求组合数转换为求阶乘,但要从$n!$提出p的倍数最后处理,
即\begin{displaymath}
	n!=\prod_{i=1}^n{i^{[(n,i)=1]}}\cdot p^{[\frac{n}{p}]}\cdot
	\prod_{i=1}^{[\frac{n}{p}]}i
\end{displaymath}
\begin{itemize}
	\item 第一部分:前$[\frac{n}{p_i^{c_i}}]$块的答案相等,计算整块后快速幂,
	      末尾不完整的块暴力计算。
	\item 第二部分:由于组合数中分子分母都有因子$p$,单独拆出来算。

	统计次数代码:
	\begin{lstlisting}
		int cnt = 0;
		for(int i=n/p;i;i/=p)
			cnt+=i;
	\end{lstlisting}
	\item 第三部分:成为了一个新阶乘,递归解决。
\end{itemize}
对于单个质数幂计算复杂度$O(p_i^{c_i})$。

以上内容参考了Candy?的博客\footnote{[Lucas定理]【学习笔记】 - Candy?
	\url{https://www.cnblogs.com/candy99/p/6637629.html}}。

\section{康托展开}
\index{C!Cantor Expansion}
康托展开可用于将一个排列映射到一个自然数(在所有排列中的字典序排名,从0开始)。

排列$P$的康托展开为$\displaystyle \sum_{i=2}^n{(i-1)!a_i}$,
其中$a_i$为位置$i$后小于$P_i$的数的个数,即$\displaystyle \sum_{j=1}^{i-1}[P_i>P_j]$。

相反已知字典序排名可以将其映射到一个排列,这就是逆康托展开。首先根据排名值计算出$a$数组,
数组$a$是唯一确定的,因为$a_i<i$而前一项的系数$i!> (i-1)!a_i$,从前到后取模确定。
计算出数组$a$后,维护一个支持删除和询问第$k$大的数据结构,从前到后确定每一位数。

\section{其它公式与定理}
\subsection{常用公式}
\begin{theorem}
    \begin{displaymath}
        \sum_{i=0}^n\binomial{i}{k}=\binomial{n+1}{k+1}
    \end{displaymath}
\end{theorem}
\begin{itemize}
    \item 证明:将等式加上$\binomial{0}{k+1}=0$,左边不断合并即为右式。
    \item 证明:将右式不断展开即为左式。
    \item 理解:在$n+1$个中选$k+1$个点,第$k+1$个点为哨兵,剩下$k$个在它之前
    的元素中选择。
\end{itemize}
\begin{theorem}
    \begin{displaymath}
        \sum_{i=0}^n{\binomial{n}{i}i}=n2^{n-1}
    \end{displaymath}
\end{theorem}
\begin{theorem}
    \begin{displaymath}
        \int_0^1{\binomial{n}{k}x^k(1-x)^{n-k}\ud x}=\frac{1}{n+1}
    \end{displaymath}
\end{theorem}
\begin{theorem}
    \begin{displaymath}
        (-1+1)^n=\sum_{k=0}^n{(-1)^k\binomial{n}{k}}=[n=0]
    \end{displaymath}
\end{theorem}
更一般地,可以得到
\begin{theorem}
    \begin{displaymath}\label{BSum}
        \sum_{k=0}^m{(-1)^k\binomial{n}{k}}=\binomial{m-n}{m}
    \end{displaymath}
\end{theorem}
证明:
\begin{lemma}[上指标反转]\label{BSL}
    \begin{displaymath}
        \binomial{n}{m}=(-1)^m\binomial{m-n-1}{m}
    \end{displaymath}
\end{lemma}
将组合数的分子取反即可证明。
由于$(-1)^i=(-1)^{-i}$,该式也可表述为
\begin{inference}
    \begin{displaymath}
        (-1)^m\binomial{n}{m}=\binomial{m-n-1}{m}
    \end{displaymath}
\end{inference}
接下来证明定理:
\begin{eqnarray*}
    \sum_{k=0}^m{(-1)^k\binomial{n}{k}}&=&
    \sum_{k=0}^m{\binomial{k-n-1}{k}}\\
    &=&\binomial{m-n}{m} \textrm{(展开后不断合并最左边两项))}
\end{eqnarray*}
\subsection{Raney引理}
\index{R!Raney Lemma}
\begin{lemma}[Raney Lemma]
	设整数序列为$A_1,A_2,\cdots,A_n$,记前缀和$S_k=\displaystyle \sum_{i=1}^k{A_k}$,
	满足$S_n=1$。则在该序列的循环表示中,有且只有一个序列满足所有前缀和均大于0。
    任何一种循环表示都和自身不同。
\end{lemma}
证明留坑待补。
\index{*TODO!Raney引理证明}

\paragraph{例题} SP9507 MARIO2 - Mario and Mushrooms

可将循环同构的序列归类,根据Raney引理,每类序列只有1个序列满足题意。

\lstinputlisting{Source/Source/Count/SP9507.cpp}

上述内容参考了兔子大天使的博客\footnote{
	Raney引理
	\url{https://blog.csdn.net/duxingstar/article/details/6406022}
}。
