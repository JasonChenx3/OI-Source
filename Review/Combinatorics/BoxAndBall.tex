\section{盒子放球问题总结}
\index{T!Twentyfold Way}
该系列问题叫做``The twentyfold way'',以下内容参考Wikipedia-EN
\footnote{
    Twelvefold way - Wikipedia
    \url{https://en.wikipedia.org/wiki/Twelvefold\_way}
}、自为风月马前卒的博客
\footnote{
    浅谈"n个球"和"m个盒子"之间的乱伦关系
    \url{https://www.cnblogs.com/zwfymqz/p/9724918.html}
}以及洛谷日报chengni的文章\footnote{
    当小球遇上盒子
    \url{https://www.luogu.org/blog/chengni5673/dang-xiao-qiu-yu-shang-he-zi}
}。这里仅摘取8种常见情况。

假设将$n$个球放入$m$个盒子中。
\subsubsection{球异,可空}
\begin{itemize}
    \item 盒异:视为染色问题,$m^n$
    \item 盒同:枚举非空盒子数,转化为球异非空盒同问题,
    $\displaystyle\sum_{i=0}^m{\stirlingB{n}{i}}$
\end{itemize}
\subsubsection{球异,非空}
\begin{itemize}
    \item 盒异:计算出盒同的所有情况后给每个盒子编号,$m!\stirlingB{n}{m}$
    \item 盒同:即第二类斯特林数,$\stirlingB{n}{m}$
\end{itemize}
\subsubsection{球同,可空}
\begin{itemize}
    \item 盒异:使用隔板法,强制给每个盒子额外一个球,$\binomial{n+m-1}{m-1}$
    \item 盒同:实际上要求的是长度为$m$的有序数列个数,数列的元素和为$n$。

    记$dp[n][m]$为划分方案数,有
    \begin{eqnarray*}
        dp[0][k]&=&dp[k][1]=1\\
        dp[i][j]&=&dp[i-j][j]+dp[i][j-1]
    \end{eqnarray*}

    状态转移方程解释:考虑$j$个盒子中是否有空盒

    \begin{itemize}
        \item 有空盒:新增一个空盒,贡献$dp[i][j-1]$
        \item 无空盒:向每个盒子放入一个球使其非空,贡献$dp[i-j][j]$
    \end{itemize}
\end{itemize}
\subsubsection{球同,非空}
\begin{itemize}
    \item 盒异:使用隔板法,$\binomial{n-1}{m-1}$
    \item 盒同:强制给每个盒子一个球后转化为可空,$dp[n-m][m]$
\end{itemize}
