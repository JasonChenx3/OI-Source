\section{置换群}
\index{P!Permutation Groups}
{\bfseries 置换}是从$[1,n]$到$[1,n]$的一一映射。

置换可以分解为多个循环,计算循环相关数据的方法为:枚举每一个节点
\begin{enumerate}
    \item 若该节点已被访问,则跳过;
    \item 顺着该节点对应的目标节点不断跳跃,标记已访问,直至跳跃到已访问点(即出发点)为止。
    \item 这个环就是一个循环。
\end{enumerate}
\begin{theorem}
    若对于一个置换有$n$个循环,长度分别为$l_1,l_2,\cdots,l_n$,
    则该置换的循环节长度为$lcm(l_1,l_2,\cdots,l_n)$。
\end{theorem}
\paragraph{不动点}
若一个状态$S$经由置换$f$置换后的状态与原状态相同,则状态$S$为$f$的不动点。
\paragraph{等价关系}
对于一个置换集合$F$,若状态$S$能经由$F$中的置换变为状态$S'$,则称$S$与$S'$等价。
\paragraph{等价类}
满足等价关系的状态属于同一等价类。

\subsubsection{Burnside引理}
\index{B!Burnside's Lemma}
\begin{lemma}[Burnside's Lemma]
    等价类数目为置换群$G$中所有置换的不动点数目的平均值。
    \begin{displaymath}
        |X/G|=\frac{1}{|G|}\sum_{g\in G}|X^g|
    \end{displaymath}
\end{lemma}
上述定理证明留坑待补。
\index{*TODO!证明Burnside引理}
\subsubsection{Polya定理}
\index{P!Pólya Enumeration Theorem}

\begin{theorem}[Pólya Enumeration Theorem]
    若对每一个节点进行$m$染色,置换$g$有$c(g)$个循环,则染色方案
    等价类数目为$\displaystyle \frac{1}{|G|}\sum_{g\in G}m^{c(g)}$。
\end{theorem}

证明:一个循环内所有的节点颜色相同,不同循环颜色的选择是独立的,每一个循环颜色选择
方案对应一个不动点,根据乘法原理可知$|X^g|=m^{c(g)}$。

以上内容参考了QAQqwe的博客\footnote{Burnside引理与Polya定理
\url{https://blog.csdn.net/liangzhaoyang1/article/details/72639208}}与
Wikipedia-EN\footnote{
    Burnside's lemma - Wikipedia
    \url{https://en.wikipedia.org/wiki/Burnside\%27s\_lemma}

    Pólya enumeration theorem - Wikipedia
    \url{https://en.wikipedia.org/wiki/P\%C3\%B3lya\_enumeration\_theorem}
}。

\subsubsection{常见题型}
题型来自My\_ACM\_Dream的博客\footnote{polya|burnside定理的一些总结\\
\url{https://blog.csdn.net/My\_ACM\_Dream/article/details/45312365}}。

\paragraph{正方形旋转}
n*n正方形染色:
\begin{itemize}
    \item 旋转$0^\circ$,循环节数$n^2$。
	\item 旋转$90^\circ/270^\circ$,若$n$为偶数,循环节数$\frac{n^2}{4}$;
	若$n$为奇数,循环节数$\frac{n^2-1}{4}+1$。
    \item 旋转$180^\circ$,若$n$为偶数,循环节数$\frac{n^2}{2}$;若$n$为奇数,循环
    节数$\frac{n^2-1}{2}+1$。
\end{itemize}
奇偶循环节数不同的原因是因为$n$为奇数时中间的点自成一个循环节。
\paragraph{正方形反射(对称)}
\begin{tabular}{|c|c|c|}
	\hline
			 & 对角反射& 对边中点反射\\
	\hline
	$n$为奇数 & $\frac{n^2-n}{2}+n$& $\frac{n^2-n}{2}+n$ \\
	\hline
	$n$为偶数 & $\frac{n^2-n}{2}+n$& $\frac{n^2}{2}$\\
	\hline
\end{tabular}
\paragraph{环形旋转}
对于一个有$n$个点的环,旋转$i$个点的置换的循环节数为$(n,i)$。

证明:$i$最小乘上$\frac{n}{(n,i)}$才会被$n$整除,所以每一个循环节的长度为
$\frac{n}{(n,i)}$,循环节个数为$(n,i)$。
\paragraph{环形对称翻转}
\begin{itemize}
	\item $n$为奇数:只有n种置换(以一点一边中点为对称轴),循环节数为
	$[\frac{n}{2}]+1$。
	\item $n$为偶数:\begin{itemize}
		\item 边边中点:$\frac{n}{2}$种,循环节数为$\frac{n}{2}$。
		\item 点点:$\frac{n}{2}$种,循环节数为$\frac{n}{2}+1$。
	\end{itemize}
\end{itemize}
\paragraph{正方体旋转}
注意是{\bfseries 棱边}置换。
\begin{itemize}
	\item 自身不变,置换1种,循环节12个,长度1;
	\item 以对面中心为轴,旋转角为$90^\circ,180^\circ,270^\circ$,
	轴有3种选择,共9种置换。
	\begin{itemize}
		\item $90^\circ/270^\circ$:循环节3个,长度4。
		\item $180^\circ$:循环节6个,长度2。
	\end{itemize}
    \item 以对边中点为轴,旋转角为$180^\circ$,有6对边,置换数为6,
    有5个长度为2的循环和2个长度为1的循环。
    \item 以对顶点为轴,旋转角为$120^\circ,240^\circ$,有4对点,置换数为8,
    均有4个长度为3的循环。
\end{itemize}
总置换数24。
\paragraph{$n$较小}
\begin{itemize}
    \item 颜色不限:裸Polya解决。
    \item 颜色限制:裸Burnside解决。
\end{itemize}
\paragraph{环形旋转且$n$较大}
枚举循环节数(即$d=(n,i)$),利用欧拉函数与容斥解决。
\paragraph{有染色限制}
使用dp与矩阵快速幂解决。
