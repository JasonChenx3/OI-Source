\section{置换群}
\index{P!Permutation Groups}
{\bfseries 置换}是从$[1,n]$到$[1,n]$的双射。

一个置换可以分解为多个内部循环,计算循环相关信息的方法为:

枚举每一个节点:
\begin{enumerate}
	\item 若该节点已被访问,则跳过;
	\item 顺着该节点对应的目标节点不断跳跃,给访问到的点打标记,直至跳跃到已访问点
	      (即出发点)为止。
	\item 这个环就是一个循环。
\end{enumerate}
\begin{theorem}
	若一个置换内部有$n$个循环,长度分别为$l_1,l_2,\cdots,l_n$,
	则该置换的循环节长度为$lcm(l_1,l_2,\cdots,l_n)$。
\end{theorem}
\paragraph{不动点}
若一个状态$S$经由置换$f$置换后的状态与原状态相同,则状态$S$为$f$的不动点。
\paragraph{等价关系}
对于一个置换集合$F$,若状态$S$能经由$F$中的置换变为状态$S'$,则称$S$与$S'$等价。
\paragraph{等价类}
满足等价关系的状态属于同一等价类。

\subsection{Burnside引理}
\index{B!Burnside's Lemma}
\begin{lemma}[Burnside's Lemma]
	等价类数目为置换群$G$中所有置换的不动点数目的平均值。
	\begin{displaymath}
		|X/G|=\frac{1}{|G|}\sum_{g\in G}|X^g|
	\end{displaymath}
\end{lemma}
上述定理证明留坑待补。
\index{*TODO!证明Burnside引理}
\subsection{Pólya定理}
\index{P!Pólya Enumeration Theorem}

\begin{theorem}[Pólya Enumeration Theorem]
	若对每一个节点进行$m$染色,置换$g$有$c(g)$个循环,则染色方案
	等价类数目为$\displaystyle \frac{1}{|G|}\sum_{g\in G}m^{c(g)}$。
\end{theorem}

证明:要使状态为不动点,必须让同循环内的节点同颜色,循环的颜色选择独立,根据乘法原理
可知$|X^g|=m^{c(g)}$。

以上内容参考了QAQqwe的博客\footnote{Burnside引理与Polya定理
	\url{https://blog.csdn.net/liangzhaoyang1/article/details/72639208}}与
Wikipedia-EN\footnote{
	Burnside's lemma - Wikipedia
	\url{https://en.wikipedia.org/wiki/Burnside\%27s\_lemma}

	Pólya enumeration theorem - Wikipedia
	\url{https://en.wikipedia.org/wiki/P\%C3\%B3lya\_enumeration\_theorem}
}。

\subsection{常见题型}
题型来自My\_ACM\_Dream的博客\footnote{polya|burnside定理的一些总结\\
	\url{https://blog.csdn.net/My\_ACM\_Dream/article/details/45312365}}。

\paragraph{正方形旋转}
n*n正方形染色:
\begin{itemize}
	\item 旋转$0^\circ$,循环数$n^2$。
	\item 旋转$90^\circ/270^\circ$,若$n$为偶数,循环数$\frac{n^2}{4}$;
	      若$n$为奇数,循环数$\frac{n^2-1}{4}+1$。
	\item 旋转$180^\circ$,若$n$为偶数,循环数$\frac{n^2}{2}$;若$n$为奇数,循环数
		  $\frac{n^2-1}{2}+1$。
\end{itemize}
奇偶循环数不同的原因是因为当$n$为奇数时中间的点自成一个循环。
\paragraph{正方形反射(对称)}
\begin{tabular}{|c|c|c|}
	\hline
	          & 对角反射            & 对边中点反射        \\
	\hline
	$n$为奇数 & $\frac{n^2-n}{2}+n$ & $\frac{n^2-n}{2}+n$ \\
	\hline
	$n$为偶数 & $\frac{n^2-n}{2}+n$ & $\frac{n^2}{2}$     \\
	\hline
\end{tabular}
\paragraph{环形旋转}
对于一个有$n$个点的环,旋转$i$个点的置换的循环数为$(n,i)$。

证明:每个循环的长度为$\frac{[n,i]}{i}=\frac{n}{(n,i)}$,循环数为$(n,i)$。
\paragraph{环形对称翻转}
\begin{itemize}
	\item $n$为奇数:只有n种置换(以一点一边中点为对称轴),循环数为
	      $[\frac{n}{2}]+1$。
	\item $n$为偶数,按照对称轴分类:
		  \begin{itemize}
		      \item 边边中点:$\frac{n}{2}$种,循环数为$\frac{n}{2}$。
		      \item 点点:$\frac{n}{2}$种,循环数为$\frac{n}{2}+1$。
	      \end{itemize}
\end{itemize}
\paragraph{正方体旋转}
注意是{\bfseries 棱边}置换。
\begin{itemize}
	\item 自身不变,置换1种,循环12个,长度1;
	\item 以对面中心为轴,旋转角为$90^\circ,180^\circ,270^\circ$,
	      轴有3种选择,共9种置换。
	      \begin{itemize}
		      \item $90^\circ/270^\circ$:循环3个,长度4。
		      \item $180^\circ$:循环6个,长度2。
	      \end{itemize}
	\item 以对边中点为轴,旋转角为$180^\circ$,有6对边,置换数为6,
	      有5个长度为2的循环和2个长度为1的循环。
	\item 以对顶点为轴,旋转角为$120^\circ,240^\circ$,有4对点,置换数为8,
	      均有4个长度为3的循环。
\end{itemize}
总置换数24。

例题:UVA10601 Cubes

暴力(矩阵变换+枚举排列检查不动点):
\lstinputlisting{Source/Source/Groop/UVA10601.cpp}

正解:
首先统计每种颜色的数量,然后分类讨论,每种情况的贡献为置换数*循环方案数。
若一种置换的所有循环长度相等,则使用排列组合知识求解(solve函数);否则枚举
短循环的颜色,令剩余循环长度相等,对剩余颜色和循环调用solve求解。
\lstinputlisting{Source/Source/Groop/UVA10601X.cpp}

\paragraph{$n$较小}
\begin{itemize}
	\item 颜色不限:裸Polya解决。
	\item 颜色限制:裸Burnside解决。
\end{itemize}
\paragraph{环形旋转且$n$较大}
枚举循环数(即$d=(n,i)$),利用欧拉函数与容斥解决。
\paragraph{有染色限制}
使用dp与矩阵快速幂解决。

\subsection{完全图染色问题}
有一个由$n$个点($n\leq 60$)组成的完全图,每条边可被染成$m$种颜色
中的一种,求本质不同的染色图数。

若使用Pólya定理解决,可以发现$|G|$的规模达到$N!$,显然枚举置换过不了。

考虑将点置换转换为边置换,即$(u,v)\rightarrow (P_u,P_v)$。

然后将其分为2种情况:
\begin{itemize}
	\item 若点置换内有一个循环长度为$L$,则对应边置换内
	有$\left[\frac{L}{2}\right]$个循环。

	证明:所有端点在点置换循环上距离(定义为最短路)相等的边构成一个等价类,
	而等价类数目有$\left[\frac{L}{2}\right]$个。

	\item 若点置换内有两个循环长度分别为$L_1,L_2$,则对应边置换内有
	$(L_1,L_2)$个循环。

	证明:使两个循环都置换到原状态需要$[L_1,L_2]$次,即每个循环的大小。
	而总边数为$L_1L_2$,因此循环数为$(L_1,L_2)$。
\end{itemize}

接下来考虑枚举点置换的循环,记这些循环的长度为$L_1,L_2,\cdots,L_k$。
首先DFS枚举保证$L_1\geq L_2 \geq \cdots \geq L_k>0$,即求$n$的划分方案。

然后对于某个划分可以由上文方法求出对应置换的循环个数,接着考虑该划分对应的置换数。
总置换数为$N!$,将$n$个点分配给这些循环,去掉内部排列导致的方案重复,要除以
$\displaystyle \prod_{i=1}^k{L_i!}$。对于每一个循环,一旦首位固定,其它位的排列顺序
不同也会构造出不同的置换,方案数再乘以$\displaystyle \prod_{i=1}^k{(L_i-1)!}$。实际上
只需除以$\displaystyle \prod_{i=1}^k{L_i}$,可以理解为一个固定的有向环可以任意指定起点。
等长循环的排列顺序会造成方案重复,记长度为$L$的循环数为$c_L$,再除以
$\displaystyle \prod_{i=1}^n{c_i!}$。最终置换数为
\begin{displaymath}
	\frac{N!}{\prod_{i=1}^k{L_i}\prod_{i=1}^n{c_i!}}
\end{displaymath}
分子的$N!$与Burnside引理的系数相消。

代码([HNOI2009]图的同构记数):
\lstinputlisting{Source/Source/Groop/4727.cpp}

上述内容参考了陈瑜希的论文《Pólya计数法的应用》。
\subsection{置换的群性质}
注意单个置换可以生成群,且满足结合律,因此。。。

\begin{itemize}
	\item 置换快速幂
	\item 置换BSGS求离散对数
\end{itemize}
