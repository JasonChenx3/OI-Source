\chapter{集合论~群论}
\section{集合论定理}
\begin{theorem}
	\begin{displaymath}
		A\oplus B=A\cup B - A\cap B
	\end{displaymath}
\end{theorem}
\index{D!De Morgan's Laws}
\begin{theorem}[De Morgan's Laws]\label{DML}
	\begin{eqnarray*}
		\overline{A\cup B}=\overline{A}\cap \overline{B} \\
		\overline{A\cap B}=\overline{A}\cup \overline{B}
	\end{eqnarray*}
\end{theorem}
\index{I!Inclusion–exclusion Principle}
\begin{theorem}[Inclusion–exclusion Principle]\label{IEP}
	\begin{displaymath}
		\left|\bigcup_{i=1}^n{A_i}\right|=
		\sum_{\emptyset \neq J\subseteq \{1,2,\ldots,n\}}{(-1)^{|J|-1}
			\left|\bigcap_{j\in J}{A_j}\right|}
	\end{displaymath}
\end{theorem}

由定理~\ref{DML}与~\ref{IEP}可得

\begin{theorem}
	\begin{displaymath}
		\left|\bigcap_{i=1}^n\overline{A_i}\right|=
		\left|\overline{\bigcup_{i=1}^n{A_i}}\right|=
		|U|+\sum_{\emptyset \neq J\subseteq \{1,2,\ldots,n\}}{(-1)^{|J|}
			\left|\bigcap_{j\in J}{A_j}\right|}
	\end{displaymath}
\end{theorem}

以上内容参考了Wikipedia-EN\footnote{Inclusion–exclusion principle - Wikipedia
	\url{https://en.wikipedia.org/wiki/Inclusion\%E2\%80\%93exclusion\_principle}

	De Morgan's laws - Wikipedia
    \url{https://en.wikipedia.org/wiki/De\_Morgan\%27s\_laws}}。

\section{拉格朗日定理}
\index{L!Lagrange's Theorem}
\begin{theorem}[Lagrange's Theorem]\label{LT}
	若$(S,\oplus)$是一个有限群,$(S',\oplus)$是$(S,\oplus)$的子群,则
	$|S'|$是$|S|$的约数。
\end{theorem}
证明留坑待补。
\index{*TODO!拉格朗日定理证明}

\section{置换群}
\index{P!Permutation Groups}
{\bfseries 置换}是从$[1,n]$到$[1,n]$的一一映射。

置换可以分解为多个循环,计算循环相关数据的方法为:枚举每一个节点
\begin{enumerate}
    \item 若该节点已被访问,则跳过;
    \item 顺着该节点对应的目标节点不断跳跃,标记已访问,直至跳跃到已访问点(即出发点)为止。
    \item 这个环就是一个循环。
\end{enumerate}
\begin{theorem}
    若对于一个置换有$n$个循环,长度分别为$l_1,l_2,\ldots,l_n$,
    则该置换的循环节长度为$lcm(l_1,l_2,\ldots,l_n)$。
\end{theorem}
\paragraph{不动点}
若一个状态$S$经由置换$f$置换后的状态与原状态相同,则状态$S$为$f$的不动点。
\paragraph{等价关系}
对于一个置换集合$F$,若状态$S$能经由$F$中的置换变为状态$S'$,则称$S$与$S'$等价。
\paragraph{等价类}
满足等价关系的状态属于同一等价类。

\subsubsection{Burnside引理}
\index{B!Burnside's Lemma}
\begin{lemma}[Burnside's Lemma]
    等价类数目为置换群$G$中所有置换的不动点数目的平均值。
    \begin{displaymath}
        |X/G|=\frac{1}{|G|}\sum_{g\in G}|X^g|
    \end{displaymath}
\end{lemma}
上述定理证明留坑待补。
\index{*TODO!证明Burnside引理}
\subsubsection{Polya定理}
\index{P!Pólya Enumeration Theorem}

\begin{theorem}[Pólya Enumeration Theorem]
    若对每一个节点进行$m$染色,置换$g$有$c(g)$个循环,则染色方案
    等价类数目为$\displaystyle \frac{1}{|G|}\sum_{g\in G}m^{c(g)}$。
\end{theorem}

证明:一个循环内所有的节点颜色相同,不同循环颜色的选择是独立的,每一个循环颜色选择
方案对应一个不动点,根据乘法原理可知$|X^g|=m^{c(g)}$。

以上内容参考了QAQqwe的博客\footnote{Burnside引理与Polya定理
\url{https://blog.csdn.net/liangzhaoyang1/article/details/72639208}}与
Wikipedia-EN\footnote{
    Burnside's lemma - Wikipedia
    \url{https://en.wikipedia.org/wiki/Burnside\%27s\_lemma}

    Pólya enumeration theorem - Wikipedia
    \url{https://en.wikipedia.org/wiki/P\%C3\%B3lya\_enumeration\_theorem}
}。

\subsubsection{常见题型}
题型来自My\_ACM\_Dream的博客\footnote{polya|burnside定理的一些总结\\
\url{https://blog.csdn.net/My\_ACM\_Dream/article/details/45312365}}。

\paragraph{正方形旋转}
n*n正方形染色:
\begin{itemize}
    \item 旋转0度,循环节数$n^2$。
    \item 旋转90/270度,若$n$为偶数,循环节数$\frac{n^2}{4}$;若$n$为奇数,
    循环节数$\frac{n^2-1}{4}+1$。
    \item 旋转180度,若$n$为偶数,循环节数$\frac{n^2}{2}$;若$n$为奇数,循环
    节数$\frac{n^2-1}{2}+1$。
\end{itemize}
奇偶循环节数不同的原因是因为$n$为奇数时中间的点自成一个循环节。
\paragraph{环形旋转}
\paragraph{环形翻转}
\paragraph{正方体旋转}

