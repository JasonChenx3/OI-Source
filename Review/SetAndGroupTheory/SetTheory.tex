\section{集合论定理}
\begin{theorem}[对称差]
	\begin{displaymath}
		A\oplus B=(A-B)\cup(B-A)=A\cup B - A\cap B
	\end{displaymath}
\end{theorem}
\index{D!De Morgan's Laws}
\begin{theorem}[De Morgan's Laws]\label{DML}
	\begin{eqnarray*}
		\overline{A\cup B}=\overline{A}\cap \overline{B} \\
		\overline{A\cap B}=\overline{A}\cup \overline{B}
	\end{eqnarray*}
\end{theorem}
\index{I!Inclusion–exclusion Principle}
\begin{theorem}[Inclusion–exclusion Principle]\label{IEP}
	\begin{displaymath}
		\left|\bigcup_{i=1}^n{A_i}\right|=
		\sum_{\emptyset \neq J\subseteq \{1,2,\cdots,n\}}{(-1)^{|J|-1}
			\left|\bigcap_{j\in J}{A_j}\right|}
	\end{displaymath}
\end{theorem}

容斥原理用来求集合并的大小,为了求集合交的大小,可以使用补集转换思想,
由定理~\ref{DML}与~\ref{IEP}可得

\begin{theorem}\label{ExDML}
	\begin{displaymath}
		\left|\bigcap_{i=1}^n{A_i}\right|=
		\left|\overline{\bigcup_{i=1}^n{\overline{A_i}}}\right|=
		|U|+\sum_{\emptyset \neq J\subseteq \{1,2,\cdots,n\}}{(-1)^{|J|}
			\left|\bigcap_{j\in J}{\overline{A_j}}\right|}
	\end{displaymath}
\end{theorem}

以上内容参考了Wikipedia-EN\footnote{Inclusion–exclusion principle - Wikipedia\\
	\url{https://en.wikipedia.org/wiki/Inclusion\%E2\%80\%93exclusion\_principle}

	De Morgan's laws - Wikipedia
	\url{https://en.wikipedia.org/wiki/De\_Morgan\%27s\_laws}}以及
国家集训队2013论文集《浅谈容斥原理》。
\subsection{模意义统计方案}
若要求恰好满足$k$个条件的方案数,且这些条件是等价的。考虑使用至少满足$k$个条件的方案数
容斥求出,后者由于限制条件较松,很容易使用NTT等方法得到。

记$g(x)$为至少满足$x$个条件的方案数,那么$g(i),i\geq k$构成的每种方案都对应
$\binomial{i}{k}$种满足$k$个条件的方案。由容斥可得
$ans=\displaystyle \sum_{i=k}^n{(-1)^{i-k}\binomial{i}{k}g(i)}$。
