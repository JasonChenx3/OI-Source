\section{思路与技巧}
\subsection{二分/三分}
若目标函数具有单调性,或者题目要求``最大值最小/最小值最大'',一般
考虑二分答案然后检查其合法性。

若目标函数为凸(尤其是一些计算几何题),一般使用三分法判断函数的``轮廓'',
以达到缩小搜索范围的目的。使用斐波那契法(或者0.618法)在区间$[0,1]$选取点
$\frac{f_{n-2}}{f_n},\frac{f_{n-1}}{f_n}$作为比较点或许会更玄学。对于整数
三分,在范围缩小到6以内时暴力枚举。
\subsection{补集转化}
根据``正难则反''的哲学,若正向思考不好解决,则考虑它的反面。尤其在
计数问题中考虑所求计数集合的补集。此外在并查集中引入补集的概念也是
个不错的思考方向。
\subsection{莫队}
对于离线的(修改+)查询问题,可以使用莫队算法来维护一些难以合并维护的信息
(比如区间颜色数)。

其思想是在能够快速对当前区间的左右端点进行移动并维护相应信息的情况下,
尽可能减少端点的移动距离。最优移动路径的计算可表示为二维(三维)曼哈顿距离
最短哈密尔顿路径问题。一般使用简单的分块思想来达到较好的复杂度\CJKsout{(O(能过))}。

\subsubsection{例子}
若能$O(1)$对端点进行移动,且区间大小与查询/修改量同阶(设为$n$),则对查询
所涉及到的所有区间,对每个位置每$\sqrt{n}$分块编号,然后对查询以左端点所在块
编号为第一关键字,右端点位置为第二关键字排序(若需要支持修改操作,则第二关键字为右端点块
所在编号,第三关键字为时间)。处理询问时先移动到指定区间(若需要修改则移动到指定时间),
然后统计该询问的答案。

对于仅询问的问题,该方法的复杂度为$O(n\sqrt{n})$。当左端点在$\sqrt{n}$个块时,
每块右端点最多移动$n$;而当左端点跨块转移时,右端点最多移动$n$,所以右端点移动
$O(n\sqrt{n})$。由于查询量与区间大小同阶,且左端点单次最多移动$2\sqrt{n}$,
左端点的移动也是$O(n\sqrt{n})$的。
\subsubsection{分块大小}
一般思路是推出各部分的复杂度,然后使用均值不等式调到最优复杂度。带修改时
注意查询移动与修改移动复杂度的平衡。
\paragraph{区间大小与查询量不同阶}
设区间大小为$n$,询问规模为$m$,块大小为$k$。右端点块内转移与跨块转移数
为$O(\frac{n}{k}\cdot n)$,左端点转移数为$O(mk)$,利用均值不等式可得
当$k=\frac{n}{\sqrt{m}}$时取得最优复杂度$O(n\sqrt{m})$。
\paragraph{修改+查询}
\begin{itemize}
	\item 左端点移动次数$O(nk)$;
	\item 右端点移动次数$O(nk)$;
	\item 修改移动次数$O(\left(\frac{n}{k}\right)^2\cdot n)$;
\end{itemize}
总复杂度为$O(nk+\left(\frac{n}{k}\right)^2\cdot n)$,令其导数为0,
可得$k=n^\frac{2}{3}$时最优,时间复杂度为$O(n^\frac{5}{3})$。
\subsubsection{树上莫队}
该方法用于维护树的链上信息,维护子树信息可以使用Dsu On Tree,参见第~\ref{DOT}节。

一种思路将树映射为括号序(不分+-),然后按照处理普通序列的方法来做。
考虑点$x,y$所映射的区间(设$L[x]<L[y]$),若$x$是$y$的祖先(使用LCA判定),
则为$[L[x],L[y]]$,否则为$[R[x],L[y]]$。为了抵消掉途经子树的贡献,维护一个$vis$
数组,根据$vis$判断当前途经点是否在贡献中,然后相应地加上或减去贡献,取反$vis$值。
{\bfseries 注意查询答案时对于LCA处要特判,若$x$不是$y$的祖先,则要临时加入LCA,
查询完毕后删去。因为此时LCA的左右括号均不在区间内,而从x到LCA的链全因右括号产生贡献,
从LCA到y的链全因左括号产生贡献。}

另一种思路是对树进行分块。分块算法参考第~\ref{dividing}节。分块后按照左右端点
所在块的编号排序,然后查询时在树上跑。

例题:WC2013 糖果公园
\footnote{https://www.luogu.org/problemnew/show/P4074}

代码如下:
\lstinputlisting{Source/Templates/TMD.cpp}
\subsubsection{奇偶排序优化}
常规的查询区间排序是这样的:
\begin{lstlisting}
    bool operator<(const Query& rhs) const {
        return bid[l]!=bid[rhs.l]?l<rhs.l:r<rhs.r;
    }
\end{lstlisting}

这里有一个更好的排序方法:
\begin{lstlisting}
    bool operator<(const Query& rhs) const {
        return bid[l]!=bid[rhs.l]?l<rhs.l:
            (bid[l]&1?r>rhs.r:r<rhs.r);
    }
\end{lstlisting}

也就是使两个相邻块内的询问右端点移动方向不同。将右端点移动路径由``[左->右]->[左->右]''
改为``[左->右]->[右->左]'',减少了跨块转移的步数。注意块编号从偶数开始。

该方法参考了洛谷日报第48期codesonic的文章\footnote{
	莫队算法初探 - \#include<codesonic>\\
	\url{https://www.luogu.org/blog/codesonic/Mosalgorithm}
}与RabbitHu的博客\footnote{
	胡小兔的良心莫队教程:莫队、带修改莫队、树上莫队 - 胡小兔
	\url{https://www.cnblogs.com/RabbitHu/p/MoDuiTutorial.html}
}。
\subsubsection{回滚莫队}
回滚莫队适用于莫队的删除操作无法快速地更新答案的情况(比如维护最值,使用可删堆无法保证更新
复杂度)。既然删除操作不好处理,干脆仅使用加入操作维护当前区间。考虑朴素区间排序方式,在左
端点块标号相同的情况下,右端点单调递增,而左端点所在的块大小为$O(\sqrt{n})$。那么可以将
左端点所在块单独处理,块右边的端点移动仅有插入操作,总时间复杂度$O(n\sqrt{n})$。每次处理
完右边后,暴力枚举查询区间落在左端点所在块内的元素尝试更新答案(更新次数最多为$O(\sqrt{n})$,
对于维护最值问题可以将当前答案单独存储而不是全局修改,这样就不会影响到下一次查询),然后
消除这些元素的影响(回滚莫队使其不影响最值的维护,其它数据很容易维护)。对于左右端点在同一块内
的特殊情况,由于块大小为$O(\sqrt{n})$,直接暴力维护。

\paragraph{例题 「JOISC 2014 Day1」 历史研究}
回滚莫队经典题,参考代码:

\lstinputlisting{Source/Source/Block/LOJ2874.cpp}

对于删除容易插入难的操作也可以这样做。

上述内容参考了yashem66的博客\footnote{
	BZOJ4241 历史研究 (分块 回滚莫队-教程向)
	\url{https://blog.csdn.net/qq\_33330876/article/details/73522230}
}。
\subsection{分块}\label{dividing}
\subsubsection{序列分块}
一般将序列分为多块,维护块内信息,区间查询/修改时整块处理,左右剩余元素暴力。
块的大小根据具体情况而定(整块/零散复杂度平衡,修改/查询复杂度平衡)。
{\bfseries 一定要考虑区间端点在同一块内的情况。}

实现时块内暴力共有3种情况,最好抽象出一个函数专门解决此类情况。

{\bfseries 血泪史:THUWC2019 D1T1中我使用了分块做法,但是先前没发现块在边界的情况要特判,即1与n虽然不是理论块的端点但是是实际块的端点,导致两端的块被重复统计,浪费了不少调试时间。因此在左右端在边界上时需强制不统计左右端。}
\paragraph{带插入序列分块}

例题:数列分块入门 6 by hzwer

插入操作集中在同一块内容易被卡,因此可以选择每插入$\sqrt{n}$个数将所有数重新分块,
每次重构的复杂度为$O(n)$,保证了块的大小相对平均,并且重构总复杂度与查询复杂度相同。

不知为何每查询+修改$\sqrt{n}$次重建比插入$\sqrt{n}$次快得多。
\paragraph{区间众数问题}

例题:数列分块入门 9 by hzwer

\begin{theorem}
	有两个可重集合$A,B$,记$mode(A)$为$A$的众数,则\\
	$mode(A\cup B)\in \{mode(A)\}\cup B$。
\end{theorem}

证明:若$mode(A\cup B)\neq mode(A)$且$mode(A\cup B)\not \in B$,则
$mode(A\cup B)\in A$,$mode(A\cup B)$在$A$中的出现次数肯定不超过$mode(A)$的
出现次数,与$mode(A\cup B)$自身定义矛盾。

这个定理启示了我们可以维护整块(注意是所有整块连起来的块)的众数,剩下的可能的众数一定在
不完整块中出现,枚举非完整块内的数字查询其出现个数。

考虑仅询问的情况,首先预处理数组$A[i][j]$表示第$i$块到第$j$块的众数,枚举$i$向后扫记录
扫描区间内每种数(需要提前离散化)的个数可以在$O(n\sqrt{n})$内处理完毕。对于非完整块,
由于它们的大小不超过$2\sqrt{n}$,根据定理可以枚举每个数查询其在查询区间内的出现次数
($O(n\lg n)$预处理每种数出现的位置,$O(\lg n)$二分查找区间内的位置,这种方式以根号大小
分块仍不为最优,最优分块大小为$\sqrt{\frac{n}{\lg n}}$。当然也可以使用分块预处理
$B[i][j][k]$表示第$i$块的第$j$个位置前等于$k$的位置数,做到$O(n\sqrt{n})$预处理,
$O(1)$查询)。

修改操作留坑待补。
\index{*TODO!区间众数修改操作}

上述内容参考了陈立杰的《区间众数解题报告》。
\subsubsection{树分块}
\paragraph{王室联邦分块法}
DFS遍历儿子时子树节点数累积大于等于$B$,则将其当做一块。其余节点并入父亲所在块,
保证块的大小$\in [B,3B]$,且块的直径不超过$B$,但块不连通。
\index{*TODO!王室联邦分块法的块大小}

代码如下:
\lstinputlisting{Source/Templates/TreeDivide.cpp}
\paragraph{size分块法}
DFS时若父亲所在块还未到达指定size,将自身加入;否则新开一个块。
保证块的大小,连通性还有直径大小。

树分块相关内容参考了nimphy的博客\footnote{
	【初识】树上分块 - nimphy
	\url{https://www.cnblogs.com/hua-dong/p/8275227.html}
}。
\subsubsection{平衡修改与查询的复杂度}
例如对于规模为$n$的元素,修改复杂度为$O(n\lg n)$(比如暴力重建),而查询复杂度
为$O(\lg n)$。此时可以考虑每$\sqrt{n}$个元素分块,单次修改/查询复杂度均为
$O(\sqrt{n}\lg n)$。
\subsection{MITM}
\index{M!Meet In The Middle}
其主要思想是双向搜索(左+右/BSGS),然后使用hash表来查询满足题意
的集合。双向BFS搜索可以有效地减少状态数。

例题:Luogu P2324 [SCOI2005]骑士精神

暴力搜索的状态数为$3^{15}$数量级,但是双向BFS搜索的状态数为$2*3^8$数量级,
可以通过该题。每轮迭代中,两边各走一步,同时进行检测,这样做找到一组可行解就可以
返回了。

\lstinputlisting{Source/Source/Search/2324.cpp}
\subsection{倍增}
通过多一个$\lg$预处理跳跃$2^k$步的信息,以达到$\lg$级快速移动的目的。

\subsubsection{倍增优化连边}
有时需要从一个点到树上的一条链连边,那么可以记$P[i][j]$表示从点$i$开始往上$2^j$
层的链对应的节点,然后将其连无用边到子链。这样可以保证每次连边是$O(\lg n)$的。
\subsection{随机化}
对于一些时间复杂度为期望复杂度的算法,需要对数据顺序随机化来避免被卡。

对于枚举点对求最值的问题,在允许的运行时间内随机生成点对并计算也可以骗到不少分。
\subsection{按位拆分}
对于位运算相关问题,若位与位之间独立,则可以考虑按位拆分计算,这种做法可以
降低时间复杂度,简化代码。
\subsection{扫描线}
若某条件仅存在于一个区间/时间段中,则可在起点与终点的后一点处打添加/删除标记,
查询时先处理完当前时间的所有添加/删除的标记,再计算答案。
对于一些计算几何题也可以考虑使用该方法。

\subsubsection{逆扫描线}
若需要维护每个时间段内元素集合的信息,带有插入和删除操作,可以考虑将其看做元素存在于
一段时间区间,使用线段树(分治)维护区间插入。
\subsection{差分}
\begin{itemize}
	\item 对于离线区间加法,可以在起点与终点后一点处打标记,最后做一遍前缀和。
	\item 对于序列相邻位置的不等式约束,有时也可以使用差分来转化问题。
	\item 对于具有单调性的序列,比如前缀/后缀最值,考虑差分后讨论修改位置。
\end{itemize}
\subsection{双指针法}
在区间最值问题中,若区间移动存在单调性(即对于每个左端点所对应的最优右端点是单调的),
则对于一个固定的左端点,使右端点移动到最优位置,然后将左端点移动一格,继续移动右端点。
此法的复杂度为$O(n)$。在两个区间上的组合问题也是如此(比如凸包矢量和问题)。
\subsection{优先队列维护长序列}\label{PQS}
若要取出超长/无穷序列的前k小值,且序列中较大的值的方案可以由较小的值的方案构造,
就可以预先加入原始的(不能被其它方案构造)方案到优先队列中,取出时加入其后继方案
到优先队列中。此法可保证优先队列的规模与算法时间复杂度在可接受的范围内。

有时甚至可以直接使用预排序+队列来代替优先队列(即$a\rightarrow a',b\rightarrow b',
	a<b\Rightarrow a'<b'$时)。

更加形式化的描述:这些方案构造出了一个DAG,如果某个方案被选中,它的前驱必定被选中。
每次选取一个方案后将它的后继加入优先队列。如果这个DAG是一棵树,就不需要去重,最好构造
出可树形转移的方案表示。

\subsection{集合选数最值问题}
\subsubsection{类型一}
给定$n$个可重集合,在每个集合中选取一个数,求这些选择方案中数的和(积)前$k$大(小)值。

当$n=2$时,沿用上面的做法可以解决。但是当方案的维数增加时,这种方法就不再有效了。
思考的方向仍然时构造一棵树,满足每个点只有一个前驱,以及较少的常数个后继。

首先DAG的根肯定是每个集合都贪心选择最大值。对每个集合内的元素排序,可以用
$(p_1,p_2,\cdots,p_n)$表示集合$i$选择第$p_i$个数,但是这样的状态不好转移,需要去重且
后继数过多。考虑当前变动集合$i$的选择,前面的集合全部固定,后面的集合全部取最大值。那么可以
使用状态$(i,j,s)$表示当前改变集合$i$,选择第$j$大的数,该方案的价值为$s$(这样就不必
存储前面集合的选择方案)。然后得到如下后继:
\begin{itemize}
	\item 当前集合后移:若$j<|S_i|$,存在后继$(i,j+1,s-A[i][j]+A[i][j+1])$;
	\item 改变下一个集合:若$i<n,2\leq |S_{i+1}|$,存在后继
	$(i+1,2,s-A[i+1][1]+A[i+1][2])$;
	\item 注意到变换下一个集合时当前集合选择的数肯定不是最大值,但存在这种方案。
	所以指定当$j=2,i<n,2\leq |S_{i+1}|$时当前集合选择最大值,然后改变下一个集合。
	即存在后继$(i+1,2,s-A[i][2]+A[i][1]-A[i+1][1]+A[i+1][2])$。此时并不能保证
	其后继不大于前驱,对集合以$A[i][1]-A[i][2]$为关键字升序排序就可以解决这个问题。
\end{itemize}

这种状态设置方式保证了前驱大于等于后继且前驱唯一,同时后继数可控而少。

{\bfseries 注意如果以贪心选取最大值的方案为根,第三种后继的假设不成立,会出现重复方案。
因此要以状态$(1,1,s_2)$为DAG根,最大值单独统计。}

例题:第十三届北航程序设计竞赛预赛 M 最优卡组
\lstinputlisting{Source/Source/Greed/LOJ6254.cpp}

\subsubsection{类型二}
给定一个可重集合,选择$m$个数作为一种方案,求价值前$k$大的方案。

同样使用按位更改的方法,不过要按顺序记录每个被选数的位置。这样的状态转移是一个DAG,
存在重复状态。可以再加一维$i$表示$i+1,\cdots m$的位置都被确定,当前状态转移时
要么将锁定位置前移并修改,要么修改位置$i$选择的数。

\paragraph{特殊情况}
若方案的价值定义为它们的积且存在负元素,可以考虑将正负数分开计算,然后枚举负数个数,
根据负数个数的奇偶性决定取负数积的绝对值的前$k$大/小值。最后使用类似双指针法根据计算
出的正负数两个堆得到所需答案。

例题:SGU 421 k-th Product

该内容参考了tkandi的博客\footnote{
	集合选数最值一类问题
	\url{https://www.cnblogs.com/tkandi/p/9375509.html}
}。
\subsection{二进制分组}\label{BinIns}
对于在线向序列右端插入元素的操作(不修改/删除),可以按照最大规模固定规模分块后建块,
询问时对每块暴力查询。假设重建块的复杂度为$O(n\lg n)$,查询复杂度为$O(\lg n)$,
询问量$q$与元素最大规模$n$同阶。离线问题则可以考虑对时间分治,当然不管哪种方法都要求
插入的贡献独立。

考虑将当前序列长度进行二进制拆分,然后令每个拆分值对应一个块的大小。
即新加入一个元素作为独立块后,不断检查块序列最右端两个块大小是否相等,相等则合并
为一块并重建(注意要先标记范围再合并(可以$O(1)$得到范围,见下文),每次插入最多合并一次)。
此法保证了查询时块的数目是$\lg n$的,比固定大小分块算法更优。

接下来证明插入复杂度:

插入第$k$个元素后,需要合并的元素数为$lowbit(k)$,设$n$为$2$的幂,有
\begin{eqnarray*}
	T(n)&=&\sum_{k=1}^n {lowbit(k)\cdot \lg lowbit(k)}\\
	&=&\sum_{k=1}^{\lg n} {\frac{n}{2^{k+1}}\cdot k\cdot 2^k}\\
	&\leq&\sum_{k=1}^{\lg n} {n \lg n}\\
	&=&O(n\lg^2 n)
\end{eqnarray*}
\subsection{整体二分}
对于某些离线问题,若其询问仅一种且可二分,整体二分是个不错的选择。
其思想是将所有询问的二分操作放在一起,分治将询问分组到两个区间中。
分治时可以顺便处理修改操作。

记整体二分主过程为$solve(beg,end,l,r)$,其中二分询问/查询为$[beg,end)$,
二分答案范围为$[l,r]$。

步骤如下:
\begin{enumerate}
	\item 若区间元素为空则返回;
	\item 若$l==r$,设置区间内元素的答案为$l$并退出;
	\item 令$m=(l+r)/2$;
	\item 施加$[l,m]$内的修改,累积到每个询问的贡献(注意要保证这部分的复杂度与
	      区间长度相关);
	\item 对于修改按照位置分为左右集合,对于询问按照当前累积贡献与目标值的关系
	      将其分为2个集合;
	\item 递归分治左右集合。
\end{enumerate}

注意修改与查询之间也是有时间先后关系的,可以调用std::stable\_partition完成稳定划分。
\subsection{cdq分治}\label{CDQ}
cdq分治用来解决偏序问题(支持动态偏序问题,但仍然要求离线)。

步骤如下($solve(l,r)$):
\begin{enumerate}
	\item 若区间元素为空则返回;
	\item 令$m=(l+r)/2$;
	\item 递归分治处理左右集合;
	\item 计算一边集合对另一边集合的影响(比如左边元素对右边元素答案的影响,
	      左边修改对右边查询的影响,有时左右元素会互相影响)。
\end{enumerate}

对于静态偏序问题,可以将第一维当做时间,转化为动态偏序问题。
对于动态偏序问题,保证第一维按照时间排序,递归合并时边计算影响边进行归并排序,
使其排序后位置从小到大,修改优先于查询(要保证两集合内部元素的时间先后关系,即排序要稳定)。
\subsection{Kernelization}
\index{*TODO!Kernelization}
参见\url{https://www.zhihu.com/question/272303098/answer/367368615}。
\subsection{启发式合并}
如果某个区间内需要的数据结构可以用子区间的数据结构合并,且规模与区间长度成正比,
就可以考虑使用启发式合并。每次合并时将较小的数据结构拆开,插入大的数据结构中。
每合并一次至少会使数据结构扩大一倍(针对较小数据结构而言),合并次数为$O(\lg n)$,
插入次数为$O(n\lg n)$。此法适用于平衡树、链表等只支持单点插入的数据结构,线段树则
有另外一套简单的合并方法。
\subsection{启发式分治}
区间$[l,r]$是否符合条件与区间的特殊值相关,求符合条件的区间数。

如果区间中的某个数的特殊性可以决定跨越它的区间是否符合条件,那么就可以
从两边开始向中间寻找这个数,然后递归处理被划分的两个区间。每次划分的复杂度取决于
左右两端哪端更快找到特殊值,时间复杂度$O(n\lg n)$。实际时间复杂度还取决与跨数
区间的处理复杂度,一般使用可持久化数据结构预处理整个区间,然后枚举较小区间的端点
查询与较大区间的贡献,一般时间复杂度$O(n\lg^2 n)$。

例题:
\begin{itemize}
	\item UVA1608/[Cerc2012]Non-boring sequences
	\item 「Luogu4755」Beautiful Pair
	\item 「LOJ6198」谢特 (SA的height数组的性质使该方法很适用于此种问题)
\end{itemize}

上述内容参考了DSL\_HN\_2002的博客\footnote{
	「随笔」一种基于启发式思想的分治策略——启发式分裂
	\url{https://blog.csdn.net/DSL\_HN\_2002/article/details/81193576}
}。
\subsection{分段打表}
需求:少量大规模单点查询,查询上界已知,支持快速转移。比如求$1e9$范围内的模意义阶乘。

可以写一个打表程序,每隔$1e5$输出一个点,然后将其导入源程序,查询时寻找最近的点,
从该点开始转移。
\subsection{注意事项/常见转化/思想}
\begin{itemize}
	\item 求解区间相交/包含/相离问题时首先判断这三种情况是否合法。若相交不合法,可以
	      考虑将区间缩点。对于序列问题考虑其置换$b_{a_i}=i$。(WYX'S BLOG
	      \footnote{[CTSC2018]青蕈领主\url{http://blog-wayne.com/2018/05/16/523/}})
	\item 「雅礼集训 2017 Day8」价:最小割建图时考虑与inf加减作为边权。
	\item 「TJOI / HEOI2016」字符串:前缀$\rightarrow$后缀,使用后缀系列算法解决。
	\item 删除+可离线$\rightarrow$逆序加入
	\item 「LNOI2014」LCA、HNOI2015开店 :两点的LCA深度可以理解为一个点到根的链
	      标记+1,统计另一个点到根的标记总和。
	\item 「LibreOJ Round \#6」花火:将点的编号与点值视为二维平面上的一个点
	      $P_i=(i,h_i)$,若交换$i,j$则减少了$P_i,P_j$矩形内的点数(不含边界)*2+1。
	      可以推出$h_i$一定是前缀最大值,$h_j$一定是后缀最小值,否则存在更优的矩形。
	      求出前缀最大点集$L$与后缀最小点集$R$后,设选择的矩形为$(i,j)$,考虑点$k$被包含
	      在$(i,j)$内的条件。记$l$为前缀最大点集中在$k$左上方的点的最左位置,$r$为后缀最小点集
	      中在$k$右下方的最右位置,这两个均可以二分处理。那么条件为$i\in [l,k-1],j\in [k+1,j]$,
	      将问题转化为每个点$k$覆盖一个矩形,求点的最大覆盖值。使用扫描线+线段树解决。最后答案为
	      逆序对-最大覆盖数*2(可非相邻交换1次与减少1对逆序对抵消)。
	\item 「LibreOJ Round \#11」Misaka Network 与任务:需要大量同指数快速幂且已知
	      底数范围时使用线性筛预处理。
	\item 解题时先考虑不带修改的情况。
	\item 利用时间存储数据。
	\item THUWC2019 D2T1:对于统计树上所有路径的权值和问题,除了点分治外,一个比较简单
	      的思路是考虑DFS统计一条边两边节点与这条边的贡献。由于在换边时点集变化比较大,可以考虑
	      统计当前遍历的点/边与已遍历边/点的贡献。注意DFS从边进入子树与从边走出子树时边的权要修改
	      (子树反向)。
	\item 最短路、网络流问题在时空复杂度无法承受时考虑动态加边。
	\item 询问参数局部修改时考虑从上一个最优解开始迭代。
	\item 区间范围没有保证$l\leq r$时要注意swap。
	\item 节点到根的路径上的点权值+1,统计某个点的权值。\CJKsout{树链剖分+区间修改单点查询。}
	使用树上差分直接在该点修改,查询时查询子树权值和,可以$O(n)$预处理$O(1)$查询,比原方法
	修改少$\lg^2 n$,查询少$\lg n$。链上操作类似。

	一般化+在线:链上节点权值+1,统计某个点的权值。同样采用树上差分,在两端点处+1,在lca和lca父亲
	处-1。单点修改与查询子树权值和使用树状数组解决。
	\item 多个卷积后的函数之和可以在点值乘法时就相加,最后只做一次IDFT。
	\item 区间gcd问题:注意到$gcd(a,b,c)=gcd(a,b-a,c-b)$,对原数组差分后仅需支持单点修改,单点查询
	和区间gcd操作。差分后可以支持区间加减操作。
	\item 支持有后效性的可加减区间操作,询问历史版本和:维护两个数组$A,B$,在$A$上记录所有区间操作,
	在$B$上记录若该区间操作从0时刻开始与从现在开始的贡献差。若当前询问时刻为$t$,则答案为$t*A-B$。
	\label{HistorySum}
	\item 无标号的方案数可转化为有标号的方案数解决,最后将答案除以每种元素个数的阶乘之积。
	\item 写题时忽视背景描述,读清题目,看一遍小样例,确定明白题意后想题。想题时先多推性质,
	然后每部分选择最简数据结构/算法实现,确定算法正确且复杂度正确后再敲代码。
	\item 对字符进行重映射时注意字典序。\CJKsout{FWC2019D1T2 AGCT把我最后的40分也给送了。}
	\item 给一些元素A分配不重复元素B:考虑枚举足够的元素B,然后给两种元素连边跑二分图匹配/费用流。
	\item 要确保暴力算法/小数据特判的正确性。\CJKsout{FWC2019D2T1 特判写炸+捆绑测试=AC
	->20分}
	\item 构造问题中要大胆猜想,利用随机性质缩小搜索范围,对于序列构造问题考虑排序贪心,对于树的构造
	问题考虑菊花树/二叉树。
	\item DP:考虑元素的组合是否要求有序。比较数据范围,合并本质相同的元素,降低时间复杂度。
	\item $n$很大,求第$n$项:若答案是次数与其他项有关的关于$n$的多项式,求出小数据答案后多项式
	插值。若答案是次数与$n$有关的多项式,考虑矩阵快速幂。
	\item 每个物品有存在区间,查询在某个时刻$t$的答案。一般使用离线+扫描线解决。若存在区间定长为$p$,
	则可以将原问题转化为查询$(t-p,t]$出现的物品的贡献。记区间长度为$L$,在$L$上每$p$个时间取一个
	关键点,可以发现任意查询区间都恰好覆盖一个关键点。预处理从每个关键点开始向左右$p$个时间的答案。
	查询时定位到对应关键点,进而定位到预处理左右区间的答案,快速合并得到答案。预处理的长度是$O(L)$的。
	猫树也是这个思想。
	\item 如果离线处理时空间不够,那就分多次离线。注意有些问题无法分割(比如组合问题)。
	\item 对于区间内的数的最大值为定值的限制,可以将其表示为区间内的数不大于该定值的方案扣除区间内的数
	不大于该定值-1的方案。
	\item 矩阵快速幂时若时间复杂度不优则观察矩阵的性质。
	\item 与排序有关的问题:考虑枚举某个数$x$,将整个序列表示为01序列。序列在全局下有序,说明相邻的数
	之间都是有序的。
	\item 「HAOI2018」字串覆盖:根据数据范围提示讨论子串大小,不重叠子串长则出现次数少,分两种情况使用
	不同方法处理。「雅礼集训 2017 Day1」字符串:$k*q$小于等于定值$1e5$,并且当$k$较小与$q$较小时都有
	较优的做法,分情况处理。一般来说,若存在两个量的乘积不超过定值,并且存在两种高效算法分别主要与这两个
	量有关,那么就可以考虑对询问分类处理。
	\item 最长公共子序列:dp数组单调且相邻项之差不超过1时考虑对数组差分,得到01序列,然后上bitset加速
	转移。
	\item 对于对相邻项/行列/对称位置有约束的矩阵构造问题,考虑两个方向的斜线以及分块递归构造。
	\item 注意FFT与NTT计算的是循环卷积,可以利用此性质优化常数。
	\item 「LibreOJ β Round \#7」匹配字符串:对于类似于$f_i=af_{i-1}+bf_{i-m}$
	这类的表达式,可以将其视作一个DAG,其中$i$向$i+1$连一条权值为$a$的边,向$i+m$
	连一条权值为$b$的边,定义一条路径的权值为边权之积。若$f_0=1$,则求$f_n$即为求
	从点0到点$n$的路径权值和。那么考虑枚举走$i\rightarrow i+m$的边数,使用可空
	隔板法,有
	\begin{displaymath}
		f_n=\sum_{i=0}^{\lfloor\frac{n}{m}\rfloor}{\binomial{n-im+i}{i}a^{n-im}b^i}
	\end{displaymath}
	\item 在模费马素数意义下,由于其$\varphi$为2的幂,且以任意整数为生成元做模意义
	乘法构成的群的阶都是它的因子。对于某个常底数,它的阶可能极小,可以根据指数分类计算。
	例如2模65537的循环节大小为32。
	\item 做概率期望类dp时,若每一步都贪心选取最优方案,需要从后往前dp。
	\item CF553E Kyoya and Train:dp不好按照点转移,但发现时间t是递增的,状态
	可以拆成分层图,因此使用cdq分治FFT转移。
	\item $B'(x)A(x)=A'(x)\Rightarrow B(x)=\ln A(x)$。
	\item 求某个元素必选的最优方案:分别对前缀与后缀dp,再组合dp。
	\item 用无限次修改最优化某个量:寻找修改中的不变量。
	\item 统计区间或链上颜色数:考虑每个颜色的贡献,对每种颜色分别统计。
	\item 考虑是否仅依赖相邻位置的关系,序列$\Rightarrow$相邻位,网格图$\Rightarrow$
	相邻行。
	\item APIO2018铁人两项:树上指定点集两两树上路径点权和:考虑每个点权的贡献。
	\item 当元素之间无先后顺序要求时,可以考虑按照元素的某个属性排序。
	\item 对于等比数列求和问题,若不存在乘法逆元,则考虑分治计算或使用矩阵记录历史版本和,
	再做快速幂。
	\item 树上某点的儿子的子树大小只有$O(\sqrt{n})$种,可以合并以优化复杂度。
	\item 边在满足指定条件时连通:
	\begin{itemize}
		\item 单向限制$(w\leq a)$+强制在线:重构树
		\item 双向限制$(a\leq w \leq b)$+可离线:扫描线+LCT
	\end{itemize}
	\item 连通块计数
	\begin{itemize}
		\item 树上连通块个数=点数-边数(统计无向图连通块个数需要去环,根据题目需要以
		权值/编号为关键字删掉环上的某条边,一般使用LCT维护)
		\item 网格图四连通块个数=格子数-1*2矩形数-2*1矩形数+2*2矩形数+环数
	\end{itemize}
	\item 注意最小权最少连通块最多只有n-1条边(MST),在合并时可以使用Kruskal保留
	有用边。
	\item $\ln \frac{1}{1-x}=\displaystyle \sum_{i=1}^\infty{\frac{x^i}{i}}$,
	在做多项式ln前考虑生成函数的封闭形式是否可以直接计算。
	\item 最大化$k$个连续子段和(【POJ Challenge】生日礼物):
	\begin{itemize}
		\item 费用流:$S\rightarrow[1,n]$连边$(1,0)$,$i\rightarrow i+1$
		连边$(1,A[i])$,$[2,n+1]\rightarrow T$连边$(1,0)$,控制最大流为$k$。
		\item 线段树模拟费用流:注意到此题费用流建图的结构比较简单,考虑增广的过程。
		选取最长增广路径相当于查询区间最大子段和,增广的过程相当于区间取反(下次如果
		选取了这一段就相当于退流),用线段树支持这两种操作。

		选取不超过$k$个正权区间即为答案(朴素费用流可以走空区间)。
		\item 贪心+优先队列:首先将同号的缩点,然后去除两边的负数。考虑选取所有的正块,
		然后将块数降为$k$。那么可以先将所有正块加入答案,再将所有块的绝对值加入优先队列,
		每次选取权值最小的删除,答案减去它的权值。删除正块相当于舍弃它,删除负块相当于
		选取它,然后可以合并两边的正块。注意一个块一旦被选择,它的两边都不能被选择。用
		一个链表维护:如果当前块不为边界,则保留左右块,将左右块节点删除,可能选择左右块节点
		优,需要支持后悔操作(模拟费用流一般都需要),权值为左右块权值和减去自身权值,将其加入
		优先队列中;否则左右块与自己都不能再被选择,将其从链表中删去。
	\end{itemize}
	\item 对于整体取反操作,可以同时维护两个方面的信息,取反时直接swap。
	\item 对于区间内选取最佳位置问题,可以将询问按照左端点排序,然后从右到左加入每个位置,
	使用单调栈维护位置作为答案的区间,二分查询回答询问。
	\item 对于复制数据结构的问题,考虑复制的每个元素中待统计信息的占比,修改时大部分的转移
	是相同的。
	\item 方差计算的转化:\begin{displaymath}
		\sum{(x_i-\bar{x})^2}=\sum{x_i^2}-n\bar{x}^2
	\end{displaymath}
\end{itemize}
\subsection{比赛注意事项}
\subsubsection{Linux/GCC工具}
鉴于THUWC2019Day3工程题的CRC32校验可以直接使用系统内置工具。。。
\begin{itemize}
	\item hashsum/md5sum/shasum/cksum/crc32/sum:计算文件校验和

	{\bfseries 注意校验文本时一定要使用不换行输出,即echo -n xxx | xxxsum。}
	\item python:本地高精度计算/binascii.crc32/hashlib.md5
	\item size:查看可执行文件静态分配内存大小
	\item diff:文件比较,要加入-s-b选项以忽略多余空格和换行,完全相同时输出信息
	\item perf(Linux不一定自带):底层性能分析

	主要性能指标为IPC,Cache命中率,分支命中率。所以命令为perf~stat~-e~
	cache-misses,branch-misses,instructions,branches,cache-references,
	cpu-cycles~./a.out
	\item time:测量程序运行时间
	\item ulimit -s size:开栈
\end{itemize}
\subsubsection{用于Debug的编译命令}
\begin{itemize}
	\item -ftrapv 有符号算术运算溢出检测(需要时再开,因为有时候会利用自然溢出)
	\item -Wall 最高级别警告
	\item -Wextra 扩展警告
	\item -fsanitize=undefined 未定义行为检测
	\item -fsanitize=address 访问越界检测
	\item -ggdb3 生成适宜GDB的调试信息
\end{itemize}
\subsubsection{进程信息查询}
在/proc/self/status文件中有程序运行信息,其中VmPeak指示峰值虚拟内存。使用时在程序退出
前将这个文件的内容重定向至stderr。
\subsubsection{代码注意事项}
\begin{itemize}
	\item 提前退出计算时注意数据清空:
	\begin{itemize}
		\item setjmp/longjmp:不要使用任何RTTI数据结构,全部的longjmp由一个函数
		earlyExit调用,函数内执行清空操作
		\item break/return:使用RTTI技术
		\item throw/try-catch:别用这个
	\end{itemize}
	\item 个人喜欢使用但是C++98不支持的:
	\begin{itemize}
		\item std::vector<>::data()
		\item fmax/fmin
		\item fma
		\item <random>
		\item <chrono>
		\item 委托构造函数
	\end{itemize}
	\item 不要使用while(n$--$)这种语句,宁可用for(int i=0;i<n;++i)代替。因为并不知道
		  将来是否要用到$i$或$n$,还可能导致莫名其妙的错误。
	\item int内二分时,要注意$l+r$是否溢出,一般使用$l+((r-l)>>1)$代替。
	\item 尽量不要使用与标准库函数相同的名称,哪怕没有using~namespace~std;(尤其
	是STL,编译器的错误信息能让人抓狂)。
	\item 边读入边处理时不要跳出循环,若要跳出循环则需继续读入以跳过冗余数据。
	\item std::priority\_queue<>的清空使用.swap()完成
	\item 删除std::multiset内的单个元素时要使用迭代器
	\item std::set的插入最好使用hint与区间。
	\item (分段)打表时注意源代码不能超过100KB。
	\item 不要使用std::bitset的set与test,它们会执行越界检查。不过operator[]的性能
	尚未验证,因为它们的返回值是bool的Proxy Class------std::bitset::reference。
	\item 跨平台scanf/printf:最保险的方式是手写输入输出。另一种方式是使用<cinttypes>
	提供的PRI/SCN*宏。例如不管long long输入输出是\%lld还是\%I64d,都可以使用SCNd64/
	PRId64表示。
	\item 需要稳定划分时不能使用维护左区间end位置与swap操作实现,必须使用额外缓冲区,
	调用std::stable\_partition。
	\item set与map的插入与删除操作不影响之前取得的迭代器的有效性。
	\item 调试输出时重定向至stderr,即使忘记删掉也可以降低风险。
	\item \%c输入时不会跳不可见字符,可以在其前面加空格或者使用\%s
	\item 使用strtol/strtod时,要么一次性读入整个文件,要么读完整段后再调用,
	直接在缓冲区上搞可能会导致RE与读入错误。
	\item 避免出现Undefined behavior导致优化后程序错误。例如有符号整数溢出:要
	利用自然溢出特性时强转至unsigned做加法。
\end{itemize}
\subsection{本节注记}
2013年许昊然的国家集训队论文答辩《浅谈数据结构题的几个非经典解法——<Claymore>命题报告》
中提到了不少奇妙的解法。
