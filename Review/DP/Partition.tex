\section{分治DP}
例题:「雅礼集训 2017 Day5」珠宝

这是一个经典的背包问题,但是按照背包做会TLE。发现代价$c$很小,考虑按照$c$分类
然后转移,花费$kc$的代价时贪心地选取价值前$k$大的物品(我的思路止步于此)。

由于同一条dp依赖链上的位置模$c$同余,可以考虑再按照位置模$c$分类,将每一条链拆开处理。
注意到选取前$k$大物品的代价的增长率是单调非增的,尝试验证该dp是否有决策单调性。

采用反证法,设两个同类dp点$i,j$,满足$i<j$,记它们的转移决策点分别为$P_i,P_j$,
满足$P_i>P_j$。记原先的dp数组为$dp$,前$k$大前缀和数组为$w$。根据决策点的定义,有:
\begin{eqnarray*}
    dp[P_i]+w[i-P_i]>dp[P_j]+W[i-P_j]\\
    \Rightarrow dp[P_i]-dp[P_j]>W[i-P_j]-w[i-P_i]\\
    dp[P_i]+w[j-P_i]\leq dp[P_j]+W[j-P_j]\\
    \Rightarrow dp[P_i]-dp[P_j]\leq W[j-P_j]-w[j-P_i]\\
\end{eqnarray*}

由不等式传递性可得$W[i-P_j]-w[i-P_i]<W[j-P_j]-w[j-P_i]$,由于$i<j$且增长率单调非增,
与该式产生矛盾,因此转移点是单调的。

那么可以写一个分治程序$solve(l,r,L,R)$,表示处理$[l,r]$之间的dp值,转移区间在$L,R$。
每次在转移区间内扫一遍求出$mid$的dp值,得到转移点,最后左右递归处理,时间复杂度$O(ck\lg k)$。

代码:
\lstinputlisting{Source/Source/DP/LOJ6039.cpp}

上述内容参考了ShichengXiao的博客\footnote{
    DP及其优化
    \url{https://www.cnblogs.com/ShichengXiao/p/9501386.html}
}。
