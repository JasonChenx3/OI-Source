\section{动态dp}
动态dp与常规dp的区别就是加上了多次修改与询问。
\subsection{序列DDP}
\subsubsection{区间最值问题}
一般使用线段树维护区间信息。
\paragraph{例题 小白逛公园}
使用线段树维护区间最大子段和。
\lstinputlisting{Source/Source/DDP/4513.cpp}
\subsection{树上DDP}
树上DDP一般使用线段树+树链剖分+矩阵乘法。

考虑动态最大带权独立集问题:
\subsubsection{描述dp转移}
记$f_u$为不选点$u$的子树最大带权独立集,
$g_u$为选点$v$的子树最大带权独立集。

那么对于点$u$的子树$T_u$有
\begin{eqnarray*}
    f_u&=&\sum_{(u,v)\in T_u}{max(f_v,g_v)}\\
    g_u&=&w_u+\sum_{(u,v)\in T_u}{f_v}
\end{eqnarray*}

类似于算法导论\cite{ITA3}中解决所有节点对最短路径问题时
介绍的类矩阵乘法,这里把对单个儿子$v$的转移视为左乘一个由点$u$当前dp状态决定
的转移矩阵,把$(f_v,g_v)$视作向量。其中矩阵乘法的定义需要修改:乘法变为加法,
加法变为取max,这个新的乘法操作仍然满足结合律。单位阵$I_n$的主对角线上为0,其余
元素为$-\infty$。

所以转移式如下:
\begin{displaymath}
    \left[
    \begin{array}{cc}
        f_u&f_u\\
        g_u&-\infty
    \end{array}\right]
    \left[
    \begin{array}{c}
        f_v\\g_v
    \end{array}\right]  =
    \left[
    \begin{array}{c}
        f'_u\\g'_u
    \end{array}\right]
\end{displaymath}

考虑一条链的情况,链头的dp向量即为链上所有点按顺序组成的矩阵链之积(不是向量?视作
在链尾后再加一个空点,即向量{\bfseries 0})。
每个点的矩阵均为考虑自身权值与分支后的转移矩阵。

此时已经将dp转移转化为矩阵乘法的形式。
\subsubsection{修改与查询}
暴力算法每一次都要维护从修改点到根的点权,转化为矩阵乘法后也不例外。
考虑对整棵树进行树链剖分,暴力跳重链维护轻儿子对父亲的贡献,查询时线段树
维护区间矩阵乘积,把重链的贡献施加到父亲上。注意跳重链时根据树链剖分的性质,
该操作不超过$O(\lg n)$次,因此每次修改的复杂度为$O(\lg^2 n)$。
查询时间复杂度$O(m\lg^2n)$。

代码如下:
\lstinputlisting{Source/Templates/TDDP.cpp}

对于其它动态树型dp问题,其关键是将dp转化为矩阵乘法的形式,然后
套树链剖分+线段树解决。为了保证更新轻儿子对父亲矩阵的贡献的复杂度,
不能重新对所有轻儿子dp,而应该把自己未修改前的矩阵(考虑重链)对父亲
转移矩阵的贡献扣去,然后施加变换更新线段树,重新计算自身矩阵然后修改
父亲的矩阵。所以对矩阵的修改需要``延迟更新''。

上述内容参考了小蒟蒻yyb的博客\footnote{
    动态dp
    \url{http://www.cnblogs.com/cjyyb/p/10031947.html}
}。
\subsection{全局平衡二叉树}
这个黑科技出自2007年Yang Zhe的论文《SPOJ375 QTREE解法的一些研究》\cite{GBT}。

考虑把树链剖分+线段树换成LCT,直接维护dp值可以把查询时间复杂度降到
$O(m\lg n)$。但是LCT的常数太大,其表现不如树链剖分+线段树。事实上
我们并不需要LCT的link-cut功能,因此可以考虑把树静态化,即构造一个
能够暴力向上更新的数据结构。

全局平衡二叉树就能满足这一要求。建树过程很简单:
\begin{enumerate}
    \item 树链剖分求出重儿子;
    \item 给每个节点附上权值,权值为轻儿子子树大小之和+1。
    \item 对于每条重链找整条链的带权重心,把重心当做bst的根,
    然后递归两边继续找带权重心建bst。
\end{enumerate}

暴力更新即自底向上沿着bst和虚边更新。注意我们只需要维护每棵bst的先序遍历矩阵积,
对应每条重链的矩阵积。可以发现无论经过bst的边还是虚边,子树大小至少增大1倍(经过
bst的情况可类比点分治性质~\ref{WPP})。所以树高为$O(\lg n)$,查询复杂度降为
$O(m\lg n)$。

模板:
\lstinputlisting{Source/Templates/GBT.cpp}

上述内容参考了shadowice1984的博客\footnote{
    题解 P4643 【【模板】动态dp】 - 某菜鸡的blog\\
    \url{https://www.luogu.org/blog/ShadowassIIXVIIIIV/solution-p4643}
}。
