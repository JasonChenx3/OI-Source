\section{动态dp}
动态dp与常规dp的区别就是加上了多次修改与询问。
\subsection{序列DDP}
\subsubsection{区间最值问题}
一般使用线段树维护区间信息。
\paragraph{例题 小白逛公园}
使用线段树维护区间最大子段和。
\lstinputlisting{Source/Source/DDP/4513.cpp}
\subsection{树上DDP}
树上DDP一般使用线段树+树链剖分+矩阵乘法。

考虑动态最大带权独立集问题:
\subsubsection{描述dp转移}
记$f_u$为不选点$u$的子树最大带权独立集,
$g_u$为选点$v$的子树最大带权独立集。

那么对于点$u$的子树$T_u$有
\begin{eqnarray*}
    f_u&=&\sum_{(u,v)\in T_u}{max(f_v,g_v)}\\
    g_u&=&w_u+\sum_{(u,v)\in T_u}{f_v}
\end{eqnarray*}

类似于算法导论\cite{ITA3}中解决所有节点对最短路径问题时
介绍的类矩阵乘法,这里把对单个儿子$v$的转移视为左乘一个由点$u$当前dp状态决定
的转移矩阵,把$(f_v,g_v)$视作向量。其中矩阵乘法的定义需要修改:乘法变为加法,
加法变为取max,这个新的乘法操作仍然满足结合律。单位阵$I_n$的主对角线上为0,其余
元素为$-\infty$。

所以转移式如下:
\begin{displaymath}
    \left[
    \begin{array}{cc}
        f_u&f_u\\
        g_u&-\infty
    \end{array}\right]
    \left[
    \begin{array}{c}
        f_v\\g_v
    \end{array}\right]  =
    \left[
    \begin{array}{c}
        f'_u\\g'_u
    \end{array}\right]
\end{displaymath}

此时已经将dp转移转化为矩阵乘法的形式。
\subsubsection{修改}
\subsection{全局平衡二叉树}
\subsection{基于链的分治优化树上动态DP}
