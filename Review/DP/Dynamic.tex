\section{动态dp}
动态dp与常规dp的区别就是加上了多次修改与询问。

序列DDP一般使用线段树维护区间信息,这里不详述。下文的重点是树上DDP。
\subsection{常规树上DDP方法}
树上DDP一般使用线段树+树链剖分+矩阵乘法。

考虑动态最大带权独立集问题:
\subsubsection{描述dp转移}
记$f_u$为不选点$u$的子树最大带权独立集,
$g_u$为选点$v$的子树最大带权独立集。

那么对于点$u$的子树$T_u$有
\begin{eqnarray*}
    f_u&=&\sum_{(u,v)\in T_u}{max(f_v,g_v)}\\
    g_u&=&w_u+\sum_{(u,v)\in T_u}{f_v}
\end{eqnarray*}

类似于算法导论\cite{ITA3}中解决所有节点对最短路径问题时
介绍的类矩阵乘法,这里把对单个儿子$v$的转移视为左乘一个由点$u$当前dp状态决定
的转移矩阵,把$(f_v,g_v)$视作向量。其中矩阵乘法的定义需要修改:乘法变为加法,
加法变为取max,这个新的乘法操作仍然满足结合律。单位阵$I_n$的主对角线上为0,其余
元素为$-\infty$。

所以转移式如下:
\begin{displaymath}
    \left[
    \begin{array}{cc}
        f_u&f_u\\
        g_u&-\infty
    \end{array}\right]
    \left[
    \begin{array}{c}
        f_v\\g_v
    \end{array}\right]  =
    \left[
    \begin{array}{c}
        f'_u\\g'_u
    \end{array}\right]
\end{displaymath}

考虑一条链的情况,链头的dp向量即为链上所有点按顺序组成的矩阵链之积(不是向量?视作
在链尾后再加一个空点,即向量{\bfseries 0})。
每个点的矩阵均为考虑自身权值与分支后的转移矩阵。

此时已经将dp转移转化为矩阵乘法的形式。

\paragraph{注意事项}
\begin{itemize}
    \item 「NOIP2018」保卫王国:构造出矩阵后转移矩阵内一定有一列与向量的表示相同,
    取dp值时从这里取。更准确地说,要乘上末尾的哨兵向量观察取值。
    \item 「SDOI2017」切树游戏:在构造矩阵时不要吝啬矩阵大小,可以加入一些常量辅助
    转移,然后模拟多次矩阵相乘,找出不变量并省略(尤其是常量所在位置),以节省时间和空间。
\end{itemize}

\subsubsection{修改与查询}
暴力算法每一次都要维护从修改点到根的点权,转化为矩阵乘法后也不例外。
考虑对整棵树进行树链剖分,暴力跳重链维护轻儿子对父亲的贡献,查询时线段树
查询区间矩阵乘积,把自己所在重链的后代的贡献施加到自身矩阵,计算出真实的转移矩阵。
注意跳重链时根据树链剖分的性质,该操作不超过$O(\lg n)$次,因此每次修改的复杂度
为$O(\lg^2 n)$。

代码如下:
\lstinputlisting{Source/Templates/TDDP.cpp}

为了保证更新轻儿子对父亲矩阵的贡献的复杂度,不能重新对所有轻儿子dp,
而应该把自己未修改前的{\bfseries 真实转移矩阵}对父亲转移矩阵的贡献扣去,
然后施加变换更新线段树,重新计算自身真实转移矩阵然后修改父亲的矩阵。所以对
矩阵的修改需要``延迟更新''。

对于其它动态树形dp问题,其关键是将dp转化为矩阵乘法的形式,然后
套树链剖分+线段树解决。

上述内容参考了小蒟蒻yyb的博客\footnote{
    动态dp
    \url{http://www.cnblogs.com/cjyyb/p/10031947.html}
}。
\subsection{全局平衡二叉树}
这个黑科技出自2007年Yang Zhe的论文《SPOJ375 QTREE解法的一些研究》\cite{GBT}。

考虑把树链剖分+线段树换成LCT,直接维护dp值可以把查询时间复杂度降到
$O(m\lg n)$。但是LCT的常数太大,其表现不如树链剖分+线段树。事实上
我们并不需要LCT的link-cut功能,因此可以考虑把树静态化,即构造一个
能够暴力向上更新的数据结构。

全局平衡二叉树就能满足这一要求。建树过程很简单:
\begin{enumerate}
    \item 树链剖分求出重儿子;
    \item 给每个节点附上权值,权值为轻儿子子树大小之和+1。
    \item 对于每条重链找整条链的带权重心,把重心当做bst的根,
    然后递归两边继续找带权重心建bst。
\end{enumerate}

暴力更新即自底向上沿着bst和虚边更新。注意我们只需要维护每棵bst的先序遍历矩阵积,
对应每条重链的矩阵积。可以发现无论经过的是bst的边还是虚边,子树大小至少增大1倍(经过
bst的情况可类比点分治性质~\ref{WPPA})。所以树高为$O(\lg n)$,查询复杂度降为
$O(m\lg n)$。

模板:
\lstinputlisting{Source/Templates/GBT.cpp}

这种写法算法速度快、代码简单、易于理解,推荐使用此法。

上述内容参考了shadowice1984的博客\footnote{
    题解 P4643 【【模板】动态dp】 - 某菜鸡的blog\\
    \url{https://www.luogu.org/blog/ShadowassIIXVIIIIV/solution-p4643}
}。

\subsection{子树DP值查询}
如果需要向bzoj4712 洪水那样查询某个子树内的dp值,不能直接使用子树根节点的信息。
因为整条链是一棵二叉树,这个根节点的矩阵积还包括其左儿子的贡献,但这一部分并不是
它的子树。实际答案应该为自身及其后继的积,实现时只有往右移才算入这个祖先的贡献。

参考代码:
\begin{lstlisting}
Mat res;
bool init = false;
int last = T[x].l;
while(true) {
    if(T[x].l == last) {
        if(init)
            res = res * T[x].mat;
        else
            res = T[x].mat, init = true;
        if(T[x].r)
            res = res * T[T[x].r].mul;
    }
    last = x;
    if(isRoot(x))
        break;
    else
        x = T[x].p;
}
\end{lstlisting}
\subsection{固定思路}
\begin{enumerate}
    \item 转移方式:推导dp转移方程,确定最少需要的信息,组成向量
    \item 矩阵表达:将父亲$u$转移儿子$v$的信息写成矩阵左乘向量的形式,矩阵与$u$相关,向量
    与$v$相关。若转移无法被表达,则尝试给向量增加新的维度(而不是让矩阵变为非方阵),
    新维度的信息用常量填充($0,1,\pm \infty$)。
    \item 确定有效项:模拟多次矩阵乘法,直至不变量的个数稳定(即这个由变量组成的多元组满足
    乘法封闭)。省去不变量,硬编码矩阵乘法。对于相等项可以只保留一份
    ({\bfseries 必须保证正确性})。
    \item 结果读取:构造链尾哨兵,右乘转移矩阵(使用已确定有效项的矩阵而不是最初的转移矩阵),
    根据向量元素的实际意义得到结果。
    \item 轻儿子DP值转移到重链上的方式:最好满足可减性(最好不要是乘法)。注意+ with
    max/min意义下乘法时不要为了使用int而给+操作做clamp,因为在替换时需要做信息减法,
    结果是错误的。对于不满足信息可减性的,与LCT维护子树信息的做法一样,给每个点开一个
    数据结构维护轻儿子的信息和。

    Update:由于每个点最多只给一个父亲贡献自己作为轻儿子的信息,可以将每个节点的所有轻儿子
    平铺到序列上,记录其区间,然后使用线段树维护。
    \item 单位阵构造(Optional,用于子树DP值查询):按照方程模拟确定$I$的每一项,如果
    不存在单位阵,乘法时记录是否已初始化的flag。
\end{enumerate}
