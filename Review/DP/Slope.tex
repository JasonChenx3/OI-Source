\section{斜率优化}\label{Slope}
\subsection{推导}
当状态转移方程为如下形式时:
\begin{displaymath}
    dp[i]=min(a[i]*b[j]+c[i]+d[j])
\end{displaymath}
考虑使用斜率优化。

以下推导假设$a[i]$单调递减且$b[j]$单调递增:

设决策点$j<k<i$,且从点$k$转移到$i$不差于从点$j$转移到$i$,
易证从点$k$转移到$i+1$同样不差于从点$j$转移到$i+1$。称该性质为决策单调性。

接下来考虑点$k$不比点$j$更差的条件:
\begin{eqnarray*}
    a[i]*b[j]+c[i]+d[j]&\geq&a[i]*b[k]+c[i]+d[k]\\
    \Rightarrow -a[i]&\geq&\frac{d[k]-d[j]}{b[k]-b[j]}
\end{eqnarray*}

记斜率
\begin{displaymath}
    slope(i,j)=\frac{d[j]-d[i]}{b[j]-b[i]}
\end{displaymath}

斜率可以使用单调队列维护,记$q[b]$为队首,$q[e-1]$为队尾:
\begin{itemize}
    \item 若$-a[i]\geq slope(q[b],q[b+1])$,则表明$q[b+1]$不比$q[b]$更差,
    弹出$q[b]$。
    \item 若$slope(q[e-2],q[e-1])\geq slope(q[e-1],i)$,则说明若
    $q[e-2]$被弹出,$q[e-1]$一定被弹出,所以$q[e-1]$无效,可以先弹出。
\end{itemize}

从``形''的角度理解,单调队列维护了一个下凸壳。
\subsection{应用}
主要过程就是研究决策单调性满足的条件,然后选取适当的数据结构维护信息,快速dp。

实际应用中需注意以下几点:
\begin{itemize}
    \item 比较斜率时尽量使用乘法避免精度误差,提高效率,要考虑变号时的符号问题
    \sout{(反正也就两处符号,面向样例编程就行了)}。
    \item 若$a[i]$单调,使用单调队列,否则使用二分法(使用类似线性规划的思路找到切点)。
    \item 若$b[i]$单调,使用单调队列,否则使用平衡树维护凸壳。

    {\bfseries 血泪史:「CEOI2017」Building Bridges

    事实上动态凸壳的维护不是很好处理,因为浮点数的精度问题不好解决。我调了3个多小时还是
    WA(更悲惨的是总共只WA一半,但每组捆绑测试都有测试点WA)。可以考虑维护动态半平面交,
    毕竟HPI还是比较成熟的,由于半平面只有插入,使用第~\ref{BinIns}节所述的二进制分组
    解决。事实证明HPI+二进制分组法数值稳定性比较好(一遍AC,速度比动态凸包快,代码比动态
    凸包短)。
    }
\end{itemize}

以上内容参考了MashiroSky的博客\footnote{斜率优化学习笔记 - MashiroSky
    \url{https://www.cnblogs.com/MashiroSky/p/6009685.html}
}。
