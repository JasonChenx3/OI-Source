\section{矩阵快速幂优化}
\subsection{常规矩阵快速幂}
若dp状态转移方程满足如下形式:
\begin{displaymath}
    dp[i]=\sum_{j=1}^k{c_idp[i-j]}
\end{displaymath}
或对于图满足如下形式:
\begin{displaymath}
    dp[d][i][j]=\sum_{(i,k)\in E,(k,j)\in E}{dp[d-1][i][k]\cdot dp[d-1][k][j]}
\end{displaymath}
则可以使用矩阵快速幂优化。

计算$k*k$的转移矩阵$A$,dp初始值为$1*k$的向量$v_0$。
构造$A$,使其每乘一次$A$,向量表示的区间后移一格,那么
$A[i][j]$表示其在做一次乘法后将第$i$点的值贡献到第$j$点中的权值。

矩阵乘法满足结合律,因此可以使用矩阵快速幂进行计算。

\paragraph{例题} 「NOI2007」 生成树计数

使用基尔霍夫定理计算肯定是不现实的(\sout{不过+k=3时的斐波那契数列可以水到80分})。
注意到连边是局部的,且$k$很小。可以使用状态来表示最近$k$个节点的连通性,然后枚举树边
转移。至于连通性的表示,可以使用最小表示法DFS预处理,即每个节点对应一个编号,且$i$必须
在$i+1$之前出现,相同编号的节点连通,并且这些状态保证了之前的链合法。当$k=5$时,对应的
集合划分数目(贝尔数)为52,使用矩阵快速幂转移。

参考代码:
\lstinputlisting{Source/Source/DP/LOJ2356.cpp}

\subsection{矩阵链乘秩分解}

若大小为$n*n$的矩阵$A$可表示为大小为$n*k$的矩阵$B$与大小为
$k*n$的矩阵$C$的乘积,其中$k\ll n$。
那么可以将$A$的幂拆开,错位结合,计算$k*k$的矩阵$D=CB$,对$D$快速幂后
计算需要的值{\bfseries (答案向量为$v_0AD^{c-1}B$,尽量按需计算结果)}。

\paragraph{例题} bzoj3583 杰杰的女性朋友

使用上述方法优化矩阵快速幂的效率。此外还存在一个问题,矩阵$A$的$k$次幂求的是走$k$
步的方案数的转移矩阵,但是答案要的矩阵为矩阵幂求和。因此我们可以对于每次询问再加一个
累加计数器,自己向自己连边,对应点向自己连边,最后单独求出起点到自己的方案数。即再开
一个``信道''?,然后在新开的信道上,终点的出边,计数器的入边与出边均设为1。

还有另一种方法:注意到这里求的是矩阵的等比数列之和。可以将数列对半分,然后可以提出
$(1+A^d)$的因子,子问题的规模减半。细节比较多,这里不详细写。

参考代码:
\lstinputlisting{Source/Source/DP/bzoj3583.cpp}

\subsection{dp步伐不一致时的解决方案}
例题~LOJ\#2180. 「BJOI2017」魔法咒语

如果单次转移最多需要跳跃$k$步($k$为小常数),可以给每个状态$S_0$引入$k-1$个``延迟状态''
$S_i$,若有状态$S$跳跃$k$步到达状态$T$,则实际转移为
$S_0\rightarrow T_{k-1} \rightarrow T_{k-2} \rightarrow \cdots \rightarrow T_0$。
注意后面的转移链是固定的,可以单独预处理。而第一个转移与原来的处理方式相同,只要根据跳跃
步数计算需要延迟转移的时间,然后连到链上对应的节点。统计时仍然只统计$S_0$,因为延迟状态
的贡献是不完整的。

\subsection{矩阵对角化加速快速幂}
有时会推出一些比较简单的矩阵(尤其是三角矩阵与对称矩阵),这时可以快速找到矩阵的特征值,
计算出特征向量,将矩阵表示为$A=PDP^{-1}$的形式,其中$P$是可逆矩阵,由特征向量组成,
$D$为对应特征值组成的对角矩阵。

那么$A^n$可以表示为$PD^nP^{-1}$的形式,中间的部分可以快速幂求出,两边的部分要根据具体
情况讨论。

\paragraph{例题} CF923E Perpetual Subtraction

记初始向量为$P_0$,转移矩阵为$A$,若用矩阵左乘表示转移,则答案为$A^nP_0$。

现在写出转移矩阵$A$:
\begin{displaymath}
    \left[
    \begin{array}{ccccc}
        1&\frac{1}{2}&\frac{1}{3}&\cdots&\frac{1}{n+1}\\
         &\frac{1}{2}&\frac{1}{3}&\cdots&\frac{1}{n+1}\\
         &           &\frac{1}{3}&\cdots&\frac{1}{n+1}\\
         &           &           &\ddots&\vdots\\
         &           &           &      &\frac{1}{n+1}\\
    \end{array}
    \right]
\end{displaymath}

这是一个上三角矩阵,不过即使有稀疏矩阵优化,$O(n^3\lg M)$的快速幂仍然无法承受。

由于该矩阵的特殊性,考虑对角化该矩阵以加速矩阵幂的计算。易知该矩阵的特征向量为
$1,\frac{1}{2},\cdots,\frac{1}{n+1}$。接下来代入$Av=\lambda v$求出
特征向量$v$并组出$P$:经过小矩阵的推导得特征值$\frac{1}{1+x}$的特征向量为
$[(-1)^i\binomial{i}{0},(-1)^{i+1}\binomial{i}{1},\cdots (-1)^{i+n}\binomial{i}{n}]^T$。
继续打表求出逆矩阵,然后将矩阵乘法展开,得到卷积的形式,使用NTT解决。具体证明参见参考链接。

该方法参考了yhx-12243的博客\footnote{
    [Codeforces923E/947E]Perpetual Subtraction\\
    \url{https://yhx-12243.github.io/OI-transit/records/cf923E\%3Bcf947E.html}
}。

对角化很简单,但是寻找特征向量与逆矩阵是困难的。
