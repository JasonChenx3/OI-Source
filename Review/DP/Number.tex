\section{数位动规}
问题一般是询问区间内数位满足指定要求的数的个数。
区间计数可转换为前缀和差分,因此原可问题转换为询问$n$
以内的满足指定要求的数的个数。

一般思路如下:
\begin{enumerate}
	\item 将$n$拆位为$n_k,n_{k-1},\cdots,n_1$;
	\item 从最低位开始dp一直做到最高位,第一维一般是首位数字;
	\item 从最低位开始统计到次高位,因为这部分答案是满的;
	\item 从最高位开始做到最低位,假设做到第$i$位,表示处理该位以前
	      的位都固定,枚举当前位$j<n_i$,使用预处理的dp值统计入答案。
\end{enumerate}

\paragraph{优化} 若固定的位不满足要求(比如对相邻位有要求)则直接返回
当前统计的答案(因为后续dp的数字都是不合法的,可以忽略)。
