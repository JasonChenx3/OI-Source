\chapter{动态规划}
\minitoc
\section{背包优化}
\subsection{完全背包优化}
\subsubsection{排序筛选}
若一种物品比另一种物品代价更大,收益更低,直接排除。
\subsection{多重背包优化}
\subsubsection{完全背包转换}
若该物品的数量已经超过最大装载量,直接将其当做完全背包,单种物品$O(V)$转移。
\subsubsection{二进制优化}
将数量按$2^0,2^1,2^2,\cdots,2^k,rem$拆分为多个物品,然后做01背包,
单种物品$O(V\lg c)$转移。
\subsubsection{单调队列优化}
朴素的多重背包状态转移方程为:
\begin{eqnarray*}
    end&=&min(c[i],j/v[i])\\
    dp[i][j]&=&max(dp[i-1][j-k*v[i]]+k*w[i]),0\leq k \leq end
\end{eqnarray*}
令$a=j/v[i],b=j\%v[i]$,将方程转换为:
\begin{displaymath}
    dp[i][b+a*v[i]]=max(dp[i-1][b+k*v[i]]-k*w[i])+a*w[i],a-end\leq k \leq a
\end{displaymath}
此时$k$表示比$a$少取$k$件。

此时$max$部分只与$k$的取值有关,可以使用单调队列优化。
每次转移时,枚举$b$,对$k$做单调队列,根据转移区间的移动把队首弹出队列。
单种物品$O(V)$转移。

该方法参考了soloier的博客\footnote{单调队列优化多重背包\\
\url{https://blog.csdn.net/sinat\_34943123/article/details/52857327}}。

\subsubsection{例题}
Luogu P1776 宝物筛选\_NOI导刊2010提高(02)
\footnote{【P1776】宝物筛选\_NOI导刊2010提高(02) - 洛谷
\url{https://www.luogu.org/problemnew/show/P1776}}

使用了上述的完全背包转换和单调队列优化。

\lstinputlisting[title=Luogu 1776]{DP/MultiBag.cpp}

\section{数位动规}
问题一般是询问区间内数位满足指定要求的数的个数。
区间计数可转换为前缀和差分,因此原可问题转换为询问$n$
以内的满足指定要求的数的个数。

一般思路如下:
\begin{enumerate}
	\item 将$n$拆位为$n_k,n_{k-1},\cdots,n_1$;
	\item 从最低位开始dp一直做到最高位,第一维一般是首位数字;
	\item 从最低位开始统计到次高位,因为这部分答案是满的;
	\item 从最高位开始做到最低位,假设做到第$i$位,表示处理该位以前
	      的位都固定,枚举当前位$j<n_i$,使用预处理的dp值统计入答案。
\end{enumerate}

\paragraph{优化} 若固定的位不满足要求(比如对相邻位有要求)则直接返回
当前统计的答案(因为后续dp的数字都是不合法的,可以忽略)。

\section{基于连通性状态压缩的动态规划/轮廓线DP/插头DP}
\subsection{简单回路问题}
例题:ural 1519

给定一个棋盘,某些格子不能经过,其余格子必须经过,求有多少条简单回路。
\subsubsection{最小表示法}
最小表示法用于表示行内格子的连通性。有两种方法:
\begin{itemize}
    \item 有障碍的格子标记为0,连通的格子标记为同一个数,并且$i$比$i+1$更早出现。
    预处理连通状态时使用DFS计算,同构于集合划分问题,状态总量为贝尔数。
    \item 有障碍的格子标记为0,其余格子标记为与其连通的最左格子的列号。
\end{itemize}

下文均使用第一种方法。存储时将其编码为$k$进制数,$k\geq$连通块个数+1。为了提高运算效率,
将$k$对齐至2的幂。
\subsubsection{状态的表示}
在例题中,每个格子与4个相邻格子都有可能连通,连接的部分称为``插头''。由于转移是
逐行进行的,下一行的插头受到上一行下插头的影响,因此每行需要用二进制表示当前行对应位置
是否有下插头。由于最后需要围成一个回路,还要维护行内格子之间的连通性。因此使用$F(i,j,k)$
表示前$i$行,下插头状态为$j$,连通性为$k$的方案数。注意到若该格子没有下插头,则它的连通性
不影响下一行的插头。那么干脆直接记录下插头的连通性,若不存在下插头则标记为0。

鉴于逐格递推比逐行递推更有优势,这里仅记录逐格递推。

用状态$F(i,j,k)$表示当前处理到第$i$行,已经处理完前$i-1$行和第$i$行前$j$列的格子,插头
连通性为$k$的方案数。已决策的格子与未决策格子之间的分界线称为``轮廓线''。在转移的过程中,
除了$n$个下插头外,第$i$格到第$i+1$格还有一个右插头。$k$从左到右表示$n+1$个插头的连通性。

考虑$k$需要使用的进制,由于一个回路最多有$m/2$个连通块,在例题中至少要用7进制,使用8进制
更加快速。
\subsubsection{状态的转移}
\paragraph{一般转移}
逐格递推,可以发现每移动一格最多有2个插头被改动。

枚举当前格子的插头,状态共有3种转移:

下文的``相接位置''指代轮廓线与当前格子的两条邻接边,``对应位''指转移前相接位置的位与
转移后新边的位。
\begin{itemize}
    \item 新建连通分量:相接位置没有右插头和下插头,当前格子有右插头和下插头,将对应位置
    置为新的标号,然后重新$O(n)$扫描以保持最小表示法。
    \item 合并连通分量:当前格子有左插头和上插头。若对应的右插头和下插头未连通,则将含有
    这两个标号的所有位置标记为同一标号,并重新扫描,再将对应位置0。若其已连通,则只允许在
    最后一个非障碍格子中合并为一条回路。
    \item 保持连通分量:相接位置有右插头或下插头恰好一个,当前格子有上或左
    插头恰好一个,有下或右插头恰好一个。不需要重新扫描,可以$O(1)$转移。
\end{itemize}
\paragraph{障碍格子处理}
当遇到障碍格子时,相接位置必须没有插头,由于该格子不能铺线,对应位置上没有插头,仍然置为0,
所以状态不变。
\paragraph{跨行处理}
当从上一行的最后一个格子转移到下一行的第一个格子时,不能从有右插头的状态转移。实现时
保证上一行的最后一个格子不能产生右插头,转移到首格时移位处理(将高位的竖线删除,再添加新的
无插头竖线到低位,实际操作只要左移一个位置。{\bfseries 如果更高位还有存储数据,要注意保留。})。
\paragraph{始末状态}
初始状态和终止状态都没有插头,值为0。
\paragraph{小结}
在推导状态转移时需要注意以下方面:
\begin{itemize}
    \item 状态表示是否存储足够的信息
    \item 连通分量的三种变化情况
    \item 移动格子时要保证所有裸露的插头都在轮廓线上
    \item 转移的目标状态表示的是{\bfseries 轮廓线上插头}的连通性。
    \item 转移完状态后是否需要重新扫描以保持最小表示法的性质
    \item 障碍格子和行首尾格子的处理
    \item 始末状态
    \item 从可行性与最优性对无效状态进行剪枝
\end{itemize}
\subsubsection{程序的实现与优化}
直接枚举状态会产生大量的无效状态,因此使用队列从初始状态开始转移。枚举每个格子,
枚举循环队列中的状态,计算出可以转移的目标状态。使用Hash表存储当前已经转移过的状态的
dp值,Hash表的size不必太大,每转移一个格子开一个新的Hash表。

参考代码:
\lstinputlisting{Source/Templates/Link.cpp}
\subsubsection{简单回路问题与括号表示法}
事实上对于简单回路问题,裸露的插头必须在轮廓线上,轮廓线上的插头必定两两匹配。
又因为求的是简单回路,匹配的插头不会交叉。那么可以使用一个括号序列来表示连通性,
0表示没有插头,1表示左端,2表示右端,使用4进制表示状态。

再次根据连通分量的变化讨论转移:
\begin{itemize}
    \item 新建连通分量:要求相接位置没有插头,转移时将对应位分别置为1和2。
    \item 合并连通分量:需要按照相接位置的插头是左括号还是右括号讨论,记
    右插头为$A$,下插头为$B$:
    \begin{itemize}
        \item $A$左$B$左:将$B$对应的右括号修改为左括号
        \item $A$左$B$右:此时将连成一条回路,当其为最后一个无障碍格子时才转移
        \item $A$右$B$左:不需要额外修改
        \item $A$右$B$右:将$A$对应的左括号修改为右括号
    \end{itemize}
    最后将对应位都置为0。
    \item 保持连通分量:直接继承有插头位置的左右标号
\end{itemize}

此法思维难度低,实现简单,程序速度快。但是括号表示法的局限性较大,参见下文的广义表示法。

参考代码:
\lstinputlisting{Source/Source/Link/5056.cpp}
\subsubsection{非回路问题转化为简单回路问题}
求从棋盘中一个特殊点经过所有非障碍点走到另一个特殊点的方案数。

可以尝试额外构造一条宽度为1的路径使其连通,新棋盘的一条回路对应了原棋盘的一条路径。
\subsection{简单路径问题}
给定一个棋盘,某些格子不能经过,其余格子必须经过,求有多少条简单路径。

此时非轮廓线上有不超过2个裸露插头。若使用最小表示法,则还要记录每个插头与路径端点的连通情况。
若使用括号表示法,则再引入标号3指示独立插头,说明这个插头连接着路径的一端,转移时需要保证任意
时刻轮廓线上的独立插头不超过2个。
\subsection{最大化回路/路径/连通块点权和}
\begin{itemize}
    \item 回路:注意仅在连为回路的情况更新答案。由于允许有些格子不经过,会导致回路外还可能
    存在一些孤立的路径,需要特判连为回路后轮廓线上是否均没有插头。
    \item 路径:额外使用2bit记录轮廓线上方裸露插头的个数,同时引入独立插头标记。
    注意要始终保证裸露插头的个数$\leq 2$。读入时要顺便计算只选取一个格子的方案。
    \item 连通块:仅记录$m$个格子的连通性。特别考虑移动轮廓线后消失的格子(即当前格子的上方)
    ,若它被选中且当前格子不选,需要保证它至少与轮廓线上的其它格子连通。
\end{itemize}

由于要求至少选择一个格子/回路,要在计算的过程中更新答案。必要时单独计算选取一个格子的特殊情况。
已经形成回路/路径的方案,不必加入下一次迭代。
\subsection{广义括号表示法}
括号表示法的局限性在于其只能表示在轮廓线上最多有2个插头的连通分量。

将最左插头标记为``('',最右插头标记为``)'',中间的插头标记为``)('',独立插头标记为``()'',
能够匹配的括号对应的插头是连通的。
\subsection{棋盘染色问题}
棋盘上格子的连通性取决于棋盘格子的颜色,因此需要额外记录轮廓线上棋盘格子的颜色。
\subsection{局部连通性加速DP}
有些题目会给出一个特殊的构图方法,其连边具有局部性质。那么就可以讨论局部连通性计算转移矩阵,
使用矩阵快速幂加速DP。

上述内容参考了IOI2018国家集训队论文集中陈丹琦的《基于连通性状态压缩的动态规划问题》。

\section{单调队列优化}
当状态转移方程为如下形式时
\begin{displaymath}
    dp[i][j]=max(dp[i-1][j-k]+w(i,j,k))+c(i,j),l \leq k \leq r
\end{displaymath}
可以使用单调队列优化。

该状态转移方程满足如下性质:
\begin{itemize}
	\item 转移区间是连续的。
	\item 当前转移位置到转移区间有距离限制。
\end{itemize}

对每个$i$进行dp的步骤如下:
\begin{enumerate}
	\item 初始化空队列;
	\item 计算自己到队首的距离,弹出超出转移距离的队首;
	\item 计算新进入转移区间的转移点的权(即$max$的内容);
	\item 为了维护队列的单调性,不断地从队尾弹出不比当前转移点更优的旧转移点
	      (即使同样优也要弹出,因为当前转移点肯定比旧转移点更晚弹出);
	\item 此时队首为最优值,加上常数后即为dp最优值。
\end{enumerate}

\section{斜率优化}
\subsection{推导}
当状态转移方程为如下形式时:
\begin{displaymath}
    dp[i]=min(a[i]*b[j]+c[i]+d[j])
\end{displaymath}
使用斜率优化。

以下推导假设$a[i]$递减且$b[j]$递增。

设决策点$j<k<i$,且从点$k$转移到$i$优于点$j$,
易证从点$k$转移到$i+1$同样优于点$j$,这就是决策单调性。

接下来考虑点$k$比点$j$更优的条件:
\begin{eqnarray*}
    a[i]*b[j]+c[i]+d[j]&\geq&a[i]*b[k]+c[i]+d[k]\\
    \Leftrightarrow -a[i]&\geq&\frac{d[j]-d[k]}{b[j]-b[k]}
\end{eqnarray*}

记斜率
\begin{displaymath}
    slope(i,j)=\frac{d[i]-d[j]}{b[i]-b[j]}
\end{displaymath}

可以发现使用单调队列维护,记$q[b]$为队首,$q[e-1]$为队尾:
\begin{itemize}
    \item 若$-a[i]\geq slope(q[b],q[b+1])$,则表明$q[b+1]$比$q[b]$更优,
    弹出$q[b]$。
    \item 若$slope(q[e-2],q[e-1])\geq slope(q[e-1],i)$,则说明若
    $q[e-2]$被弹出,$q[e-1]$一定被弹出,所以$q[e-1]$无效,可以先弹出。
\end{itemize}

从``形''的角度理解,单调队列相当于维护了一个下凸壳。
\subsection{应用}
主要过程就是研究决策单调性满足的条件,然后选取适当的数据结构维护信息,快速dp。

实际应用中需注意以下几点:
\begin{itemize}
    \item 比较斜率时尽量使用乘法。
    \item 若斜率单调,使用单调队列,否则使用二分法(使用类似线性规划的思路找到切点)。
    \item 若横坐标单调,使用单调队列,否则使用平衡树维护凸壳。
\end{itemize}

以上内容参考了MashiroSky的博客\footnote{斜率优化学习笔记 - MashiroSky
    \url{https://www.cnblogs.com/MashiroSky/p/6009685.html}
}。

\section{四边形不等式优化}\label{Quad}
在解区间动规题时通常会推出以下状态转移方程:
\begin{displaymath}
    dp[i][j]=min(dp[i][k]+dp[k+1][j])+w[i][j],i\leq k <j
\end{displaymath}
此时可考虑使用四边形不等式将$O(n^3)$优化到$O(n^2)$。

接下来定义两个性质:
\paragraph{区间包含的单调性}
对于$a\leq b\leq c\leq d$,有$f(b,c)\leq f(a,d)$。
\paragraph{四边形不等式}
对于$a\leq b\leq c\leq d$,有$f(a,d)+f(b,c)\geq f(a,c)+f(b,d)$。

\begin{theorem}
    若$w$函数满足上述两个性质,则$dp$函数满足四边形不等式性质。
\end{theorem}

\begin{theorem}
    若$dp$函数满足四边形不等式,设$s(i,j)$为$dp[i][j]$的转移点$k$,有
    $s(i,j)\leq s(i,j+1)\leq s(i+1,j+1)$。
\end{theorem}

根据该定理可以缩小转移点的搜索范围:
\begin{displaymath}
    dp[i][j]=min(dp[i][k]+dp[k+1][j])+w[i][j],s[i][j-1]\leq k \leq s[i+1][j]
\end{displaymath}

\paragraph{复杂度分析} 对于每一个左端点,最多扫一遍右边所有点,
因此时间复杂度降为$O(n^2)$。

\paragraph{优化} 计算$w[i][j]$时要考虑区间统计方面的优化(如前缀和)。

以上内容参考了XDU\_Skyline的博客\footnote{动态规划专题小结:四边形不等式优化
    \url{https://blog.csdn.net/u014800748/article/details/45750737}
}。

\section{矩阵快速幂优化}
\subsection{常规矩阵快速幂}
若dp状态转移方程满足如下形式:
\begin{displaymath}
    dp[i]=\sum_{j=1}^k{c_idp[i-j]}
\end{displaymath}
或对于图满足如下形式:
\begin{displaymath}
    dp[d][i][j]=\sum_{(i,k)\in E,(k,j)\in E}{dp[d-1][i][k]\cdot dp[d-1][k][j]}
\end{displaymath}
则可以使用矩阵快速幂优化。

计算$k*k$的转移矩阵$A$,dp初始值为$1*k$的向量$v_0$。
构造$A$,使其每乘一次$A$,向量表示的区间后移一格,那么
$A[i][j]$表示其在做一次乘法后将第$i$点的值贡献到第$j$点中的权值。

矩阵乘法满足结合律,因此可以使用矩阵快速幂进行计算。

\subsection{矩阵链乘因式分解}

若大小为$n*n$的矩阵$A$可表示为大小为$n*k$的矩阵$B$与大小为
$k*n$的矩阵$C$的乘积,其中$k\ll n$。
那么可以将$A$的幂拆开,错位结合,计算$k*k$的矩阵$D=CB$,对$D$快速幂后
计算需要的值{\bfseries (答案向量为$v_0AD^{c-1}B$,尽量按需计算结果)}。

\paragraph{例题} bzoj3583 杰杰的女性朋友

使用上述方法优化矩阵快速幂的效率。此外还存在一个问题,矩阵$A$的$k$次幂求的是走$k$
步的方案数的转移矩阵,但是答案要的矩阵为矩阵幂求和。因此我们可以对于每次询问再加一个
累加计数器,自己向自己连边,对应点向自己连边,最后单独求出起点到自己的方案数。

\subsection{dp步伐不一致时的解决方案}
例题 LOJ\#2180. 「BJOI2017」魔法咒语

如果单次转移最多需要跳跃$k$步($k$为小常数),可以给每个状态$S_0$引入$k-1$个``延迟状态''
$S_i$,若有状态$S$跳跃$k$步到达状态$T$,则实际转移为
$S\rightarrow T_{k-1} \rightarrow T_{k-2} \rightarrow \cdots \rightarrow T_0$。
注意后面的转移链是固定的,可以单独预处理。而第一个转移与原来的处理方式相同,只要根据跳跃
步数计算需要延迟转移的时间,然后连到链上对应的节点。统计时仍然只统计$S_0$,因为延迟状态
的贡献是不完整的。

\section{动态dp}
动态dp与常规dp的区别就是加上了多次修改与询问。

序列DDP一般使用线段树维护区间信息,这里不详述。下文的重点是树上DDP。
\subsection{常规树上DDP方法}
树上DDP一般使用线段树+树链剖分+矩阵乘法。

考虑动态最大带权独立集问题:
\subsubsection{描述dp转移}
记$f_u$为不选点$u$的子树最大带权独立集,
$g_u$为选点$v$的子树最大带权独立集。

那么对于点$u$的子树$T_u$有
\begin{eqnarray*}
    f_u&=&\sum_{(u,v)\in T_u}{max(f_v,g_v)}\\
    g_u&=&w_u+\sum_{(u,v)\in T_u}{f_v}
\end{eqnarray*}

类似于算法导论\cite{ITA3}中解决所有节点对最短路径问题时
介绍的类矩阵乘法,这里把对单个儿子$v$的转移视为左乘一个由点$u$当前dp状态决定
的转移矩阵,把$(f_v,g_v)$视作向量。其中矩阵乘法的定义需要修改:乘法变为加法,
加法变为取max,这个新的乘法操作仍然满足结合律。单位阵$I_n$的主对角线上为0,其余
元素为$-\infty$。

所以转移式如下:
\begin{displaymath}
    \left[
    \begin{array}{cc}
        f_u&f_u\\
        g_u&-\infty
    \end{array}\right]
    \left[
    \begin{array}{c}
        f_v\\g_v
    \end{array}\right]  =
    \left[
    \begin{array}{c}
        f'_u\\g'_u
    \end{array}\right]
\end{displaymath}

考虑一条链的情况,链头的dp向量即为链上所有点按顺序组成的矩阵链之积(不是向量?视作
在链尾后再加一个空点,即向量{\bfseries 0})。
每个点的矩阵均为考虑自身权值与分支后的转移矩阵。

此时已经将dp转移转化为矩阵乘法的形式。

\paragraph{注意事项}
\begin{itemize}
    \item 「NOIP2018」保卫王国:构造出矩阵后转移矩阵内一定有一列与向量的表示相同,
    取dp值时从这里取。更准确地说,要乘上末尾的哨兵向量观察取值。
    \item 「SDOI2017」切树游戏:在构造矩阵时不要吝啬矩阵大小,可以加入一些常量辅助
    转移,然后模拟多次矩阵相乘,找出不变量并省略(尤其是常量所在位置),以节省时间和空间。
\end{itemize}

\subsubsection{修改与查询}
暴力算法每一次都要维护从修改点到根的点权,转化为矩阵乘法后也不例外。
考虑对整棵树进行树链剖分,暴力跳重链维护轻儿子对父亲的贡献,查询时线段树
查询区间矩阵乘积,把自己所在重链的后代的贡献施加到自身矩阵,计算出真实的转移矩阵。
注意跳重链时根据树链剖分的性质,该操作不超过$O(\lg n)$次,因此每次修改的复杂度
为$O(\lg^2 n)$。

代码如下:
\lstinputlisting{Source/Templates/TDDP.cpp}

为了保证更新轻儿子对父亲矩阵的贡献的复杂度,不能重新对所有轻儿子dp,
而应该把自己未修改前的{\bfseries 真实转移矩阵}对父亲转移矩阵的贡献扣去,
然后施加变换更新线段树,重新计算自身真实转移矩阵然后修改父亲的矩阵。所以对
矩阵的修改需要``延迟更新''。

对于其它动态树形dp问题,其关键是将dp转化为矩阵乘法的形式,然后
套树链剖分+线段树解决。

上述内容参考了小蒟蒻yyb的博客\footnote{
    动态dp
    \url{http://www.cnblogs.com/cjyyb/p/10031947.html}
}。
\subsection{全局平衡二叉树}
这个黑科技出自2007年Yang Zhe的论文《SPOJ375 QTREE解法的一些研究》\cite{GBT}。

考虑把树链剖分+线段树换成LCT,直接维护dp值可以把查询时间复杂度降到
$O(m\lg n)$。但是LCT的常数太大,其表现不如树链剖分+线段树。事实上
我们并不需要LCT的link-cut功能,因此可以考虑把树静态化,即构造一个
能够暴力向上更新的数据结构。

全局平衡二叉树就能满足这一要求。建树过程很简单:
\begin{enumerate}
    \item 树链剖分求出重儿子;
    \item 给每个节点附上权值,权值为轻儿子子树大小之和+1。
    \item 对于每条重链找整条链的带权重心,把重心当做bst的根,
    然后递归两边继续找带权重心建bst。
\end{enumerate}

暴力更新即自底向上沿着bst和虚边更新。注意我们只需要维护每棵bst的先序遍历矩阵积,
对应每条重链的矩阵积。可以发现无论经过的是bst的边还是虚边,子树大小至少增大1倍(经过
bst的情况可类比点分治性质~\ref{WPPA})。所以树高为$O(\lg n)$,查询复杂度降为
$O(m\lg n)$。

模板:
\lstinputlisting{Source/Templates/GBT.cpp}

这种写法算法速度快、代码简单、易于理解,推荐使用此法。

上述内容参考了shadowice1984的博客\footnote{
    题解 P4643 【【模板】动态dp】 - 某菜鸡的blog\\
    \url{https://www.luogu.org/blog/ShadowassIIXVIIIIV/solution-p4643}
}。

\subsection{子树DP值查询}
如果需要向bzoj4712 洪水那样查询某个子树内的dp值,不能直接使用子树根节点的信息。
因为整条链是一棵二叉树,这个根节点的矩阵积还包括其左儿子的贡献,但这一部分并不是
它的子树。实际答案应该为自身及其后继的积,实现时只有往右移才算入这个祖先的贡献。

参考代码:
\begin{lstlisting}
Mat res;
bool init = false;
int last = T[x].l;
while(true) {
    if(T[x].l == last) {
        if(init)
            res = res * T[x].mat;
        else
            res = T[x].mat, init = true;
        if(T[x].r)
            res = res * T[T[x].r].mul;
    }
    last = x;
    if(isRoot(x))
        break;
    else
        x = T[x].p;
}
\end{lstlisting}
\subsection{固定思路}
\begin{enumerate}
    \item 转移方式:推导dp转移方程,确定最少需要的信息,组成向量
    \item 矩阵表达:将父亲$u$转移儿子$v$的信息写成矩阵左乘向量的形式,矩阵与$u$相关,向量
    与$v$相关。若转移无法被表达,则尝试给向量增加新的维度(而不是让矩阵变为非方阵),
    新维度的信息用常量填充($0,1,\pm \infty$)。
    \item 确定有效项:模拟多次矩阵乘法,直至不变量的个数稳定(即这个由变量组成的多元组满足
    乘法封闭)。省去不变量,硬编码矩阵乘法。对于相等项可以只保留一份
    ({\bfseries 必须保证正确性})。
    \item 结果读取:构造链尾哨兵,右乘转移矩阵(使用已确定有效项的矩阵而不是最初的转移矩阵),
    根据向量元素的实际意义得到结果。
    \item 轻儿子DP值转移到重链上的方式:最好满足可减性(最好不要是乘法)。注意+ with
    max/min意义下乘法时不要为了使用int而给+操作做clamp,因为在替换时需要做信息减法,
    结果是错误的。对于不满足信息可减性的,与LCT维护子树信息的做法一样,给每个点开一个
    数据结构维护轻儿子的信息和。

    Update:由于每个点最多只给一个父亲贡献自己作为轻儿子的信息,可以将每个节点的所有轻儿子
    平铺到序列上,记录其区间,然后使用线段树维护。
    \item 单位阵构造(Optional,用于子树DP值查询):按照方程模拟确定$I$的每一项,如果
    不存在单位阵,乘法时记录是否已初始化的flag。
\end{enumerate}

\section{决策单调性DP}
斜率优化与四边形不等式是决策单调性DP的特殊情形,参见第~\ref{Slope}节与
第~\ref{Quad}节。

在分析、实现、Debug难度、错误率方面,分治DP<决策二分栈DP<斜率优化DP。在题目条件与
运行时间允许的情况下,优先使用分治DP。
\subsection{双指针优化}
对于dp方程为$f_i=max/min\{w_{j,i}\}$的问题,记$f_i$的转移决策点为$P_i$,
若对于任意$i<j$,满足$P_i<P_j$,且对于任意$i$的可行转移点集合$P$,有
$P_i=P_{min}$,那么可以维护2个指针,一个指示$i$,一个指示转移决策点,
当不在可行转移点集合时才移动决策点,时间复杂度$O(n)$。
\subsection{决策二分栈DP}
该方法适用于dp方程为$f_i=max/min\{g_j+w_{j,i}\}$,其中$g_j$是一些与$j$有关的函数,
可能包含$f_j$,且对于任意$i<j$,满足$P_i<P_j$,$w_{j,i}$可以快速算出。

由于这类题无法贪心,必须存储决策转移点。不过注意到$i$对应的决策转移点随着$i$的递增而递增,
反过来每个决策转移点映射了一段$i$的区间。那么可以维护一个单调栈/队列,内部元素均为当前
有效决策转移点(还未计算转移的位置落在它们的管辖区间内)。转移点之间维护一个分界点,使用二分
计算(若函数较简单也可以直接求交点,还可以用牛顿迭代法)。添加新转移点时计算新点与栈顶/队尾的
转移点的分界点,若产生交叉则将其弹出。计算转移决策点时也使用二分计算。时间复杂度$O(n\lg n)$。

在二分决策点$x<y$的分界点时,左端点可设为$x$而不是1,因为$f_i$只可能从$j\leq i$的位置
转移。{\bfseries 注意对于连续决策函数,即使其DP值需要取整,二分分界点时也要使用浮点计算。}

\paragraph{例题} [NOI2009]诗人小G

参考代码:
\lstinputlisting{Source/Source/DP/1912.cpp}

上述内容参考了FlashHu的博客\footnote{
    DP的各种优化(动态规划,决策单调性,斜率优化,带权二分,单调栈,单调队列)\\
    \url{https://www.cnblogs.com/flashhu/p/9480669.html}
}。
\subsection{分治DP}
该方法适用于$f_i$之间独立且决策二分栈DP中$w_{j,i}$无法快速计算或不好写的情况。

例题:「雅礼集训 2017 Day5」珠宝

这是一个经典的背包问题,但是按照背包做会TLE。发现代价$c$很小,考虑按照$c$分类
然后转移,花费$kc$的代价时贪心地选取价值前$k$大的物品(我的思路止步于此)。

由于同一条dp依赖链上的位置模$c$同余,可以考虑再按照位置模$c$分类,将每一条链拆开处理。
注意到选取前$k$大物品的代价的增长率是单调非增的,尝试验证该dp是否有决策单调性。

采用反证法,设两个同类dp点$i,j$,满足$i<j$,记它们的转移决策点分别为$P_i,P_j$,
满足$P_i>P_j$。记原先的dp数组为$dp$,前$k$大前缀和数组为$w$。根据决策点的定义,有:
\begin{eqnarray*}
    dp[P_i]+w[i-P_i]>dp[P_j]+W[i-P_j]\\
    \Rightarrow dp[P_i]-dp[P_j]>W[i-P_j]-w[i-P_i]\\
    dp[P_i]+w[j-P_i]\leq dp[P_j]+W[j-P_j]\\
    \Rightarrow dp[P_i]-dp[P_j]\leq W[j-P_j]-w[j-P_i]\\
\end{eqnarray*}

由不等式传递性可得$W[i-P_j]-w[i-P_i]<W[j-P_j]-w[j-P_i]$,由于$i<j$且增长率单调非增,
与该式产生矛盾,因此转移点是单调的。

那么可以写一个分治程序$solve(l,r,L,R)$,表示处理$[l,r]$之间的dp值,转移区间在$L,R$。
每次在转移区间内扫一遍求出$mid$的dp值,得到转移点,最后左右递归处理,时间复杂度$O(ck\lg k)$。

代码:
\lstinputlisting{Source/Source/DP/LOJ6039.cpp}

上述内容参考了ShichengXiao的博客\footnote{
    DP及其优化\\
    \url{https://www.cnblogs.com/ShichengXiao/p/9501386.html}
}。

注意$w_{j,i}$无法快速算出但是可以快速转移(类似莫队)时,要预处理一部分区间以保证
分治复杂度。{\bfseries 准确地说,分治复杂度要严格与决策区间长度+处理区间长度相关。
单层分治的复杂度表达式不能是不同类端点之差的形式。}

例题:CF868F Yet Another Minimization Problem

推出DP方程,滚动第一维,发现转移具有决策单调性,并且无法快速计算$w_{j,i}$,考虑使用分治
解决。记$solve(l,r,b,e)$表示计算$[l,r]$的dp值,决策区间在$[b,e]$,记$m=(l+r)/2$。
原做法:记$end=min(e,m)$,暴力加入$(end,m]$内的元素,然后从$end$到$n$边计算dp边添加
新元素。复杂度不保的地方在于每次都暴力加入$(end,m]$内的元素。

正确姿势:注意到$b<l$,向左递归时都要用到区间$[b,l)$,考虑每次调用$solve$前都已加入
$[b,l)$的元素。接下来分析3个子过程,记单层分治复杂度中与处理区间长度相关的项为$A$,
与决策区间长度相关的项为$B$:

\begin{itemize}
    \item 寻找$m$的决策点$p$:该过程需要用到区间$[b,m]$,首先再加入$[l,m]$的元素,
    然后从$b$开始移动向$end$缩减区间计算dp,最后还原区间$[b,end]$,删除$[l,m]$。
    时间复杂度$A+B+B+A$。
    \item 准备左递归要预处理的区间:预处理部分已就绪,直接递归。
    \item 准备右递归要预处理的区间:需要将区间$[b,l)$迁移至区间$[p,m)$,这个过程
    可视为左端点和右端点的移动,因此复杂度仍然是靠谱的。最后仍然要退回$[b,l)$给调用端
    用,时间复杂度$B+A+A+B$。
\end{itemize}

参考代码:
\lstinputlisting{Source/Source/DP/CF868F.cpp}

该内容参考了FlashHu的题解\footnote{
    洛谷CF868F Yet Another Minimization Problem(动态规划,决策单调性,分治)\\
    \url{https://www.cnblogs.com/flashhu/p/9495839.html}
}。
\subsection{决策单调性快速判断}
该内容参考了FlashHu的博客\footnote{
    不失一般性和快捷性地判定决策单调(洛谷P1912 [NOI2009]诗人小G)(动态规划,决策单调性,单调队列)\\
    \url{https://www.cnblogs.com/flashhu/p/9521094.html}
}。

定义决策函数$F_j(i)$为从$j$转移到$i$的dp值。

具有决策单调性的DP有以下特征:

\begin{itemize}
    \item 决策函数是直线(使用斜率优化)。
    \item 发现某些部分分决策函数是直线,可以使用斜率优化做。部分分有一定的提示作用。
    \item 两个决策函数只有一个交点。
    \item 决策函数的导函数单调:
    \begin{itemize}
        \item 导函数单调递增,求最大值/单调递减,求最小值:单调栈
        \item 导函数单调递增,求最小值/单调递减,求最大值:单调队列
    \end{itemize}
    \item 决策函数可表示为多个满足决策单调性的子决策函数之和。
    \item 输出决策点发现决策具有单调性。
    \item 带绝对值的决策函数拆成两种情况后均具有单调性。遇到关于下标之差绝对值的决策函数,
    拆成两次DP来做,每次强制单方向转移。
\end{itemize}

\section{凸包}
\index{C!Convex Hull}
\subsection{极角序凸包}
经典算法是Graham扫描法\index{G!Graham Scan}。
算法步骤如下:
\begin{itemize}
	\item 选择一个纵坐标最低的点(若有多个选横坐标最小)加入凸包,以此为
	      原点按极角对其他点排序;
	\item 按照极角序加入每一个节点,保持凸包相邻3个节点的凸性质,注意在
	      一条直线上时选择距离较远的点。
\end{itemize}
\subsection{水平序凸包}
极角序计算凸包容易由于$atan2$的精度问题而造成错误,并且不易处理共线问题
(始边要求从远到近,终边要求从近到远)。
考虑对横坐标进行排序,分别计算其凸包的上凸壳和下凸壳,最后合并即可。

代码如下(逆时针顺序):
\begin{lstlisting}
Vec P[size],C[size];
void convexHull(int n) {
    std::sort(P+1,P+n+1,[](const Vec& a,const Vec& b) {
        return a.x<b.x;
    });
    int top=1;
    C[1]=P[1];
    for(int i=2;i<=n;++i) {
        while(top>=2 && cross(C[top]-C[top-1],
            P[i]-C[top-1])<eps)
            --top;
        C[++top]=P[i];
    }
    for(int i=n-1;i>=1;--i) {
        while(top>=2 && cross(C[top]-C[top-1],
            P[i]-C[top-1])<eps)
            --top;
        C[++top]=P[i];
    }
}
\end{lstlisting}
\subsection{在线凸包}
在线凸包即每次向点集中加入新点,求当前凸包的某些信息。
经典思路是按照极角序(选择凸包内的定点作为基准点,因为凸包会越来越大,所以可以取前3个点的重心)
将凸包上的点存储在set上。加入点时在set上查询极角序相邻的点组成的边,判断加入该点后
这条边变为的两条边是否是凸的,如果是凸的继续更新左右两边,直至局部全为凸。注意跨x负半轴
的情况(极角为$-pi$与$+pi$附近的点相邻)。
\subsubsection{水平序在线凸包}
同理维护上下凸壳,根据dwjshift的博客\footnote{
	实现水平序动态凸包的小技巧 « dwjshift's Blog
	\url{http://dwjshift.logdown.com/posts/285072}
}所述,可以对在维护上凸壳时将其横纵坐标取负,只要考虑维护下凸壳。
\subsection{凸包矢量和}
已知点集$A,B$,求点集$C={P_1+P_2|P_1\in A \land P_2\in B}$的凸包。

显然该凸包与点集$A,B$凸包相加的凸包相同,容易想到预处理点集$A,B$的凸包后对
将每对凸包上的点之和加入集合做凸包。在点随机分布的情况下,这种方法的时间复杂度为$O(n \lg n)$,
因为凸包的期望规模为$\Theta(\lg n)$。但这种方法会被卡成$O(n^2 \lg n)$,当
所有点都在凸包上时。

考虑优化对两个凸包做加法的过程,容易发现按照一个方向构造新凸包时,在原凸包选择
点的顺序是单调的。所以使用双指针法维护当前选择的点,每次判断哪个凸包上的指针后移,
就能在$O(n)$复杂度内完成合并。注意在两个候选点在一条直线上时选择较远的点。
\subsection{凸包合并}
合并后的凸包有一部分是两个凸包上点的连边,一部分是凸包上一段链。
$P_i,Q_j$连边当且仅当:
\begin{itemize}
	\item $P_i$与$Q_j$是并踵点对;
	\item $P_{i-1},P_{i+1},Q_{i-1},Q_{i+1}$在$P_i-Q_j$的同一侧。
\end{itemize}
使用旋转卡壳法可在线性时间内合并凸包。
该方法参考了ACMaker的博客\footnote{
	旋转卡壳——合并凸包
	\url{https://blog.csdn.net/ACMaker/article/details/3561150}
}。
\subsection{稀疏包分布}
\begin{itemize}
	\item 若点在圆面上均匀分布,则凸包期望规模为$\Theta(n^{1/3})$。
	\item 若点在凸多边形内部取得,凸包期望规模为$\Theta(\lg n)$。
	\item 若点根据二维正态分布取得,凸包期望规模为$\Theta(\sqrt{\lg n})$。
\end{itemize}
该内容来自算法导论\cite{ITA3}思考题33-5。
\subsection{二维最小乘积生成树}
选择$n-1$条边使得图连通,最小化所选边第一权值和与第二权值和的乘积。

解法:将第一权值和当做横坐标,第二权值和当做纵坐标,每一棵生成树对应一个点。
首先分别求出第一权值和和第二权值和的MST,标记这两棵MST对应点$A,B$。那么最优解
肯定在以$A,B$为端点的下凸壳上,考虑计算到$AB$距离最远的点$C$,用它更新答案。
它肯定在凸壳上并且排除了大部分解。然后对$AC,CB$也递归计算即可。

接下来讨论如何计算最远点$C$:

首先由于$AB$长度固定,可以转化为求$S_{\triangle ABC}$的面积最大值。
使用叉积可得
\begin{eqnarray*}
	2S_{\triangle ABC}&=&cross(\overrightarrow{AC},\overrightarrow{AB})\\
	&=&(C.x-A.x)*(B.y-A.y)-(B.x-A.x)*(C.y-A.y)\\
	&=&C.x*(B.y-A.y)-C.y*(B.x-A.x)+Constant
\end{eqnarray*}
计算权值后可转化为计算最大生成树,为了简便可将其系数取反同样求MST。
递归结束的条件为找不到满足要求的点(即面积不为正)。

最小乘积最大匹配也使用类似做法。对于高维情况,将其改为求到超平面上最远距离。

上述内容参考了空灰冰魂的博客\footnote{
	【BZOJ2395】【Balkan 2011】Timeismoney 最小乘积生成树
	\url{https://blog.csdn.net/vmurder/article/details/46828379}
}。
\subsection{三维凸包}
这里使用$O(n^2)$增量法计算三维凸包。通过链表来维护面以便快速删除,以及
使用邻接矩阵维护点的有向连接对应的面的编号便于更新。

该算法的主要思想是检查并删除``可视面'',然后将未封闭的边缘与新点连边重新构成封闭立体。
\begin{itemize}
	\item 首先选择四个不共面的点,组成四面体,将面加入链表(注意顶点顺序要朝外,
	      即CCW顺序。可以计算重心到该面的有向体积,使其为负)。
	\item 对于其余点,不断加入凸包,枚举每一个面,若该面到新点的有向体积为正,则
	      删除原来的面,检查三条边的邻接面,直到有向体积为负,加入新面。
\end{itemize}

代码如下:
\lstinputlisting{Source/Unclassified/Done/4724.cpp}

该方法参考了\_sunshine的博客\footnote{
	hdu 4266 三维凸包(增量法)
	\url{https://www.cnblogs.com/-sunshine/archive/2012/08/25/2656794.html}
}。
\subsection{快速凸包}
\index{Q!Quick Hull}
高维凸包不易理解且使用增量法的$O(n^2)$复杂度较大,所以在此
介绍快速凸包算法。在平均情况下复杂度为$O(n\lg n)$,最坏情况$O(n^2)$,与
快速排序类似。

主要思想是每次选择到超平面的最远点,该点肯定在凸包上,并且该点与超平面将整个集合分割为$d+1$个子集,
其中该点与超平面组成的体内的点直接被排除,以达到快速求解的目的。

构造初始超平面时先选择一个点,再选择离这个点最远的点构成线,再选择离这条线最远的
点构成三角平面,以此类推。构造线也可以使用两个坐标为极值的点。还有一种随机化方法:
不断随机构造一个超平面,计算到这个超平面的有向距离的最大值与最小值,取出对应的点,
直到选出$d$个不重复的点。

二维凸包代码:
\lstinputlisting{Source/Templates/QuickHull2D.cpp}
三维凸包代码(不正确):
\lstinputlisting{Source/Templates/QuickHull3D.cpp}
计算表面积时四点共面的情况不太好处理。为了处理四点共面的情况,
可以给每个点的坐标值加一些``扰动''。

\index{*TODO!修复三维QuickHull板子}

上述内容参考了Wikipedia-EN\footnote{
    Quickhull - Wikipedia
    \url{https://en.wikipedia.org/wiki/Quickhull}
}。

\section{特殊DP}
\subsection{LCIS}
LCIS即Longest Common Increasing Subsequence,最长公共上升子序列。
注意这个问题不能用LCS+LIS解决。

输入长度为$n$的数组A和长度为$m$的数组B。
\subsubsection{常规dp}
记$dp[i][j]$为数组A的前$i$个元素与数组B的前$j$个元素中,以$B[j]$为结尾的最长
公共上升子序列长度。

接下来分类讨论状态转移方程:
\begin{itemize}
    \item $A[i]=B[j]$,则考虑将$A[i]$与$B[j]$匹配。显然转移位置的第一维为
    $i-1$,第二维只能枚举$B[k]<B[j]$转移。
    \item 若不匹配则直接继承$dp[i-1][j]$的值。
\end{itemize}

综上所述,状态转移方程为:
\begin{displaymath}
    dp[i][j]=\left\{\begin{array}{lr}
        dp[i-1][j]&\textrm{if~}A[i]\neq B[j]\\
        max\left\{dp[i-1][k]\right\}+1 (0\leq k <j \wedge B[k]<B[j])&
        \textrm{if~}A[i]=B[j]
    \end{array}\right.
\end{displaymath}
时间复杂度$O(nm^2)$。
\subsubsection{优化}
注意到第一层循环后$A[i]$是固定的,而枚举取$max$的情况仅在$A[i]=B[j]$时出现。
因此可以在第一层循环内维护一个最优转移值,当$A[i]>B[j]$时更新该值,当$A[i]=B[j]$时
用该值更新,时间复杂度$O(nm)$。如果不需要输出方案还可以滚动数组节省空间。

上述内容参考了ojnQ的博客\footnote{
    LCIS 最长公共上升子序列问题DP算法及优化
    \url{https://www.cnblogs.com/WArobot/p/7479431.html}
}。
\subsection{Dilworth定理}
\CJKsout{对于一个偏序集,最少链划分等于最长反链长度。}

通俗地讲,举个例子,把一个序列划分成最少的上升子序列数等于最长不升子序列长度。
\subsection{杨氏图表(排序矩阵)}
\index{Y!Young Tableau}
杨氏图表是一个二维表,若某个位置没有元素,则它的右边与下边没有元素;否则若它的
右边与下边有元素,其值比它自身大。

$1\cdots n$组成的杨氏图表个数为数列A000085,递推式为
$f(0)=f(1)=1,f(n)=f(n-1)+(n-1)f(n-2)$。

\subsubsection{钩长公式}
\index{H!Hook Length Formula}
该公式用来计算给定杨氏图表的形状的方案数。

定义某个位置的钩长$h_k$为它下边和它右边的格子数+1,则方案数为
$\frac{n!}{\displaystyle \prod_{k=1}^n{h_k}}$。
\subsubsection{作为数据结构}
\paragraph{查找元素x的位置}
从矩阵右上角开始寻找,保证元素$x$要么不存在,要么在当前元素的左下角。

算法步骤如下:
\begin{enumerate}
    \item 若当前元素$=x$,返回;
    \item 若当前元素$>x$,向左移动一格,因为下方的数都比自己大;
    \item 若当前元素$<x$,向下移动一格,因为左边的数都比自己小。
\end{enumerate}

记矩阵规模为$m*n$,时间复杂度$O(m+n)$。此时杨氏图表像一个平衡树,
每次选取一个key缩小搜索范围。
\paragraph{查找第k大值}
首先二分值找到大于矩阵内$k$个格子的值$x$,然后在矩阵中找最大的小于$x$的值。
计算小于指定数的格子数与找最大的小于指定数的格子均类似上述做法。时间复杂度
$O((m+n)\lg (m+n))$。

上述内容参考了acdreamers的博客\footnote{
    杨氏矩阵与钩子公式\\
    \url{https://blog.csdn.net/acdreamers/article/details/14549077}
}与Wikipedia-EN\footnote{
    Young tableau\\
    \url{https://en.wikipedia.org/wiki/Young\_tableau}
}。
\subsection{GarsiaWachs算法}
\index{G!Garsia–Wachs Algorithm}
该算法用来解决相邻石子合并问题,与该问题同构的还有最优二叉搜索树问题。
该问题有一个$O(n^2)$的朴素区间dp解法。

算法步骤如下:
\begin{enumerate}
    \item 从左往右找到第一个$k$,满足$w[k-1] \leq w[k+1]$,为了简化算法插入两个哨兵
    $w[0]=w[n+1]=\infty$。
    \item 合并$w[k-1]$与$w[k]$,记新节点的权值为$w_{new}$,向前寻找第一个大于$w_{new}$
    的位置$j$,将其插入$j$的后面。
    \item 迭代直至只剩一个节点。
\end{enumerate}

这个算法的复杂度仍然是$O(n^2)$的。注意每次找到$k$后,处理它只需要用到之前的信息。
那么可以考虑逐个插入元素,当前的$k-1$满足条件时迭代地维护序列。全部插入完毕后,由于
尾部的$\infty$,尾部的元素会不断被合并。可以发现在任意时刻,序列都是由一堆单调不增的子序列
构成,使用平衡树维护可以达到$O(n\lg n)$的复杂度(为什么没人写?难道是$O(n^2)$能过的原因吗?)。

\CJKsout{这里没有参考代码,我也不想写平衡树。}

上述内容参考了acdreamers的博客\footnote{
    石子合并(GarsiaWachs算法)\\
    \url{https://blog.csdn.net/acdreamers/article/details/18043897}
}。

\section{杂记}
\begin{itemize}
    \item 我原先写区间dp时首先枚举长度,然后枚举左端点:
    \begin{lstlisting}
    for(int len=1;len<=n;++len)
        for(int l=1,r=len;r<=n;++r)
    \end{lstlisting}

    这样子长度较小的区间先于长度较大的区间dp完毕。

    不过还有另一种写法:
    \begin{lstlisting}
    for(int l=n;l>=1;--l)
        for(int r=l;r<=n;++r)
    \end{lstlisting}

    这种区间枚举方式代码简短,在缓存友好方面倒是没有什么优势。
\end{itemize}
