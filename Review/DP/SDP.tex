\section{特殊DP}
\subsection{LCIS}
LCIS即Longest Common Increasing Subsequence,最长公共上升子序列。
注意这个问题不能用LCS+LIS解决。

输入长度为$n$的数组A和长度为$m$的数组B。
\subsubsection{常规dp}
记$dp[i][j]$为数组A的前$i$个元素与数组B的前$j$个元素中,以$B[j]$为结尾的最长
公共上升子序列长度。

接下来分类讨论状态转移方程:
\begin{itemize}
    \item $A[i]=B[j]$,则考虑将$A[i]$与$B[j]$匹配。显然转移位置的第一维为
    $i-1$,第二维只能枚举$B[k]<B[j]$转移。
    \item 若不匹配则直接继承$dp[i-1][j]$的值。
\end{itemize}

综上所述,状态转移方程为:
\begin{displaymath}
    dp[i][j]=\left\{\begin{array}{lr}
        dp[i-1][j]&\textrm{if~}A[i]\neq B[j]\\
        max\left\{dp[i-1][k]\right\}+1 (0\leq k <j \wedge B[k]<B[j])&
        \textrm{if~}A[i]=B[j]
    \end{array}\right.
\end{displaymath}
时间复杂度$O(nm^2)$。
\subsubsection{优化}
注意到第一层循环后$A[i]$是固定的,而枚举取$max$的情况仅在$A[i]=B[j]$时出现。
因此可以在第一层循环内维护一个最优转移值,当$A[i]>B[j]$时更新该值,当$A[i]=B[j]$时
用该值更新,时间复杂度$O(nm)$。如果不需要输出方案还可以滚动数组节省空间。

上述内容参考了ojnQ的博客\footnote{
    LCIS 最长公共上升子序列问题DP算法及优化
    \url{https://www.cnblogs.com/WArobot/p/7479431.html}
}。
\subsection{Dilworth定理}
\CJKsout{对于一个偏序集,最少链划分等于最长反链长度。}

通俗地讲,举个例子,把一个序列划分成最少的上升子序列数等于最长不升子序列长度。
\subsection{杨氏图表(排序矩阵)}
\index{Y!Young Tableau}
杨氏图表是一个二维表,若某个位置没有元素,则它的右边与下边没有元素;否则若它的
右边与下边有元素,其值比它自身大。

$1\cdots n$组成的杨氏图表个数为数列A000085,递推式为
$f(0)=f(1)=1,f(n)=f(n-1)+(n-1)f(n-2)$。

\subsubsection{钩长公式}
\index{H!Hook Length Formula}
该公式用来计算给定杨氏图表的形状的方案数。

定义某个位置的钩长$h_k$为它下边和它右边的格子数+1,则方案数为
$\frac{n!}{\displaystyle \prod_{k=1}^n{h_k}}$。
\subsubsection{作为数据结构}
\paragraph{查找元素x的位置}
从矩阵右上角开始寻找,保证元素$x$要么不存在,要么在当前元素的左下角。

算法步骤如下:
\begin{enumerate}
    \item 若当前元素$=x$,返回;
    \item 若当前元素$>x$,向左移动一格,因为下方的数都比自己大;
    \item 若当前元素$<x$,向下移动一格,因为左边的数都比自己小。
\end{enumerate}

记矩阵规模为$m*n$,时间复杂度$O(m+n)$。此时杨氏图表像一个平衡树,
每次选取一个key缩小搜索范围。
\paragraph{查找第k大值}
首先二分值找到大于矩阵内$k$个格子的值$x$,然后在矩阵中找最大的小于$x$的值。
计算小于指定数的格子数与找最大的小于指定数的格子均类似上述做法。时间复杂度
$O((m+n)\lg (m+n))$。

上述内容参考了acdreamers的博客\footnote{
    杨氏矩阵与钩子公式\\
    \url{https://blog.csdn.net/acdreamers/article/details/14549077}
}与Wikipedia-EN\footnote{
    Young tableau\\
    \url{https://en.wikipedia.org/wiki/Young\_tableau}
}。
\subsection{GarsiaWachs算法}
\index{G!Garsia–Wachs Algorithm}
该算法用来解决相邻石子合并问题,与该问题同构的还有最优二叉搜索树问题。
该问题有一个$O(n^2)$的朴素区间dp解法。

算法步骤如下:
\begin{enumerate}
    \item 从左往右找到第一个$k$,满足$w[k-1] \leq w[k+1]$,为了简化算法插入两个哨兵
    $w[0]=w[n+1]=\infty$。
    \item 合并$w[k-1]$与$w[k]$,记新节点的权值为$w_{new}$,向前寻找第一个大于$w_{new}$
    的位置$j$,将其插入$j$的后面。
    \item 迭代直至只剩一个节点。
\end{enumerate}

这个算法的复杂度仍然是$O(n^2)$的。注意每次找到$k$后,处理它只需要用到之前的信息。
那么可以考虑逐个插入元素,当前的$k-1$满足条件时迭代地维护序列。全部插入完毕后,由于
尾部的$\infty$,尾部的元素会不断被合并。可以发现在任意时刻,序列都是由一堆单调不增的子序列
构成,使用平衡树维护可以达到$O(n\lg n)$的复杂度(为什么没人写?难道是$O(n^2)$能过的原因吗?)。

\CJKsout{这里没有参考代码,我也不想写平衡树。}

上述内容参考了acdreamers的博客\footnote{
    石子合并(GarsiaWachs算法)\\
    \url{https://blog.csdn.net/acdreamers/article/details/18043897}
}。
\subsection{双单调性优化}
例题:[POI2010]KLO-Blocks

将该问题转化为找到最长的平均数$\geq k$的连续段,将每个数$-k$再做前缀和$s$后,
问题进一步转化为对于每个$r$,找到最小的$l$满足$s[r]\geq s[l]$。

注意对于两个决策点$l<r$,若$s[r]\geq s[l]$,则$r$一定不会被选择。因此可以维护
一个单调栈(没有距离限制,无法移动头部),$s$的值随着栈内元素下标的增加而下降。
dp时在单调栈中二分查找。时间复杂度$O(n\lg n)$。

注意对于两个查询点$l<r$,若$s[r]\geq s[l]$,$l$的决策点必定可转移$r$,并且
得到更优的答案,那么$l$就不用查询了。这时需要查询的点也组成一个单调栈,$s$的值
随着栈内元素下标的增加而下降(注意这两个栈不同,差别在于对$s[l]==s[r]$的处理,
决策栈保留前者,而查询栈保留后者)。

发现查询栈中的$s$是下降的,决策点单调右移,用双指针扫描支持$O(n)$查询。

该方法参考了kczno1的题解\footnote{
    题解 P3503 【[POI2010]KLO-Blocks】\\
    \url{https://www.luogu.org/blog/user9168/solution-p3503}
}。
