\section{背包优化}
\subsection{bitset优化}
有些简单的背包问题可以使用位运算转移,std::bitset可以减小常数。
\subsection{完全背包优化}
\subsubsection{排序筛选}
若一种物品比另一种物品代价更大,收益更低,直接排除。
\subsection{多重背包优化}
\subsubsection{完全背包转换}
若该物品的数量已经超过最大装载量,直接将其当做完全背包,单种物品$O(V)$转移。
\subsubsection{二进制优化}
将数量按$2^0,2^1,2^2,\cdots,2^k,rem$拆分为多个物品,然后做01背包,
单种物品$O(V\lg c)$转移。
\subsubsection{单调队列优化}
朴素的多重背包状态转移方程为:
\begin{eqnarray*}
    end&=&min(c[i],j/v[i])\\
    dp[i][j]&=&max(dp[i-1][j-k*v[i]]+k*w[i]),0\leq k \leq end
\end{eqnarray*}
令$a=j/v[i],b=j\%v[i]$,将方程转换为:
\begin{displaymath}
    dp[i][b+a*v[i]]=max(dp[i-1][b+k*v[i]]-k*w[i])+a*w[i],a-end\leq k \leq a
\end{displaymath}
此时$k$表示比$a$少取$k$件。

此时$max$部分只与$k$的取值有关,可以使用单调队列优化。
每次转移时,枚举$b$,对$k$做单调队列,根据转移区间的移动把队首弹出队列。
单种物品$O(V)$转移。

该方法参考了soloier的博客\footnote{单调队列优化多重背包\\
\url{https://blog.csdn.net/sinat\_34943123/article/details/52857327}}。

\subsubsection{例题}
Luogu P1776 宝物筛选\_NOI导刊2010提高(02)
\footnote{【P1776】宝物筛选\_NOI导刊2010提高(02) - 洛谷
\url{https://www.luogu.org/problemnew/show/P1776}}

使用了上述的完全背包转换和单调队列优化。

\lstinputlisting[title=Luogu 1776]{DP/MultiBag.cpp}
