\section{决策单调性DP}
斜率优化与四边形不等式是决策单调性DP的特殊情形,参见第~\ref{Slope}节与
第~\ref{Quad}节。
\subsection{双指针优化}
对于dp方程为$f_i=max/min\{w_{j,i}\}$的问题,记$f_i$的转移决策点为$P_i$,
若对于任意$i<j$,满足$P_i<P_j$,且对于任意$i$的可行转移点集合$P$,有
$P_i=P_{min}$,那么可以维护2个指针,一个指示$i$,一个指示转移决策点,
当不在可行转移点集合时才移动决策点,时间复杂度$O(n)$。
\subsection{决策二分栈DP}
该方法适用于dp方程为$f_i=max/min\{g_j+w_{j,i}\}$,其中$g_j$是一些与$j$有关的函数,
可能包含$f_j$,且对于任意$i<j$,满足$P_i<P_j$,$w_{j,i}$可以$O(1)$算出。

由于这类题无法贪心,必须存储决策转移点。不过注意到$i$对应的决策转移点随着$i$的递增而递增,
反过来每个决策转移点映射了一段$i$的区间。那么可以维护一个单调栈/队列,内部元素均为当前
有效决策转移点(还未计算转移的位置落在它们的管辖区间内)。转移点之间维护一个分界点,使用二分
计算。添加新转移点时计算新点与栈顶/队尾的转移点的分界点,若产生交叉则将其弹出。计算转移决策
点时也使用二分计算。时间复杂度$O(n\lg n)$。

在二分决策点$x<y$的分界点时,左端点可设为$x$而不是1,因为$f_i$只可能从$j\leq i$的位置
转移。

\paragraph{例题} [NOI2009]诗人小G

参考代码:
\lstinputlisting{Source/Source/DP/1912.cpp}

上述内容参考了FlashHu的博客\footnote{
    DP的各种优化(动态规划,决策单调性,斜率优化,带权二分,单调栈,单调队列)\\
    \url{https://www.cnblogs.com/flashhu/p/9480669.html}
}。
\subsection{分治DP}
该方法适用于$f_i$之间独立且决策二分栈DP中$w_{j,i}$无法快速计算的情况。

例题:「雅礼集训 2017 Day5」珠宝

这是一个经典的背包问题,但是按照背包做会TLE。发现代价$c$很小,考虑按照$c$分类
然后转移,花费$kc$的代价时贪心地选取价值前$k$大的物品(我的思路止步于此)。

由于同一条dp依赖链上的位置模$c$同余,可以考虑再按照位置模$c$分类,将每一条链拆开处理。
注意到选取前$k$大物品的代价的增长率是单调非增的,尝试验证该dp是否有决策单调性。

采用反证法,设两个同类dp点$i,j$,满足$i<j$,记它们的转移决策点分别为$P_i,P_j$,
满足$P_i>P_j$。记原先的dp数组为$dp$,前$k$大前缀和数组为$w$。根据决策点的定义,有:
\begin{eqnarray*}
    dp[P_i]+w[i-P_i]>dp[P_j]+W[i-P_j]\\
    \Rightarrow dp[P_i]-dp[P_j]>W[i-P_j]-w[i-P_i]\\
    dp[P_i]+w[j-P_i]\leq dp[P_j]+W[j-P_j]\\
    \Rightarrow dp[P_i]-dp[P_j]\leq W[j-P_j]-w[j-P_i]\\
\end{eqnarray*}

由不等式传递性可得$W[i-P_j]-w[i-P_i]<W[j-P_j]-w[j-P_i]$,由于$i<j$且增长率单调非增,
与该式产生矛盾,因此转移点是单调的。

那么可以写一个分治程序$solve(l,r,L,R)$,表示处理$[l,r]$之间的dp值,转移区间在$L,R$。
每次在转移区间内扫一遍求出$mid$的dp值,得到转移点,最后左右递归处理,时间复杂度$O(ck\lg k)$。

代码:
\lstinputlisting{Source/Source/DP/LOJ6039.cpp}

上述内容参考了ShichengXiao的博客\footnote{
    DP及其优化\\
    \url{https://www.cnblogs.com/ShichengXiao/p/9501386.html}
}。
\subsection{决策单调性快速判断}
该内容参考了FlashHu的博客\footnote{
    不失一般性和快捷性地判定决策单调(洛谷P1912 [NOI2009]诗人小G)(动态规划,决策单调性,单调队列)\\
    \url{https://www.cnblogs.com/flashhu/p/9521094.html}
}。

具有决策单调性的DP有以下特征:

\begin{itemize}
    \item 转移函数是直线(使用斜率优化)。
    \item 发现某些部分分转移函数是直线,可以使用斜率优化做。部分分有一定的提示作用。
    \item 两个决策函数(与决策点$j$相关的,以$i$为自变量的函数)只有一个交点。
    \item 决策函数的导函数单调:
    \begin{itemize}
        \item 导函数单调递增,求最大值/单调递减,求最小值:单调栈
        \item 导函数单调递增,求最小值/单调递减,求最大值:单调队列
    \end{itemize}
    \item 可表示为多个满足决策单调性的子决策函数之和。
\end{itemize}
