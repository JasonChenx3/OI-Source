\section{最小二乘逼近}
\index{L!Least Squares}
该方法用来确定基函数的系数以拟合曲线。

设有$m$个数据点$(x_1,y_1),(x_2,y_2),\cdots,(x_m,y_m)$,
$n$个基函数$f_1,f_2,\cdots,f_n$
记向量$y=(y_1,y_2.\cdots,y_m)$,基函数值矩阵$A$
为\begin{displaymath}
    A=\left[
    \begin{array}{cccc}
    f_1(x_1)&f_2(x_1)&\cdots&f_n(x_1)\\
    f_1(x_2)&f_2(x_2)&\cdots&f_n(x_2)\\
    \vdots&\vdots&\ddots&\vdots\\
    f_1(x_m)&f_2(x_m)&\cdots&f_n(x_m)
    \end{array}
    \right]
\end{displaymath}
设系数向量为$c$,则有误差向量$\eta=Ac-y$。

使用最小二乘的思想,令$\|\eta\|^2$最小,
即最小化
\begin{displaymath}
    \|Ac-y\|=\sum_{i=1}^m{\left(\sum_{j=1}^n{a_{ij}c_j}-y_i\right)}^2
\end{displaymath}

对向量$c$中的每个元素微分并让结果为0,
即对于单个元素$c_k$,有
\begin{displaymath}
    \frac{\ud \|\eta\|^2}{\ud c_k}=\sum_{i=1}^m{2\left(
        \sum_{j=1}^n{a_{ij}c_j-y_i}\right)a_{ik}}=0
\end{displaymath}
将$n$个等式结合在一起,发现这是个向量*矩阵的形式,可得新的矩阵方程$(Ac-y)^TA=0$,
记$A^+=(A^TA)^{-1}A^T$为矩阵$A$的伪逆,有$c=A^+y$。

以上内容参考了算法导论\cite{ITA3} 28.3节。
