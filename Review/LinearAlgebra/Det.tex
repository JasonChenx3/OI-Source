\section{行列式}
\subsection{定义与性质}
记$A_{[ij]}$为矩阵$A$去掉第$i$行第$j$列后的矩阵,$\tau(P)$为序列$P$的逆序对数
,$P=(p_1,p_2,\cdots,p_n)$为$1\ldots n$的排列。
那么$n*n$矩阵$A$的行列式定义为:
\begin{eqnarray}
	det(A)&=&\sum_{P=(p_1,p_2,\cdots,p_n)}
	{(-1)^{\tau(P)}\prod_{i=1}^n{a_{ip_i}}}\label{DetA}\\
	&=&\left\{
	\begin{array}{ll}
		a_{11}                                                    & \textrm{~if~$n=1$} \\
		\displaystyle \sum_{j=1}^n{(-1)^{1+j}a_{1j}det(A_{[1j]})} & \textrm{~if~$n>1$}
	\end{array}
	\right.\label{DetB}
\end{eqnarray}

式~\ref{DetA}可理解为选择$n$个不同行且不同列的元素,逆序对数为偶数的加,奇数的减。

式~\ref{DetB}是式~\ref{DetA}的递归形式。

记$A_{ij}=(-1)^{i+j}det(A_{[ij]})$为元素$a_{ij}$的{\bfseries 代数余子式}。

行列式拥有如下性质:
\begin{property}
	$det(A)=det(A^T)$
\end{property}

证明:根据式~\ref{DetA}可知,行和列是无关的,转置后$det(A)$不变。
\begin{property}
	若矩阵$A$的某一行/列为0,则$det(A)=0$。
\end{property}

证明:根据式~\ref{DetA}可知每个排列中矩阵元素必有一个0,所以$det(A)=0$。
\begin{property}
	矩阵$A$的任意一行/列乘以$\lambda$,$det(A)$乘以$\lambda$。
\end{property}

证明:根据式~\ref{DetA}可知每个排列中有一个矩阵元素乘以$\lambda$,则$det(A)$放大$\lambda$。
\begin{property}
    若矩阵$C$的第$i$行/列可分解为$c=a+b$的形式,
    则$det(C)$可分解为分别包含这两个向量,其余元素与矩阵$C$相同的矩阵的行列式之和。
\end{property}
\begin{inference}
	将矩阵$A$中某一行/列元素加$\lambda$倍到另一行/列上,$det(A)$不变。
\end{inference}

将$det(A')$拆分,有$det(A')=det(A)+det(A_{add})=det(A)$(因为$A_{add}$有线性相关向量)。
\begin{property}
	交换$A$任意两行/列,$det(A)$变号。
\end{property}

证明:交换会使逆序对数奇偶性改变(通过计算相邻交换步数可证明),$det(A)$变号。
\begin{property}
	det(AB)=det(A)det(B)
\end{property}
\begin{theorem}
	$det(A)=0\Leftrightarrow$矩阵$A$奇异
\end{theorem}

证明:$det(A)=0$意味着至少有2个系数向量是线性相关的,矩阵$A$是欠定方程组的系数矩阵,没有逆矩阵。

代数余子式有如下性质:
\begin{property}
    \begin{eqnarray*}
        det(A)&=&\sum_{j=1}^n{a_{ij}A_{ij}} ~i=1,2,\cdots,n\\
        &=&\sum_{i=1}^n{a_{ij}A_{ij}} ~j=1,2,\cdots,n\\
    \end{eqnarray*}
\end{property}
\begin{property}
    \begin{displaymath}
        \forall i!=j,\sum_{k=1}^n{a_{ik}A_{jk}}=0
    \end{displaymath}
\end{property}
\subsection{求行列式}
\begin{theorem}\label{LTAD}
	若矩阵$A$为上三角矩阵,则$det(A)$为主对角线上元素之积。
\end{theorem}

证明:唯一矩阵元素积可能非0的排列方案只有在主对角线上。

高斯消元后根据定理~\ref{LTAD}计算行列式,注意在初等变换操作中交换行时要变号
{\bfseries (尤其是在计算模意义下有向生成树计数时)}。

以上内容参考了算法导论~\cite{ITA3} 附录D 矩阵。

若矩阵比较特殊,有较多的非零项,可以选取0的个数最多的一行/列展开。
