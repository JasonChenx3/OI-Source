\section{LUP分解}
\subsection{基本原理}
要求解线性方程组$Ax=b$,对系数矩阵$A$进行LUP分解:
\begin{displaymath}
	PA=LU
\end{displaymath}
其中$P$为置换矩阵,$L$为下三角矩阵,$U$为上三角矩阵。

将$Ax=b$乘上$P$,得$PAx=Pb$,然后用$PA=LU$代换得$LUx=Pb$。
设$y=Ux$,有$Ly=Pb$,可以使用类似于~\ref{LSE}节$O(n^2)$求解$y$,
然后再次使用该方法求出$x$。

\subsection{LUP分解}
\subsubsection{LU分解}
考虑不会出现不需要换主元的情况(比如对称正定矩阵),即$P=I$。

运用矩阵代数将A分解:
\begin{eqnarray*}
	A&=&\left[\begin{array}{c|ccc}
			a_{11} & a_{12} & \cdots & a_{1n} \\
			\hline
			a_{21} & a_{22} & \cdots & a_{2n} \\
			\vdots & \vdots & \ddots & \vdots \\
			a_{n1} & a_{n2} & \cdots & a_{nn}
        \end{array} \right]\\
    &=&\left[\begin{array}{cc}
        a_{11}&w^T\\
        v&A'
    \end{array}\right]\\
    &=&\left[\begin{array}{cc}
        1&0\\
        v/a_{11}&I_{n-1}
    \end{array}\right]
    \left[\begin{array}{cc}
        a_{11}&w^T\\
        0&A'-vw^T/a_{11}
    \end{array}\right]\\
    &=&\left[\begin{array}{cc}
        1&0\\
        v/a_{11}&I_{n-1}
    \end{array}\right]
    \left[\begin{array}{cc}
        a_{11}&w^T\\
        0&L'U'
    \end{array}\right] \textrm{~(递归分解子矩阵)}\\
    &=&\left[\begin{array}{cc}
        1&0\\
        v/a_{11}&L'
    \end{array}\right]
    \left[\begin{array}{cc}
        a_{11}&w^T\\
        0&U'
    \end{array}\right]\textrm{~(左右矩阵分别为L,U)}\\
    &=&LU
\end{eqnarray*}

求出矩阵的外围部分与子矩阵后将子矩阵递归分解。

\subsubsection{LUP分解实现}
设换主元时把第1行(因为要递归分解)与第$k$行交换的置换矩阵为$Q$,则$QA$
可以进行LU分解,即\begin{displaymath}
    QA=\left[\begin{array}{cc}
        1&0\\
        v/a_{k1}&I_{n-1}
    \end{array}\right]
    \left[\begin{array}{cc}
        a_{k1}&w^T\\
        0&A'-vw^T/a_{k1}
    \end{array}\right]
\end{displaymath}
设子矩阵满足$P'(A'-vw^T/a_{k1})=L'U'$,与$Q$相乘得到置换矩阵$P$,即
\begin{displaymath}
    P=\left[\begin{array}{cc}
        1&0\\
        0&P'
    \end{array}\right]Q
\end{displaymath}
继续变换:
\begin{eqnarray*}
    PA&=&\left[\begin{array}{cc}
        1&0\\
        0&P'
    \end{array}\right]
    \left[\begin{array}{cc}
        1&0\\
        v/a_{k1}&I_{n-1}
    \end{array}\right]
    \left[\begin{array}{cc}
        a_{k1}&w^T\\
        0&A'-vw^T/a_{k1}
    \end{array}\right]\\
    &=&\left[\begin{array}{cc}
        1&0\\
        P'v/a_{k1}&P'
    \end{array}\right]
    \left[\begin{array}{cc}
        a_{k1}&w^T\\
        0&A'-vw^T/a_{k1}
    \end{array}\right]\\
    &=&\left[\begin{array}{cc}
        1&0\\
        P'v/a_{k1}&I_{n-1}
    \end{array}\right]
    \left[\begin{array}{cc}
        a_{k1}&w^T\\
        0&P'(A'-vw^T/a_{k1})
    \end{array}\right]\\
    &=&\left[\begin{array}{cc}
        1&0\\
        P'v/a_{k1}&I_{n-1}
    \end{array}\right]
    \left[\begin{array}{cc}
        a_{k1}&w^T\\
        0&L'U'
    \end{array}\right]\\
    &=&\left[\begin{array}{cc}
        1&0\\
        P'v/a_{k1}&L'
    \end{array}\right]
    \left[\begin{array}{cc}
        a_{k1}&w^T\\
        0&U'
    \end{array}\right]\\
    &=&LU
\end{eqnarray*}

\begin{lstlisting}
typedef double FT;
const FT eps=1e-8;
FT A[size][size],B[size];
bool LUP(int n){
    for(int i=1;i<=n;++i) {
        int x=i;
        for(int j=i+1;j<=n;++j)
            if(fabs(A[j][i])>fabs(A[x][i]))
                x=j;
        if(fabs(A[x][i])<eps)return false;
        if(i!=x){
            for(int j=1;j<=n;++j)
                std::swap(A[i][j],A[x][j]);
            std::swap(B[i],B[x]);
        }
        for(int j=i+1;j<=n;++j) {
            A[j][i]/=A[i][i];
            FT fac=A[j][i];
            for(int k=i+1;k<=n;++k)
                A[j][k]-=fac*A[i][k];
        }
    }
    return true;
}
\end{lstlisting}
{\bfseries 注意L与U同时覆盖于A数组上,即}
\begin{displaymath}
	a_{ij}=\left\{\begin{array}{ll}
		l_{ij} & \textrm{if $i>j$}     \\
		u_{ij} & \textrm{if $i\leq j$}
	\end{array}\right.
\end{displaymath}
\subsection{正向/反向替换}
原理同~\ref{LSE}节,代码如下:
\begin{lstlisting}
FT Y[size],X[size];
void solve(int n) {
    for(int i=1;i<=n;++i) {
        FT sum=B[i];
        for(int j=1;j<i;++j)
            sum-=A[i][j]*Y[j];
        Y[j]=sum;
    }
    for(int i=n;i>=1;--i) {
        FT sum=Y[i];
        for(int j=i+1;j<=n;++j)
            sum-=A[i][j]*X[j];
        X[i]=sum/A[i][i];
    }
}
\end{lstlisting}
\subsection{LUP分解求逆矩阵}
同~\ref{InvMatGauss}节所述,求解$n$次线性方程组得到逆矩阵。LUP分解的好处
在于只需维护置换矩阵$P$而不是维护变换矩阵$PI$(这两个矩阵的意义不同)。
\lstinputlisting[title=InvMatLUP]{Source/Templates/LUP.cpp}

以上内容参考了算法导论\cite{ITA3} 第28章 矩阵运算。
