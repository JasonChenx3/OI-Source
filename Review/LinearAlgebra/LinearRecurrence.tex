\section{常系数齐次线性递推}
\index{L!Linear Difference Equation}
给定系数$a_0,a_2,\cdots,a_k$,有一个信号${f_n}$,满足$k$阶齐次线性差分方程
$\displaystyle \sum_{i=0}^k{a_if_{n-i}}=0$对所有$n$成立。现在给定信号
${f_n}$中的连续$k$项,求信号的任意一项。

一般$a_0$取1,其余系数取反,那么有$\displaystyle f_n=\sum_{i=1}^k{a_if_{n-i}}$,
假设给定了$f_0,f_1,\cdots,f_{k-1}$的值,现在要求出$f_n$的值。

如果$k$足够小,那么很容易构造转移矩阵,记向量$F_i$表示$(f_{i+k-1},f_{i+k-2},\cdots,f_i)$,
很容易构造$k*k$的转移矩阵$A$,其中$i$向$i+1$转移系数为1,$i$向$1$转移系数为$a_i$。即
$A[i+1][i]=1,A[1][i]=a_i$,那么有$F_n=A^nF_0$。使用矩阵快速幂可得到
$O(k^3\lg n)$的算法。

注意原等式左右向量取第$k$项仍然成立,即$(F_n)_{[k]}=(A^nF_0)_{[k]}$。左边就是$f_n$,
右边就是以$A^n$的第$k$行为系数的$f_{0,\cdots,k-1}$的线性组合。记这些系数为
$c_{0,\cdots,k-1}$,同样将$f_{0,\cdots,k-1}$表示为$A^nF_0$的形式,有
$A^n=\displaystyle \sum_{i=0}^{k-1}{c_iA^i}$,记右式为矩阵$A$的多项式表达$R(M)$。
设存在矩阵多项式$F(M),G(M)$,满足$G(M)$的次数为$k$,$F(M)G(M)$的次数为$n$,
且$M^n=F(M)G(M)+R(M)$。根据Cayley-Hamilton定理,若$G(M)$为矩阵$A$的特征多项式,
其次数恰好为$k$且$G(A)=0$。由于$G(M)$的次数为$k$,$R(M)\equiv M^n\pmod{G(M)}$,
多项式取模可求出$R(M)$。并且由于$G(A)=0$,此时有$A^n=R(A)$。拿到$R(M)$后就可以直接$O(k)$
求值。

接下来讨论如何构造出矩阵$A$的特征多项式$G(M)$。根据定义有
$G(\lambda)=\textrm{det}(\lambda I_k-A)$,多项式高斯消元求行列式十分麻烦。
注意到矩阵$\lambda I_k-A$的特殊性,将第一行每一项的代数余子式求和,去除第一行第$i$
列后都会得到一个下三角矩阵,行列式值为对角线上元素之积,同时代数余子式的$(-1)^{i+1}$项
与子矩阵的$-1$项恰好抵消。综上所述,$G(\lambda)=\lambda^k-a_1\lambda^{k-1}-\cdots-a_k$。

因此该方法的性能瓶颈在多项式取模上,时间复杂度$O(k\lg k\lg n)$。引入$\lg n$的原因是
我们无法直接构造一个多项式$x^n$然后以$n$的规模做取模,由于该多项式较为简单,可以使用类似
模意义快速幂的方法做模$R(M)$意义下的多项式快速幂。当$k$较小时,可以考虑暴力取模,时间
复杂度$O(k^2\lg n)$。

好写的优化技巧:
\begin{itemize}
    \item 预处理出$G(x)$与$G_{rev}^{-1}(x)$的点值表达。
    \item 需要取模时才实例化取模。
\end{itemize}

参考代码:
\lstinputlisting{Source/Templates/LR.cpp}

上述内容参考了《线性代数及其应用》\cite{LAIA5}4.8节以及shadowice1984的博客
\footnote{
    题解 P4723 【【模板】线性递推】
    \url{https://www.luogu.org/blog/ShadowassIIXVIIIIV/solution-p4723}
}。
