\section{高斯消元}
\subsection{变换为上三角矩阵}
高斯消元的思路很简单:每次通过初等变换消去第$i$行第$i$列下面的项,
最终变换为一个上三角矩阵。
\begin{lstlisting}[title=gauss]
typedef double FT;
const FT eps=1e-8;
FT A[size][size];
bool gauss(int n) {
    for(int i=1;i<=n;++i) {
        int x=i;
        for(int j=i+1;j<=n;++j)
            if(fabs(A[j][i])>fabs(A[x][i]))
                x=j;
        if(fabs(A[x][i])<eps)return false;
        if(x!=i) {
            for(int j=i;j<=n;++j)
                std::swap(A[i][j],A[x][j]);
        }
        for(int j=i+1;j<=n;++j) {
            FT fac=A[j][i]/A[i][i];
            for(int k=i;k<=n;++k)
                A[j][k]-=fac*A[i][k];
        }
    }
    return true;
}
\end{lstlisting}
时间复杂度$O(n^3)$。
\subsubsection{精度}
对于实数运算,选择绝对值最大数作为主元,以减小误差。
\subsubsection{优化}
可以观察矩阵的特殊性来优化消元复杂度,对于方程间联系不大的情况使用
迭代解小规模线性方程组。
\subsection{求解线性方程组}\label{LSE}
即求解$Ax=b$的向量$x$。
注意到上三角矩阵的最后一行是一元方程,解出后倒数第二行还是是一元方程,因此可以不断逆推得到所有解。
注意对矩阵$A$的操作也要应用到向量$b$上(实质为两边同时乘上变换矩阵)。
时间复杂度$O(n^2)$。
\begin{lstlisting}
for(int i=n;i>=1;--i) {
    FT sum=B[i];
    for(int j=i+1;j<=n;++j)
        sum-=A[i][j]*X[j];
    X[i]=sum/A[i][i];
}
\end{lstlisting}
\subsection{求解逆矩阵}\label{InvMatGauss}
设将$A$变换为上三角矩阵的变换矩阵为$P$,求逆矩阵即求解方程$PAA^{-1}=PI$,
由矩阵乘法的定义可知,将$A^{-1}$与$PI$按列拆分,即得到$(PA)A_i^{-1}=(PI)_i$
的形式,按照求解线性方程组的方法解出逆矩阵的每一列。

由于要额外维护$PI$,所以常数较LUP分解大。

\subsubsection{板子}

【P4783】【模板】矩阵求逆 - 洛谷
\footnote{\url{https://www.luogu.org/problemnew/show/P4783}}
\lstinputlisting[title=InvMatGauss]{LinearAlgebra/InvMatGauss.cpp}
时间复杂度$O(n^3)$。

《线性代数及其应用》中提到使用高斯消元法求逆矩阵的另一种方法:将待求逆矩阵$A$与单位矩阵$I$并为
矩阵$[A~I]$,然后使用高斯消元法将$A$消为上三角矩阵,再进一步消为单位矩阵,右边的单位矩阵做同样
的初等行变换,最后的结果为$[I~A^{-1}]$。

证明:设变换矩阵为$P$,有$AP=I$,根据IMT可知$A^{-1}=P$,并且此时矩阵的右半部分为$IP=P$。
