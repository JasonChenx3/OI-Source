\section{带花树}
\index{B!Blossom Algorithm}
带花树主要用来解决无向图最大匹配与完美匹配问题。
\subsection{无向图最大匹配}
尝试使用匈牙利算法计算无向图最大匹配,即每次选取一个未匹配点,
以此为树根,DFS遍历出交错树。

定义交错树上的``奇点''为距树根距离为奇数的点,``偶点''为距树根
距离为偶数的点。

那么对于二分图有如下两种情况:
\begin{itemize}
    \item 奇点连到偶点:一定走匹配边。
    \item 偶点连到奇点:一定走未匹配边。
\end{itemize}

但是对于无向图来说有一点不同:
\begin{itemize}
    \item 奇点连到偶点:一定走匹配边。
    \item 偶点连到奇点:一定走未匹配边。
    \item 偶点连到偶点:一定走未匹配边。此时我们把交错树上的两个偶点的
    交错路径以及偶点到偶点的边形成的奇环称为``花'',树根称为``花托''。
\end{itemize}

考虑跨过偶点到偶点的边,可以发现原先的奇点可以变为偶点,然后可以走未匹配边延伸出
更多的交错路径。既然花上的点都可以是偶点,不妨将其直接缩为一个偶点,这就是``缩花''。
一朵花至少有3个点,因此最多缩花$\frac{|V|}{2}$次,这里使用并查集缩花。

其余步骤与匈牙利算法类似:

遍历未匹配点尝试建立交错树以寻找增广路径:

使用BFS遍历,队列仅维护偶点。
\begin{itemize}
    \item 若走到未访问点:
    \begin{itemize}
        \item 若该点已匹配:
        \item 若该点未匹配:找到增广路径,翻转链上边的匹配状态后返回true。
    \end{itemize}
    \item 若走到已访问点:
    \begin{itemize}
        \item 若该点为奇点:由于该点已访问,什么也不要做。
        \item 若该点为偶点:缩花,参数为两个偶点标号$u,v$。
    \end{itemize}
\end{itemize}

缩花:
\begin{enumerate}
    \item 使用DSU找出花托,即$LCA(u,v)$。
    \item 两遍调用缩花的一边并把奇点标记为偶点,加入队列。
\end{enumerate}

假设并查集复杂度为$O(1)$,算法时间复杂度$O(|V||E|)$。
二分图最大匹配中的Greedy Matching仍然适用。

模板:
\lstinputlisting{Source/Templates/Blossom.cpp}

\subsection{Micali-Vazirani Algorithm}
\index{M!Micali-Vazirani Algorithm}
该算法也用于求无向图最大匹配,时间复杂度$O(\sqrt{|V|}|E|)$。

由于找不到中文资料,暂时先坑着。
参考Silvio Micali与Vijay V. Vazirani的论文\cite{MMG}。
\subsection{近线性复杂度的随机匹配算法}
这里介绍的是2011年由Anant Jindal, Gazal Kochar, Manjish Pal
提出的基于马尔科夫链和Glauber Dynamics的随机算法,参见论文\cite{AGM}。

算法步骤如下(调用下述算法$10\lg n$次):
\begin{enumerate}
    \item 令$M_0$为任意匹配,$p=2^{|E|}$。
    \item 迭代$k=10|E|\lg|V|$次更新$M_i$。
    \begin{enumerate}
        \item 随机取边$e$;
        \item 令$M'$有$\frac{p}{1+p}$的概率为$M_i\cup \{e\}$,
        有$\frac{1}{1+p}$的概率为$M_i \backslash \{e\}$。
        \item 若$M'$为匹配则令$M_{i+1}=M'$,否则$M_{i+1}=M_i$。
    \end{enumerate}
    \item 输出$M_k$。
\end{enumerate}

时间复杂度$O(|E|\lg^2|V|)$,算法正确率为
$1-\left(\frac{169}{189}\right)^{10\lg n}$。

模板(不正确,待修复):

为了防止错过最优解,我还加入了备份操作,最坏时间复杂度$O(|V|^2)$。

\lstinputlisting{Source/Templates/AGM.cpp}

\index{*TODO!最大匹配随机算法实现}
\subsection{无向图最大权匹配}
似乎并没有比较短的实现。。。暂时弃疗。

\index{*TODO!无向图最大权匹配}

上述内容参考了江任捷的演算法筆記\footnote{
    演算法筆記 - Matching
    \url{http://www.csie.ntnu.edu.tw/\~u91029/Matching.html}
}、permui\footnote{
    一般图最大匹配-带花树算法
    \url{https://www.cnblogs.com/owenyu/p/6858508.html}
}和hcsoso\footnote{
    Maximum matching in general graphs
    \url{https://finiteplayground.wordpress.com/2011/07/15/
    maximum-matching-in-general-graphs/}
}的博客。
