\section{杂讲}
\subsection{竞赛图}
\index{T!Tournament}
竞赛图是一个无向完全图被定向后得到的图。
\begin{theorem}
	竞赛图缩点后是一条链。
\end{theorem}
\subsubsection{竞赛图判定}
竞赛图可以用来指示两两选手比赛的胜负,判定比分是否
合法即判定是否存在合法的竞赛图。
\index{L!Landau's Theorem}
\begin{theorem}[Landau's Theorem]
	对于一个有序的竞赛图度数序列/得分序列$0\leq s_1 \leq s_2 \leq \cdots \leq s_n$,
	有$\displaystyle \forall 1\leq k\leq n,\sum_{i=1}^k{s_i}\geq \binomial{k}{2}$
	,当$k=n$时等号必须成立。
\end{theorem}
对度数/胜利场数排序后逐个判断即可。
\subsection{最小平均值环}
对于一个有向图,找出平均值最小的环。

类似于分数规划的思想,对平均值进行二分,将所有边权减去二分值,
若存在负环则说明存在环的平均值小于该二分值,不断二分即可。
\subsection{平面图性质}
\index{P!Planar Graph}
以下仅讨论$V\geq 3$的情况:
\begin{property}
	$E\leq 3V-6$
\end{property}
\begin{property}
	$F\leq 2V-4$
\end{property}
使用这些性质可以限制边数以加速平面图判定。

上述内容参考了Wikipedia-EN\footnote{Planar graph - Wikipedia
	\url{https://en.wikipedia.org/wiki/Planar\_graph}
}
\subsection{拓扑排序判环}
若拓扑排序无法使得所有点都入队则说明存在环。
可以通过枚举点,将其度数-1取得删掉一条边的效果。
例如CF915D Almost Acyclic Graph:
\lstinputlisting{Source/Source/TopSort/CF915D.cpp}
\subsection{Lindström–Gessel–Viennot Lemma}
\index{L!Lindström–Gessel\\–Viennot Lemma}
给定一个DAG,以及$n$个起点$a_1,a_2,\cdots,a_n$和对应终点$b_1,b_2,\cdots,b_n$,
求这$n$条点不相交(包括终点)路径的方案数。

根据Lindström–Gessel–Viennot Lemma,记$e(a_i,b_j)$为$a_i\rightarrow b_j$的
路径方案数,答案为\begin{displaymath}
	det\left(\left[\begin{array}{cccc}
			e(a_1,b_1) & e(a_1,b_2) & \cdots & e(a_1,b_n) \\
			e(a_2,b_1) & e(a_2,b_2) & \cdots & e(a_2,b_n) \\
			\vdots     & \vdots     & \ddots & \vdots     \\
			e(a_n,b_1) & e(a_n,b_2) & \cdots & e(a_2,b_n) \\
		\end{array}\right]\right)
\end{displaymath}
\subsubsection{推广}
实际上$e(a_i,b_j)$为$a_i\rightarrow b_j$的所有路径上边权积之和,
类似Matrix-Tree定理的讨论可扩展到边权相关问题。

上述内容参考了Wikipedia-EN\footnote{
	Lindström–Gessel–Viennot lemma - Wikipedia
	\url{https://en.wikipedia.org/wiki/Lindstr\%C3\%B6m\%E2\%80\%93Gessel\%E2\%80\%93Viennot\_lemma}
}。
