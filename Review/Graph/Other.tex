\section{杂讲}
\subsection{竞赛图}
\index{T!Tournament}
竞赛图是一个无向完全图被定向后得到的图。
\begin{theorem}
    竞赛图缩点后是一条链。
\end{theorem}
\subsubsection{竞赛图判定}
竞赛图可以用来指示两两选手比赛的胜负,判定比分是否
合法即判定是否存在合法的竞赛图。
\index{L!Landau's Theorem}
\begin{theorem}[Landau's Theorem]
对于一个有序的竞赛图度数序列/得分序列$0\leq s_1 \leq s_2 \leq \cdots \leq s_n$,
有$\displaystyle \forall 1\leq k\leq n,\sum_{i=1}^k{s_i}\geq \binomial{k}{2}$
,当$k=n$时等号必须成立。
\end{theorem}
对度数/胜利场数排序后逐个判断即可。
\subsection{最小平均值环}
对于一个有向图,找出平均值最小的环。

类似于分数规划的思想,对平均值进行二分,将所有边权减去二分值,
若存在负环则说明存在环的平均值小于该二分值,不断二分即可。
\subsection{平面图性质}
\index{P!Planar Graph}
以下仅讨论$V\geq 3$的情况:
\begin{property}
    $E\leq 3V-6$
\end{property}
\begin{property}
    $F\leq 2V-4$
\end{property}
使用这些性质可以限制边数以加速平面图判定。

上述内容参考了Wikipedia-EN\footnote{Planar graph - Wikipedia
    \url{https://en.wikipedia.org/wiki/Planar\_graph}
}
\subsection{拓扑排序判环}
若拓扑排序无法使得所有点都入队则说明存在环。
可以通过枚举点,将其度数-1取得删掉一条边的效果。
例如CF915D Almost Acyclic Graph:
\lstinputlisting{Source/Source/TopSort/CF915D.cpp}
