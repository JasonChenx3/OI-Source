\section{图上路径}
\subsection{欧拉路径与欧拉回路}
\index{E!Eulerian Path}
这两种路径都要求经过每条边有且只有一次,后者要求起点和终点相同。
\subsubsection{无向图欧拉路径}
\paragraph{判定} 无向图欧拉路径的奇点个数为0(欧拉回路)或2。
\index{H!Hierholzer's Algorithm}
这里使用Hierholzer算法,时间复杂度$O(E)$。

首先统计奇点个数,若奇点个数为0则任意选取点作为起点,
奇点个数为2则任意选取奇点之一作为起点。

接下来从起点开始DFS:
\begin{enumerate}
    \item 选取与点$u$相连的边$(u,v)$,删除$(u,v),(v,u)$,然后$DFS(u)$。
    \item 向队列中加入点$u$。
\end{enumerate}
遍历边时可以使用类似于Dinic的当前弧优化,若要求字典序最小,使用$std::multiset$
存边。
\subsubsection{有向图欧拉回路}
有向图欧拉路径有0(欧拉回路)或2个点不满足入度=出度的要求。
然后继续使用Hierholzer算法,注意输出时要反向(因为DFS是反向放入队列的)。

板子(UOJ117):
\lstinputlisting{Graph/EulerPath.cpp}
\subsubsection{混合图欧拉回路}
对于既有无向边又有有向边的图,主要思路是对无向边进行定向,使其满足入度=出度,
然后转换为有向边欧拉回路问题来做。

首先对无向边进行随机定向,把图转换为有向图,然后统计每个点入度与出度,
若入度与出度奇偶性不同则无解(每次调整方向都会导致它们的差变动2,无法调整至相等)。

接下来利用网络流的自动调整功能,从超级源向每个入度<出度的点连流量为
$\frac{\textrm{出度}-\textrm{入度}}{2}$的边,从超级汇向每个入度>出度的点连流量为
$\frac{\textrm{入度}-\textrm{出度}}{2}$的边,对于原图中定向为$u\rightarrow v$
的边从$u$到$v$连一条流量为1的边。最后跑最大流,若满流则说明有解。无向边对应网络流的双边
中残余流量非0的边就是该无向边定向后的边。此时满足每个点的入度=出度,跑有向边欧拉回路求得
方案。

\paragraph{正确性证明} 每次对一条增广路进行增广时,相当于把路径上所有的边反向,
路径中间的点的入度和出度保持不变,而起点的入度+1,出度-1,终点的入度-1,出度+1。
如果最大流满流,则所有的点都已经满足入度=出度了。此时如果这条边有流量,说明它所
对应的无向边应该定向为自己的反向。

以上内容参考了ajcxsu\footnote{[模板][持续更新]欧拉回路与欧拉路径浅析
    \url{https://www.cnblogs.com/acxblog/p/7390301.html}
}和commonc\footnote{混合图的欧拉回路
    \url{https://blog.csdn.net/commonc/article/details/52442882}
}的博客。
\subsection{哈密尔顿回路}
\index{H!Hamiltonian Path}
哈密尔顿路径要求经过每个点有且只有一次,哈密尔顿回路还要求起点和终点相同。

\subsubsection{判定}
\begin{itemize}
    \item \begin{theorem}[Dirac's Theorem]
        若大小为$n$的无向图的每个节点的度数都大于等于$\frac{n}{2}$,则
        存在哈密尔顿回路。
    \end{theorem}
    \item \begin{theorem}
        任意竞赛图存在哈密尔顿路径,若此图强连通则存在哈密尔顿回路。
    \end{theorem}
\end{itemize}
构造只能靠暴力DFS了。。。
\subsubsection{旅行商问题TSP}
\index{T!Travelling Salesman Problem}
旅行商问题求的是无向完全图的最短哈密尔顿回路,是NP问题。

若距离函数满足三角不等式(比如欧几里得距离),存在其解不超过
最优解两倍的近似算法:对该图求MST,DFS该MST,遍历的路径
去掉重复点后就是其解。时间复杂度$O(V^2)$。

还存在其解不超过最优解$\frac{3}{2}$的近似算法(Christofides算法):
\index{C!Christofides Algorithm}
\begin{itemize}
    \item 求该图的MST;
    \item 对MST中度数为奇数的点做无向图最小权完美匹配,将匹配边加入MST;
    \item 求出欧拉回路;
    \item 按照欧拉回路走,跳过重复节点。
\end{itemize}
时间复杂度$O(V^3)$。

该方法参考了Wikipedia-EN\footnote{Christofides algorithm - Wikipedia\\
    \url{https://en.wikipedia.org/wiki/Christofides\_algorithm}
}。

若不满足三角不等式,可以使用$O(V^2)$的最近邻算法,即每次DFS都贪心地选取
最短的边递归。
