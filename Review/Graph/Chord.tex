\section{弦图}
以下所述的图均为无向图。
\subsection{相关概念}
\subsubsection{团}
\index{C!Clique}
图$G$的子图$G'=(V',E')$,$G'$是$V'$的完全图。
\subsubsection{极大团}
\index{M!Maximal Clique}
一个不是其它团的子集的团。
\subsubsection{最大团}
\index{M!Maximum Clique}
点数最大的团。
\subsubsection{最小染色}
\index{M!Minimum Coloring}
$\chi(G)$为使得相邻点颜色不同的最小颜色数。
\subsubsection{弦}
\index{C!Chord}
连接环中不相邻两点的边。
\subsubsection{弦图}
\index{C!Chordal Graph}
图中任意长度大于3的环都至少有一根弦。
\subsubsection{诱导子图}
\index{I!Included Subgraph}
诱导子图$G'=(V',E')$,其中$V'\subseteq V,E'=\{(u,v)|u,v\in V',(u,v)\in E\}$。
\begin{lemma}
    弦图的诱导子图仍然是弦图。
\end{lemma}
\subsection{弦图判定}
\subsubsection{单纯点}
\index{S!Simplicial Vertex}
点$u$为单纯点当且仅当点$u$以及其邻接点构成的诱导子图是一个团。
\begin{lemma}
    弦图至少有一个单纯点,若其不为完全图则至少有两个不相邻的单纯点。
\end{lemma}
\subsubsection{完美消除序列}
\index{P!Perfect Elimination Ordering}
点集的序列$v_1,v_2,\cdots,v_n$是完美消除序列当且仅当$v_i$在$\{v_i,v_{i+1},\cdots,v_n\}$
的诱导子图上是单纯点。
\begin{theorem}
    图$G$是弦图当且仅当其存在完美消除序列。
\end{theorem}
充分性:使用数学归纳法假设弦图的真诱导子图有完美消除序列,由上文两个引理得
该弦图的完美消除序列可由单纯点$u$+剩余点的诱导子图的完美消除序列得到。

必要性:设出现在完美消除序列中的某点为$u$,根据完美消除序列的定义得与$u$相连的所有点中
点对之间有连边,故不存在长度$>3$的无弦环。
\subsubsection{朴素判定算法}
每次找到一个单纯点$v$并加入完美消除序列中,然后删除$v$及其连边。
直到所有点都被删除或找不到单纯点为止。时间复杂度$O(n^4)$。
\subsubsection{最大势算法}
维护每个点是否被标号以及相邻标号点的数量。每次选择未被标号且相邻标号点数量
最多的点标号。序列顺序与标号顺序相反。使用链表而不是优先队列实现,时空复杂度为$O(n+m)$。

\begin{theorem}
    若该图是弦图,则最大势算法生成的是完美消除序列。
\end{theorem}
证明留坑待补。
\index{*TODO!最大势算法正确性证明}

接下来要判断这个序列是否为完美消除序列。设$v_i$与${v_{i+1},v_{i+2},\cdots,v_n}$中
的$u_1,u_2,\cdots,u_k$相邻,仅需检查$u_1$是否与$u_2,\cdots,u_k$是否全相邻
(若全相邻则不仅保证了$u_1$到$u_2,\cdots,u_k$的相邻,
还会触发$u_2$到$u_3,\cdots,u_k$的检查,以此类推可以遍历到整个团的边)。
时间复杂度$O(n+m)$。

模板:
\lstinputlisting{Source/Templates/MCS.cpp}
\subsection{弦图的极大团}
\subsection{弦图的点染色}
\subsection{弦图的最大独立集与最小点覆盖}
\subsection{弦图的点染色}
\subsection{区间图}
上述内容参考了陈丹琦在WC2009上的讲稿《弦图与区间图》\cite{chord}。
