\section{仙人掌与圆方树}
\subsection{仙人掌}
\paragraph{仙人掌} 任意一条边最多只存在于一个环中的无向连通图叫做仙人掌。
\subsubsection{仙人掌的判定}
假设整个图已经连通,可以先对仙人掌进行DFS,记录DFS序与点的深度。然后
树上差分求出经过这条边的环数。判定的同时还可以把仙人掌去环变成森林。
\begin{lstlisting}
void initGraph(int n) {
    memset(last, n);
    cnt = 1;
}
bool flag[size];
int p[size], d[size], id[size], icnt;
void DFS(int u) {
    id[++icnt] = u;
    flag[u] = true;
    for(int i = last[u]; i; i = E[i].nxt) {
        int v = E[i].to;
        if(!flag[v]) {
            d[v] = d[u] + 1;
            p[v] = u;
            DFS(v);
        }
    }
}
int tag[size];
bool graph2Forest(int n) {
    icnt = 0;
    memset(flag, n);
    DFS(1);
    if(icnt != n)
        return false;
    memset(tag, n);
    for(int i = 2; i <= cnt; i += 2) {
        int u = E[i].to, v = E[i ^ 1].to;
        if(d[u] < d[v])
            std::swap(u, v);
        if(p[u] != v)
            ++tag[u], --tag[v];
    }
    initGraph(n);
    for(int i = n; i >= 1; --i) {
        int u = id[i];
        tag[p[u]] += tag[u];
        if(tag[u] == 0) {
            if(p[u]) {
                addEdge(u, p[u]);
                addEdge(p[u], u);
            }
        } else if(tag[u] > 1)
            return false;
    }
    return true;
}
\end{lstlisting}
如果graph2Forest返回false则说明这个图不是仙人掌。若返回true
则建出一个去掉环的森林。
\subsubsection{DFS树dp法}
对于简单的仙人掌问题可以使用DFS树做法:

首先可以使用类似Tarjan的算法判断是否出现了环,然后对于环和桥分别dp,
将环的信息记录在环在DFS树上最浅的节点上(即在回到环上最浅点时
另外dp),这样向上转移时就和桥一样了。

模板:
\begin{lstlisting}
int d[size], p[size], dfn[size], low[size], icnt = 0;
void DFS(int u) {
    dfn[u] = low[u] = ++icnt;
    for(int i = last[u]; i; i = E[i].nxt) {
        int v = E[i].to;
        if(v == p[u])
            continue;
        if(!dfn[v]) {
            p[v] = u;
            d[v] = d[u] + 1;
            DFS(v);
        }
        low[u] = std::min(low[u], low[v]);
        if(low[v] > dfn[u])
            //转移桥边 v->u
    }
    for(int i = last[u]; i; i = E[i].nxt) {
        int v = E[i].to;
        if(p[v] != u && dfn[u] < dfn[v])
            solveRing(u, v);//计算环 v->u
    }
}
\end{lstlisting}
\subsection{圆方树}
圆方树是解决仙人掌问题的利器,主要思想是把仙人掌构造为一棵树,然后使用
熟悉的树上操作来处理。
\subsubsection{构造}
在Tarjan时连树边(圆圆边),在环上深度最浅的点上处理环,把环上的节点(圆点)
都连到新建的点(方点)上(圆方边)。

\begin{lstlisting}
int dfn[size],low[size],d[size],p[size],ncnt;
void solveRing(int u,int v) {
    int siz=d[v]-d[u]+1;
    int id=++ncnt;
    for(int i=v;i!=u;i=p[i])
        //add id->i
    //add u->id
}
void tarjan(int u) {
    static int icnt=0;
    dfn[u]=low[u]=++icnt;
    for(int i=last[u];i;i=E[i].nxt) {
        int v=E[i].to;
        if(v==p[u])
            continue;
        if(!dfn[v]) {
            p[v]=u;
            d[v]=d[u]+1;
            DFS(v);
            low[u]=std::min(low[u],low[v]);
        }
        else low[u]=std::min(low[u],dfn[v]);
        if(dfn[u]<low[v])
            //add u->v
    }
    for(int i=last[u];i;i=E[i].nxt) {
        int v=E[i].to;
        if(v!=p[u] && p[v]!=u && dfn[u]<dfn[v])
            solveRing(u,v);
    }
}
ncnt=n;
\end{lstlisting}

还有一种简洁的建树方法:
\begin{lstlisting}
int dfn[size],p[size],icnt=0,ncnt;
bool ring[size];
void DFS(int u) {
    dfn[u]=++icnt;
    for(int i=last[u];i;i=E[i].nxt) {
        int v=E[i].to;
        if(v==p[u])
            continue;
        if(dfn[v]) {
            if(dfn[v]<dfn[u]) {
                int id=++ncnt;
                for(int j=u;j!=v;j=p[j])
                    addEdge(id,j),ring[j]=true;
                addEdge(v,id),ring[v]=true;
            }
        }
        else {
            p[v]=u;
            ring[u]=false;
            DFS(v);
            if(!ring[u])
                addEdge(u,v);
        }
    }
}
ncnt=n;
\end{lstlisting}

参见Iking123的博客\footnote{
    圆方树/广义圆方树学习小记(gradually update...)\\
   \url{https://blog.csdn.net/qq\_36551189/article/details/81047872}
}。

\subsubsection{性质}
\paragraph{子仙人掌} 以$r$为根的仙人掌上点$p$的子仙人掌为去掉$p$到$r$的简单路径后
点$p$所在的连通块。
\begin{property}
    以$r$为根仙人掌上点$p$的子仙人掌对应圆方树上点$p$的子树。
\end{property}
\begin{property}
    圆方树不存在方点与方点的连边。
\end{property}

\subsubsection{应用}\label{CSTA}
\paragraph{最短路}
把圆圆边的权值设为原边权值,圆方边的权值设为当前点到方点父亲的最短距离
(可以维护到树根的距离来计算),方圆边的权值设为0。

类比树上距离的做法,树链剖分后对每个询问求LCA。
\begin{itemize}
    \item 若LCA为圆点,则按照树上距离的方法解决。
    \item 若LCA为方点,则说明该路径经过了环上的两个点,但走哪一侧还未知。
    首先计算询问的这两个点分别在哪棵子树中,计算两条路径的答案,接下来只要考虑
    环上路径。
    询问子树时需要对在LCA环上的祖先是否为重儿子进行分类,jump函数计算该点的编号:
    \begin{lstlisting}
    int jump(int u,int lca) {
        int res;
        while(top[u]!=top[lca]) {
            res=top[u];
            u=p[top[u]];
        }
        return u==lca?res:son[lca];
    }
    \end{lstlisting}
    预先在计算方圆边距离时标记走的方向,然后分类讨论计算环上两点的最短路。
\end{itemize}
如此可以保证经过环的时候走的是最短路。
\paragraph{点分治}
注意处理方点时的复杂度,一般将方点设为环的大小,点分治时找带权重心。
\index{*TODO!仙人掌典型应用}
\subsection{广义圆方树}
对于每个点双,将点双内的点(称为圆点)连到新点(称为方点)。
\begin{lstlisting}
int dfn[size], low[size], st[size], timeStamp = 0,
    top = 0, nsiz;
void tarjan(int u) {
    dfn[u] = low[u] = ++timeStamp;
    st[++top] = u;
    for(int i = g1.last[u]; i; i = g1.E[i].nxt) {
        int v = g1.E[i].to;
        if(dfn[v])
            low[u] = std::min(low[u], dfn[v]);
        else {
            tarjan(v);
            low[u] = std::min(low[u], low[v]);
            if(dfn[u] <= low[v]) {
                int s = ++nsiz, p;
                g2.addEdge(u, s);
                do {
                    p = st[top--];
                    g2.addEdge(s, p);
                } while(p != v);
            }
        }
    }
}
nsiz = n;
\end{lstlisting}
广义圆方树的点与边数的最大规模都是$2V$级别的(圆点与方点相间,
并且连成一棵树)。

关键在于考虑圆点和方点的权值和边权,注意更新方点时不要计算父亲的贡献,
仅计算子树的贡献,在另外的计算过程中考虑父亲的贡献,以保证更新的复杂度。

上述内容参考了小蒟蒻yyb\footnote{仙人掌\&圆方树学习笔记
    \url{https://www.cnblogs.com/cjyyb/p/9098400.html}
}和immortalCO\footnote{圆方树——处理仙人掌的利器
    \url{http://immortalco.blog.uoj.ac/blog/1955}
}的博客。
