\section{2-SAT}
\subsection{问题描述}
有若干个布尔变量,对于由多个AND连接的若干个OR子表达式,且子表达式的操作数为
布尔变量或者其否定,
判断是否存在一组对布尔变量的赋值,使得这个布尔表达式的值为1。

这类问题一般描述为一个变量为真/假限制了另一个变量必须为真/假,求使得
所有限制被满足的一组变量赋值方案。对每个布尔变量拆点,可以将这些限制
表示为有向图。将命题中的条件向结论连边,同时对逆否命题的条件和结论连边。

比如要求变量$X$为真时变量$Y$必定为真,首先连边$X_1\rightarrow Y_1$;其逆否命题为
$Y$为假时$X$必定为假,连边$Y_0\rightarrow X_0$。

对于强制某一变量为真/假的需求,独立出来不好做,考虑沿用连边的思路。如果强制变量$X$为真,就
连边$X_0\rightarrow X_1$。
\subsection{可行性判定}
对这个图求强连通分量,发现同一强连通分量的点同时被选或不被选(分量里有一个点被选,根据
边的意义和强连通分量的定义,它可以把这个强连通分量内的点染色为被选)。

因此可以对该图求强连通分量,若$X_0$与$X_1$在同一个强连通分量内则为无解。
\subsection{构造方案}
以下方法的正确性证明留坑待补。
\index{*TODO!2-SAT构造方案正确性证明}
\subsubsection{DFS染色法}
首先对其进行缩点,标记每个强连通分量的对立分量,并连反图。

按照拓扑序处理每个强连通分量,DFS对立分量:
\begin{enumerate}
    \item 若自己已被标记则返回;
    \item 将自己标记为不选择,将对立分量标记为选择;
    \item DFS递归标记所连的点。
\end{enumerate}
查看每个点所在强连通分量的标记来输出方案。
\subsubsection{更简洁的方法}
由于tarjan算法标记强连通分量的顺序为自底向上,而上述方法的顺序同样也是
自底向上。因此对于每个布尔变量,选取所在强连通分量标号小的布尔值。

该方法参考了TRTTG的博客\footnote{2-SAT问题的方案输出
\url{https://www.cnblogs.com/TheRoadToTheGold/p/8436948.html\#\_label1}}。

{\bfseries 由于01变量的染色过程是连续的,因此当$X_0$与$X_1$较后一个的颜色确定后
就可以进行判断。这种方式可以避免在非法解上浪费时间。判断过程要放在if(dfn[i])外!!!}
\subsection{前后缀优化建图}
例题 PA2010 Riddle

有$n$个城镇,隶属于$k$个郡,城镇之间有$m$条连边。现在要将这些城镇中设立首都,
满足任意一条边之间至少有一个城镇为首都,且每个郡最多有一个首都,求是否存在合法方案。

点覆盖的约束很好解决,但是限制每个郡最多有一个首都比较困难,因为常规思路的连边达到
$O(n^2)$级别。考虑如何每个点只连常数条边就可以把不选的约束传递到同一个郡内的所有节点。
记布尔变量$x_i$表示城镇$i$是否为首都,记布尔变量$x_i'$表示在其所在郡的城镇序列中,
城镇$i$及其之前的城镇是否被选。根据单点与前缀、前缀与前缀的关系连边,就可以把约束快速
传递。由于前缀与前缀之间的约束关系可以将约束后传,不必对后缀连边。

该方法参考了ZigZagK的博客\footnote{
    【前后缀优化建图+2-SAT】BZOJ3495(PA2010)[Riddle]题解\\
    \url{https://blog.csdn.net/zzkksunboy/article/details/76285426}
}。
