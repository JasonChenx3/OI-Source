\section{2-SAT}
\subsection{问题描述}
有若干个布尔变量,对于由多个AND连接的若干个OR子表达式,且子表达式的操作数为
布尔变量或者其否定,
判断是否存在一组对布尔变量的赋值,使得这个布尔表达式的值为1.

这类问题一般描述为一个变量为真/假限制了另一个变量必须为真/假。求使得
所有限制被满足的一组变量赋值方案。对每个布尔变量拆点,可以将这些限制
表示为有向图。将命题中的条件和结论连边,同时对逆否命题的条件和结论连边。

比如要求变量$X$为真时变量$Y$必定为真,首先连边$X_1->Y_1$;其逆否命题为
$Y$为假时$X$必定为假,连边$Y_0->X_0$。
\subsection{可行性判定}
对这个图求强连通分量,发现同一强连通分量的点同时被选或不被选(分量里有一个点被选,根据
边的意义和强连通分量的定义,它可以把这个强连通分量内的点染色为被选)。

所以对建出的图求强连通分量,若$X_0$与$X_1$在同一个强连通分量内则为无解。
\subsection{构造方案}
以下方法的正确性证明留坑待补。
\index{*TODO!2-SAT构造方案正确性证明}
\subsubsection{DFS染色法}
首先对其进行缩点,标记每个强连通分量的对立分量,并连反图。

按照拓扑序处理每个强连通分量,DFS对立分量:
\begin{enumerate}
    \item 若自己已被标记则返回;
    \item 将自己标记为不选择,将对立分量标记为选择;
    \item DFS递归标记所连的点。
\end{enumerate}
对于每个点查看其强连通分量的标记即可输出方案。
\subsubsection{更简洁的方法}
由于tarjan算法标记强连通分量的顺序为自底向上,而上述方法的顺序同样也是
自底向上选择的,因此对于每个布尔变量,哪个值的强连通分量标号小就选择哪个。

该方法参考了TRTTG的博客\footnote{2-SAT问题的方案输出
\url{https://www.cnblogs.com/TheRoadToTheGold/p/8436948.html\#\_label1}}。
