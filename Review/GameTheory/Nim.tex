\section{Nim系列游戏}

\subsection{Nim游戏}

\index{N!Nim}

普通Nim游戏的定义:
有两个玩家轮流从许多堆中移除对象。在每个回合中,玩家选择一个非空的堆,可以移除任何数量
的对象,但至少移除一个对象。无法操作的玩家为败者。

此类游戏可看做是Bash游戏的特殊化。

\begin{theorem}
	$SG_{Nim}(x)=x$
\end{theorem}

证明略。

\subsection{Bash游戏}

\index{B!Bash Nim}

Bash游戏与普通Nim游戏的区别是增加了每次最多移除k个对象的限制。

\begin{theorem}
	$SG_{Bash}(x)=x \bmod (k+1)$
\end{theorem}

证明略。

\subsection{NimK游戏}

\index{N!NimK}

NimK游戏与普通Nim游戏的区别是每次可以从不超过k个堆中移除任意数目对象。

\begin{theorem}\label{NimK}
	将每堆对象的数目拆位,若每位上1的个数$\bmod (k+1)$均为0,则必败,反之必胜。
\end{theorem}

记忆:普通Nim游戏可理解为mod 2的情况。

算法正确性证明:

对于当前玩家,可以通过如下步骤赢得游戏:

策略就是不断地转移到每位上1的个数$\bmod (k+1)$均为0的状态。

设$D0[i]$与$D1[i]$为已标记堆中第i位为0和1的个数。

\begin{enumerate}
	\item 选取非0最高位W;
	\item 找到一个第W位为1,且未标记的堆,将该堆标记,把它的第W位改为0,
	并更新$D0[1\sim W-1],D1[1\sim W-1]$;
	\item 如果$S[i]$非0,且$S[i]+D0[i]>k || S[i]-D1[i]<1$,则可以
	通过修改已标记的堆将$S[i]$变为0。
	\item 如果$S[i]$均为0,则结束,否则重复步骤1。
\end{enumerate}

保证修改的堆数小于k,我暂时没想到证明。\index{TODO!NimK正确性证明}

所以当每位均为0时,当前玩家要么已经无法操作,要么必须转移至必胜态,因此该状态为必败态。

定理~\ref{NimK}得证。

\subsection{Anti Nim}

\index{A!Anti Nim}

不能操作的玩家胜利。

\begin{theorem}\label{AntiNim}
	先手必胜当且仅当满足以下条件之一:
	\begin{enumerate}
		\item $SG(x)=0$ 且所有堆的对象数都为1
		\item $SG(x)\not=0$ 且至少有一堆对象数大于1
	\end{enumerate}

\end{theorem}

证明:
定义对象数为1的叫A堆,大于1的叫B堆。

\begin{enumerate}
	\item 若所有堆均为A堆,则奇数堆先手必败,反之必胜。
	\item 若B堆数等于1,显然$SG(x)\not=0$,则可根据堆的总数确定取该堆的数目,
	      使下一状态为情况1的奇数堆,所以先手必胜。
	\item 若B堆数大于1,则
	      \begin{enumerate}
			  \item 若$SG(x)=0$,则必须留下超过2个B堆并使$SG(x')\not=0$,否则
			  会使对方进入情况2的必胜态。
			  \item 若$SG(x)\not=0$,则根据Nim游戏的理论(必胜态->必败态),
			  存在一种方法转移至情况3的子情况1。
	      \end{enumerate}
	      若玩家处于情况3的子情况2中,则可以在有限次回合内使对方无法转移至子情况2,
	      因此该状态为必胜态。
\end{enumerate}

定理~\ref{AntiNim}得证。

\subsection{阶梯博弈(Staircase Nim)}

阶梯博弈的定义:有多个阶梯,从左到右编号为$1\sim n$,1号阶梯的左边为地面,
玩家每次可以将某阶梯的石子移动至其左边的阶梯,当某玩家无法移动时(即所有石子
都在地面),该玩家失败。

\index{S!Staircase Nim}

\begin{theorem}\label{StaircaseNim}
	阶梯博弈问题等价于奇数号阶梯的Nim博弈。
\end{theorem}

证明:

假设己方是先手:

\begin{enumerate}
	\item 当对方移动奇数号阶梯的石子到偶数号阶梯时,我们按照Nim游戏的策略从奇
	数号阶梯。
	\item 当对方移动偶数号阶梯的石子到奇数号阶梯时,我们将其移动的石子移动到偶
	数号阶梯,抵消对方的操作。
\end{enumerate}

可以将移动到偶数号阶梯看做被移除。使用上述策略可以使状态与偶数号阶梯的石子数无关,
定理~\ref{StaircaseNim}得证。

\subsubsection{例题}

Luogu P3480 [POI2009]KAM-Pebbles\footnote{\url{https://www.luogu.org/problemnew/show/P3480}}

对于这题可将原条件通过差分转换为阶梯博弈模型($A_i \geq A_i-1 \Leftrightarrow
	A_i-A_i-1 \geq 0$)。

\lstinputlisting[title=Luogu P3480]{Source/Source/'Game Theory'/3480.cpp}

出题灵感:Anti BashK游戏

以上内容参考了forezxl\footnote{anti-Nim游戏(反Nim游戏)简介
	\url{https://blog.csdn.net/a1799342217/article/details/78274410}},
hehedad\footnote{关于nimk类型博弈的详细理解与解释
	\url{https://blog.csdn.net/chenshibo17/article/details/79783523}},
我爱AI\_AI爱我\footnote{阶梯博弈算法详解
    \url{https://blog.csdn.net/qq\_30241305/article/details/51956518}}的博客。
