\section{Nim系列游戏}

\subsection{Nim游戏}

\index{N!Nim}

普通Nim游戏的定义:
有两个玩家轮流从许多堆中移除对象。在每个回合中,玩家选择一个非空的堆,可以移除任何数量
的对象,但要移动至少一个对象。无法操作的玩家为败者。

此类游戏可看做是Bash游戏的特殊化。

\begin{theorem}
	$SG_{Nim}(x)=x$
\end{theorem}

证明略。

\subsection{Bash游戏}

\index{B!Bash Nim}

Bash游戏与普通Nim游戏的区别是增加了每次最多移除k个对象的限制。

\begin{theorem}
	$SG_{Bash}(x)=x \bmod (k+1)$
\end{theorem}

证明略。

\subsection{NimK游戏}

\index{N!NimK}

NimK游戏与普通Nim游戏的区别是每次可以从不超过k个堆中移除任意数目对象。

\begin{theorem}\label{NimK}
	此状态为必败态当且仅当将每堆对象的数目拆位,每位上1的个数$S[i] \bmod (k+1)$均为0。
\end{theorem}

记忆:普通Nim游戏可理解为mod 2的情况。

算法正确性证明:

当前玩家可以通过如下步骤赢得游戏:

策略是不断地转移到每位上1的个数$\bmod (k+1)$均为0的状态。

设$D0[i]$与$D1[i]$为已标记堆中第i位为0和1的个数。

\begin{enumerate}
	\item 选取$S[i]$非0最高位W;
	\item 找到一个第W位为1,且未标记的堆,将该堆标记,把它的第W位改为0,
	      并更新$D0[1\sim W-1],D1[1\sim W-1]$;
	\item 如果$S[i]$非0,且$S[i]+D0[i]>k || S[i]-D1[i]<1$,则可以
	      通过修改已标记的堆将$S[i]$变为0。
	\item 如果$S[i]$均为0,则结束循环,否则重复步骤1。
\end{enumerate}

模$k+1$保证了修改的堆数不超过$k$,因为如果已标记了$k$个堆,$S[i]+D0[i]>k ||
	S[i]-D1[i]<1$必定成立,无需再标记新堆。

所以当每位均为0时,当前玩家要么已经无法操作,要么必须转移至必胜态(对方可根据上述方法
转移回必败态),因此该状态为必败态。

定理~\ref{NimK}得证。

\subsection{Anti Nim}

\index{A!Anti Nim}

不能操作的玩家胜利。

\begin{theorem}\label{AntiNim}
	先手必胜当且仅当满足以下条件之一:
	\begin{enumerate}
		\item $SG(x)=0$ 且所有堆的对象数都为1
		\item $SG(x)\not=0$ 且至少有一堆对象数大于1
	\end{enumerate}

\end{theorem}

证明:
定义对象数为1的叫A堆,大于1的叫B堆。

\begin{enumerate}
	\item 若所有堆均为A堆,则奇数堆先手必败,反之必胜。
	\item 若B堆数等于1,显然$SG(x)\not=0$,则可根据堆的总数确定取完该堆还是剩1个,
	      使下一状态为情况1的必败态,所以先手必胜。
	\item 若B堆数大于1,则
	      \begin{enumerate}
		      \item 若$SG(x)=0$,则必须留下超过2个B堆并使$SG(x')\not=0$,否则
		            会使对方进入情况2的必胜态。
		      \item 若$SG(x)\not=0$,则根据Nim游戏的理论(必胜态->必败态),
		            存在一种方法转移至情况3的子情况(a)。
	      \end{enumerate}
	      若玩家处于情况3的子情况(b)中,则可以在有限次回合内使对方无法转移至子情况(b),
	      因此该状态为必胜态。
\end{enumerate}

定理~\ref{AntiNim}得证。

\subsection{阶梯博弈(Staircase Nim)}
\index{S!Staircase Nim}
阶梯博弈的定义:有多个阶梯,从左到右编号为$1\sim n$,1号阶梯的左边为地面,
阶梯上有若干石子,玩家每次可以将某阶梯的石子移动至其左边的下一级阶梯,
当某玩家无法移动石子时(即所有石子都在地面),该玩家失败。

\begin{theorem}\label{StaircaseNim}
	阶梯博弈问题等价于奇数号阶梯的Nim博弈。
\end{theorem}

证明:假设己方是先手:

\begin{enumerate}
	\item 当对方移动奇数号阶梯的石子到偶数号阶梯时,我们按照Nim游戏的策略从奇
	      数号阶梯移动石子到偶数号阶梯(等价于移除石子)。
	\item 当对方移动偶数号阶梯的石子到奇数号阶梯时,我们将其移动的石子移动到偶
	      数号阶梯,抵消对方的操作。
\end{enumerate}

可以将移动到偶数号阶梯看做被移除(最终移动到地面)。使用上述策略可以使状态与偶数号阶梯
的石子数无关,定理~\ref{StaircaseNim}得证。

\subsubsection{例题}

Luogu P3480 [POI2009]KAM-Pebbles\footnote{\url{https://www.luogu.org/problemnew/show/P3480}}

对于这题可将原条件通过差分转换为阶梯博弈模型($A_i \geq A_i-1 \Leftrightarrow
	A_i-A_i-1 \geq 0$)。

\lstinputlisting[title=Luogu P3480]{Source/Source/'Game Theory'/3480.cpp}

出题灵感:Anti BashK游戏

\subsection{威佐夫博奕}
\index{W!Wythoff's Game}
有两堆石子,玩家可从一个堆中拿出石子或者从两个堆中拿出相同数量的石子,不能操作者输。

我们可以使用SG定理写一个打表程序:
\lstinputlisting{Source/Source/'Game Theory'/2252test.cpp}

打印出20以内的必败点:
0 0
1 2
3 5
4 7
6 10
8 13
9 15
11 18
12 20

可以发现每一个必败点的石子数满足较小数是之前未出现过的最小自然数,较大数与较小数
之差递增。
进一步研究可以发现两数之差与较小数之比近似等于黄金分割比。

必败点的局面构成的序列为Beatty序列,两个数列可由两个无理数$\alpha,\beta$生成,
即$\{\lfloor\alpha n\rfloor\},\{\lfloor\beta n\rfloor\}$,
其中$\frac{1}{\alpha}+\frac{1}{\beta}=1$。
Rayleigh Theorem描述了Beatty序列中任意一个正整数仅在某个数列中出现一次。
由两个数列的关系可得$(\alpha+1)n=\beta n$,联立解得$\alpha=\frac{\sqrt{5}+1}{2}$。
因此只要一行代码就可以判断必败点:$a==static\_cast<int>((b-a)*1.618\cdots)$。

\subsection{斐波那契博弈}
\index{F!Fibonacci Nim}
仅一堆石子,先手不能把所有石子取完,至少取1颗。每一步玩家可以取的石子数
不超过上一次取的石子数的2倍,取出最后一颗石子的玩家胜利。

\begin{theorem}
	当且仅当石子数为斐波那契数时,该局面为必败态。
\end{theorem}

\paragraph{证明}
首先证明当石子数为斐波那契数时,该局面为必败态。

记$f_i$为第$i$个斐波那契数。
显然当$i\leq 2$时,该局面为必败态。
使用归纳法,假设对于$i<k$,该结论成立。
当$i=k$时,此时将石子看成两部分,即$f_k=f_{k-1}+f_{k-2}$。由于$f_{k-1}<2f_{k-2}$,
先手肯定不能取完$f_{k-2}$的部分。由假设可知后手取得了$f_{k-2}$部分的最后一颗石子,
考虑后手此时取的石子数。由贪心(后手要刚好取完$f_{k-2}$部分)可知当先手第一次取的石子
数$\geq \frac{f_{k-2}}{3}$时,后手取的石子数$\leq \frac{2f_{k-2}}{3}$,此时后手
取完$f_{k-2}$部分后,先手最多取得$\frac{4f_{k-2}}{3}<f_{k-1}$,即先手仍然不能一次性
取完$f_{k-1}$部分,由假设可知先手必败。

然后证明当石子数不为斐波那契数时,该局面为必胜态。
\begin{lemma}[Zeckendorf's Theorem]
	任意正整数可分解为多个不连续的斐波那契数之和。
\end{lemma}
每次用尽可能大的斐波那契数将石子拆分为多个部分。先手可将数量最小的部分取完,
由于斐波那契数的不连续性,后手不可能取完倒数第二部分的石子。此时后手处于这一部分
石子的必败态,即先手将取得这部分石子的最后一颗。对于每一堆都如此操作就能赢得胜利。

以上内容参考了forezxl\footnote{anti-Nim游戏(反Nim游戏)简介
	\url{https://blog.csdn.net/a1799342217/article/details/78274410}},
hehedad\footnote{关于nimk类型博弈的详细理解与解释
	\url{https://blog.csdn.net/chenshibo17/article/details/79783523}},
我爱AI\_AI爱我\footnote{阶梯博弈算法详解
	\url{https://blog.csdn.net/qq\_30241305/article/details/51956518}}
和dgq8211\footnote{
	斐波那契博弈(Fibonacci Nim) - nyist\_xiaod
	\url{https://blog.csdn.net/dgq8211/article/details/7602807}
}的博客,
以及百度百科(终于发现比维基更全的词条了)\footnote{
	威佐夫博弈\_百度百科
	\url{https://baike.baidu.com/item/
		\%E5\%A8\%81\%E4\%BD\%90\%E5\%A4\%AB\%E5\%8D\%9A\%E5\%BC\%88/
		19858256?fr=aladdin\#3}
}。
