\section{SG函数与SG定理}

\subsection{适用范围}

一切Impartial Combinatorial Games都等价于Nim游戏,可以使用SG函数
解决。\index{I!Impartial Combinatorial\\ Games}

该类游戏拥有如下特征:\footnote{参见 Impartial game - Wikipedia
	\url{https://en.wikipedia.org/wiki/Impartial_game}}

\begin{itemize}

	\item 两个玩家轮流操作
	\item 当有一名玩家无法操作时,游戏结束
	\item 游戏会在有限次操作后结束(状态转移图是一个DAG)
	\item 游戏对双方是公平的,所有操作必须能够由双方完成(即当双方都采取最优策略
	      时,游戏的胜负只取决于先后手)
	\item 双方在开局前已知道关于游戏的全部信息,并在游戏时采用最优策略。

\end{itemize}

\subsection{SG函数}

接下来给出必胜点和必败点的定义(前提是双方均采用最优策略):

\begin{itemize}
	\item 必胜点(N-Position):处于该状态的玩家必胜\index{N!N-Position}
	\item 必败点(P-Position):处于该状态的玩家必败\index{P!P-Position}
\end{itemize}

必胜点与必败点有如下性质:

\begin{itemize}
	\item \begin{property}
		      终结点为必败点
	      \end{property}
	\item \begin{property}
		      必败点的下一状态必然为必胜点(某玩家必败,等价于无论他如何操作都使另一
		      玩家必胜)
	      \end{property}
	\item \begin{property}
		      从必胜点出发至少有一种方式进入必败点(该玩家的最优策略就是使状态转移到
		      必败点)
	      \end{property}
\end{itemize}

要判断哪个玩家必胜,只要使用SG函数和SG定理计算出先手所在状态(即初始状态)是必胜点
还是必败点即可。

SG函数的定义如下:$SG(x)=mex(S(x))$\index{S!SpragueGrundy Function}

其中$S(x)$是状态x的后继状态的SG函数值的集合,$mex(S)$是没有出现在集合$S$中的最
小非负整数。

以Nim游戏为例,可根据函数定义计算SG值:

\lstinputlisting[title=NimSG]{GameTheory/NimSG.cpp}

\subsection{SG定理}

\index{S!SpragueGrundy Theorem}

\begin{theorem}[SpragueGrundy Theorem A]
	\bfseries 假设当某玩家无法操作时,该玩家失败。\mdseries
	若$SG(x)=0$,则该状态为必败态,否则为必胜态。
\end{theorem}

归纳证明:假设该定理对状态x后继的状态成立,则

\begin{itemize}
	\item 若$SG(x)>0$,则说明存在一个后继状态$y$,使得$SG(y)=0$,因为$y$为必败
	      态,所以$x$为必胜态。
	\item 若$SG(x)=0$,则说明对$x$的任意后继状态$y$,都有$SG(y)>0$,因为$y$为
	      必胜态,所以$x$为必败态。
\end{itemize}

\begin{theorem}[SpragueGrundy Theorem B]\label{SGB}
	游戏的SG函数值等于各子游戏函数值的Nim和(即xor和)。
\end{theorem}

设$S_X$为$X$后继状态的集合,$b=SG(X_1)\oplus \cdots \oplus SG(X_N)$。

该定理可分为两个引理证明:

\begin{enumerate}
	\item
	      \begin{lemma}\label{SGBL1}
		      $\forall_{a\in N,a<b},\exists_{X'\in S_X},SG(X')=a$
	      \end{lemma}

	      归纳证明:首先假设该引理对子游戏成立。

	      设$d=b\oplus a$,$d$的最高位为k,则存在$SG(X_i)$的第k位为1
	      ($d$的那一位由奇数个$SG(X_i)$贡献)。

	      所以$SG(X_i)\oplus d<SG(X_i)$,由假设得$\exists_{X_i'\in
			      S_{X_i}},SG(X_i')=SG(X_i)\oplus d$。

	      结合$a=b\oplus d$得$a=SG(X_1)\oplus \cdots \oplus SG(X_i')
		      \oplus \cdots \oplus SG(X_N)$

	      因为$X_1,\cdots,X_i',\cdots,X_N\in S_X$,所以引理~\ref{SGBL1}得证。

	\item
	      \begin{lemma}\label{SGBL2}
		      $\forall_{X'\in S_X},SG(X')\not=b$
	      \end{lemma}

	      反证法:假设SG定理对子状态成立(这里貌似不严谨),\\且$\exists_{X' \in S_X},SG(X')=b$。

	      那么就有$SG(X_i')=SG(X_i)$,由$mex$函数的定义可得$1$

	      两式矛盾,引理~\ref{SGBL2}得证。

\end{enumerate}

以上内容参考了Angel\_Kitty\footnote{SG函数和SG定理[详解] Angel\_Kitty
\url{https://www.cnblogs.com/ECJTUACM-873284962/p/6921829.html}}与
PhilipsWeng\footnote{SG定理
	\url{https://blog.csdn.net/PhilipsWeng/article/details/48395375}}的博客。

\subsubsection{例题}

Luogu P2575 高手过招\footnote{\url{https://www.luogu.org/problemnew/show/P2575}}

每行棋子可视为一个子游戏,状压后使用DFS计算SG函数值,然后利用定理~\ref{SGB}计算整个
游戏的SG函数值即可。

\lstinputlisting[title=Luogu P2575]{Source/Source/'Game Theory'/2575.cpp}
