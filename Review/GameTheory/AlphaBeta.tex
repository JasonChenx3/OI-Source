\section{Alpha-Beta剪枝}

\subsection{Min-Max搜索}

\index{M!Min-Max Search}

核心在于搜索出对双方来说最好的操作。在实践中可以将对手的最优估价函数取反,
这样可以避免写区分双方的复杂代码。

伪代码如下:
\begin{lstlisting}[title=Min-Max Search]
    int DFS(State s) {
        if(s.end())return s.eval();
        else {
            int res=-inf;
            for(auto nxt:s)
                res = std::max(res,-DFS(nxt));
            return res;
        }
    }
\end{lstlisting}

\subsubsection{例题}

Luogu P4363 [九省联考2018]一双木棋chess
\footnote{\url{https://www.luogu.org/problemnew/show/P4363}}

状压后DFS即可,需要记忆化搜索。

\lstinputlisting[title=Luogu P4363]{Source/Unclassified/Done/4363.cpp}

\subsection{Alpha-Beta剪枝}

\index{A!Alpha-Beta Search}

Min-Max搜索运行时需要检查整棵博弈树,而Alpha-Beta能够抛弃掉一些显然不是最优的子树,
提高搜索速度。核心思想是当你已经可以明确这个操作已经得不到更优解时,就不需要再搜下去了。

设$\alpha$为当前搜索到的下界,$\beta$为上界。

伪代码如下:
\begin{lstlisting}[title=Alpha-Beta 剪枝]
    int DFS(State s,int alpha,int beta) {
        if(s.end())return s.eval();
        else {
            for(auto nxt:s) {
                int val = -DFS(nxt,-beta,-alpha);// 1
                if(val >= beta)// 2
                    return beta;
                if(val > alpha)
                    alpha = val;
            }
            return alpha;
        }
    }
    int res = DFS(init,-inf,inf);
\end{lstlisting}

Alpha-Beta剪枝与Min-Max搜索的区别有二:

\begin{enumerate}
    \item 对于对手来说,权值是相反数,因此$\alpha'=-\beta,\beta'=-\alpha$。
    \item $val$会更新$\alpha$的值,所以$val>=\beta$等价于$\alpha>\beta$,
    已没有继续搜索的必要(对手要么只能接受$-\beta$,要么可以避开这条路径(因为
    $\beta$值是由上一层节点递归计算同级节点得到的结果,决定权在上级))。
\end{enumerate}

Alpha-Beta的效率严重依赖于搜索顺序:

\begin{itemize}
	\item 在最坏情况下(无法剪枝),该算法与Min-Max搜索相同,效率极低。
	\item 在最优情况下(每次都从最优着法开始搜索),每层的搜索分支达到$\sqrt{n}$,因此搜索
	      深度可以达到原来的两倍。
\end{itemize}

分支规模大小待证明。\index{TODO!Alpha-Beta分支规模大小}

以上内容参考了象棋百科全书\footnote{计算机博弈 - 象棋百科全书
	\url{http://www.xqbase.com/computer.htm}}
