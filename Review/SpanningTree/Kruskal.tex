\section{Kruskal重构树}
\subsection{构造}
在做Kruskal时,将合并两点所在集合的操作改为将两点所在集合
连到一个新的点上(代表边),新点的权为边权,代表点的叶子节点的权为0,
这样就可以构造出一棵Kruskal重构树,用来处理边权最值问题。

由构造过程可推出Kruskal重构树的性质:

\begin{property}
	最小生成树与Kruskal重构树两点路径上的边权(点权)最大值相等。
\end{property}
\begin{property}\label{KRTP2}
	Kruskal重构树中子节点的权小于等于父节点的权。
\end{property}
由上述性质可得到在Kruskal重构树中询问边权最大值的方法:计算
两点的LCA,LCA的点权就是原生成树上边权最大值。

以上内容参考了Coco\_T\_的博客\footnote{
	bzoj3732 Network(Kruskal重构树)
	\url{https://blog.csdn.net/wu_tongtong/article/details/77601523}
}。
\subsection{例题}
Luogu P4768 [NOI2018]归程\footnote{
【P4768】[NOI2018]归程 - 洛谷
\url{https://www.luogu.org/problemnew/show/P4768}
}

很容易看出这题求的是汽车可到达节点到原节点的最短路的最小值。

首先使用Dijkstra预处理最短路({\bfseries 不要用SPFA!!!不要用SPFA!!!
不要用SPFA!!!})。

\subsubsection{可持久化并查集法}
一个很简单的思路是按照海拔高度从大到小连边,使用可持久化并查集记录连通性,同时
维护连通块内最短路最小值,预处理后查询集合最值即可,注意可持久化线段树的规模。

\subsubsection{Kruskal重构树法}
下面是Kruskal重构树的做法:

由性质~\ref{KRTP2}不难想到对该图按海拔高度做最大生成树,这样就保证了
Kruskal重构树的父节点点权小于等于子节点点权,使用倍增法可以跳到满足要求的
的最浅的祖先。这个祖先就代表了满足海拔要求的最大连通集合。对每个点存储子树叶
子节点(即生成树中的点)的最短路最小值即可支持$O(\lg n)$查询。

代码如下:
\lstinputlisting{Source/Unclassified/Done/4768.cpp}

\subsection{最小瓶颈路}\label{MBP}
最小瓶颈路求的是无向图中指定两点的路径的边权最大值的最小值。

无向图中任意两点的最小瓶颈路肯定在最小生成树上,建出
Kruskal重构树,根据性质~\ref{KRTP2},两点的LCA即为边权最大值。

\subsection{次小生成树}
求解次小生成树的思路是使用非树边替换链上最大边,按照~\ref{MBP}节所述
可以简洁地求出链上边权最大值。对于严格次小生成树,对每个点维护子树严格
次小值也是很简单的事情(比线段树不知道高到哪里去了)。
