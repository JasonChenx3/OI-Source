\section{斯坦纳树}
\index{S!Steiner Tree}
最小斯坦纳树求的是将指定的大小为$k$的点集连通的最小代价。

计算步骤如下:
使用状态压缩来描述指定点集的连通状态,
令$dp[s][i]$为以$i$为根,连通状态为$s$的最小代价。

转移方法有两种:
\begin{itemize}
    \item 两个不相交集合与同一个点连接,即
    \begin{displaymath}
        dp[s][i]=min\left\{dp[t][i]+dp[s-t][i]\right\},t\in s
    \end{displaymath}
    \item 给集合连入一条新边:
    \begin{displaymath}
        dp[s][i]=min\left\{dp[s][j]+w(i,j)\right\},(i,j)\in E
    \end{displaymath}
\end{itemize}

首先令$dp[1<<j][P[i]]=0$,初始化单个点的情况。

可以从小到大枚举连通集合:
\begin{enumerate}
    \item 枚举子集更新$dp[i][]$。
    \item 将$dp[][s]\neq \infty$的点入队,使用SPFA或Dijkstra更新,注意不要
    改变连通状态。
\end{enumerate}

枚举子集更新的复杂度为$\displaystyle n\sum_{i=0}^k{\binomial{k}{i}2^i}=
n\cdot(1+2)^k=n\cdot 3^k$。

代码如下:
\begin{lstlisting}
int dp[1<<maxk][size];
int solve(int n,int k) {
    memset(dp,0x3f,sizeof(dp[0])<<k);
    for(int j=0;j<k;++j)
        for(int i=1;i<=n;++i)
            dp[1<<j][P[i]]=0;
    int end=1<<k;
    for(int s=0;s<end;++s) {
        for(int t=s&(s-1);t;t=s&(t-1))
            for(int i=1;i<=n;++i)
                dp[s][i]=std::min(dp[s][i],
                    dp[t][i]+dp[s^t][i]);
        for(int i=1;i<=n;++i)
            if(dp[s][i]!=inf)
                //push
        SSSP(s);
    }
    int ans=inf;
    for(int i=1;i<=n;++i)
        ans=std::min(ans,dp[end-1][i]);
    return ans;
}
\end{lstlisting}

\subsubsection{多组斯坦纳树}
给定多个这样的点集,求最小代价。

重新回想$dp[s][i]$的实际意义而不必考虑s中连通节点的组别,$dp[s][i]$表示的
是我们关注的节点连通状态为$s$,连通块以$i$为根的最小代价。

那么最终结果由一些连通块组成,把有整组连通的方案当做连通块按照组别状压,再做一次DP。
