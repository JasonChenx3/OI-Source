\section{最小生成树}
\subsection{Kruskal与LCT}
Luogu P4234 最小差值生成树\footnote{【P4234】最小差值生成树 - 洛谷
	\url{https://www.luogu.org/problemnew/show/P4234}
}

按照Kruskal的处理顺序处理生成树,当遇到环时删掉环上权值最小的边,然后连上自己。

\subsection{Prim算法}

从一个点开始贪心地连接最短的可扩展边,直至每个点都被连接。

代码:
\begin{lstlisting}
struct Info{
    int u,d;
    Info(int u,int d):u(u),d(d){}
    bool operator<(const Info& rhs) const {
        return d>rhs.d;
    }
};
bool flag[size];
int prim(int n) {
    flag[1]=true;
    std::priority_queue<Info> heap;
    for(int i=last[1];i;i=E[i].nxt)
        heap.push(Info(E[i].to,E[i].w));
    int cnt=1,sum=0;
    while(cnt<n && heap.size()) {
        int u=heap.top().u;
        int d=heap.top().d;
        heap.pop();
        if(!flag[u]) {
            flag[u]=true;
            sum+=E[i].w;
            for(int i=last[u];i;i=E[i].nxt) {
                int v=E[i].to;
                if(!flag[v])
                    heap.push(Info(v,E[i].w));
            }
        }
    }
    return cnt==n?sum:-1;
}
\end{lstlisting}

当边数较大时,可以考虑使用Prim或优先队列+Kruskal(参见第~\ref{PQS}\\节)。

\subsection{次小生成树}
步骤如下:
\begin{enumerate}
	\item 构造最小生成树,标记并连接树边;
	\item 对生成树进行树链剖分,构造线段树,使其可以$O(\lg n)$查询到链上最大权;
	\item 枚举非树边,计算其替代链上的最大边后的代价,更新答案。
\end{enumerate}

对于严格次小生成树,需要维护链上最大权与严格次大权。

\subsection{生成树性质}
以下性质用于解决最小生成树计数问题:
\begin{property}
	最小生成树中,不同权值边的数量固定。
\end{property}
\begin{property}
	Kruskal算法中,处理完某一权值的边后,连通块相同。
\end{property}
\subsubsection{例题}

Luogu P4208 [JSOI2008]最小生成树计数\footnote{
【P4208】[JSOI2008]最小生成树计数 - 洛谷
\url{https://www.luogu.org/problemnew/show/P4208}
}

根据以上两个性质可以把每种权值的边分别计算,然后使用乘法原理组合即为答案。

步骤如下:
\begin{enumerate}
	\item 首先使用常规Kruskal计算每种权值边的数量$c_i$以及等权边的分布区间;
	\item 对于每一种权值,计算用完$c_i$条边的方案数。
	      由于题中每种权值的边数不超过10,可以枚举每条边选还是不选,
          在回溯时直接令连接的两点的根的父亲重设为自己,时间复杂度近似为$O(2^{c_i})$。
          模拟对该权值边做Kruskal的过程。
	\item 把每种权值的方案数乘起来就是方案数。
\end{enumerate}

代码如下:
\lstinputlisting{Source/Unclassified/Done/4208.cpp}

另一种方法:每次计算小连通块->大连通块的过程,对于每个大连通块单独使用
Matrix-Tree定理求方案数,方案数之积即为答案。

最小生成树计数问题参考了clover\_hxy的博客\footnote{
bzoj 1016: [JSOI2008]最小生成树计数 (矩阵树定理+最小生成树)
\url{https://blog.csdn.net/clover\_hxy/article/details/69397184}
}。
