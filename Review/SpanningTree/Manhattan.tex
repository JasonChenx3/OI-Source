\section{曼哈顿距离MST}
\subsection{分析}
由于$E=O(V^2)$,显然不能使用常规Kruskal法。

考虑曼哈顿距离的特性,以每个点为原点建坐标系,将坐标系每$45^\circ$分成一块区域,
发现这个点到每个区域最多只会连离它最近的点,因此只需预处理每个点到八个区域的最近点,
边数缩减为$O(V)$。根据对称性可只考虑一半的区域,最多加入$4V$条边。

接下来思考如何查询最近点(仅处理右半区域,以下左右分别指逆时针和顺时针):

\begin{itemize}
	\item 首先考虑y正半轴向右$45^\circ$的点,发现若以点$P$为原点,
	      落在该区域的点$Q$满足:
	      \begin{eqnarray*}
		      x_P&\leq& x_Q\\
		      x_Q-y_Q&\leq& x_P-y_P\\
	      \end{eqnarray*}
	      仅一个不等式满足等号就足够了,因为其它区域可以覆盖到这个边界。

	      因为$|PQ|=(x_Q+y_Q)-(x_P+y_P)$,所以要查询满足要求且$x+y$的取最小时
	      对应的点。

	      按照$x$从大到小排序,对$x-y$离散,使用树状数组维护$x+y$前缀最小值与对应的点。
	\item 对于其它三个区域的点,可以使用坐标变换得到(连续变换):
	      \begin{enumerate}
		      \item y正半轴向右:直接做;
		      \item x正半轴向左:交换$x,y$;
		      \item x正半轴向右:$x$取反;
		      \item y负半轴向左:交换$x,y$。
	      \end{enumerate}
\end{itemize}

以上内容参考了GGBeng的博客\footnote{曼哈顿距离最小生成树
	\url{https://www.cnblogs.com/xzxl/p/7237246.html}
}。
\subsection{实现}
\begin{lstlisting}
struct Node {
    int id,val;
    Node(int id,int val):id(id),val(val) {}
    void update(const Node& rhs) {
        if(val>rhs.val)
            *this=rhs;
    }
} T[size];
void reset(int siz) {
    for(int i=1;i<=siz;++i)
        T[i].id=-1,T[i].val=1<<30;
}
void modify(int x,int siz,const Node& val) {
    while(x<=siz) {
        T[x].update(val);
        x+=x&-x;
    }
}
Node query(int x) {
    Node val(-1,1<<30);
    while(x) {
        val.update(T[x]);
        x-=x&-x;
    }
    return val;
}
int A[size];
int find(int x,int siz) {
    return std::lower_bound(A+1,A+siz+1,x)-A;
}
struct Pos {
    int x,y,id;
    bool operator<(const Pos& rhs) const {
        return x>rhs.x;
    }
} P[size];
struct Edge {
    int u,v,w;
    Edge(int u,int v,int w):u(u),v(v),w(w) {}
} E[4*size];
int buildGraph(int n) {
    int ecnt=0;
    for(int d=0;d<4;++d) {
        if(d==2) {
            for(int i=1;i<=n;++i)
                P[i].x=-P[i].x;
        }
        else if(d!=0) {
            for(int i=1;i<=n;++i)
                std::swap(P[i].x,P[i].y);
        }
        for(int i=1;i<=n;++i)
            A[i]=P[i].x-P[i].y;
        std::sort(A+1,A+n+1);
        int siz=std::unique(A+1,A+n+1)-(A+1);
        reset(siz);
        std::sort(P+1,P+n+1);
        for(int i=1;i<=n;++i) {
            int pos=find(P[i].x-P[i].y,siz);
            Node p=query(pos);
            int sum=P[i].x+P[i].y;
            if(p.id!=-1)
                E[++ecnt]=Edge(P[i].id,p.id,p.val-sum);
            modify(pos,Node(P[i].id,sum));
        }
    }
    return ecnt;
}
\end{lstlisting}
