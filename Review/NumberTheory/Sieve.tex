\section{低于线性时间复杂度筛法}
这类筛法主要用于计算大数据规模积性函数求和。
\subsection{约数函数前缀和}
求$\displaystyle \sum_{i=1}^n{\sigma(i)},n\leq 10^{12}$。
\begin{eqnarray*}
    \sum_{i=1}^n{\sigma(i)}&=&\sum_{i=1}^n{\sum_{d|i}d}\\
    &=&\sum_{d=1}^n{d[\frac{n}{d}]}
\end{eqnarray*}
由于$[\frac{n}{d}]$存在许多连续相同的值,使用根号分块法可做到$O(\sqrt{n})$。
\subsection{欧拉函数前缀和}
求$\displaystyle \sum_{i=1}^n{\varphi(i)},n\leq 10^{11}$。
由定理~\ref{PhiT}可得
$\displaystyle \varphi(n)=n-\sum_{d|n,d<n}{\varphi(d)}$。
\begin{eqnarray*}
    ans(n)&=&\sum_{i=1}^n{\varphi(i)}\\
    &=&\sum_{i=1}^n{\left(i-\sum_{d|i,d<i}{\varphi(d)}\right)}\\
    &=&\frac{n(n+1)}{2}-\sum_{i=2}^{n}{\sum_{d|i,d<i}{\varphi(d)}}\\
    &=&\frac{n(n+1)}{2}-\sum_{\frac{i}{d}=2}^n
    {\sum_{d=1}^{[\frac{n}{\frac{i}{d}}]}{\varphi(d)}}\\
    &=&\frac{n(n+1)}{2}-\sum_{t=2}^n
    {\sum_{d=1}^{[\frac{n}{t}]}{\varphi(d)}}\\
    &=&\frac{n(n+1)}{2}-\sum_{t=2}^n{ans([\frac{n}{t}])}
\end{eqnarray*}
同理使用分块+递归询问区间和来计算答案。为了降低复杂度,应该先线性筛预处理前一部分值。
当预处理$k=n^\frac{2}{3}$时可以取到复杂度$T(n)=O(n^\frac{2}{3})$。
\subsection{莫比乌斯函数前缀和}
求$\displaystyle \sum_{i=1}^n{\mu(i)},n\leq 10^{11}$。
由定理~\ref{MobiusT}可得
$\displaystyle \mu(n)=[n=1]-\sum_{d|n,d<n}{\mu(d)}$。
\begin{eqnarray*}
    ans(n)&=&\sum_{i=1}^n{\mu(i)}\\
    &=&\sum_{i=1}^n{\left([i=1]-\sum_{d|i,d<i}{\mu(d)}\right)}\\
    &=&1-\sum_{i=1}^n{\sum_{d|i,d<i}{\mu(d)}}\\
    &=&1-\sum_{t=2}^n{ans([\frac{n}{t}])}
\end{eqnarray*}
\subsection{其它函数前缀和}
主要思路是使用狄利克雷卷积构造出一个简单的前缀和函数,且用于卷积的另一个函数也容易计算。

令$\displaystyle A(n)=\sum_{i=1}^n\frac{i}{(n,i)}$,求
$\displaystyle \sum_{i=1}^n{A(n)},n\leq 10^{9}$。

先化简$A(n)$:
\begin{eqnarray*}
    A(n)&=&\sum_{i=1}^n\frac{i}{(n,i)}\\
    &=&\sum_{d|n}{\sum_{i=1}^n{[(n,i)=d]\cdot\frac{i}{d}}}\\
    &=&\sum_{d|n}{\sum_{\frac{i}{d}=1}^{\frac{n}{d}}
    {[(\frac{n}{d},\frac{i}{d})=1]\cdot\frac{i}{d}}}\\
    &=&\frac{1}{2}\left(1+\sum_{d|n}{d\cdot\varphi(d)}\right)
\end{eqnarray*}

那么答案即为$\displaystyle \frac{1}{2}\left(n+\sum_{t=1}^n
{\sum_{d=1}^{[\frac{n}{t}]}{d\cdot\varphi(d)}}\right)$。

考虑计算$\displaystyle \sum_{d=1}^n{d\cdot\varphi(d)}$的值:

易知$(id\cdot\varphi)*id=id^2$,因为\begin{displaymath}
    \sum_{d|n}d\cdot*\varphi(d)\cdot\frac{n}{d}=
    n\cdot\sum_{d|n}\varphi(d)=n^2
\end{displaymath}

所以有\begin{eqnarray*}
    \frac{n(n+1)(2n+1)}{6}&=&\sum_{i=1}^n{(id\cdot\varphi)*id}\\
    &=&\sum_{t=1}^n{t\cdot\sum_{d=1}^{[\frac{n}{t}]}{d\cdot\varphi(d)}}
\end{eqnarray*}

从等式中提出要求的那一项即可。
\subsection{复杂度分析}
为了得到较优的复杂度,需要设置合适的预处理大小。
\index{*TODO!筛法复杂度分析}

以上例题来自skywalkert的博客\footnote{浅谈一类积性函数的前缀和\\
\url{https://blog.csdn.net/skywalkert/article/details/50500009}}。
