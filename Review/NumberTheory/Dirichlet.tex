\section{狄利克雷卷积,狄利克雷逆与莫比乌斯反演}
\subsection{狄利克雷卷积}
\index{D!Dirichlet Convolution}
对于数论函数$f,g$,定义狄利克雷卷积
\begin{displaymath}
	(f*g)(n)=\sum_{d|n}{f(d)g(\frac{n}{d})}=\sum_{ab=n}{f(a)g(b)}
\end{displaymath}
由积性函数集合与狄利克雷卷积组成的群的乘法单位元为元函数$\epsilon$。

狄利克雷卷积有如下性质:
\begin{eqnarray*}
	\textrm{结合律} & (f*g)*h=f*(g*h)\\
	\textrm{分配律} & f*(g+h)=f*g+f*h\\
	\textrm{交换律} & f*g=g*f;\\
	\textrm{单位元} & f*\epsilon=\epsilon*f=f。
\end{eqnarray*}
\subsection{狄利克雷逆}
\index{D!Dirichlet Inverse}
已知数论函数$f$,求$g=f^{-1}$,满足$f*g=\epsilon$。
\begin{itemize}
	\item 当$n=1$时,有$(f*g)(1)=f(1)g(1)=\epsilon(1)=1$,
	      解得$g(1)=\frac{1}{f(1)}$。
	\item 当$n>1$时,
	      有$\displaystyle (f*g)(n)=\sum_{ab=n}{f(a)g(b)}=\epsilon(n)=0$,
	      解得$\displaystyle g(n)=\frac{-1}{f(1)}
		      \sum_{d|n,d<n}{f(\frac{n}{d})g(d)}$。
\end{itemize}
\subsubsection{狄利克雷逆性质}
\begin{property}
	积性函数的狄利克雷逆仍然是积性函数。
\end{property}
\begin{property}
	若数论函数$f,g$为积性函数,则$(f*g)^{-1}=f^{-1}*g^{-1}$。
\end{property}
\begin{property}\label{CMFP}
	积性函数$f$为完全积性函数当且仅当$f^{-1}(n)=\mu(n)f(n)$。
\end{property}
证明:\begin{eqnarray*}
	(f*f^{-1})(n)&=&\sum_{ab=n}{f(a)f^{-1}(b)}\\
	&=&\sum_{ab=n}(f(a)\mu(b)f(b))\\
	&=&f(n)\sum_{d|n}{\mu(d)}\\
	&=&f(n)\epsilon(n)\\
	&=&\epsilon(n)
\end{eqnarray*}
\subsubsection{常见数论函数及其狄利克雷逆}
\begin{itemize}
	\item $1*\mu=\epsilon$\\
	      参见定理~\ref{MobiusT}的证明。
	\item $id^\alpha*(\mu\cdot id^\alpha)=\epsilon$\\
	      根据性质~\ref{CMFP}可证明。
	\item $\displaystyle \varphi*(\sum_{d|n}{\mu(d)d})=\epsilon$\\
	      由定理~\ref{PhiT}可得$id=\varphi*1$,两边同时乘上$\mu$
	      可得$id*\mu=\varphi$,所以$\varphi^{-1}=id^{-1}*\mu^{-1}=id^{-1}*1$。
	\item $\sigma_\alpha*(\sum_{d|n}{\mu(d)\mu(\frac{n}{d})d^\alpha})=\epsilon$

	      $\sigma_\alpha=id^\alpha*1$可推出
	      $(\sigma_\alpha)^{-1}=(id^\alpha)^{-1}*\mu$
\end{itemize}
以上内容参考了Wikipedia-EN\footnote{Dirichlet convolution - Wikipedia\\
	\url{https://en.wikipedia.org/wiki/Dirichlet\_convolution}}。
\subsection{莫比乌斯反演}
\index{M!Möbius Inversion}
\begin{theorem}
	对于数论函数$f,g$,满足$\displaystyle g(n)=\sum_{d|n}f(d)$,则有
	\begin{displaymath}
		f(n)=\sum_{d|n}\mu(d)g(\frac{n}{d})
	\end{displaymath}
\end{theorem}
莫比乌斯反演可表示为若$g=f*1$则$f=\mu*g$。
证明:将$g=f*1$两边同时乘上$\mu$即可。
证明源自Wikipedia-EN\footnote{Möbius inversion formula - Wikipedia\\
	\url{https://en.wikipedia.org/wiki/Mobius_inversion}}。
\subsection{常见技巧}
\begin{itemize}
	\item
	      对于数论函数$g,f$,
	      \begin{displaymath}
		      g(n)=\sum_{n|d}{f(d)}\Rightarrow
		      f(n)=\sum_{n|d}{\mu(d)g(\frac{d}{n})}
	      \end{displaymath}
	\item
	      若$\displaystyle n=\prod_{i=1}^m{{p_i}^{c_i}},g(n)=\sum_{d|n}{f(d)}$
	      且$f$为积性函数,将$g$看做$f*1$可知$g$也是积性函数,则$g(n)=\prod_{i=1}^m
		      {\sum_{j=0}^{c_i}{f(p_i^j)}}$。
	\item 交换内外求和顺序。
	\item 枚举倍数,最大公约数等有共性的值并换元。
	\item 在化简前缀和函数时可能会遇到如下式子:
		\begin{eqnarray*}
			ans(n)&=&\sum_{i=1}^n{f(i)}\\
			&=&A(n)+B(n)\sum_{i=1}^n{\sum_{d|i}{f(d)}}\\
			&=&A(n)+B(n)\sum_{\frac{i}{d}=1}^n{\sum_{d=1}^{[\frac{n}{\frac{i}{d}}]}{f(d)}}\\
			&=&A(n)+B(n)\sum_{t=1}^n{\sum_{d=1}^{[\frac{n}{t}]}{f(d)}}\\
			&=&A(n)+B(n)\sum_{t=1}^n{ans([\frac{n}{t}])}
	      \end{eqnarray*}
		线性筛预处理一部分前缀和(一般预处理规模为$n^{2/3}$,最终时间复杂度
		$O(n^{2/3})$,大规模前缀和使用根号分块法递归计算。

		注意这里可以使用存储Trick来Cache计算结果(多次询问使用map或时间戳数组
		清零,下面只讨论单次询问的情况)。设预处理了前$k$个前缀和,其中$k\geq \sqrt{n}$。
		那么$[\frac{n}{t}]>k$的值不超过$\sqrt{n}$个,并且$t$不同对应的值也不同。所以
		可以以$t$为下标把计算结果存入另一个数组中。
	\item 同时除以最大公约数使其互质,然后套用$\varphi$。
	\item $\displaystyle [gcd(i,j)=1]=\sum_{k|gcd(i,j)}\mu(k)=
		      \sum_{k|i,k|j}\mu(k)$
	\item $\displaystyle \sum_{i=1}^n{i}=
		      \sum_{i=1}^n{\sum_{d|i}\varphi(d)}=
		      \sum_{d=1}^n{\varphi(d)\cdot[\frac{n}{d}]}$
	\item $(id\cdot\varphi)*id=id^2$
\end{itemize}
更多技巧待补充。
