\section{原根}\label{PrimitiveRoot}
\index{P!Primitive Root}
\subsection{基本定义与定理}
\subsubsection{数论阶}
设$n>1,(a,n)=1$,记$\delta_n(a)$为使得$a^r\equiv 1 \pmod{n}$
成立的最小正整数$r$,称为$a$对模$n$的阶。

\begin{theorem}
	设$n>1,(a,n)=1$,若$a^x\equiv 1 \pmod{n}$,则有$\delta_n(a)\mid x$。
\end{theorem}

\subsubsection{原根}
若$\delta_n(a)=\varphi(n)$,则称$a$为模$n$的一个原根。

若$a$为模$n$的原根,则对于$0\leq i< \varphi(n)$,有$a^i\bmod{n}$两两不同。

\begin{theorem}
	如果模$n$有原根,则它一共有$\varphi(\varphi(n))$个原根。
\end{theorem}

\begin{theorem}
	设$p$为奇素数,$n=2,4,2p^i,4p^i\Leftrightarrow$模$n$有原根。
\end{theorem}

\subsection{求模$n$的原根}

对$\varphi(n)$进行质因数分解,
对于$\displaystyle \varphi(n)=\prod_{i=1}^m{p_i^{c_i}}$,若
恒有$g^\frac{\varphi(n)}{p_i}\not\equiv 1 \pmod{n}$,则$g$为模$n$的一个原根。

以上内容参考了mosquito\_zm的博客\footnote{原根
	\url{https://blog.csdn.net/mosquito\_zm/article/details/77227570}}。
\subsection{原根的应用}
\begin{itemize}
	\item 在NTT中用于推算主单位根。
	\item 将乘积恒定转换为幂次和恒定后NTT(见例题[SDOI2015]序列统计)。
\end{itemize}
\paragraph{例题} [SDOI2015]序列统计\footnote{【P3321】[SDOI2015]序列统计 - 洛谷
\url{https://www.luogu.org/problemnew/show/P3321}}

将$x$映射为$g^i$,由生成函数法推导出要求多项式幂,最后将答案逆映射即可。

\lstinputlisting[title=luogu P3321]{Source/Source/'FFT NTT'/3321.cpp}
