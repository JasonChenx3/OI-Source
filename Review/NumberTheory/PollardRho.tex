\section{Pollard Rho启发式因子分解}
\index{P!Pollard Rho}
Pollard Rho算法可以期望在$\Theta(\sqrt{p})$次算术运算内得到$n$的一个小因子$p$。

\subsection{利用Birthday Trick提高效率}
\subsubsection{朴素随机算法}
考虑每次随机选择一个数$x\in [2,n-1]$,若$x|n$,则$x$为$n$的一个因子,最坏情况下
($n$为两素数之积)期望测试次数为$\frac{n-2}{2}$。
\subsubsection{Birthday Trick}
\index{B!Birthday Trick}
可以把问题转换为选取$k$个数$x$,每次询问是否存在$(x_i-x_j)|n$。
根据生日悖论,当$k$达到$\Theta(\sqrt{n})$级别时,可以期望得到一组$(i,j)$满足该条件。
具体证明参见算法导论\cite{ITA3}第5.4.1节中采用指示器随机变量的分析。

询问是否存在$(x_i-x_j)|n$仍然不够高效,可以转换为询问是否存在$gcd(x_i-x_j,n)>1$。
只需要选取约$\sqrt[4]{n}$个数(因子大小为$O(\sqrt{n})$)。

\index{*TODO!PollardRho时间复杂度证明}

Pollard Rho采取如下策略:
\begin{enumerate}
    \item 在区间$[2,n-1]$中随机选取$k$个数;
    \item 询问是否存在$gcd(x_i-x_j,n)>1$。
\end{enumerate}

在实践中把$k$个数存下是不可能的,可以每次把相邻的随机数当做$x_i,x_j$来测试。
有一个简单有效的伪随机数生成函数:$f(x)=(x^2+c) \bmod n$。

\subsection{Floyd判圈法}
某些输入可能会导致出现在找到某个因子前,伪随机数生成过程已掉入死循环的情况。
遇到这种情况,应该及时退出迭代,并修改随机数种子$x_0$与偏移$c$重新迭代。

如何判断是否掉入死循环呢?这里使用Floyd发明的方法:令$A_0=B_0=x_0$,同时迭代生成随机数,
$A_{i+1}=f(A_i),B_{i+1}=f(f(B_i))$,当$A_i=B_i$时,$B$至少比$A$多走了一圈,
说明已经掉入了循环。

代码实现:
\begin{lstlisting}[title=Pollard Rho]
int f(int x,int c,int n) {
    return (asInt64(x)*x+c)%n;
}
int pollardRhoImpl(int n,int seed) {
    int c=rand()%n;
    int a=f(seed,c,n),b=f(a,c,n);
    while(a!=b) {
        int d=gcd(iabs(a-b),n);
        if(d!=1 && d!=n)
            return d;
        a=f(a,c,n),b=f(f(b,c,n),c,n);
    }
    return 0;
}
int pollardRho(int x) {
    int d=0;
    do d=pollardRhoImpl(x,rand()%n);
    while(!d);
    return d;
}
\end{lstlisting}

求gcd时可以累乘多次一并求gcd以减少常数。

以上内容参考了{\bfseries A Quick Tutorial on Pollard's Rho Algorithm}
\footnote{原文地址
\url{www.cs.colorado.edu/~srirams/classes/doku.php/pollard\_rho\_tutorial}
\\中文翻译
\url{http://files.cnblogs.com/files/Doggu/Pollard-rho算法详解.pdf}
}。
