\section{辗转相除法GCD}
\subsection{裴蜀定理}
\index{B!Bézout's Theorem}
\begin{theorem}[Bézout's Theorem]\label{BT}
	对于任意$a,b\in \mathbb{Z}$,关于$x,y$的线性不定方程(裴蜀方程)
	$ax+by=c$有无穷多整数解$(x,y)$当且仅当$(a,b)|c$。特别地,
	一定存在$(x,y)$使得$ax+by=(a,b)$成立。
\end{theorem}

由此可得推论:

\begin{inference}
	$a,b$互质的充要条件是存在整数$(x,y)$使得$ax+by=1$。
\end{inference}

接下来证明一定存在$(x,y)$使得$ax+by=(a,b)$成立:

设$s$是$a$和$b$线性组合集中的最小正元素,对于某个整数组$(x,y)$有$ax+by=s$,
令$q=[a/s],r=a \bmod s=a-q(ax+by)=a(1-qx)+b(-qy)$,所以$r$也是一个线性组合。
因为$s$是线性组合集中的最小正元素,且$0\leq r \le s$,所以$r=0$,可得$s|a$。
同理$s|b$,因此$s$是$a,b$的公约数,可得$(a,b) \geq s$。因为$(a,b)|ax+by$
且$s>0$,所以$(a,b) \leq s$。结合$(a,b) \geq s$与$(a,b) \leq s$可得
$s=(a,b)$。

至于无穷多整数解嘛。。。拿最小公倍数调一调初始解$(x,y)$即可。

证明参考了霜刃未曾试的博客\footnote{关于裴蜀定理的一些证明\\
	\url{https://blog.csdn.net/discreeter/article/details/69833579}}与
算法导论\cite{ITA3}第31.1节定理31.2的证明。
\subsection{exgcd}
由定理~\ref{BT}可知一定存在整数解$(x,y)$满足$ax+by=(a,b)$,如何构造
出一组解呢?

$exgcd$(扩展欧几里得算法)可求出一组特殊的整数解。

先上代码:

\begin{lstlisting}[title=exgcd]
    void exgcd(int a,int b,int& x,int& y,int& d) {
        if(b) {
            exgcd(b,a%b,y,x,d);
            y-=a/b*x;
        }
        else x=1,y=0,d=a;
    }
\end{lstlisting}

该算法由朴素$gcd$修改而来,因此同样讨论两种情况:

\begin{itemize}
	\item $b=0$时,$gcd$值为$a$,因此有$a=1*a+0*b$。
	\item $b\neq 0$时,考虑递归计算返回的一组解$(y',x')$满足
	      $by'+(a-[\frac{a}{b}]b)x'=d$,可变形为$ax'+b(y'-[\frac{a}{b}]x')=d$,
	      因此本次递归返回$(x',y'-[\frac{a}{b}]x')$。
\end{itemize}

\subsection{位运算gcd}

本节内容源自算法导论\cite{ITA3}思考题33-1。

\subsubsection{原理}

首先为了避免除法,将辗转相除法改为更相减损术,可以利用
$gcd$函数的以下性质:

\begin{itemize}
	\item \begin{property}\label{GCDC1}
		      若$a,b$均为偶数,则$gcd(a,b)=2*gcd(a/2,b/2)$。
	      \end{property}
	\item \begin{property}\label{GCDC2}
		      若$a$是奇数,$b$是偶数,则$gcd(a,b)=gcd(a,b/2)$。
	      \end{property}
	\item \begin{property}\label{GCDC3}
		      若$a,b$均为奇数,则$gcd(a,b)=gcd((a-b)/2,b)$。
	      \end{property}
\end{itemize}

特别地,当$gcd(x,y)$的参数中存在0,则返回$x|y$。

算法步骤如下:
\begin{enumerate}
	\item 特判$x,y$中存在0的情况;
	\item 根据性质~\ref{GCDC1},先消去$a,b$的2的幂的公因子,记录幂次$k$;
	\item 使$a$变为奇数以便于利用性质~\ref{GCDC2},同时去除2的幂的
	      公因子避免重复计算;
	\item 利用性质~\ref{GCDC2}使$b$变为奇数;
	\item 保持$a<b$以便利用性质~\ref{GCDC3};
	\item 利用性质~\ref{GCDC3}使$b'=b-a$;
	\item 若此时$b$为0,则返回$a*2^k$,否则重复第4步。
\end{enumerate}

实际上就是利用性质~\ref{GCDC1}与~\ref{GCDC2}对更相减损术进行了优化。

\subsubsection{位扫描优化}

如果某个数末尾有多个0,则可以直接使用右移k位代替不断右移1位。
下面是统计末尾0的个数k的方法:

\begin{itemize}
	\item GCC自带了对位扫描指令的封装,即$\_\_builtin\_$系列函数,
	      直接使用$\_\_builtin\_ctz$函数即可。
	\item Sean Eron Anderson 的\emph{Bit Twiddling Hacks}
	      \footnote{\url{http://graphics.stanford.edu/~seander/bithacks.html}}
	      中Counting consecutive trailing zero bits (or finding bit indices)
	      一节介绍了计算末尾0个数的多种方法,这里只给出multiply and lookup法:
	      \begin{lstlisting}[title=countTZ]
    int countTZ(unsigned int x) {
        static const int LUT[32] = {
            0, 1, 28, 2, 29, 14, 24, 3,
            30, 22, 20, 15, 25, 17, 4, 8,
            31, 27, 13, 23, 21, 19, 16, 7,
            26, 12, 18, 6, 11, 5, 10, 9
        };
        return LUT[(((v & -v) * 0x077CB531U)) >> 27];
    }
    \end{lstlisting}
	      这里的神奇常数{\bfseries 0x077CB531U}与De Bruijn Sequences
	      \index{D!De Bruijn Sequences}有关,在此不详细介绍。
	      LUT可预处理得出(但是不太好记住0x077CB531U)。
	      \index{*Constant!De Bruijn Sequences常数\\{\bfseries 0x077CB531U}}
\end{itemize}

\subsubsection{实现}

设$conutTZ(x)$返回$x$末尾0的个数,代码如下:

\begin{lstlisting}[title=gcdX]
int gcdX(int a,int b) {
    if(a && b) {
        int off=countTZ(a|b);
        a>>=countTZ(a);
        do {
            b>>=countTZ(b);
            if(a>b)std::swap(a,b);
            b-=a;
        }while(b);
        return a<<off;
    }
    return a|b;
}
\end{lstlisting}
