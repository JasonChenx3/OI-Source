\section{欧拉定理}
\subsection{费马小定理}\label{FLTS}
\index{F!Fermat's Little Theorem}
\begin{theorem}[Fermat's Little Theorem]\label{FLT}
	~\\
	$\forall p \textrm{ is a prime number},a\in \mathbb{Z},a^p \equiv a \pmod{p}$
\end{theorem}

该定理是定理~\ref{ET}的特殊化,不证。

由定理~\ref{FLT}可得:

\begin{inference}
	$a^{-1} \equiv a^{p-2} \pmod{p}$
\end{inference}

可使用快速幂在$O(lgp)$的复杂度下求某个数的逆元。

\subsection{线性推逆元}

如果需要获得$a\in [1,p)$内模$p$的逆元,复杂度为$O(plgp)$的逐个快速幂
并不是最优方法。

首先有$1^{-1}\equiv 1 \pmod{p}$。

令$p=qa+r$,其中$q=[\frac{p}{a}],r=p \bmod a$。

再把$p \equiv 0 \pmod{q}$中的$p$用$qa+r$代替,两边同时乘上$(ar)^{-1}$,
移项得$a^{-1}\equiv -qr^{-1} \pmod{q}$,即
$a^{-1}\equiv -[\frac{p}{a}](p \bmod a)^{-1} \pmod{p}$。

代码如下:
\begin{lstlisting}[title=inv]
inv[1]=1;
for(int i=2;i<=n;++i)
    inv[i]=asInt64(mod-mod/i)*inv[mod%i]%mod;
\end{lstlisting}

以上内容参考了Miskcoo的博客\footnote{[数论]线性求所有逆元的方法 – Miskcoo's Space
	\url{http://blog.miskcoo.com/2014/09/linear-find-all-invert}}

\subsection{欧拉定理}
\index{E!Euler's Theorem}
\begin{theorem}[Euler's Theorem]\label{ET}
	~\\
	对于任意互质正整数对$(a,n)$,有$a^{\varphi(n)} \equiv 1 \pmod{n}$
\end{theorem}
证明:

令$S=\{[x]_n\in Z_n|(a,n)=1\}$(由与$n$互质的模$n$剩余类组成的集合),
它与$\cdot_n$构成整数模$n$乘法群,$(S,\cdot_n)$的阶为$\varphi(n)$。

接着有两种证明思路:
\begin{itemize}
	\item 对于任意一个与$n$互质的正整数$a$,$a$的幂模$n$的值$a,a^2,\cdots,a^k$
	      构成了一个子群,其中$a^k\equiv 1 \pmod{n}$。

	      根据定理~\ref{LT},有$k|\varphi(n)$,令$M=\varphi(n)/k$,有
		  $a^{\varphi(n)}=a^{kM}=(a^k)^M\equiv 1^M\equiv 1 \pmod{n}$。
	\item 根据定义得对于$[x]_n\in S$和$S$中的所有元素$[a_1]_n,[a_2]_n,\cdots,
	[a_{\varphi(n)}]_n$,$[x]_n \cdot_n [a_i]_n$组成的集合仍然是$S$,因此有$x^{\varphi(n)}
	[a_1]_n[a_2]_n\cdots[a_{\varphi(n)}]_n= (x[a_1]_n)(x[a_2]_n)\cdots
	(x[a_{\varphi(n)}]_n)\equiv[a_1]_n[a_2]_n\cdots[a_{\varphi(n)}]_n\pmod{n}$,
	两边消去可得$x^{\varphi(n)}\equiv 1\pmod{n}$。
\end{itemize}

上述证明源自Wikipedia-EN\footnote{Euler's theorem - Wikipedia
	\url{https://en.wikipedia.org/wiki/Euler's\_theorem}}和Eden Harder
	的博客\footnote{RSA 加密周边 - Eden Harder
		\url{http://edenharder.is-programmer.com/posts/43247.html}}。
\subsection{扩展欧拉定理}
\begin{theorem}\label{exEuler}
	$\forall a\in \mathbb{Z},x,m\in \mathbb{Z^+},x\geq \varphi(m)
		,a^x\equiv a^{x \bmod \varphi(m)+\varphi(m)} \pmod{m}$
\end{theorem}

\begin{lemma}\label{EEL1}
	\begin{displaymath}
		\left\{
		\begin{array}{l}
			x\equiv y \pmod{m_1} \\
			x\equiv y \pmod{m_2}
		\end{array}
		\right.
		\Rightarrow x\equiv y \pmod{lcm(m_1,m_2)}
	\end{displaymath}
\end{lemma}

证明:
\begin{displaymath}
	\left\{
	\begin{array}{l}
		x\equiv y \pmod{m_1} \\
		x\equiv y \pmod{m_2}
	\end{array}
	\right.
	\Rightarrow
	\left\{
	\begin{array}{l}
		x+c_1m_1=y \\
		x+c_2m_2=y
	\end{array}
	\right.
\end{displaymath}
\begin{displaymath}
	\Rightarrow
	c_1m_1=c_2m_2=k\cdot lcm(m_1,m_2)
	\Rightarrow
\end{displaymath}
\begin{displaymath}
	x \equiv y \pmod{lcm(m_1,m_2)}
\end{displaymath}

\begin{inference}
	当$a,b$互质时,$x\equiv y \pmod{ab}$
\end{inference}

\begin{inference}
	\begin{displaymath}
		\left\{
		\begin{array}{l}
			x\equiv y \pmod{m_1} \\
			\cdots               \\
			x\equiv y \pmod{m_n}
		\end{array}
		\right.
		\Rightarrow x\equiv y \pmod{lcm(m_1,\cdots,m_n)}
	\end{displaymath}
\end{inference}

\begin{lemma}\label{EEL2}
	\begin{displaymath}
		\forall p\textrm{ is a prime number},q\in \mathbb{Z^+},q>1,
		\varphi(p^q)\geq q.
	\end{displaymath}
\end{lemma}

证明:首先有$\varphi(p^q)=(p-1)p^{q-1}$,当$p$固定时,$q$取2使得$\varphi(p^q)-q$
最小,但该值仍非负。当且仅当$p=2,q=2$时,$\varphi(p^q)=q$。

接下来证明定理~\ref{exEuler}:

首先证明当$m$为素数$p$的幂$(m=p^q)$时成立:
\begin{itemize}
	\item 若$gcd(a,p)=1$,则$gcd(a,p^q)=1$,根据欧拉定理可证在该情况下成立;
	\item 若$gcd(a,p)=p$,由适用范围可知$x\geq q$,由引理~\ref{EEL2}可知
	      $x \bmod \varphi(p^q) + \varphi(p^q) \geq q$,因此
	      $a^x\equiv 0 \equiv a^{x \bmod \varphi(p^q)+\varphi(p^q)} \pmod{p^q}$
\end{itemize}

对于任意$m$,可根据算术基本定理将其分解为素数幂之积。因为$\varphi(p^q)|\varphi(m)$,所以有
$a^x\equiv a^{x \bmod \varphi(p^q)+\varphi(p^q)}
	\equiv a^{x \bmod \varphi(m)+\varphi(m)} \pmod{p^q}$。
根据引理~\ref{EEL1}及其推论合并这些式子可证明该定理。

以上证明源自后缀自动机·张的文章\footnote{微小的欧拉定理EXT证明
	\url{https://zhuanlan.zhihu.com/p/24902174}}。
