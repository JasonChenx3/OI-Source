\section{中国剩余定理CRT}
\subsection{CRT}
\index{C!Chinese Remainder Theorem}
\begin{theorem}[Chinese Remainder Theorem]
    对于模线性方程组:
    \begin{displaymath}
        \left\{\begin{array}{l}
            x \equiv a_1 \pmod{n_1}\\
            x \equiv a_2 \pmod{n_2}\\
            \ldots\\
            x \equiv a_k \pmod{n_k}
        \end{array}\right.
    \end{displaymath}\\
    其中$n_1,n_2,\ldots,n_k$两两互质,令$N=\prod_{i=1}^k{n_i}$,
    该模线性方程组在$[0,N)$内有唯一解。
\end{theorem}
如何求解该线性方程组呢?和拉格朗日插值法的思路相同,对于每一个方程都给最终的解
贡献一个$x_i$,满足
\begin{displaymath}
x_i \bmod n_j =
\left\{\begin{array}{ll}
0 & \textrm{if $i\neq j$}\\
a_i & \textrm{if $i=j$}
\end{array}\right.
\end{displaymath}
答案即为$\sum_{i=1}^n{x_i} \bmod N$。
考虑$i\neq j$时$x_i$应该整除$n_j$,因此$x_i$应该有系数$M=N/n_i$;当$i=j$时,
$x_i$应该有系数$a_i$,为了抵消$M$带来的影响,再乘上$M$模$n_i$的乘法逆元即可(
由于$n$两两互质,$M$与$n_i$也互质,根据定理~\ref{ET},保证其乘法逆元存在)。
\subsection{ExCRT}
当$n$不满足两两互质的条件时,可能会找不到其乘法逆元。
所以我们采用另一种思路求解方程:每次选择两个方程将其合并,直到只剩一个方程为止。

考虑两个方程组成的方程组:
\begin{displaymath}
    \left\{\begin{array}{l}
        x \equiv a_1 \pmod{n_1}\\
        x \equiv a_2 \pmod{n_2}\\
    \end{array}\right.
\end{displaymath}
等价于
\begin{eqnarray}
    x=a_1+k_1n_1\label{CRTE}\\
    x=a_2+k_2n_2
\end{eqnarray}
移项得$k_1n_1-k_2n_2=a_2-a_1$,可以使用
$exgcd$求出$c_1n_1+c_2n_2=gcd(n_1,n_2)$的各项参数。根据定理~\ref{BT},
若$gcd(n_1,n_2)\not\mid(a_2-a_1)$则该方程组无解。等比例缩放方程求出$k1$,
带入方程~\ref{CRTE}反推出$x0$,得到新的模线性方程$x \equiv x0
\pmod{lcm(n_1,n_2)}$。
