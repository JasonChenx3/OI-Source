\section{BSGS}
\index{B!Baby Step Giant Step}
BSGS法(Baby Step Giant Step)用来求解类似于$a^x\equiv b\pmod{P}$的方程。
\subsection{BSGS}
普通BSGS仅考虑$P$为质数的情况。

以下为求解最小非负整数解的方法:

首先根据定理~\ref{ET}可知$x$的最小非负整数解小于$\varphi(P)=P-1$。将x的值域
分为高低两部分,分别用$O(\sqrt{P})$的复杂度枚举值即可。

\begin{enumerate}
    \item 若$a$为$P$的倍数,则特判$b$是否为$0$,算法结束;
    \item 令$m=\lceil\sqrt{P}\rceil,x=im-j$,移项得$x^im\equiv bx^j\pmod{P}$;
    \item 枚举$bx^j$的值,按$j$从小到大覆盖存入HashTable;
    \item 枚举$(x^m)^i$的值,按$i$从小到大在HashTable中查询,存在则返回$im-j$;
    \item 返回无解。
\end{enumerate}

\subsection{ExBSGS}
ExBSGS可解决$a,P$不互质的问题。主要思路是将原方程化为普通BSGS可解决的方程。

记化简后方程为$Aa^{x-B}\equiv b\pmod{P}$,化简步骤如下:
\begin{enumerate}
    \item 将$A,B$初始化为$1,0$;
    \item 令$d=(a,P)$,
    \begin{itemize}
        \item 若$d\mid b$,则提出一个$d$,即$A*=a/d,b/=d,P/=d,++B$;
        \item 若$d\nmid b$,则特判$b$是否为$A$,$b=A$则$x=B$,$b\neq A$则无解;
    \end{itemize}
    \item 重复第2步直至$d=1$。
\end{enumerate}
令$x=im+j+B$然后按照普通BSGS做即可,注意在BSGS前要暴力检查$x\in[0,B)$的解。

以上内容参考了ZigZagK的博客\footnote{BSGS及扩展BSGS
\url{https://blog.csdn.net/zzkksunboy/article/details/73162229}}。
