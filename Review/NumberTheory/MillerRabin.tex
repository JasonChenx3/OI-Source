\section{Miller Rabin随机性素性测试}
\index{M!Miller Rabin}
\subsection{朴素算法}
根据~\ref{FLTS}中所述的费马定理,若要测试$p$是否为素数,
选取基数$a\in [2,p)$,检查$a^(p-1)\equiv 1 \pmod{p}$是否对
所有的$a$均成立即可。
\subsection{Miller-Rabin}
考虑每次随机选取多个$a$,当有$k$个$a$满足时,出错的概率最多为$2^{-k}$,详见
算法导论\cite{ITA3}第31.8节定理31.39。
Miller-Rabin使用了上述思想,并且使用$witness$测试来代替朴素的检查算法以避免把
\emph{Carmichael}数当做素数。

$witness(x,base)$返回$true$当$base$可以证明$x$是合数。
\begin{lstlisting}[title=witness]
bool witness(int x, int base) {
    int end = x - 1;
    int c = countTZ(end);
    int t = powm(base, end >> c, x);
    while(c--) {
        int ct = mulm(t, t, x);
        if(t != 1 && t != x - 1 && ct == 1)
            return true;//case 1
        t = ct;
    }
    return t != 1;//case 2
}
\end{lstlisting}
接下来证明正确性:
\begin{itemize}
	\item 如果从case~1处返回$true$,则说明找到了模$x$意义
            下$1$的一个非平凡平方根。

          \begin{theorem}\label{WITNESST}
              如果p是一个奇素数且$e\geq 1$,则方程

              \begin{equation}\label{sqrm}
                   x^2 \equiv 1 \pmod{p^e}
              \end{equation}
              只有两个解$x=\pm 1$。
	      \end{theorem}

          证明:方程~\ref{sqrm}等价于$p^e|(x-1)(x+1)$。因为$p \ge 2$,所以
          $p|(x-1)$与$p|(x+1)$仅有一个成立(否则$p|((x+1)-(x-1))=2$),
          两个解为$x=\pm 1$。

	      \begin{inference}
		      如果模$n$意义下存在$1$的非平凡平方根,则$n$为合数。
	      \end{inference}

	      证明:根据定理~\ref{WITNESST}的逆否命题,$n$不可能为奇素数的幂,且对于
	      $n=1,2$均不存在非平方平方根,因此$n$必为合数。

	      根据该推论可得case~1有证据证明$x$为合数。
	\item 如果从case~2处返回$true$,则说明$x$不满足费马定理。
\end{itemize}
以上内容参考了算法导论\cite{ITA3}第31.8节。
\subsection{实现细节}
\begin{itemize}
    \item 在MillerRabin之前可先用前几个质数筛掉大部分的合数。
    \item 如果数据范围在4759123141内,相当于$uint32_t$范围内,
    只用2,7,61为基数即可。
    \item 如果数据范围在$10^{16}$内,使用2,3,7,61,24251为基数,
    唯一的强伪素数为46856248255981。
\end{itemize}

以上内容参考了Matrix67的博客\footnote{
	数论部分第一节:素数与素性测试 | Matrix67: The Aha Moments
	\url{http://www.matrix67.com/blog/archives/234}}。
