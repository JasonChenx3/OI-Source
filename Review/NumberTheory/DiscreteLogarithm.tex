\section{离散对数问题}
\index{D!Discrete Logarithm}
离散对数问题是求出方程$a^x\equiv b\pmod{p}$的可行解/最小非负整数解。
\subsection{BSGS}\label{BSGS}
\index{B!Baby Step Giant Step}
\subsubsection{BSGS}
普通BSGS仅考虑$P$为素数的情况。

以下为求解最小非负整数解的方法:

首先根据定理~\ref{ET}可知$x$的最小非负整数解小于$\varphi(P)=P-1$。将x表示为
$\sqrt{P}$进制数,分别用$O(\sqrt{P})$的复杂度枚举值,一半存入HashTable,另一半
查询是否有匹配值。注意参数的枚举顺序。

\begin{enumerate}
    \item 若$a$为$P$的倍数,则特判$b$是否为$0$,算法结束;
    \item 令$m=\lceil\sqrt{P}\rceil,x=im-j$,移项得$a^{im}\equiv ba^j\pmod{P}$;
    \item 枚举$ba^j$的值,按$j$从小到大{\bfseries 覆盖}存入HashTable;
    \item 枚举$(a^m)^i$的值,按$i$从小到大在HashTable中查询,存在则返回$im-j$;
    \item 返回无解。
\end{enumerate}

{\bfseries 今后将采用~\ref{HashTable}节的双重散列+开放寻址法,严禁
使用std::unordered\_map。}
\subsubsection{ExBSGS}
ExBSGS可解决$a,P$不互质的问题。主要思路是将原方程化为普通BSGS可解决的方程。

记化简后方程为$Aa^{x-B}\equiv b\pmod{P}$,化简步骤如下:
\begin{enumerate}
    \item 将$A,B$初始化为$1,0$;
    \item 令$d=(a,P)$,
    \begin{itemize}
        \item 若$d\mid b$,则提出一个因子$d$,即$A*=a/d,b/=d,P/=d,++B$;
        \item 若$d\nmid b$,则特判$b$是否为$A$,$b=A$则$x=B$,$b\neq A$则无解;
    \end{itemize}
    \item 重复第2步直至$d=1$。
\end{enumerate}
令$x=im-j+B$转化为普通BSGS,{\bfseries 注意在BSGS前要暴力检查$x\in[0,B)$是否可行}。

代码如下:
\lstinputlisting{Source/Templates/exBSGS.cpp}

以上内容参考了ZigZagK的博客\footnote{BSGS及扩展BSGS\\
\url{https://blog.csdn.net/zzkksunboy/article/details/73162229}}。
\subsection{单有原根模数多询问离散对数问题}
对于这类问题,每次都计算一次HashTable十分浪费。
考虑求出原根$g$,令$a=g^A,b=g^B$,原式转化为$Ax\equiv B\pmod{\varphi(n)}$,
使用exgcd快速解决。由于求$a,b$的对数时的底数固定为$g$,可以预处理$g$的幂,也可以
预处理$g$的BSGS哈希表(哈希表的大小不必开根号,可以根据模数与询问规模决定)。

若模数$p$非常大,参见LOJ6542\url{https://loj.ac/problem/6542},留坑待补。

\index{*TODO!单模数多询问离散对数}
\subsection{Pollard rho算法}
\index{P!Pollard's rho Algorithm\\ for Logarithms}

该算法的主要思路在于找到两组指数$(y_1,y_2),(z_1,z_2)$,使得
$a^{y_1}b^{y_2}\equiv a^{z_1}b^{z_2}$。然后将其转化为求解模线性方程组
$(z_2-y_2)x\equiv y_1-z_1\pmod{N}$,这里的$N$是循环群的阶。exgcd轻松解决,
不过需要注意的是要检查每个周期的解的合法性(最小非负整数解不一定是可行解)。

若模数有原根且不知道循环群的阶,可以求出$n$的原根$g$,那么阶$N=\varphi(n)$。

找指数的过程类似于因子分解的PollardRho算法,即使用随机函数来生成两组指数,
同时使用Folyd判圈法终止算法。

具体做法是将当前乘积$v$的集合划分到3个集合,不同集合的行为不同,分别为$a$的指数+1,$b$
的指数+1,$a,b$的指数翻倍。

该算法的期望复杂度是$O(\sqrt{n})$,代码量小,但是慢于BSGS。不过$O(1)$的空间使得它
在求解$n$较大的情况时较BSGS有绝对优势。

代码如下:
\lstinputlisting{Source/Templates/PollardRhoDL.cpp}

上述内容参考了Wikipedia-EN\footnote{
    Pollard's rho algorithm for logarithms
    \url{https://en.wikipedia.org/wiki/Pollard's\_rho\_algorithm\_for\_logarithms}
}。
\subsection{Pohlig–Hellman算法}
\index{P!Pohlig–Hellman Algorithm}
该算法的思路基于群论,需要知道循环群的阶。如果循环群的阶是一个smooth number,即可以
分解为小素数幂之积,就能够将其分解为小规模的子问题。如果模数有原根,就可以沿用上面的做法。
记循环群的阶$n=\displaystyle \prod{p_i^{e^i}}$,该算法可以将复杂度降到
$O(\displaystyle \sum{e_i(\lg n+\sqrt{p_i})})$,空间复杂度$O(\sqrt{p_{max}})$。

\subsubsection{阶为素数幂的情况}

记素数幂为$p^e$,算法步骤如下:

\begin{enumerate}
    \item 初始化$x_0=0$。
    \item 计算$G=g^{p^{e-1}}$,根据定理~\ref{LT},生成元$G$对应的循环群的阶为$p$。
    \item 对于$k=0,1,\cdots,e-1$,计算$b_k=(g^{-x_k}b)^{p^{e-1-k}}$,BSGS求解
    $G^{d_k}\equiv b_k\pmod{n}$(注意模数是$n$,阶为$p$,时空复杂度$O(\sqrt{p}$)
    ,最后累加$x_{k+1}=x_k+p^kd_k$。
    \item 返回$x_e$。
\end{enumerate}

算法正确性留坑待补。
\index{*TODO!Pohlig-Hellman算法正确性证明}

\subsubsection{一般情况}
首先将阶$n$因式分解,求出模数的原根$g$。然后逐个计算$g_i=g^{n/p_i^{e_i}}$,构造出
阶为$p^e$形式的群,对应地$b_i=b^{n/p_i^{e_i}}$,使用上述方法求出
$g^{x_i}\equiv b_i\pmod{p_i^{e^i}}$的解,最后使用CRT合并答案。

代码如下(针对单模数多询问重新平衡,以空间换时间):
\lstinputlisting{Source/Templates/PohligHellman.cpp}

上述内容参考了Wikipedia-EN\footnote{
    Pohlig–Hellman algorithm\\
    \url{https://en.wikipedia.org/wiki/Pohlig-Hellman\_algorithm}
}。
\subsection{对1求离散对数}
沿用判断原根的思路,可行解$x$一定是$\varphi(n)$的某个因子,最坏复杂度
$O(\sqrt{n}\lg n)$,一般情况下比BSGS更加简洁高效。

该方法源自Dance Of Faith的博客\footnote{
    【科技】原根的快速判断方法以及对1求离散对数的另一种想法%\\
    \url{https://www.cnblogs.com/Dance-Of-Faith/p/9905786.html}
}。
