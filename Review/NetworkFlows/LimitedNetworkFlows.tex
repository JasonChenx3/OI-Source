\section{带上下界网络流}
前置知识:最大流与费用流。
\subsection{无源汇有上下界可行流}\label{LFA}
最大流只能求带上界的网络流,考虑通过修改建图方式使边的流量下界为0,流量上界对应变为
上界-下界。若在修改后的图中,边$e$使流量从点$u$流出$f$个单位,则实际上边$e$通过
$f+low_e$单位的流量,这时需要额外给点$u$提供$low_e$单位的流量,同理流入流量时也需要
额外接收$low_e$单位的流量。考虑设立虚拟节点$S,T$与顶点连附加边额外补偿/接收顶点的额外
流量,最后做$S-T$的最大流使附加边满流。若附加边满流则说明可以满足每条边的流量下界限制,
存在可行解。此时每条边的实际流量即为图中流量+流量下界。
建图:对于一条边$(u,v,low,up)$,连边$(S,v,low),(u,T,low),(u,v,up-low)$。

\subsubsection{建图优化}

普通方法建图需要连$3E$条边,但我们发现如果顶点$u$同时连接$S,T$,两条附加边的流量
可以抵消掉一部分。因此可以在建图时预处理数组$d[i]$,维护$S$到$v$的流量与$v$
到$T$的流量之差,最后$O(n)$选择$S$或$T$连边。使用此方法可以将边数降到$V+E$。

\subsubsection{费用流}

如果按照常规方法连边,由于附加边的费用不相等,不能使用上述优化合并附加边。由于当且仅当
附加边满流时才存在可行解,可以先算上附加边的费用,连附加边时把费用当成0,就可以继续使用
建图优化了。

\subsection{有上下界带源汇点可行流}

若点$u$需要凭空生成流量$f$,从$S$向$u$连流量为$f$的边;
若点$u$需要凭空消灭流量$f$,从$u$向$T$连流量为$f$的边;
其余做法与子节~\ref{LFA}相同。

\subsection{有上下界带一组无限流量源汇点可行流}

易知$S$的流出等于$T$的流入,因此加边$(T,S,inf)$后做法与
子节~\ref{LFA}相同。最终该边的流量即为$S$到$T$的总流量。

\subsubsection{最大流}

\begin{enumerate}
    \item 先求出可行流;
    \item 在残量网络上跑$S->T$的最大流;
    \item 答案即为可行流中$T->S$的流量+最大流流量。
\end{enumerate}

正确性证明:

\begin{property}\label{LFB}
    求解最大流时不会修改源汇点路径之外的边的流量。
\end{property}

因此步骤2仍然能够保证该流是一个可行流(附加边不变)。

\subsubsection{最小流}

\begin{enumerate}
    \item 先求出可行流;
    \item 在残量网络上跑$T->S$的最大流;
    \item 答案即为可行流中$T->S$的流量-最大流流量。
    $T->S$的最大流相当于将$S->T$的流量减至最小。
\end{enumerate}

利用性质~\ref{LFB}我们仍然能保证该流是一个可行流。

上述内容参考了F.W.Nietzsche\footnote{有上下界的、有多组源汇的、网络流、费用流问题 - F.W.Nietzsche
\url{https://www.cnblogs.com/nietzsche-oier/p/8185805.html}}与
liu\_runda\footnote{有上下界的网络流学习笔记 - liu\_runda
\url{https://www.cnblogs.com/liu-runda/p/6262832.html}}的博客。
