\section{带上下界网络流}
前置知识:最大流与费用流。
\subsection{无源汇有上下界可行流}\label{LFA}
最大流只能求带上界的网络流,因此可以通过设立虚拟节点$S,T$额外给顶点
补偿流量下界,然后就可以让下界变为0,做$S-T$的最大流即可。
即对于一条边$(u,v,low,up)$,连边$(S,v,low),(u,T,low),(u,v,up-low)$。

做最大流是为了使附加边满流,也就是可以满足每条边的流量下界。每条边的实际流量即
为图中流量+流量下界。

\subsubsection{优化}

普通方法建图需要连$3E$条边,但我们发现如果顶点$u$同时连接$S,T$,那么两边的流量
是可以同时抵消掉的,因此建图时可以预处理一个数组$d[i]$,表示$S$到$v$的流量与$v$
到$T$的流量之差,最后$O(n)$选择$S$或$T$连边即可。此方法可以讲边数降到$V+E$。

\subsubsection{费用流}

如果按照常规方法连边,就不能使用上述优化。由于只有当附加边满流才合法,因此可以先
算上附加边的费用,然后连边时把费用当成0,就可以使用优化了。

\subsection{有上下界带源汇点可行流}

从虚拟源汇点向源汇点连边即可,其余做法与子节~\ref{LFA}相同。

\subsection{有上下界带一组无限流量源汇点可行流}

易知$S$的流出到$T$的流入,因此加边$(T,S,inf)$后做法与
子节~\ref{LFA}相同。最终该边的流量即为$S$到$T$的总流量。

\subsubsection{最大流}

\begin{enumerate}
    \item 先求出可行流;
    \item 在残量网络上跑$S->T$的最大流;
    \item 答案即为可行流中$T->S$的流量+最大流流量。
\end{enumerate}

正确性证明:

\begin{property}\label{LFB}
    求解最大流时不会修改源汇点路径之外的边的流量。
\end{property}

因此步骤2仍然能够保证该流是一个可行流(附加边不变)。

\subsubsection{最小流}

\begin{enumerate}
    \item 先求出可行流;
    \item 在残量网络上跑$T->S$的最大流;
    \item 答案即为可行流中$T->S$的流量-最大流流量。
    $T->S$的最大流相当于将$S->T$的流量减至最小。
\end{enumerate}

利用性质~\ref{LFB}我们仍然能保证该流是一个可行流。

上述内容参考了F.W.Nietzsche\footnote{有上下界的、有多组源汇的、网络流、费用流问题 - F.W.Nietzsche
\url{https://www.cnblogs.com/nietzsche-oier/p/8185805.html}}与
liu\_runda\footnote{有上下界的网络流学习笔记 - liu\_runda
\url{https://www.cnblogs.com/liu-runda/p/6262832.html}}的博客。
