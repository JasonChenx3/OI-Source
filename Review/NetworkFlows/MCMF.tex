\section{费用流}
\index{M!MCMF}
从普通的EK算法扩展,既然每次增加的流量是一样的,那么我们就选择费用最小(大)
的增广路径,从而保证在得到最大流的前提下费用最小(大)。

求最短路时使用SPFA,若没有负权边尽量使用Dijkstra。

一般的建图思路是通过流量限制来保证方案合法,然后设计边的费用引导至最优代价。

\subsection{使用Dijkstra实现费用流}\label{DijMCMF}
其实即使有负权边,也是可以使用Dijkstra来求费用流的\CJKsout{(但是仍然需要SPFA)}。

核心思想是对原图适当地修改变为不带负权边的图。首先在迭代外用SPFA求从源点到每个点的最短路,
记距离为$h[u]$,满足三角不等式$h[u]+w[u][v]\geq h[v]$。将该式变形得$w[u][v]+h[u]-h[v]
\geq 0$,令左式为边的新权值,就可以在迭代中使用Dijkstra求最短路了,记实际最短距离为
$md[i]$,计算得到的最短距离为$dis[i]$。容易发现最短路径边权和中的$h[]$抵消后,可以得到
$dis[i]=md[i]+h[S]-h[i]$,其中$h[S]=0$,那么实际距离比计算距离多$h[i]$。由此可得
本次迭代产生的费用贡献为$(dis[T]+h[T])*minf$,并且需要在当前迭代结束前更新最短距离
$h'[u]=h[u]+dis[u]$。

上述内容参考了Mogician的博客\footnote{
    最大流与Dijkstra做费用流 - Mogician's blog - 洛谷博客
    \url{https://www.luogu.org/blog/Mogician/Network-Flow-Guide}}。
\subsection{多路增广费用流}
一般使用该方法作为费用流模板。

与普通费用流的差别如下:
\begin{itemize}
    \item 使用vector存边对缓存友好
    \item SPFA使用SLF带容错优化
    \item SPFA从T开始找增广路
    \item DFS多路增广从S沿着最短路跑,可以使用当前弧优化,注意不要走环
\end{itemize}

参考代码:
\lstinputlisting{Source/Templates/MCMF.cpp}

该方法来自Melacau的博客(翻fjsdfzoj时发现的)\footnote{
    【模板】板子的集合\\
    \url{https://www.cnblogs.com/Melacau/p/ban.html}
}。
\subsection{消圈定理}
该定理用来求给定流量的最小费用流。

\begin{theorem}[消圈定理]
	当前流为给定流量的最小费用流当且仅当其残量网络中没有负环。
\end{theorem}

寻找负环可以使用IDDFS-SPFA解决。

该内容参考了Sengxian的博客\footnote{
	网络流之 - 消圈定理(POJ 2175)\\
	\url{https://blog.sengxian.com/algorithms/clearcircle}
}。
\subsection{zkw费用流}
发现zkw费用流的实现比较短,在性能要求不高时使用该方法作为费用流模板。
zkw费用流一般适用于流量大,费用范围小,增广路径短的图。

zkw费用流同样使用多路增广思想,与上文的不同之处在于其最短路算法是连续执行的。
正确性证明留坑待补。
\index{*TODO!zkw费用流正确性证明}

参考代码:
\lstinputlisting{Source/Templates/LOJ102-zkwMCMF.cpp}

该内容参考了Angel\_Kitty\footnote{
    详解zkw算法解决最小费用流问题
    \url{https://www.cnblogs.com/ECJTUACM-873284962/p/7744943.html}
}和小蒟蒻yyb\footnote{
    【LOJ\#3097】[SNOI2019]通信(费用流)
    \url{https://www.cnblogs.com/cjyyb/p/10790579.html}
}的博客。
