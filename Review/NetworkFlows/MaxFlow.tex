\section{最大流}
Dinic与ISAP属于Ford-Fulkerson方法中的SAP(Shortest Augment Path)系。
而HLPP属于Push–Relabel算法。
\subsection{Dinic算法}
\index{D!Dinic}
\CJKsout{个人比较喜欢使用Dinic算法(因为我只会这个)}。

ISAP大法好!!!

Dinic的计算流程如下:
\begin{enumerate}
	\item BFS建分层图,若找不到增广路则退出;
	\item DFS在分层图上找增广路并修改流量,重复步骤1。
\end{enumerate}

时间复杂度证明:

\begin{enumerate}
	\item \begin{lemma}
		Dinic每次BFS后的阻塞流层数是递增的(即$d[t]$递增)。
	\end{lemma}
	\item 每次BFS的时间复杂度为$O(E)$。
	\item 每次DFS的时间复杂度为$O(VE)$。
\end{enumerate}

因此算法的时间复杂度为$O(V^2E)$。

在容量均为1的图上,Dinic的时间复杂度为$O(min \{ V^\frac{2}{3},E^\frac{1}{2} \} E)$,
证明:

留坑待填,参见\cite{NFTGC}。

做二分图最大匹配时Dinic跑得飞快,时间复杂度$O(\sqrt V E)$,证明:

留坑待填,参见\cite{DSNA}。

\index{*TODO!特殊图下Dinic的时间复杂度证明}

时间复杂度证明源自Wikipedia-EN\footnote{
	Dinic's algorithm - Wikipedia
	\url{https://en.wikipedia.org/wiki/Dinic\%27s\_algorithm}}以及
	permui的博客\footnote{ 最大流算法-ISAP - permui
		\url{https://www.cnblogs.com/owenyu/p/6852664.html}}
\subsubsection{优化}
\begin{itemize}
	\item 当前弧优化:每次从未遍历的边开始遍历,减少重复计算(就算前面的边没满,
	      下一次还可以增广)。
	\item 记录无法增广的点(将其深度设为-1),避免重复计算。
	\item (玄学,未测试)BFS找到一条增广路就退出,无法解释。
	\item 若图为分层图,在Dinic之前贪心预流(依旧玄学,未测试):
	      \begin{enumerate}
		      \item 从$s$开始逐层递推,计算能够流出节点$i$的流量$out[i]$;
		      \item 从$t$开始逐层倒推,计算每条边的实际流量。
	      \end{enumerate}
	      代码:

	      \lstinputlisting[title=PreFlow]{NetworkFlows/PreFlow.cpp}

	      该方法源自沐阳的博客。
	      \footnote{ZOJ-2364 Data Transmission 分层图阻塞流 Dinic+贪心预流 - 沐阳
		      \url{https://www.cnblogs.com/Lyush/p/3204099.html}}
\end{itemize}

\subsubsection{板子}

常规优化:
\lstinputlisting[title=DinicA]{Source/Templates/DinicA.cpp}

玄学优化(注意在随机数据下表现可能更差):

\begin{itemize}
	\item 伸缩操作:首先按照边的容量从大到小排序,然后枚举位按照
	$cap\geq 2^k,2^{k-1},\cdots,2^0$加边,每加一组边跑一次Dinic。
	时间复杂度$O(VE\lg C)$。
	\item 延迟加反向边:建图时仍然加正反向边,但是第一次Dinic
	时避开反向边,第二次Dinic时才考虑反向边。
	\item 不退流跑,一次性退流:BFS失败时才退流,若退流后仍然失败才退出迭代。
\end{itemize}

这些优化参见kczno1的博客\footnote{
	论如何用dinic ac 最大流 加强版
	\url{http://kczno1.blog.uoj.ac/blog/3375}}。

参考代码:

常规优化+伸缩操作+延迟加反向边(实践中还是这个比较好用):
\lstinputlisting[title=DinicB]{Source/Templates/DinicB.cpp}

kczno1的最新做法-不退流跑,一次性退流:
\lstinputlisting[title=DinicC]{Source/Templates/DinicC.cpp}

\subsubsection{当Dinic遇上LCT}

留坑待补。
\index{*TODO!Dinic with LCT}

\subsection{ISAP算法}
\index{I!Improved Shortest Augment Path}

Dinic每次BFS计算分层图的过程为找最短增广路的过程。每次BFS
重新计算层次编号$d$似乎有些浪费,因此ISAP在Dinic的基础上用
DFS直接修改层次编号的方式来优化算法。ISAP的时间复杂度仍然为$O(V^2E)$。
记数组$d[u]$为残存网络中点$u$到汇点的最短距离,为了编码方便让$d[T]=1$。

算法步骤如下:
\begin{itemize}
	\item 迭代DFS增广,若找不到满足$d[u]=d[v]+1$的可增广边则说明此时的最短路标号
	已经过时,为了让点$u$可增广,令$d[u]=min\{d[v]\}+1$。
	\item 若$d[S]>|V|$则说明已不存在简单增广路径,退出迭代。
\end{itemize}

\subsubsection{优化}
\begin{itemize}
	\item 若数组$d$被初始化为0,则DFS需要$O(n^2)$的时间来初始化
	数组$d$。可以在增广前从汇点开始BFS$O(n+m)$预处理数组$d$。
	\item gap优化:维护每种层次编号的数量$gap[d]$,若$gap[d]=0$则说明
	出现了断层,不存在新的增广路。此时简单地令$d[S]=n+1$结束算法。
	\item 类似Dinic可以使用当前弧优化,{\bfseries 但在层次标号被修改后要重置链头}。
	\item 层次标号的修改是连续的,每次增广完后$++d[u]$。
	\item 流量用完后直接退出。
\end{itemize}

板子(代码比DinicA还短而且跑得比DinicB还快):
\lstinputlisting[title=ISAP]{Source/Templates/ISAP.cpp}

{\bfseries 注意$mf=0$时直接返回不要更新层次标号。}

ISAP算法参考了permui的博客\footnote{ 最大流算法-ISAP - permui
\url{https://www.cnblogs.com/owenyu/p/6852664.html}}。

\subsection{HLPP算法}
\index{H!Highest-label push–relabel\\ algorithm}

\CJKsout{算法导论}\cite{ITA3}~\CJKsout{26.4节讲的推送-重贴标签算法是}
$O(V^3)$\CJKsout{的。。。}

HLPP算法使用``推送-重贴标签''算法,其时间复杂度为$O(V^2\sqrt{E})$。虽然时间复杂度
比Dinic优,但由于HLPP算法上界较紧,在实践中往往跑不过Dinic(加了优化后表现还行)。

\subsubsection{推送-重贴标签算法}

以水流类比网络流,每条边都是一根有流量限制的水管,允许每个点暂时存储一些多余的水,
称为超额流。特别地,源汇点可以长期存储无限多的水。其它点需要伺机将自身的超额流推送
出去,这里给每个节点再引入一个``高度''参数,规定流量只能往低处走。固定源点的高度为$V$。
当某个节点高于源点时,它的超额流将退回给源点。{\bfseries 注意高度可以达到$2V-1$}

该算法由两个基本操作组成:
\begin{itemize}
	\item ``推送'':一个节点把自己的超额流推送给高度比自己低1的节点(源点无高度差限制)。
	\item ``重贴标签'':当一个节点无法推送完超额流时,将自身高度加到
	连边有残存流量的最低邻接点的高度+1。
\end{itemize}

首先令S的出边满流,然后维护超额流节点队列,每次取出节点对其进行推送或重贴标签操作。
直至不存在超额流节点。时间复杂度$O(V^2E)$。

\subsubsection{前置重贴标签算法}

每次重贴标签时将节点移至队首,可将时间复杂度优化至$O(V^3)$。

参见算法导论\cite{ITA3}~第26.5节。

\subsubsection{HLPP实现与优化}

使用优先队列以高度为关键字维护超额流节点,每次选取最高标号的节点进行``推送-重贴标签''。

优化:
\begin{itemize}
	\item gap优化:当一个点被重贴标签后,若没有其他点拥有其原来的高度,
	高于此高度的点就无法把流量推送到汇点。将这些点的高度全部设为$V+1$使其流量
	流回源点。
	\item 高度预计算(我因此而TLE多次):将$d$初始化为每个点到汇点的最短路径长。
	{\bfseries 注意源点的高度固定为$V$。}
	\item 使用桶维护优先队列:注意到高度值的范围不大,使用桶来维护较为快速。
\end{itemize}

板子:

优先队列版:
\lstinputlisting[title=HLPPA]{Source/Templates/HLPPA.cpp}

桶版(参考PM250的代码\footnote{
	R13845988 评测详情
	\url{https://www.luogu.org/record/show?rid=13845988}
},自己不会用vector然后就用set代替了,常数大好多):
\lstinputlisting[title=HLPPB]{Source/Templates/HLPPB.cpp}

HLPP算法参考了Mr\_Spade的博客\footnote{
	网络最大流——最高标号预流推进
	\url{https://www.cnblogs.com/Mr-Spade/p/9636935.html}
}。

\subsection{最大流与最小割}

\index{M!Max-flow min-cut theorem}
\begin{theorem}[Max-flow min-cut theorem]\label{MFMCT}
	最大流=最小割。
\end{theorem}

证明:
\begin{itemize}
	\item
	\begin{lemma}\label{MCA}
		最大流$\leq$最小割
	\end{lemma}
	由于流量被割边所限制,所以最大流$\leq$任意割,所以最大流$\leq$最小割。
	\item
	\begin{lemma}\label{MCB}
		最大流$\geq$最小割
	\end{lemma}
	证明:跑完最大流后残量网络内$s$与$t$不连通,所以得到了一个割,
	即最大流$\geq$最小割。
\end{itemize}

结合引理~\ref{MCA}与~\ref{MCB}可得最大流=最小割。
\subsection{无向图最小割}
\subsubsection{Stoer-Wagner Algorithm}
\index{S!Stoer-Wagner Algorithm}
若需要求全局最小割,使用Stoer-Wagner Algorithm。

算法步骤如下:
\begin{enumerate}
	\item 任意指定一个节点作为初始点集;
	\item 查询到点集内的点边权和的最大的点集外的点;
	\item 合并最后加入的两个节点$s,t$并更新最小割;
	\item 重复第一步直至整个图被合并。
\end{enumerate}
具体做法见代码。边权可用优先队列维护,时间复杂度$O(|V||E|\lg |E|)$。

模板(SP12056 FZ10B - Nubulsa Expo):
\lstinputlisting{Source/Templates/Stoer-Wagner.cpp}

这题$|V|$比较小所以可以用邻接矩阵存图,$O(|V|^3)$解决。
\lstinputlisting{Source/Templates/Stoer-WagnerV3.cpp}

不知为何两种方法在SPOJ上都TLE了。
\index{*TODO!证明无向图最小割算法的正确性并修改模板}
上述内容参考了Oyking的博客\footnote{
	全局最小割StoerWagner算法详解
	\url{https://www.cnblogs.com/oyking/p/7339153.html}
}。
\subsubsection{流量构造法}
若指定源汇点,连边时给正反向边的残余流量都初始化为割边代价,然后跑Dinic。
\subsection{最小割性质}
该性质源自AHOI2009最小割,描述了边在最小割中的条件。

首先求出任意一个$ST$割,然后对残量网络求SCC。

按照下列定理判定:
\begin{theorem}
	\begin{itemize}
		\item 若该边满流且两端不属于同一个SCC,则该边出现在某个最小割中
		\item 若该边满流且起点与$S$在同一个SCC中,终点与$T$在同一个SCC中,则该边出现在任意最小割中
	\end{itemize}
\end{theorem}
