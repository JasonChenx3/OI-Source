\section{最大流}
\subsection{Dinic算法}
\index{D!Dinic}
个人比较喜欢使用Dinic算法(因为我只会这个)。

Dinic的计算流程如下:
\begin{enumerate}
	\item BFS建分层图,若找不到增广路则退出;
	\item DFS在分层图上找增广路并修改流量,重复步骤1。
\end{enumerate}

附上板子:
\lstinputlisting[title=Dinic]{Source/Source/'Network Flows'/2764.cpp}

时间复杂度证明:

\begin{enumerate}
	\item \begin{lemma}
		Dinic每次BFS后的阻塞流层数是递增的(即$d[t]$递增)。
	\end{lemma}
	\item 每次BFS的时间复杂度为$O(E)$。
	\item 每次DFS的时间复杂度为$O(VE)$。
\end{enumerate}

因此算法的时间复杂度为$O(V^2E)$。

在容量均为1的图上,Dinic的时间复杂度为$O(min \{ V^\frac{2}{3},E^\frac{1}{2} \} E)$,
证明:

留坑待填,参见\cite{NFTGC}。

做二分图最大匹配时Dinic跑得飞快,时间复杂度$O(\sqrt V E)$,证明:

留坑待填,参见\cite{DSNA}。

\index{TODO!特殊图下Dinic的时间复杂度证明}

时间复杂度证明源自Wikipedia-EN\footnote{
	Dinic's algorithm - Wikipedia
	\url{https://en.wikipedia.org/wiki/Dinic\%27s\_algorithm}}以及
	permui的博客\footnote{ 最大流算法-ISAP - permui
		\url{https://www.cnblogs.com/owenyu/p/6852664.html}}
\subsubsection{优化}
\begin{itemize}
	\item 当前弧优化:每次从未遍历的边开始遍历,减少重复计算(就算前面的边没满,
	      下一次还可以增广)。
	\item 记录无法增广的点(将其深度设为-1),避免重复计算。
	\item (玄学,未测试)BFS找到一条增广路就退出,无法解释。
	\item 若图为分层图,在Dinic之前贪心预流(依旧玄学,未测试):
	      \begin{enumerate}
		      \item 从$s$开始逐层递推,计算能够流出节点$i$的流量$out[i]$;
		      \item 从$t$开始逐层倒推,计算每条边的实际流量。
	      \end{enumerate}
	      代码:

	      \lstinputlisting[title=PreFlow]{NetworkFlows/PreFlow.cpp}

	      该方法源自沐阳的博客。
	      \footnote{ZOJ-2364 Data Transmission 分层图阻塞流 Dinic+贪心预流 - 沐阳
		      \url{https://www.cnblogs.com/Lyush/p/3204099.html}}
\end{itemize}

\subsubsection{当Dinic遇上LCT}

\index{D!Dinic with LCT}
留坑待补。
\index{TODO!Dinic with LCT}

\subsection{ISAP算法}
\index{I!ISAP}
留坑待补。
\index{TODO!ISAP算法}
\subsection{HLPP算法}
\index{H!HLPP}
留坑待补。
\index{TODO!HLPP算法}
\subsection{最大流与最小割}

\index{M!Max-flow min-cut theorem}
\begin{theorem}[Max-flow min-cut theorem]\label{MFMCT}
	最大流=最小割。
\end{theorem}

证明:
\begin{itemize}
	\item
	\begin{lemma}\label{MCA}
		最大流$\leq$最小割。
	\end{lemma}
	由于流量被割边所限制,所以最大流$\leq$任意割,所以最大流$\leq$最小割。
	\item
	\begin{lemma}\label{MCB}
		最大流$\geq$最小割。
	\end{lemma}
	证明:跑完最大流后残量网络内$s$与$t$不连通,所以得到了一个割,
	即最大流$\geq$最小割。
\end{itemize}

结合引理~\ref{MCA}与~\ref{MCB}可得最大流=最小割。
