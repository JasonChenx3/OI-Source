\section{常见网络流/最小割模型}
\subsection{平面图转对偶图}

平面图与对偶图的定义:
\begin{itemize}
	\item 平面图(Planar Graph):在平面上画出来可以使边与边只在顶点上相交的图。
	      \index{P!Planar Graph}
	\item 对偶图(Dual Graph):将平面图的每条边两边的区域连边而成的新平面图。
	      \index{D!Dual Graph}
\end{itemize}

记平面图$G$的对偶图为$G^*$,平面集合为$P_G$。

对偶图$G^*$有两个性质:
\begin{itemize}
	\item
	      \begin{property}
		      $G^*$中的环对应$G$中的一个割。
	      \end{property}
	\item
	      \begin{property}
		      $|P_G|=|V_{G^*}|,|E_G|=|E_{G^*}|$
	      \end{property}
\end{itemize}

实际应用时,首先连接$(s,t)$使得外部平面被分为两个平面,以获得源汇点$s',t'$(同时连
到一个点上并没有什么用),然后按照定义建图(不要连$s'$与$t'$的边)。

那么$s,t$的最小割=$s'\rightarrow t'$的最短路(即拆点前的最小环),时间复杂度降低不少。

\subsubsection{例题}

Luogu P4001 [BJOI2006]狼抓兔子\footnote{【P4001】[BJOI2006]狼抓兔子 - 洛谷
\url{https://www.luogu.org/problemnew/show/P4001}}

根据定理~\ref{MFMCT}转换为求最大流,将右上角当做起点,右下角当做终点,然后使用上述
方法转为对偶图求最短路。

\lstinputlisting[title=Luogu P4001]{Source/Unclassified/Done/4001.cpp}

\subsection{最大权闭合子图}

$S$向非负权点连容量为权值的边,负权点向$T$连容量为权值相反数的边,如果选择点$u$必须
选择点$v$,就从$u$向$v$连容量为$\infty$的边。

答案=正权值之和-最小割。

简单理解:如果割去正权点的权值,则说明舍弃该正权点,权值从答案中扣除;如果割去负权点
的权值,则说明选择之前的正权点并从答案扣除该负权值。

严格的正确性证明待补充。\index{*TODO!最大权闭合子图算法的正确性}

\subsubsection{板子}

Luogu P4174 [NOI2006]最大获利\footnote{【P4174】[NOI2006]最大获利 - 洛谷
\url{https://www.luogu.org/problemnew/show/P4174}}

\lstinputlisting[title=Luogu P4174]{Source/Source/'Network Flows'/4174.cpp}

\subsubsection{输出方案}

\begin{theorem}
    Dinic最后一次增广时可访问到的点的集合就是最优方案。
\end{theorem}

简单理解:最后一次增广后BFS必然找不到增广路。若割源点到正权点之间的
边,则访问不到该正权点,说明该正权点被舍弃,若不被割则必割后继负权点对应的边,
该点可被访问到,说明该正权点被选中;若割负权点到源点之间的边,则可访问到该负权点,
说明该负权点被选中,若不被割则必割前驱正权点对应的边,该点不可访问,说明该点被舍弃。

上述内容参考了appgle\footnote{网络流算法基本模型 - appgle
	\url{https://www.cnblogs.com/hyl2000/p/6618519.html}},
MaxMercer\footnote{关于平面图到对偶图的转化 \\
	\url{https://blog.csdn.net/MaxMercer/article/details/77976666}}和
Cold\_Chair\footnote{网络流——最大权闭合子图 \\
	\url{https://blog.csdn.net/Cold\_Chair/article/details/52841351}}
的博客。
