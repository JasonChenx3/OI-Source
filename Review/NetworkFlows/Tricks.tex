\section{技巧总结}
\subsection{最大流}
\begin{itemize}
    \item 若一个点只能被经过有限次,将其拆为入点和出点,入点到出点连流量为
    经过次数限制的边。
    \item 树形最大流可以贪心解决。
\end{itemize}
\subsection{最小割}
\begin{itemize}
    \item 最大化收益可以理解为已经拿到所有收益,最小化损失。然后将其转化为最小割解决。
    \item 使用$+\infty$边描述依赖关系,可以保证这条边不出现在最小割中。
    \item 用$S,T$与点的连边来表示点的权。
\end{itemize}
\subsection{费用流}
\begin{itemize}
    \item 要求费用最小且边数最小:类比进制的思想,实际费用乘以一个大于总边数的因子,再加上
    1作为该边边权。
    \item 若已知走一条边之前必定已经走完了另外几条边,则考虑动态加边。
    \item 对于层数较少,结构简单的图,考虑使用其它数据结构贪心模拟费用流。
    \item (待验证)判断一条边是否一定被选:在残量网络上跑SPFA,若距离差不等于边权则必选。
    \item 餐巾计划问题:
\end{itemize}

上述内容参考了胡伯涛的2007年国家集训队论文《最小割模型在信息学竞赛中的应用》
\cite{MCIOI}和租酥雨的博客\footnote{
    网络流总结\\
    \url{https://www.cnblogs.com/zhoushuyu/p/8137534.html}
}。
