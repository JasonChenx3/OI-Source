\section{二分图}
\index{B!Bipartite Graph}
\subsection{二分图判定}
\begin{property}
	二分图中不存在奇环。
\end{property}
如果存在奇环,则必有一条边的端点属于同一集合。
所以可以使用DFS染色来判定二分图,遇到矛盾则退出。

\lstinputlisting[title=BGJudge.cpp]{NetworkFlows/BGJudge.cpp}

\subsection{二分图最大匹配}

\subsubsection{匈牙利算法}
\index{H!Hungarian Algorithm}

匈牙利算法的主要步骤就是遍历左集合的每一个顶点,使得其尽可能找到一个匹配。
要为该顶点找到一个匹配,首先遍历边,如果右顶点已经有匹配,则递归尝试让该
匹配点重新找一个匹配,如果右顶点无匹配或者更换匹配成功,则这条边是一个匹配。

原则:有机会上,没机会创造机会也要上。
\footnote{Dark\_Scope 趣写算法系列之--匈牙利算法
	\url{https://blog.csdn.net/dark\_scope/article/details/8880547}}

感性的算法的正确性证明:每次递归时匹配数只增不减,且递归有权修改整个连通块
的着色情况。(似乎并没有什么说服力)。

匈牙利算法的时间复杂度为$O(VE)$,每次尝试匹配的复杂度为$O(E)$。

\index{*TODO!匈牙利算法标准描述与正确性证明}

\subsubsection{Hopcroft–Karp Algorithm}
\index{H!Hopcroft–Karp Algorithm}
暂时先坑着\sout{为什么不写Dinic呢}。
\index{*TODO!Hopcroft–Karp算法}

\subsubsection{例题}

Luogu P1129 [ZJOI2007]矩阵游戏
\footnote{\url{https://www.luogu.org/problemnew/show/P1129}}

首先用二分图最大匹配找到n个不同行且不同列的黑格子(置换矩阵P),然后就可以操作得到
目标矩阵(单位矩阵I)了。

\lstinputlisting[title=Luogu P1129]{Source/Unclassified/Done/1129.cpp}

\subsection{二分图最大权匹配 Kuhn-Munkras Algorithm}
\index{K!Kuhn-Munkras Algorithm}
\sout{先用费用流做吧,暂时先坑着。}
\subsubsection{起步}
维护每个左/右顶点的权值(称为顶标),所有节点的顶标和为答案上界。
令每个左顶点的顶标为出边边权最大值,右顶点顶标为0。

对每个顶点运行匈牙利算法,若左右顶点顶标之和等于边权,则考虑连边;
若无法为当前点找到匹配,则将访问到的左顶点顶标-1,右顶点顶标+1,
等价于使答案上界-1(DFS访问树中的叶子必为左顶点),重新为该点寻找匹配。
把任意二分图当做完全二分图(不存在的边权值为0),迭代必定会结束。

这种做法能够保证在找到最大匹配的情况下使权值和最大。
\subsubsection{优化1}
可以发现在左-1右+1后,原先等于左右顶点顶标之和的边仍然被经过,
一个简单的思路是一次性突破``瓶颈'',即令下次增广时终点位置处的某条边从
不可连边变为可连边,每次DFS增广时维护(顶标和-边权)的最小值$d$,
若匹配失败则左$-d$右$+d$。

这才是复杂度比较靠谱的算法($O(n^3)$)。
\subsubsection{优化2}
在匹配每个点时,初始化所有右顶点的松弛函数$slack$为$\infty$,然后
DFS时$slack$维护(顶标和-边权)的最小值。若匹配失败则令$d$为未访问右
顶点的$slack$函数最小值,左$-d$右$+d$,同时未访问节点的$slack-=d$。

该优化的复杂度似乎没变,但实测该方法比优化1的效率更高(3x)。
\subsubsection{优化3}
考虑记录其增广时的路径,然后将递归算法转换为非递归算法。
\begin{lstlisting}
int w[size][size],lh[size],rh[size],pair[size],
    pre[size],slack[size];
bool flag[size];
void aug(int s) {
    reset(flag);
    reset(pre);
    reset(slack,0x3f);
    pair[0]=s;
    int u=0;
    do {
        int v=pair[u],minh=inf,nxt;
        flag[u]=true;
        // `再次DFS后新访问到了点u和它的匹配点`
        // `为点v找新匹配点`
        for(int i=1;i<=n;++i)
            if(!flag[i]){
                int delta=lh[v]+rh[i]-w[v][i];
                if(delta<slack[i])
                    slack[i]=delta,pre[i]=u;
                    // `点i的匹配点有可能置换为u的匹配点,`
                    // `以腾出u的匹配点的空位`
                if(minh>slack[i])
                    minh=slack[i],nxt=i;// `点i下次将被访问`
            }
        //松弛
        for(int i=0;i<=n;++i)
            if(flag[i])lh[pair[i]]-=minh,rh[i]+=minh;
            else slack[i]-=minh;
        u=nxt;
    } while(pair[u]);// `直到找到未匹配点为止`
    // `置换匹配`
    while(u) {
        int p=pre[u];
        pair[u]=pair[p];
        u=p;
    }
}
int KM(int n) {
    for(int i=1;i<=n;++i) {
        int maxh=0;
        for(int j=1;j<=n;++j)
            maxh=std::max(maxh,w[i][j]);
        lh[i]=maxh;
    }
    reset(rh);
    reset(pair);
    for(int i=1;i<=n;++i)
        aug(i);
    int res=0;
    for(int i=1;i<=n;++i)
        res+=w[pair[i]][i];
    return res;
}
\end{lstlisting}
实测该方法比优化2的效率更高(2x)。
\index{*TODO!解释KM算法优化的合理性}
\subsection{二分图常见模型}
\subsubsection{最小点覆盖}
\index{K!König's theorem}
\begin{theorem}[König's Theorem]
	最小点覆盖数=最大匹配数。
\end{theorem}

使用反证法证明:如果有一条边两端顶点都不在最大匹配上,那么这条边可以进入最大匹配
成为一个更大的匹配边集,所以与最大匹配的假设矛盾。

\subsubsection{最大独立集}

\begin{theorem}
	最大独立集大小=顶点数-最小点覆盖数=顶点数-最大匹配数
\end{theorem}

证明:

容易发现去掉二分图中的最小点覆盖可得到一个独立集(若其不是独立集,则说明存在一条
边未被覆盖,与点覆盖的定义矛盾)。尝试以此独立集为基础扩展,可以发现若要使点覆盖
中的某个点变为独立集的点,由最小点覆盖数=最大匹配数可知,最小点覆盖的每个点都与$\geq 1$
的边相连,因此必须使不少于1个原独立集的点被删除。所以无论如何修改,最多得到与之大小
相等的独立集。

\subsubsection{DAG最小路径覆盖}

\paragraph{最小不相交路径覆盖}

将顶点拆成左右两点,若存在边$u->v$则连边$Lu->Rv$,求二分图最大匹配。

\begin{theorem}
	最小路径覆盖数=顶点数-二分图最大匹配数。
\end{theorem}

证明:二分图中每增加一个匹配,就意味着减少一条路径。

\paragraph{最小可相交路径覆盖}

先用Floyd求出传递闭包,转化为最小不相交路径覆盖问题。
因为如果要从a走到b,直接连边可以避开中间点的流量限制。

以上内容参考了罗茜\footnote{二分图详解及总结
	\url{https://www.cnblogs.com/alihenaixiao/p/4695298.html}},
justPassBy\footnote{有向无环图(DAG)的最小路径覆盖
	\url{https://www.cnblogs.com/justPassBy/p/5369930.html}}和
不可不戒\footnote{二分图:最大独立集\&最大匹配\&最小顶点覆盖
	\url{https://blog.csdn.net/lezg\_bkbj/article/details/9872189}}
的博客。
