\section{Huffman编码}
\subsection{常规Huffman}
计算步骤如下:
\begin{enumerate}
	\item 将每个字符按照(节点编号,频率)插入构造最小堆。
	\item 弹出频率最低的两个节点,将这两个节点挂在新节点下,新节点的频率
	      为这两个节点频率之和。将新节点插入堆中。
	\item 重复步骤2直至堆中只剩1个节点为止。
\end{enumerate}
\subsection{k叉Huffman}
对于2叉Huffman,可以保证最后堆中只剩1个节点,但是对于k叉Huffman来说则不一定
,所以需要`补零‘。考虑k叉Huffman每次操作都会减少$k-1$个节点,而且最后剩下1个
节点,因此补齐$(k-1 - ((n-1) \bmod (k-1))) \bmod (k-1)$个频率为0的节点即可。
\subsection{效率优化}
与NOIP2016蚯蚓同理,新插入的节点的频率是单调非减的。因此可以对原字符进行排序,然后
维护两个单调队列,取最小的一个弹出即可。
\subsection{限长Huffman}
论文\cite{LLH}给出了Package-Merge Algorithm,时间复杂度$O(nL)$。
\footnote{A fast algorithm for optimal length-limited Huffman codes\\
	\url{http://www.cse.ust.hk/mjg\_lib/bibs/DPSu/DPSu.Files/p464-larmore.pdf}
}
\index{P!Package-Merge Algorithm}
\subsubsection{Coin collector's problem}
一个硬币收藏家收藏有许多硬币,每个硬币都有它的面值和收藏价值,
现在硬币收藏家需要选择一个硬币子集,使得这个子集的面值总和$=X,X\in \mathbb{N}$,
同时收藏价值总和最小。其中面值取值为$2^{-k},0\leq k \leq m$。
\subsubsection{Package-Merge Algorithm}
\begin{enumerate}
	\item 将硬币按照面额分成多个集合,集合内按照收藏价值排序;
	\item 选取最小面额的集合内的硬币按照顺序两两合并($1+2,3+4,\cdots$)
	      后插入2倍面额的硬币集合,若剩余一个硬币则搁置;
	\item 不断合并直至合并到面值为1的集合,此时在该集合内贪心选取硬币。
\end{enumerate}
注意插入到下一面值集合的收藏价值是单调非减的,维护一个指针即可支持线性插入。
除去排序的复杂度,该算法复杂度为$O(n)$。

如果$X$不为最大面值的倍数,将$X$二进制拆分,算到某个为1的位对应的面值集合时,
选取集合内最小值加入方案并从集合内删除,继续合并。
\subsubsection{规约到限长Huffman问题}
设限长为$L$,将面值设为编码后的代价,即$2^{-1},2^{-2},\cdots,2^{-L}$,
设收藏价值为每种字符的频率,对每个面值都加入所有字符的收藏价值(预先排序),
运行该算法,最后选取前$n-1$个硬币。使用递归分治可将空间复杂度降为$O(n)$。

接下来构造标准Huffman编码:统计每个字符被选择的硬币数,作为其编码的长度(保证
长度不超过$L$),计算每个长度的字符数,然后给当前长度的字符分配位置(即计算偏移),
剩余位置给下一长度的字符使用。

代码如下:
\lstinputlisting{String/LLH.cpp}

该算法参考了Wikipedia-EN\footnote{
	Package-merge algorithm - Wikipedia
	\url{https://en.wikipedia.org/wiki/Package-merge\_algorithm}
}和pymqq的博客\footnote{
	基于二叉树和双向链表实现限制长度的最优Huffman编码
	\url{https://blog.csdn.net/pymqq/article/details/32084763}
}。
\subsection{编码字符代价不等的Huffman}
现实应用是摩尔斯电码的点和短横线的发送时间不同。
\index{*TODO!编码字符代价不等的\\Huffman}

留坑待补,参见\cite{huffman}。
