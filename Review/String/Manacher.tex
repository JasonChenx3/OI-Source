\section{Manacher算法}
\index{M!Manacher's Algorithm}
Manacher算法用来求解最长回文串问题。主要思想是利用之前的计算结果
来加速回文串的计算,从而达到线性时间复杂度。

算法步骤如下:
\begin{enumerate}
    \item 为了统一奇回文串和偶回文串,向每对相邻字符间插入一个未出现过的字符,
    接下来算法仅讨论奇回文串;
    \item 维护当前访问到的最右位置$maxr$和最右位置所对应的字符串中心$pos$,
    以及以每个位置为中心向右扩展长度$RL[i]$(从中心开始数);
    \item 对于每一个位置:
    \begin{enumerate}
        \item
        \begin{itemize}
            \item 若当前位置$i\geq maxr$,设$RL[i]=1$;
            \item 否则令$RL[i]=min(RL[pos-(pos-i)],maxr-i+1)$(因为
            此时$i$到$maxr$的部分和$i$以$pos$为轴的对称部分对称,而那个
            部分已经被处理过了)。
        \end{itemize}
        \item 不断向两端扩展增大$i$;
        \item 更新$maxr$与$pos$。
    \end{enumerate}
    \item 答案即为$RL[i]-1$的最大值。
\end{enumerate}

代码如下:
\begin{lstlisting}
char buf[size],str[2*size];
int RL[2*size];
int manacher() {
    int cnt=0;
    for(int i=0;buf[i];++i) {
        str[cnt++]='@';
        str[cnt++]=buf[i];
    }
    str[cnt++]='@';
    int maxr=0,pos=0,ans=0;
    for(int i=0;i<cnt;++i) {
        if(i<maxr)
            RL[i]=std::min(RL[2*pos-i],maxr-i+1);
        else
            RL[i]=1;
        while(i-RL[i]>=0 && i+RL[i]<cnt &&
            str[i-RL[i]]==str[i+RL[i]])
            ++RL[i];
        if(i+RL[i]-1>maxr) {
            maxr=i+RL[i]-1;
            pos=i;
        }
        ans=std::max(ans,RL[i]);
    }
    return ans-1;
}
\end{lstlisting}
