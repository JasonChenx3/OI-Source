\section{后缀树}
\index{S!Suffix Tree}
后缀树是由串S的所有后缀组成的压缩Trie。

为了避免压缩过程隐藏了某个后缀,插入每个后缀后再
插入一个不出现的字符。
\subsection{构造}
常规的构造方法是$O(n^2)$的,一般采用Ukkonen算法在线性时间内完成构造。
\index{U!Ukkonen's Algorithm}
\index{*TODO!线性构造后缀树}
留坑待补(还是去学后缀仙人掌吧)。
\subsection{广义后缀树}
广义后缀树在插入不同的串时用的结尾字符不同,以区分不同的字符串。

查询最长公共子串时要找到最深的拥有2种结束标记的节点。
\subsection{应用}
后缀树满足如下性质:
\begin{itemize}
    \item 每个节点代表一个子串。

    查询串P在串S的出现次数可以按照Trie的方法匹配,然后
    统计其所在节点的子树中的叶子节点个数。
    \item 每个非叶节点至少有两个儿子。

    统计串S的最长重复子串要找到其最深非叶子节点。
\end{itemize}
以上内容参考了v\_JULY\_v的博客\footnote{
	从Trie树(字典树)谈到后缀树(10.28修订)
	\url{https://blog.csdn.net/v\_july\_v/article/details/6897097}
}。
