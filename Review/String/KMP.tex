\section{KMP算法}
\index{K!Knuth–Morris–Pratt Algorithm}

KMP算法通过预处理nxt数组来避免重复匹配。考虑在长度为$i$的模式串前缀$S_i$
的非自身前缀$P_j$中,满足$P_j$是$S_i$后缀的$P_j$。定义$nxt[i]$为最长的满足条件的
$P_j$长度。

以下字符串下标从0开始,nxt数组下标从1开始。
\subsubsection{预处理}
可以利用之前的预处理信息跳nxt来找到最长后缀。
\begin{lstlisting}
int nxt[size];
void cook(const char* P) {
    int p=0;
    nxt[1]=0;
    for(int i=1;P[i];++i) {
        while(p && P[p]!=P[i])
            p=nxt[p];
        if(P[p]==P[i])
            ++p;
        nxt[i+1]=p;
    }
}
\end{lstlisting}
\subsubsection{匹配}
匹配和预处理的过程十分相似。
\begin{lstlisting}
void match(const char* str,int len) {
    int p=0;
    for(int i=0;str[i];++i) {
        while(p && P[p]!=str[i])
            p=nxt[p];
        if(P[p]==str[i])
            ++p;
        if(p==len)
            //match str[i-len+1...i]
    }
}
\end{lstlisting}

ExKMP已被更好理解的Z Algorithm取代,故不再补充该内容。
Z Algorithm参见第~\ref{ZA}节。
