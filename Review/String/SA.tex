\section{后缀数组}
\index{S!Suffix Array}
\subsection{倍增构造}
后缀数组$SA[i]$表示排名为$i$的后缀位置,相应地可以
得到起始位置为$i$的后缀排名$rk[i]$。

数组$rk$一般使用倍增法$O(n\lg n)$构造,然后对应地初始化$SA$。

构造步骤如下:
\begin{enumerate}
    \item 对于每一个字符初始化其排序关键字(即字符);
    \item 对关键字进行基数排序,算出此时字符串$[i\ldots i+2^k-1]$的排名;
    \item 若所有排名都不相同,直接跳出;否则将$rk[i]$与$rk[i+2^k]$作为以$i$
    为起始位置的字符串的关键字,重复步骤2;
    \item 根据$rk$数组初始化$SA$数组。
\end{enumerate}
代码如下:
\lstinputlisting{Source/Templates/SA.cpp}
\subsection{最长公共前缀}
\index{L!Longest Common Prefix}
定义$LCP(i,j)$为第$i$个后缀与第$j$个后缀的最长公共前缀。

显然LCP函数有两个性质:
\begin{itemize}
    \item $LCP(i,j)=LCP(j,i)$
    \item $LCP(i,i)=n-SA[i]+1$
\end{itemize}
所以只要考虑$LCP(i,j),i<j$的情况。
\begin{theorem}[LCP Theorem]
    $LCP(i,j)=min{LCP(k-1,k)},i<k\leq j$
\end{theorem}
要证明该定理可先证明下列引理:
\begin{lemma}[LCP Lemma]
    $LCP(i,j)=min(LCP(i,k),LCP(k,j)),i<k<j$
\end{lemma}
\paragraph{证明}
首先根据传递性显然有$LCP(i,j)\geq min(LCP(i,k),LCP(k,j))$。

其次考虑到$i,j,k$是后缀的排名且$i<j<k$,所以$LCP(i,j)>min(LCP(i\\,k),LCP(k,j))$
不可能成立,即$LCP(i,j)\leq min(LCP(i,k),LCP(k,j))$。该引理得证。

因此设$height[i]=LCP(i-1,i),height[1]=0$,求出height数组后
使用ST表即可$O(1)$询问LCP。

考虑在原串上相邻后缀的关系,显然两个后缀右移1位的$LCP'\geq LCP-1$,
快速转移后继续向后比较即可。

\begin{lstlisting}
void cook(int n) {
    int h=0;
    for(int i=1;i<=n;++i) {
        if(rk[i]==1)h=0;
        else {
            int k=SA[rk[i]-1];
            if(h)--h;
            while(buf[i+h]==buf[k+h])
                ++h;
        }
        height[rk[i]]=h;
    }
}
\end{lstlisting}
以上内容参考了Angel\_Kitty的博客\footnote{后缀数组(一堆干货) - Angel\_Kitty
    \url{https://www.cnblogs.com/ECJTUACM-873284962/p/6618870.html}
}。
\subsection{应用}
一般思想是二分LCP长度,对height数组进行分组。若遇到多串
将其用未出现字符连在一块求后缀数组。对于回文串/翻转系列问题将
反串一起加入即可。
\subsubsection{可重叠最长重复子串}
该子串一定是某两个后缀的LCP,而LCP在height数组中取最小值,因此
答案为height最大值。
\subsubsection{不可重叠最长重复子串}
二分LCP长度k,使用k对height数组进行划分,满足每块内的height值
不小于k,判断是否存在块内$SA[i]$的极差$\geq k$(此时子串不重叠)。
\subsubsection{可重叠k次最长重复子串}
二分LCP长度,对其分组,询问是否存在大小$\geq k$的块。
\subsubsection{本质不同的子串个数}
考虑按照后缀字典序加入每个后缀的前缀,每个后缀贡献了$n-SA[i]+1$个前缀,
去掉重复的$height[i]$个重复前缀,对每个后缀统计贡献即可。
\subsubsection{最长回文子串}
将串与反串用未出现字符连接,按照长度奇偶分类讨论,枚举对称中心求LCP。
\subsubsection{连续重复子串}
枚举串长$n$的因子$k$,询问字符串$[1\ldots n-k]$与$[k+1\ldots n]$是否
相等,即$LCP(rk[1],rk[k+1])=n-k$,$O(n)$预处理即可。
\subsubsection{重复次数最多的连续重复子串}
首先枚举连续重复子串长度$L$,仅考虑重复2次以上的情况,那么整个连续的串
每隔$L$个必相同,因此对起始位置$L*i,L*(i+1)$错位匹配求LCP后,若起点
不为$L$的倍数,尝试调用LCP判断左边剩余部分是否相等。求得长度后取max即为答案。
\subsubsection{最长公共子串}
将两个串拼接在一块,求两个字典序相邻但来自不同字符串的height最大值。
\subsubsection{长度$\geq k$的公共子串数}

\subsubsection{出现于不小于$k$个字符串的最长子串}
\subsubsection{在每个字符串不重叠重复的最长子串}
\subsubsection{出现或翻转后出现在每个字符串中的最长子串}

以上内容参考了罗穗骞的论文《后缀数组——处理字符串的有力工具》。
\subsection{后缀仙人掌}
\index{S!Suffix Cactus}
该方法主要用来快速构造后缀树。
