\section{后缀自动机}
\index{S!Suffix Automaton}
\subsection{描述}
后缀自动机可用来识别母串的后缀,在转移的过程中可以识别母串的子串。
SAM使用最简状态表示。SAM是一个DAG,任何从初始状态出发的路径对应一个子串。

定义$ST(str)$为接受$str$后转移到的状态,$Right(s)$为能够转移到
该状态的字符串的end位置集合(即字符串末端的下一位,且编号从0开始)。
给定状态$Right$集合的元素$r$,再确定子串长度$len$,就可以识别子串$[r-len,r)$。
发现若$l\in Right(s),r\in Right(s),l<r$,有$m\in Right(s),l\leq m \leq r$。
所以一个状态$s$映射了一段长度区间内的,末尾出现位置集合相同所有子串。
设其长度区间为$[l_s,r_s]$。

\subsubsection{状态Right集合之间的关系}
\begin{property}
若状态$a,b$的$Right$集合有交集,则必满足一个集合是另一个集合的真子集。
\end{property}
\paragraph{证明} 因为状态$a,b$表示的子串无交集,所以$[l_a,r_a]$与$[l_b,r_b]$
无交集。设$r_a<l_b$,对于任意终结点$r\in Right(a) \cap Right(b)$,有
$a$表示的所有子串是$b$表示的所有子串的后缀。那么$\forall r\in Right(b),r\in Right(a)$。
又因为$Right(a)\neq Right(b)$,所以$Right(b) \subset Right(a)$。

若$Right(s)\subset Right(fa)$,将$fa\rightarrow s$连边,构成了一棵Parent树。
可以发现$r_{fa}=l_s-1$,因此存储状态时只要存储$r_s$即可。
\subsection{构造}
与回文自动机类似,按顺序调用$extend$函数插入字符,时间复杂度$O(n)$。

对于每一个状态,维护其转移函数$nxt$,区间右端$len$,Parent树上父亲$link$。
SAM内维护一个末状态$last$,初始状态为1,表示空串。

插入步骤记为$SAM(T)\rightarrow SAM(Tx)$,即增加了Tx的后缀。
其中已加入的串为$T$,记$len(T)=L$。
步骤如下:
\begin{enumerate}
    \item 新建节点$s$,设$s.len=last.len+1$。
    \item 所有表示$T$后缀的节点,即满足位置$L\in Right(c)$的状态$c$。
    显然$ST(T)$满足该条件,因为$Right(ST(T))=\{L\}$。易知$ST(T)$在Parent
    树上到根的链上节点都有$L$,考虑这些节点。对于其中一个节点$c$,
    若要使$c$能够转移$x$,需要$\forall{r\in Right(c)},str[r]=x$。
    对于没有转移$x$的节点$c$,将其转移$x$设为$s$。显然这一段从$last$开始是连续的
    一段,找到有转移的节点后直接退出,开始讨论另一种情况。
    \item \begin{itemize}
        \item 若不存在已有转移的节点,使$s.link=1$后退出。
        \item 设有转移的节点$p$的转移节点为$q$,若$p.len+1=q.len$,令
        $s.link=q$,否则克隆节点$q$为$cq$,其中$cq.len=p.len+1$,在Parent
        树上把$q$换为$cq$,同时令$s.link=q.link=cq$。
    \end{itemize}
    \index{*TODO!理解SAM中克隆节点的原因}
    \item 更新$last=s$。
\end{enumerate}
注意SAM的点数不超过2n-2,边数不超过3n-3(转移边+Parent树边)。

代码如下:
\begin{lstlisting}
struct SAM {
    struct State {
        int nxt[26],link,len;
    } S[size*2];
    int last,siz;
    SAM():last(1),siz(1) {}
    void extend(int c) {
        int id=++siz;
        S[id].len=S[last].len+1;
        int p=last;
        while(p && !S[p].nxt[c]) {
            S[p].nxt[c]=id;
            p=S[p].link;
        }
        if(p) {
            int q=S[p].nxt[c];
            if(S[p].len+1==S[q].len)
                S[id].link=q;
            else {
                int cq=++siz;
                S[cq]=S[q];
                S[cq].len=S[p].len+1;
                while(p && S[p].nxt[c]==q) {
                    S[p].nxt[c]=cq;
                    p=S[p].link;
                }
                S[q].link=S[id].link=cq;
            }
        }
        else S[id].link=1;
        last=id;
    }
}
\end{lstlisting}
\subsection{Parent树}
注意Parent树上的父亲是儿子的后缀,因此匹配子串时可以
在转移边上跑,不行就跳Parent树的link(等同于fail树)。
\subsection{后缀树}
Parent树是反串的后缀树。
\subsection{计数问题}
对于每一个节点维护一个$right$值表示当前状态的Right集合大小。
新增节点时该节点贡献了1,但注意克隆节点并没有贡献,所以克隆后令$cq.right=0$。
最后拓扑排序dp就可以得到真实right值。

节点$s$表示了$s.len-s.link.len$个本质不同的子串,每种子串有$s.right$个。

优化:拓扑排序时可以按照len进行分层基数排序。
\subsection{线性构造后缀数组}
首先构造出SAM,发现last到根的链上的节点分别代表每一个后缀。对这些节点进行
标记,按照字典序DFS,维护DFS子串的长度$d$,通过遍历顺序得到$sa$数组。

注意对于跑单条链的情况可以使用类似并查集的方法压缩。

代码如下:
\lstinputlisting{Source/Templates/SAM2SA.cpp}
\subsection{广义SAM}
有两种构造方法:
\begin{itemize}
    \item 在线:插入一个字符串之前将$last$重置,时间复杂度为O(Trie大小*字符集大小
    +叶子节点深度和)。
    \item 离线:先建出Trie,BFS插入,插入时把父亲在SAM上的编号当做$last$,
    时间复杂度为O(Trie大小*字符集大小)。
\end{itemize}

以上内容参考了WC2012陈立杰的讲课课件《后缀自动机 Suffix Automaton》
与Candy?\footnote{[后缀自动机]【学习笔记】
    \url{https://www.cnblogs.com/candy99/p/6374177.html}
}和dwjshift\footnote{
    用SAM建广义后缀树 « dwjshift's Blog
    \url{http://dwjshift.logdown.com/posts/304570}
}的博客。
