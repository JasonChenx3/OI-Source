\section{后缀自动机}
\index{S!Suffix Automaton}
\subsection{描述}
后缀自动机(SAM)可用来识别母串的后缀,使用最简状态表示。SAM是一个DAG,任何从初始状态出发的路径
对应母串的某个子串。与其他自动机不同的是,SAM中一个状态对应多个子串。

\subsection{构造}
假设现在已经构造出了串$S$的SAM,考虑如何构造串$Sx$的SAM以识别新的子串。如果能够快速完成这个任务,
就可以逐位将字符插入SAM以构造出整个串的SAM。

为了加强理解,这里不直接按照最终的SAM描述,而是需要什么元素加入什么元素。\CJKsout{兵来将挡,水来土掩。}
记$SAM(P)$为串$P$的SAM,$S_P(i)$表示$P$的第$i$个后缀。

首先用根节点表示空串,记为$R$。

考虑由$SAM(P)$构造出$SAM(Px)$,添加字符$x$使得当前串新增后缀$x,S_P(1)x,\\\cdots,S_P(|P|)x$。那么
可以考虑新增一个节点,将$SAM(P)$表示后缀的节点向它连一条$x$转移边。注意到上一次增加的节点
(记为$last(P)$)表示了$S_P(1),\cdots,S_P(|P|)$,需要连边的只有与$R$与$last(P)$。此时每个节点
需要存储转移边$nxt[|\Sigma|]$。

注意到每个节点的相同字母转移边只能有一条,而上述方法中节点$R$一直在连边,当已有转移边时就无法转移了。
考虑向串$ab$后加字符$a$,$R$已有字符$a$的转移边。注意到若不连该边,原有的连边可以识别后缀$a$,不过
下次转移时表示后缀的节点不再是$R$与$last(P)$,而是$R,last(P)$与表示$a$的节点。可以考虑用一条链将它们
串起来,从$last(Px)$开始按照最长串长度连向上一个表示后缀的节点,此时每个节点需要存储该节点的$link$。

如何程序化地描述向新点$id$连转移边的过程呢?那就是从$last(P)$开始,若该节点没有转移边则将其连向新点,然后
根据$link$指针跳到下一个表示后缀的节点。若跳到$R$处也没有转移边,则设置新点的$link=R$,更新$last(Px)=id$,
退出程序。否则说明$SAM(P)$已经能够识别该后缀,只需设置$id$的$link$。记当前有转移边的节点为$p$,转移边
指向$q$。注意到节点$q$不一定表示$p$表示的后缀+字符$x$,需要进行判断。

考虑按照最长串长度排列的$link$链,既然链上所有节点对应了所有后缀,且链上节点代表的最长串长度从$last(P)$
递减,那么每一个节点代表的后缀长度是一个连续区间,且区间之间无缝。此外节点还有一个性质:节点编号+串长度
唯一对应一个子串。证明:若节点编号与串长度相同,而代表的子串不同,则说明这两条不重叠的路径等长,将这两条路径
从当前节点延伸到某个叶子,对应了两个长度相等且不相同的后缀,与事实矛盾。

有了这两个性质,可以推出一个结论:若节点$q$的最长串长度$len_q$恰好等于节点$p$的最长串长度$len_p+1$,
则节点$q$及其在$link$链上往$R$的方向的元素代表了剩余的新后缀
($S_P(0\cdots len_p)+x\rightarrow S_{Px}(1 \cdots len_q)$,还有链头$R$代表空串,而长度
$\geq len_p+2$的后缀已经在连转移边后可被$id$识别)。此时每个节点需要存储最长串长度$len$。若
$len_p+1=len_q$,则令$link_{id}=q$。否则考虑分割出我们需要的后缀,即把节点$q$代表的子串切割
为两部分,其中一部分的最长串长度为$len_p+1$。可以新建一个节点$cq$,由于切出的这一个子串也能转移到
新的节点,所以需要拷贝$q$的转移边。由于$len_{cq}<len_q$,需要修改$link$链为
$link_q\leftarrow cq \leftarrow q$,最后将$id$的$link$指向$cq$。注意分割后原来转移边指向$q$的
节点$p$及它在$link$链上的节点都应该改指向$cq$,因为原本它们指向的就是这一部分子串(只是因为未分离前
合并到指向$q$)。无论是修改转移边到$id$还是$cp$,都是从$last(P)$或$p$开始的连续子链,因为之前修改
转移边的操作影响的都是连续子链。

SAM的点数不超过2n-2,边数不超过3n-3(转移边+Parent树边),构造复杂度为$O(n|\Sigma|)$,证明参见
文末引用。

代码如下:
\begin{lstlisting}
struct SAM {
    struct State {
        int nxt[26],link,len;
    } S[size*2];
    int last,siz;
    SAM():last(1),siz(1) {}
    void extend(int c) {
        int id=++siz;
        S[id].len=S[last].len+1;
        int p=last;
        while(p && !S[p].nxt[c]) {
            S[p].nxt[c]=id;
            p=S[p].link;
        }
        if(p) {
            int q=S[p].nxt[c];
            if(S[p].len+1==S[q].len)
                S[id].link=q;
            else {
                int cq=++siz;
                S[cq]=S[q];
                S[cq].len=S[p].len+1;
                while(p && S[p].nxt[c]==q) {
                    S[p].nxt[c]=cq;
                    p=S[p].link;
                }
                S[q].link=S[id].link=cq;
            }
        }
        else S[id].link=1;
        last=id;
    }
}
\end{lstlisting}

\subsubsection{Right集合}
将节点$u$表示的所有子串从$u$开始沿着一条路继续向后走,走到某个后缀,再从这个后缀回退到$u$的位置,
字符串上的位置也对应倒退,最后的位置就是这些子串的某个公共尾部位置。这些公共尾部位置的集合称为$u$
的Right集合。
\subsection{Parent树的应用}
将$link$链并在一起,就得到了一棵树,称为Parent树。
\subsubsection{Parent树性质}
\begin{theorem}
    父节点的Right集合是儿子Right集合的并。
\end{theorem}

证明:父节点代表的子串同时也是儿子代表子串的后缀,所以儿子代表子串存在的地方,父节点代表子串也存在。

\begin{theorem}
    不在同一条$link$链上的节点Right集合不相交。
\end{theorem}

证明留坑待补。
\index{*TODO!SAMRight集合性质证明}

这个性质保证了可持久化线段树合并的复杂度。

Parent树自底向上Right集合逐渐变大,匹配子串长度逐渐变小。据此性质可贪心倍增跳到满足条件的最高的祖先
后利用该位置的数据查询。
\subsubsection{子串匹配}
注意Parent树上的父亲是儿子的后缀,因此匹配子串时可以
在转移边上跑,失配就跳Parent树的link(等同于fail树)。
\subsubsection{与后缀树的联系}
Parent树是反串的后缀树,因为父亲是儿子的后缀,等同于父亲的反串是儿子反串的前缀,且该树可以识别
反串的所有后缀。
\subsubsection{计数问题}
对于每一个状态维护一个$right$值表示当前状态的Right集合大小。
新增状态时该状态贡献了1,但注意克隆状态并没有贡献,所以克隆后令$cq.right=0$。
最后拓扑排序dp在Parent树上自底向上更新就可以得到真实right值。

状态$s$表示了$s.len-s.link.len$个本质不同的子串,每种子串有$s.right$个。

优化:拓扑排序时可以按照len进行分层基数排序,但是广义SAM不能使用这种方法。
\subsubsection{可持久化线段树合并维护Right集合}
有时需要判定某个终点是否在某个状态的Right集合内,可以在extend时给新建状态添加
对应的Right值(不需要太严格,子串右节点也行),然后拓扑排序进行线段树合并计算出
每个状态真正的Right集合。时间复杂度$O(n\lg n)$。

{\bfseries 若该方法用于匹配母串的某个子串的子串,在失配时沿着Parent树跳跃,注意要逐位跳,当
匹配长度等于父亲的最长后缀时才跳到父亲。例题:NOI2018 你的名字}
\subsubsection{倍增自匹配}
例题~ 「TJOI / HEOI2016」字符串

通过前缀$\rightarrow$后缀以及二分LCP长度将原问题转换为:给定子串$a$与$b$,
求长度为$m$的$b$的后缀是否出现在子串$a$中。

由于$a$与$b$在同一个串内(不在同一个串?把两个字符串用特殊字符连在一起),该问题
即为判定Right集合中含有$b$的Right值且$r_S\geq m$的状态$S$与$a$对应的Right区间
是否有交。可以在构建SAM时保存每个字符对应的$last$,这些状态一定含有对应字符的Right值。
由于Parent树上Right集合的包含关系,其祖先也有该Right值且Right集合更大,$r_S$递减,
可以使用倍增计算出极大的Right集合所对应的状态(要求有$b$的Right值且$r_S\geq m$),
然后线段树查询该状态的Right集合是否与指定区间有交。

代码:
\lstinputlisting{Source/Source/SAM/LOJ2059.cpp}
\subsection{线性构造后缀数组}
首先构造出SAM,发现last到根的链上的状态分别代表每一个后缀。对这些状态进行
标记,按照字典序DFS,维护DFS子串的长度$d$,通过遍历顺序得到$sa$数组。

注意对于跑单条链的情况要使用路径压缩优化。

代码如下:
\lstinputlisting{Source/Templates/SAM2SA.cpp}

还可以利用Parent树的性质得到更优的做法。反串的SAM的Parent树就是原串的后缀树,对后缀树按照
字典序遍历后就可以得到rank与sa数组。

这个算法的关键在于如何计算出字典序顺序,即每个节点的转移边(转移边不会重复,如果重复就会开新的公共点)。
首先沿着转移边DFS,仅找$len$比自身大1的后继$v$以避免重复遍历。然后将转移字符压入栈中,
设置$link_v$的第$len_v-len_{link_v}$个字符转移为$v$,这恰好是$link_v$到$v$转移边上的首字符。
第二次按照字典序贪心DFS一遍就可以得到后缀数组。

该方法参考了z1j1n1的博客\footnote{
    使用后缀自动机求后缀数组\\
    \url{https://www.cnblogs.com/zhujiangning/p/7999381.html}
}。
\subsection{广义SAM}
有两种构造方法:
\begin{itemize}
    \item 在线:插入一个字符串之前将$last$重置,时间复杂度为O(Trie大小*字符集大小
    +叶子状态深度和)。
    \item 离线:先建出Trie,BFS插入,插入时把父亲在SAM上的编号当做$last$,
    时间复杂度为O(Trie大小*字符集大小)。
\end{itemize}

\subsubsection{统计状态对应的模板串数}
为每个节点记录最后匹配的模板串编号,每次extend后从$last$开始沿着parent树暴力上跳,将这些节点的
$count$值加一,直到遇到被同模板串标记过节点的为止。设状态总数为$S$,最坏时间复杂度$O(S\sqrt{2S})$,
一般情况下不会达到最坏情况,有时跑得比$\lg$做法还快。在想不出更优做法时,这是一个简单有效的方法。

以上内容参考了WC2012陈立杰的讲课课件《后缀自动机 Suffix Automaton》
与Candy?\footnote{[后缀自动机]【学习笔记】
    \url{https://www.cnblogs.com/candy99/p/6374177.html}
}、dwjshift\footnote{
    用SAM建广义后缀树 $\ll$ dwjshift's Blog
    \url{http://dwjshift.logdown.com/posts/304570}
}、Mangoyang\footnote{
    一个用SAM维护多个串的根号特技
    \url{https://www.cnblogs.com/mangoyang/p/10155185.html}
}的博客。Menci的博客写得更详细,一些性质的证明请移步
\url{https://oi.men.ci/suffix-automaton-notes/}。

\subsection{序列自动机}
类比后缀自动机,序列自动机上的每条路径对应一个子序列。

\subsubsection{构造}
序列自动机的构造比较简单,即预处理$nxt[i][j]$表示位置$i$后的第一个字符$j$
出现的位置。存在一个简单的$O(n|\alpha|)$DP:
\begin{lstlisting}
for(int i=n;i>=1;--i) {
    for(int j=0;j<26;++j)
        nxt[i-1][j]=nxt[i][j];
    nxt[i-1][P[i]-'a']=i;
}
\end{lstlisting}

维护可持久化数组可以把时间复杂度降到$O(n\lg |\alpha|)$,在字符集比较大的时候
使用。
\subsubsection{应用}
下列序列数统计均指本质不同的序列。
\paragraph{子序列个数}
记$dp[i]$为从位置$i$开始的子序列个数,位置$i$的字符自成一个子序列,
并且它与$nxt[i][j]$位置的方案构成了本质不同的子序列,因此有
\begin{displaymath}
    dp[i]=1+\sum_j{dp[nxt[i][j]]}
\end{displaymath}
使用记忆化搜索或者逆序dp。
\paragraph{公共子序列个数}
例题~「FJOI2016」所有公共子序列问题

预处理出两个字符串的序列自动机后,使用记忆化搜索在序列自动机上跑。
\paragraph{回文子序列个数}
对原串和反串构建序列自动机,求这两个串的公共子序列数。

记记忆化搜索调用为$DFS(x,y)$,$x,y$分别为在这两个串上的匹配位置,有
$x\leq n+1-y$,等号成立意味着该回文序列为奇序列。但是奇序列不一定满足
其等号成立,如果记忆化搜索搜索到一个偶回文序列,删掉该回文序列中心的一个字符,
就会出现新的奇回文序列,因此在搜索时若满足$x+y<n+1$,要补上该奇序列的贡献。
\paragraph{求满足存在指定子序列要求的最长公共子序列}
给定串$A,B,C$,求$A,B$的最长公共子序列$S$,要求$C$是$S$的子序列。
再给$DFS$增加一个状态记录其在$C$上匹配的位置后记忆化搜索。

上述内容参考了pig\_dog\_baby的博客\footnote{
    序列自动机(一个数组而已...)及经典例题
    \url{https://blog.csdn.net/pig\_dog\_baby/article/details/81145857}
}。
