\section{AC自动机}
\index{A!Aho–Corasick Algorithm}
AC自动机基于Trie树实现,原理类似于KMP算法,即在Trie树上匹配字符串,
失配时根据fail指针跳到下一匹配位置。
\subsection{构造}
首先建出Trie,然后对Trie按BFS序确定fail指针,fail指针指向它的在Trie中的
最长后缀(根据Trie树的性质,这个后缀也是某个模式串的前缀),BFS序遍历可以
保证其长度最长。

为了简化找fail指针的代码,可以设一个虚拟节点0,root编号为1,令root的
fail为0,0的所有儿子全设为root,这样就可以避免判断是否到达root,
同时令找不到后缀的节点的fail为root。

\begin{lstlisting}
const int root=1;
int q[size];
void cook() {
    for(int i=0;i<26;++i)
        T[0].nxt[i]=root;
    T[root].fail=0;
    int b=0,e=1;
    q[b]=root;
    while(b!=e) {
        int u=q[b++];
        for(int i=0;i<26;++i) {
            int v=T[u].nxt[i];
            if(v) {
                int p=T[u].fail;
                while(!T[p].nxt[i])
                    p=T[p].fail;
                T[v].fail=T[p].nxt[i];
                q[e++]=v;
            }
        }
    }
}
\end{lstlisting}

还有一种更加简洁快速的cook过程。让无儿子$v$的节点$u$继承其fail的对应儿子,
省去了在Trie上不断跳fail指针的时间。这种cook方式不会影响fail树的建立。
\begin{lstlisting}
int q[size];
void cook() {
    int b = 0, e = 0;
    for(int i = 0; i < 26; ++i)
        if(T[0].nxt[i])
            q[e++] = T[0].nxt[i];
    while(b != e) {
        int u = q[b++];
        for(int i = 0; i < 26; ++i) {
            int& v = T[u].nxt[i];
            if(v) {
                T[v].fail = T[T[u].fail].nxt[i];
                q[e++] = v;
            } else
                v = T[T[u].fail].nxt[i];
        }
    }
}
\end{lstlisting}
\subsection{查询}
AC自动机的经典功能是多模匹配。
将主串的字符按顺序在自动机上跳,失配就走fail指针,
对于每一次匹配后的状态,不断跳fail到根查找是否存在
节点为某个模式串。
\begin{lstlisting}
void match(const char* str) {
    int p=0;
    while(*str) {
        int c=*str-'a';
        //注意如果使用继承fail优化就不需要跳fail了
        while(p && !T[p].nxt[c])
            p=T[p].fail;
        p=T[p].nxt[c];
        //查找匹配的模式串
        int cp=p;
        while(cp) {
            if(T[cp].cnt)
                //处理匹配字符串
            cp=T[cp].fail;
        }
        ++str;
    }
}
\end{lstlisting}
继承方式不影响查询的正确性(继承fail节点的儿子相当于跳到fail上),
而且同样避免了跳fail。
\subsection{fail树}
容易发现fail指针所指的节点所代表的字符串是自己的最长公共后缀。
从fail向自己连边,可以得到一棵fail树。将代表某个模式串的节点称为终结点,
那么fail树上某个终结点是它子树内的终结点的后缀。

\paragraph{例题~NOI2011 阿狸的打字机}
对于多次询问求第$x$个字符串在第$y$个字符串中出现次数的问题,模拟字符串$y$在AC
自动机上跑的过程,查询匹配时跳fail扫一遍相当于在fail树上将自己的权值传给父亲。
因此先建出fail树,按DFS序平铺为序列,(尽量按照字典序)让$y$在AC自动机上跑,
记录节点被经过的次数,最后求询问中所有$x$的子树权值和,使用树状数组很容易实现。
