\section{最小表示法}
最小表示法用来解决字符串的循环同构问题。一个字符串的最小表示就是它的字典序
最小的循环串。如果两个字符串的循环表示相同,说明这两个字符串循环同构。

考虑朴素算法:维护当前最小表示的起点$i$以及用于比较的起点$j$,初始$i=0,j=1$。
然后按照$S[i]$与$S[j]$的大小分类,若$S[i]=S[j]$,则逐个比较直至$S[i]\neq S[j]$;
若$S[i]<S[j]$,则说明起点$j$不可能成为答案,令$j$后移;若$S[i]>S[j]$,则说明$j$
是更优的答案,令$i=j$,$j$后移。

上述算法的低效性在于$S[i]=S[j]$的比较无法重复利用,比如串aaaaaa可以将其卡到
$O(n^2)$。考虑记录当前起点$i$与$j$的前$k$位都相同,当$S[i+k]\neq S[j+k]$时,
需要移动某一个指针,若只移动一位会导致下一次匹配时仍然重新匹配。那么需要增加指针跳跃
的幅度。设$S[i+k]>S[j+k]$,那么$i$不是最优解,由于
$S[i\ldots i+k-1]=S[j\ldots j+k-1]$,$(i,i+k]$范围内的起点也不是最优解,
因此$i$需要后移$k+1$位。由于在当前阶段内已经扫描了$k+1$次,所以$i,j$的总偏移
等于扫描次数,由于总偏移不超过$4n$,算法的时间复杂度是$O(n)$的。

算法退出的条件为$i<n \land j<n \land k<n$,若$i$或$j\geq n$,则说明扫描完毕,
另一个为合法解;否则有$k=n$,两个均为最小表示。因此可以使用$min(i,j)$作为最终解。

参考代码:
\begin{lstlisting}
int scan(int n, const char* A) {
    int i = 0, j = 1, k = 0;
    while(i < n && j < n && k < n) {
        char ci = A[add(i, k, n)],
                cj = A[add(j, k, n)];
        if(ci == cj)
            ++k;
        else {
            (ci < cj ? j : i) += k + 1;
            if(i == j)
                ++j;
            k = 0;
        }
    }
    return std::min(i, j);
}
\end{lstlisting}

上述内容参考了zy691357966的博客\footnote{
    字符串最小表示法 O(n)算法
    \url{https://blog.csdn.net/zy691357966/article/details/39854359}
}。
