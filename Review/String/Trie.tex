\section{Trie字典树}
Trie字典树利用了字符串的公共前缀信息,一般用作
搭配AC自动机或者实现可持久化01Trie回答xor最值问题。

Trie树上每个节点对应一个字符串的某个前缀,节点的LCA为
这两个节点所代表字符串的最大公共前缀。

代码如下:
\begin{lstlisting}
struct Node{
    int nxt[26],cnt;
} T[tsiz];
void insert(const char* str) {
    static int icnt=0;
    int p=0;
    while(*str) {
        int& id=T[p].nxt[*str-'a']
        if(id==0)
            id=++icnt;
        p=id;
        ++str;
    }
    ++T[p].cnt;
}
\end{lstlisting}

有时Trie的空间需求很大,此时需要考虑压缩Trie的空间,尤其是01Trie。

以01Trie的值$\leq 2^{30}$,串数$n\leq 10^6$,单个节点16Byte为例:

保守估计,开大小$nL$的Trie:457MB。

注意到最坏情况下是$2^{19}$层被填满,剩下一堆长链,此时数组大小为$2^{20}+11n$:183MB。

如果剩余链的答案很好统计,那么干脆将无分支的长链压缩为一点,该标记往往不需要多余空间。若遇到
新分支才把该剩余链向下推。此时数组大小为$2^{20}+n$:31MB。这个方法需要更多的代码,其实现要求精细。

由此可见,Trie的空间优化潜力很大,但需要冒不小的风险(比如惰性扩展的代码写挂了)。
