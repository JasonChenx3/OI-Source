\section{k短路}
\subsection{A*算法}
主要思想是使用启发式函数来修正关键字,以达到加速效果。

步骤如下:

\begin{enumerate}
	\item 将启发式函数设为当前节点到目标节点的最短距离,SSSP预处理。
	\item 维护一个优先队列(关键字为估价函数=当前距离+到目标点的最短路),加入起点。
	\item 选取估价函数最小的点松弛(不比较直接加入堆中)。根据以下定理:
	      \begin{theorem}
		      一个点第$k$次出队后,当前距离就是它到起点的第$k$短路。
	      \end{theorem}
	      在目标节点出队时计算是否为第$k$次出队,满足就返回。
	\item 若中途不返回则说明不存在第$k$短路,返回无解。
\end{enumerate}
{\bfseries 注意判断$s=t$和$s-t$不连通的情况。}

以上内容参考了Z\_Mendez的博客\footnote{A*算法—第K短路 - This is Mendez.
	\url{https://blog.csdn.net/z\_mendez/article/details/47057461}
}。
\subsection{可持久化左偏树法}
步骤如下:
\begin{enumerate}
    \item 对反图做SSSP;
    \item 在反图上建出最短路树,使用DFS递归预处理每个节点在树边路径上的
    所有置换为非树边的方案(使用小根堆维护):
    \begin{enumerate}
        \item 如果当前点不是终点
        \begin{enumerate}
            \item 选取非树边(注意有多条树边的情况,此时只能排除一条边),
            计算将当前边置换成非树边后再走最短路增加的距离
            (注意不可到达终点的情况),加入该节点的堆;
            \item 将父节点的堆合并到自己的堆,因为自己的方案也包括父亲的方案。
        \end{enumerate}
        注意:
        \begin{itemize}
            \item 合并时可以使用克隆开关或者先把自己的堆初始化为父节点的堆。
            \item 插入堆中的还有转移点的编号。
        \end{itemize}

        \item 选取树边递归预处理(注意有多条树边的情况,此时要选择所有的点未处
        理的树边)。
    \end{enumerate}
    \item 接下来计算k短路:
    \begin{enumerate}
        \item 首先考虑最短路;
        \item 使用优先队列维护小根堆(堆的节点编号,距离),把次短路的信息加入优先队列;
        \item 第$k$次从优先队列中弹出的就是第$k$短路,然后往优先队列中加入更长的路径:
        \begin{itemize}
            \item 从转移点开始不走最短路:使用转移点的堆转移,加入该堆的根即可。
            \item 更换转移点:使用原堆继续转移,将当前节点替换为左右儿子。
        \end{itemize}
        这种方法保证了优先队列的小规模,而且更新到优先队列中的方案都比原方案小保证了正确性。
    \end{enumerate}
\end{enumerate}

时间复杂度$O((V+E)\lg V+E \lg E+k \lg k)$。

模板代码(Luogu P2483 [SDOI2010]魔法猪学院\footnote{
【P2483】【模板】k短路([SDOI2010]魔法猪学院) - 洛谷
\url{https://www.luogu.org/problemnew/show/P2483}
}):

\lstinputlisting{Source/Templates/SSKTP.cpp}

以上内容参考了litble\footnote{HDU5960 可持久化左偏树 k短路问题 - litble的成(tui)长(fei)史
	\url{https://blog.csdn.net/litble/article/details/79171311}
}的博客。
