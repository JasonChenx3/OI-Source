\section{k短路}
\subsection{A*算法}
主要思想是使用启发式函数来修正关键字,以达到加速效果。

步骤如下:

\begin{enumerate}
    \item 将启发式函数设为当前节点到目标节点的最短距离,SSSP预处理。
    \item 维护一个优先队列(关键字为估价函数=当前距离+到目标点的最短路),加入起点。
    \item 选取估价函数最小的点松弛(不比较直接加入堆中)。根据以下定理:
    \begin{theorem}
        一个点第$k$次出队后,当前距离就是它到起点的第$k$短路。
    \end{theorem}
    在目标节点出队时计算是否为第$k$次出队,满足就返回。
    \item 若中途不返回则说明不存在第$k$短路,返回无解。
\end{enumerate}
{\bfseries 注意判断$s=t$和$s-t$不连通的情况。}

以上内容参考了Z\_Mendez的博客\footnote{A*算法—第K短路 - This is Mendez.
    \url{https://blog.csdn.net/z\_mendez/article/details/47057461}
}。
\subsection{可持久化左偏树法}

以上内容参考了litble\footnote{HDU5960 可持久化左偏树 k短路问题 - litble的成(tui)长(fei)史
    \url{https://blog.csdn.net/litble/article/details/79171311}
}
