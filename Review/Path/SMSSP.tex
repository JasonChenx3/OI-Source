\section{单/多源最短路}
\subsection{SPFA}
最坏时间复杂度$O(VE)$。
\paragraph{严重警告}
是SPFA让我在NOI2018Day1中滚粗的。

SPFA优化的思路基本上是使队列尽可能接近优先队列。
下列优化中``当前节点''指被松弛后将入队的节点。
\subsubsection{SLF优化}
插入队列时,若当前节点比队首距离更短则插入队首,否则插入队尾。
\subsubsection{LLL优化}
若队首超过队列平均距离则将其塞入队尾,否则进行松弛。
\subsubsection{堆优化}
沿用LLL优化的思路:使用距离尽可能小的节点来松弛。使用堆来维护。
\subsubsection{SLF带容错}
令边权和为$W$,若当前节点比队首距离多$\sqrt{W}$才插入队尾。在边权和
小的图上表现不错。
\subsubsection{mcfx优化}
在第$[2,\sqrt{V}]$次访问当前节点时插入队首,否则插入队尾。在网格图
上表现优秀,搭配SLF带容错优化效果更佳。
\subsubsection{SLF+Swap}
插入队尾后若队首大于队尾则交换首尾。

更多优化与Hack参见\url{https://www.zhihu.com/question/292283275}。
\index{*TODO!卡SPFA}
\subsection{算法优化}
若用节点编号表示状态来求最短路,要具体分析题目,阻止无效状态入队。
\subsection{Dijkstra}
使用二叉堆的最坏时间复杂度$O((V+E)\lg V)$,在稀疏图中表现良好。
稠密图可以使用$O(V^2)$暴力。
\paragraph{强烈安利}
大型考试还是使用Dijkstra算法好了。

注意Dijkstra中不方便的堆内修改可以改为加入新点(标号+松弛距离)的形式,
出堆时与数组中记录的距离值比较,匹配才进行松弛。当然由于每个节点只会松弛一次,
使用flag记录是否松弛过也可以正确运行,这个方法在扩展Dijkstra中比较好用。
\subsection{Johnson算法}
Johnson算法用于求所有节点对间最短路径。
其主要思想是将原图转换为无负权边的图。

算法步骤如下:
\begin{enumerate}
    \item 新建节点$s$,将$s$连向原图所有点,边权为0。
    \item 跑SPFA计算以$s$为源点的最短路,同时检查负权环。
    记点$u$的距离标号为$h[u]$,满足三角不等式$h[u]+w[u][v]\geq h[v]$,
    变形得$h[u]+w[u][v]-h[v]\geq 0$。令新边权为$h[u]+w[u][v]-h[v]$,
    满足Dijkstra的条件。
    \item 以每个节点为源点跑Dijkstra填充最短距离矩阵。记以点$u$为源点时,
    点$u$到点$v$的实际距离为$md[v]$,Dijkstra结果为$dis[v]$,将边权加和后得
    $dis[v]=md[v]+h[u]-h[v]$,变形得$md[v]=dis[v]-h[u]+h[v]$。
\end{enumerate}

第~\ref{DijMCMF}节所述方法就是该算法的应用之一。
该算法在稀疏图上表现比Floyd算法好,时间复杂度$O(VE\lg V)$。
