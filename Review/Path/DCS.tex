\section{差分约束系统}
差分约束系统是一堆形如$x_i-x_j\leq c_k$的不等式组成的不等式组。

若要构造$x_i-x_j=c_k$,加双向不等式约束即可。

\subsection{与最短/长路的关系}
对不等式$x_i-x_j\leq c_k$进行变形得$x_i\leq x_j+c_k$,把$x_i$看做点$i$
到源点的距离,该不等式就等价于从点$j$向点$i$连了一条距离为$c_k$边。
如此便可以将差分约束问题转换为最短路问题。最长路的分析类似。

\subsection{判断是否有可行解}

\begin{theorem}
    当图中存在负权环路时,该差分约束系统无可行解。
\end{theorem}

证明:从负权环路中拆出一条边,记边为$(u->v,w_1)$,其它边合并得$(v->u,w_2)$,
有$w_1+w_2<0$,将边还原成不等式组:
\begin{eqnarray*}
    x_v-x_u&\leq& w_1\\
    x_u-x_v&\leq& w_2\\
\end{eqnarray*}
两式相加得$0\leq w_1+w_2$,与$w_1+w_2>0$矛盾。

使用SPFA判负环有DFS与BFS两种方法。

首先用一个超级源连接所有点,保证图的连通。

\begin{itemize}
    \item DFS-SPFA:松弛时递归下去,当存在环时说明存在负权环。

    注:使用setjmp/longjmp较为方便,记得把调用过程中修改的变量设为
    {\bfseries volatile}。
    \item BFS-SPFA:当某节点{\bfseries 入队次数}(注意不是被松弛次数)大于等于节点
    总数时,存在负权环。
\end{itemize}
一般使用DFS法(跑得飞快),但是如果不存在环,DFS法的效率会比BFS法慢,
所以可以使用卡时技巧将DFS/BFS结合。

代码如下:
\lstinputlisting{Source/Templates/NegRing.cpp}

\subsubsection{例题}
Luogu P4578 [FJOI2018]所罗门王的宝藏
\footnote{【P4578】[FJOI2018]所罗门王的宝藏 - 洛谷
\url{https://www.luogu.org/problemnew/show/P4578}}

看上去可以使用高斯消元法求解系数矩阵为稀疏矩阵的线性方程组。进一步分析发现,若
将$y$方向操作的左右取反,可以得到$x_i-y_j=c_k$的形式。将该形式拆成最短路的
差分约束形式:
\begin{eqnarray*}
    x_i-y_j&\leq& c_k\\
    y_j-x_i&\leq& -c_k\\
\end{eqnarray*}
双向连边后判负环即可。

\subsection{求解}
最短路的距离就是一组可行解,对这组可行解任意加减一个常数可以得到另一个可行解。
