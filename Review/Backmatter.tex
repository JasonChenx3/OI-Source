\chapter{后记}
\subsection{初稿}
2018年10月23日

写这本笔记大概花了我两个月的时间,此时这本笔记大概有200千字(50千字为中文),我
为自己的成果感到骄傲。在此期间我系统地复习了NOIP复赛以上的知识点,学到了许多新的方法/技巧/
思路,重新理解了之前看不懂的算法,收获颇多。我将自己的想法记录于此,以便之后重新翻阅。
该笔记在内容上还是较为完整的,在接下来的时间内我会在巩固的过程中继续学习并补充新的高级算法
/数据结构,以应战省选和NOI2019。

在我作为D类队员参加了NOI2018现场赛后,我发现了自己的严重问题——知识点掌握不扎实,
导致我没能拿到该得的分数。所以我决定自己写一写复习笔记来建立出一套自己的思维体系。

同时作为目前校内在OI道路上走的最远的人(其他人初赛都过不了。。。孤单),我没有
信息学强校学生所拥有的资源——学长。有学长的指导和遗留下的资料,能够少走许多弯路。
既然自己没有学长,只好自己做学长了。这本笔记算是是我的一份心意吧,希望学校的OI事业
能够有所发展,学弟们(???我好像根本不知道有学弟)如果需要我帮助的地方,随时联系我。
我的表达能力不太好,而且有些简单的地方没有细讲,你们可以打开页底下的参考资料链接细看
(这些资料是我自己能够接受并理解的),虽然这些东西在百度和维基上都能轻易找到,但我之前
根本就不知道这些东西的存在(直到考完后才知道,已经晚了)。。。所以你们就把它当做一种索引吧。

在写这本笔记的过程中,我发现自己在使用\LaTeX{}和排版方面还存在许多问题:
\begin{itemize}
    \item 目录太长,分类过细
    \item 参考资料链接过多
    \item 不恰当的知识点分类
    \item 过长的TODO List
    \item 滥用lstlisting
    \item 书籍排版密度过低
    \item 滥用Index
    \item 未熟练使用BibTex
    \item 数学公式排版掌握不熟
    \item 缺少图片
    \item 语言表达不当或过于口语化(才发现``即可''之类的词是从其它blog上学来的)
\end{itemize}

上述问题将会在接下来几轮的review中得到改善。

在此我要感谢学校领导、老师和同学的理解与支持,感谢父母(还有我妹)对我的
大力支持。感谢写出优秀博客的OIers和Wikipedia-EN的维护者,他们为我提供了大量的
参考资料与经验总结,使我受益匪浅,我所做的不过是将它们聚集在一块而已。
\sout{差点忘了感谢CCF}。

刚开始写这本笔记时,我遇到了一件让我感到最幸福的事:我喜欢的女生向我表白了。
在此之前我已经预先认真考虑了几个月,所以当时我毫不犹豫就接受了。那个月是我最幸福的日子,
我开始改变自己,更加努力地学习,以应对将来需要承担的责任。她给了我写这本笔记的动力,她能够
理解许多别人所不能理解我的事情。我当时认为自己是这个世上最幸运的人。但好景不长,在本笔记
初稿即将完成时,我最担心的事情还是发生了——双方突然无话可说。我是预先知道这个问题会发生,
而且已经预先提醒过她了——遇到问题要多沟通,商量解决方案。但在短短几天内,她对我的态度越来
越差。最后她以一堆自相矛盾的理由离开了我,根本不给我与之沟通的机会。从前的誓言,以及我对
她的好,她似乎已经完全忘记了。我不知道她的真实意图,她或许是为了让我安心学习。我仍然相
信自己的单元测试是可靠的,她最终会回来的。我决定履行自己之前的诺言:我等她两年。我好想在
她身边给她讲解数学题;好想在她生理期时给她端杯热水,陪在她身边;好想和她一起去公园散步,
在绿道慢跑,一起看动画电影;好想在她心情不好时,自己在旁边给她疏导,给她一个拥抱;好想
静静地坐在一旁,听她弹琴唱歌;好想再让她笑着敲我的头;好想高中毕业后用自己的Renderer给她
渲一个鸽了两年的生日礼物。唉,这些愿望怕是无法实现了。现在自己还是先努力学习,认真搞OI,
考上理想的大学。\sout{得之我幸,不得我命。未来会怎么样,得看自己的造化了。}
{\bfseries I can do it!}

$\frac{\sin 4\theta}{\sin \theta}$,我等你回来。
