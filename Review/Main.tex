\documentclass[10pt,AutoFakeBold,AutoFakeSlant]{book}
%\usepackage{titletoc}
%\usepackage{titlesec}
%\usepackage{ctexcap}
\usepackage{xeCJK}
\setCJKmainfont{SimSun}
\usepackage[english]{babel}
\usepackage[T1]{fontenc}
\usepackage{listings}
\usepackage{xcolor}
\lstset{
    columns=fixed,
    frame=none,
    backgroundcolor=\color[RGB]{255,255,255},
    keywordstyle=\color[RGB]{40,40,255},
    commentstyle=\it\color[RGB]{0,96,96},
    stringstyle=\rmfamily\slshape\color[RGB]{128,0,0},
    showstringspaces=false,
    escapeinside=``,
    language=c++,
    morekeywords={alignas,continute,friend,register,true,alignof,decltype,goto,
    reinterpret_cast,try,asm,defult,if,return,typedef,auto,delete,inline,short,
    typeid,bool,do,int,signed,typename,break,double,long,sizeof,union,case,
    dynamic_cast,mutable,static,unsigned,catch,else,namespace,static_assert,using,
    char,enum,new,static_cast,virtual,char16_t,char32_t,explict,noexcept,struct,
    void,export,nullptr,switch,volatile,class,extern,operator,template,wchar_t,
    const,false,private,this,while,constexpr,float,protected,thread_local,
    const_cast,for,public,throw}
}
\usepackage{amssymb}
\usepackage{amsmath}
\usepackage[hyperfootnotes=true]{hyperref}
\usepackage{tabularx}
\usepackage{url}
\usepackage{makeidx}
\usepackage[numbib,numindex,chapter]{tocbibind}
\usepackage{lastpage}
\usepackage{array}
\makeindex
\pagestyle{headings}
\begin{document}
\newtheorem{theorem}{定理}[chapter]
\newtheorem{lemma}[theorem]{引理}
\newtheorem{property}{性质}[chapter]
\newtheorem{inference}[theorem]{推论}
\newcommand{\ud}{\mathrm{d}}
\title{OI知识点复习笔记}
\author{dtcxzyw}
\frontmatter
\maketitle
\chapter{前言}
本人写复习笔记的目的有两个:
\begin{itemize}
	\item 系统地复习知识点并挖掘一些有用的性质。
	\item 学会使用\LaTeX{}。
\end{itemize}
当前共\pageref{LastPage}页,约\input{../latex/charcnt}字。
%\renewcommand\contentsname{目录}
\tableofcontents
\mainmatter
\chapter{博弈论}
\section{SG函数与SG定理}

\subsection{适用范围}

一切Impartial Combinatorial Games都等价于Nim游戏,可以使用SG函数
解决。\index{I!Impartial Combinatorial\\ Games}

该类游戏拥有如下特征:\footnote{参见 Impartial game - Wikipedia
	\url{https://en.wikipedia.org/wiki/Impartial_game}}

\begin{itemize}

	\item 两个玩家轮流操作
	\item 当有一名玩家无法操作时,游戏结束
	\item 游戏会在有限次操作后结束(状态转移图是一个DAG)
	\item 游戏对双方是公平的,所有操作必须能够由双方完成(即当双方都采取最优策略
	      时,游戏的胜负只取决于先后手)
	\item 双方在开局前已知道关于游戏的全部信息,并在游戏时采用最优策略。

\end{itemize}

\subsection{SG函数}

接下来给出必胜点和必败点的定义(前提是双方均采用最优策略):

\begin{itemize}
	\item 必胜点(N-Position):处于该状态的玩家必胜\index{N!N-Position}
	\item 必败点(P-Position):处于该状态的玩家必败\index{P!P-Position}
\end{itemize}

必胜点与必败点有如下性质:

\begin{itemize}
	\item \begin{property}
		      终结点为必败点
	      \end{property}
	\item \begin{property}
		      必败点的下一状态必然为必胜点(某玩家必败,等价于无论他如何操作都使另一
		      玩家必胜)
	      \end{property}
	\item \begin{property}
		      从必胜点出发至少有一种方式进入必败点(该玩家的最优策略就是使状态转移到
		      必败点)
	      \end{property}
\end{itemize}

要判断哪个玩家必胜,一般使用SG函数和SG定理计算出先手所在状态(即初始状态)是必胜点
还是必败点。

SG函数的定义如下:$SG(x)=mex(S(x))$\index{S!SpragueGrundy Function}

其中$S(x)$是状态x的后继状态的SG函数值的集合,$mex(S)$是没有出现在集合$S$中的最
小非负整数。

以Nim游戏为例,可根据函数定义计算SG值:

\lstinputlisting[title=NimSG]{GameTheory/NimSG.cpp}

\subsection{SG定理}

\index{S!SpragueGrundy Theorem}

\begin{theorem}[SpragueGrundy Theorem A]
	\bfseries 假设当某玩家无法操作时,该玩家失败。\mdseries
	若$SG(x)=0$,则该状态为必败态,否则为必胜态。
\end{theorem}

归纳证明:假设该定理对状态x后继的状态成立,则

\begin{itemize}
	\item 若$SG(x)>0$,则说明存在一个后继状态$y$,使得$SG(y)=0$,因为$y$为必败
	      态,所以$x$为必胜态。
	\item 若$SG(x)=0$,则说明对$x$的任意后继状态$y$,都有$SG(y)>0$,因为$y$为
	      必胜态,所以$x$为必败态。
\end{itemize}

\begin{theorem}[SpragueGrundy Theorem B]\label{SGB}
	游戏的SG函数值等于各子游戏函数值的Nim和(即xor和)。
\end{theorem}

设$S_X$为$X$后继状态的集合,$b=SG(X_1)\oplus \cdots \oplus SG(X_N)$。

该定理可分为两个引理证明:

\begin{enumerate}
	\item
	      \begin{lemma}\label{SGBL1}
		      $\forall_{a\in N,a<b},\exists_{X'\in S_X},SG(X')=a$
	      \end{lemma}

	      归纳证明:首先假设该引理对子游戏成立。

	      设$d=b\oplus a$,$d$的最高位为k,则存在$SG(X_i)$的第k位为1
	      ($d$的那一位由奇数个$SG(X_i)$贡献)。

	      所以$SG(X_i)\oplus d<SG(X_i)$,由假设得$\exists_{X_i'\in
			      S_{X_i}},SG(X_i')=SG(X_i)\oplus d$。

	      结合$a=b\oplus d$得$a=SG(X_1)\oplus \cdots \oplus SG(X_i')
		      \oplus \cdots \oplus SG(X_N)$

	      又因为$\left\{X_1,\cdots,X_i',\cdots,X_N\right\}\in S_X$,所以引理~\ref{SGBL1}得证。

	\item
	      \begin{lemma}\label{SGBL2}
		      $\forall_{X'\in S_X},SG(X')\neq b$
	      \end{lemma}

	      反证法:假设SG定理对子状态成立(这里貌似不严谨),\\且$\exists_{X' \in S_X},SG(X')=b$。

	      那么就有$SG(X_i')=SG(X_i)$,由$mex$函数的定义可得两式矛盾,
		  引理~\ref{SGBL2}得证。

\end{enumerate}

以上内容参考了Angel\_Kitty\footnote{SG函数和SG定理[详解] Angel\_Kitty
\url{https://www.cnblogs.com/ECJTUACM-873284962/p/6921829.html}}与
PhilipsWeng\footnote{SG定理
	\url{https://blog.csdn.net/PhilipsWeng/article/details/48395375}}的博客。

\subsubsection{例题}

Luogu P2575 高手过招\footnote{\url{https://www.luogu.org/problemnew/show/P2575}}

每行棋子可视为一个子游戏,状压后使用DFS计算SG函数值,然后利用定理~\ref{SGB}计算整个
游戏的SG函数值。

\lstinputlisting[title=Luogu P2575]{Source/Source/'Game Theory'/2575.cpp}

\section{Nim系列游戏}

\subsection{Nim游戏}

普通Nim游戏的定义:
有两个玩家轮流从许多堆中移除对象。在每个回合中,玩家选择一个非空的堆,可以移除任何数量
的对象,但至少移除一个对象。无法操作的玩家为败者。

此类游戏可看做是Bash游戏的特殊化。

\begin{Theorem}
	$SG_{Nim}(x)=x$
\end{Theorem}

证明略。

\subsection{Bash游戏}

Bash游戏与普通Nim游戏的区别是增加了每次最多移除k个对象的限制。

\begin{Theorem}
	$SG_{Bash}(x)=x~mod~(k+1)$
\end{Theorem}

证明略。

\subsection{NimK游戏}

NimK游戏与普通Nim游戏的区别是每次可以从不超过k个堆中移除任意数目对象。

\begin{Theorem}\label{NimK}
	将每堆对象的数目拆位,若每位上1的个数mod(k+1)均为0,则必败,反之必胜。
\end{Theorem}

记忆:普通Nim游戏可理解为mod 2的情况。

算法正确性证明:



定理~\ref{NimK}得证。

\subsection{Anti Nim}

不能操作的玩家胜利。

\begin{Theorem}\label{AntiNim}
	先手必胜当且仅当满足以下条件之一:
	\begin{enumerate}
		\item $SG(x)=0$ 且所有堆的对象数都为1
		\item $SG(x)\not=0$ 且至少有一堆对象数大于1
	\end{enumerate}

\end{Theorem}

证明:
定义对象数为1的叫A堆,大于1的叫B堆。

\begin{enumerate}
	\item 若所有堆均为A堆,则奇数堆先手必败,反之必胜。
	\item 若B堆数等于1,显然$SG(x)\not=0$,则可根据堆的总数确定取该堆的数目,
	      使下一状态为情况1的奇数堆,所以先手必胜。
	\item 若B堆数大于1,则
	      \begin{enumerate}
		      \item 若$SG(x)=0$,则必须留下超过2个B堆并使$SG(x')\not=0$,否则会使
		            对方进入情况2的必胜态。
		      \item 若$SG(x)\not=0$,则根据Nim游戏的理论(必胜态->必败态),存在一种方法转移至情况3的子情况1。
	      \end{enumerate}
	      若玩家处于情况3的子情况2中,则可以在有限次回合内使对方无法转移至子情况2,
	      因此该状态为必胜态。
\end{enumerate}

定理~\ref{AntiNim}得证。

\subsection{阶梯博弈(Staircase Nim)}


\subsubsection{例题}

Luogu P3480 [POI2009]KAM-Pebbles\footnote{\url{https://www.luogu.org/problemnew/show/P3480}}

对于这题可将原条件通过差分转换为阶梯博弈模型($A_i \geq A_i-1 \Leftrightarrow
	A_i-A_i-1 \geq 0$)。

\lstinputlisting[title=Luogu P3480]{Source/'Game Theory'/3480.cpp}

出题灵感:Anti BashK游戏

以上内容参考了forezxl\footnote{anti-Nim游戏(反Nim游戏)简介
	\url{https://blog.csdn.net/a1799342217/article/details/78274410}}和
hehedad\footnote{关于nimk类型博弈的详细理解与解释
	\url{https://blog.csdn.net/chenshibo17/article/details/79783523}}的博客。

\section{Alpha-Beta剪枝}

\section{本章注记}
更多博弈论模型参见~博弈游戏的各种经典模型(备忘) - Randolph87 - 博客园
 \url{http://www.cnblogs.com/Randolph87/p/5804798.html},待补充。
\index{TODO!补充博弈论经典模型}

\chapter{网络流}
\section{二分图}
\index{B!Bipartite Graph}
\subsection{二分图判定}
\begin{property}
	二分图中不存在奇环。
\end{property}
如果存在奇环,则必有一条边的端点属于同一集合。
所以可以使用DFS染色来判定二分图,遇到矛盾则退出。

\lstinputlisting[title=BGJudge.cpp]{NetworkFlows/BGJudge.cpp}

\subsection{二分图最大匹配}

\subsubsection{匈牙利算法}
\index{H!Hungarian Algorithm}

匈牙利算法的主要步骤就是遍历左集合的每一个顶点,使得其尽可能找到一个匹配。
要为该顶点找到一个匹配,首先遍历边,如果右顶点已经有匹配,则递归尝试让该
匹配点重新找一个匹配,如果右顶点无匹配或者更换匹配成功,则这条边是一个匹配。

原则:有机会上,没机会创造机会也要上。
\footnote{Dark\_Scope 趣写算法系列之--匈牙利算法
	\url{https://blog.csdn.net/dark\_scope/article/details/8880547}}

感性的算法的正确性证明:每次递归时匹配数只增不减,且递归有权修改整个连通块
的着色情况。(似乎并没有什么说服力)。

匈牙利算法的时间复杂度为$O(VE)$,每次尝试匹配的复杂度为$O(E)$。

\index{*TODO!匈牙利算法标准描述与正确性证明}

\subsubsection{Greedy Matching}
可以先遍历一次图,贪心地连边,以减少尝试拆开匹配边的次数。
在图很大的时候有加速效果。

该方法参考了江任捷的演算法筆記\footnote{
    演算法筆記 - Matching
    \url{http://www.csie.ntnu.edu.tw/\~u91029/Matching.html\#4}
}。
\subsubsection{Hopcroft–Karp Algorithm}
\index{H!Hopcroft–Karp Algorithm}
暂时先坑着\sout{为什么不写Dinic呢}。
\index{*TODO!Hopcroft–Karp算法}

\subsubsection{例题}

Luogu P1129 [ZJOI2007]矩阵游戏
\footnote{\url{https://www.luogu.org/problemnew/show/P1129}}

首先用二分图最大匹配找到n个不同行且不同列的黑格子(置换矩阵P),然后就可以操作得到
目标矩阵(单位矩阵I)了。

\lstinputlisting[title=Luogu P1129]{Source/Unclassified/Done/1129.cpp}

\subsection{二分图最大权匹配 Kuhn-Munkras Algorithm}
\index{K!Kuhn-Munkras Algorithm}
\sout{先用费用流做吧,暂时先坑着。}
\subsubsection{起步}
维护每个左/右顶点的权值(称为顶标),所有节点的顶标和为答案上界。
令每个左顶点的顶标为出边边权最大值,右顶点顶标为0。

对每个顶点运行匈牙利算法,若左右顶点顶标之和等于边权,则考虑连边;
若无法为当前点找到匹配,则将访问到的左顶点顶标-1,右顶点顶标+1,
等价于使答案上界-1(DFS访问树中的叶子必为左顶点),重新为该点寻找匹配。
把任意二分图当做完全二分图(不存在的边权值为0),迭代必定会结束。

这种做法能够保证在找到最大匹配的情况下使权值和最大。
\subsubsection{优化1}
可以发现在左-1右+1后,原先边权等于左右顶点顶标之和的边仍然被经过,
一个简单的思路是一次性突破``瓶颈'',即令下次增广时终点位置处的某条边从
不可连边变为可连边,每次DFS增广时维护(顶标和-边权)的最小值$d$,
若匹配失败则左$-d$右$+d$。

这才是复杂度比较靠谱的算法($O(n^3)$)。
\subsubsection{优化2}
在匹配每个点时,初始化所有右顶点的松弛函数$slack$为$\infty$,然后
DFS时$slack$维护(顶标和-边权)的最小值。若匹配失败则令$d$为未访问右
顶点的$slack$函数最小值,左$-d$右$+d$,同时未访问节点的$slack-=d$。

该优化的复杂度不变,但实测该方法比优化1的效率更高(3x)。
\subsubsection{优化3}
考虑记录其增广时的路径,然后将递归算法转换为非递归算法。
\begin{lstlisting}
int w[size][size],lh[size],rh[size],pair[size],
    pre[size],slack[size];
bool flag[size];
void aug(int s) {
    reset(flag);
    reset(pre);
    reset(slack,0x3f);
    pair[0]=s;
    int u=0;
    do {
        int v=pair[u],minh=inf,nxt;
        flag[u]=true;
        // `再次DFS后新访问到了点u和它的匹配点`
        // `为点v找新匹配点`
        for(int i=1;i<=n;++i)
            if(!flag[i]){
                int delta=lh[v]+rh[i]-w[v][i];
                if(delta<slack[i])
                    slack[i]=delta,pre[i]=u;
                    // `点i的匹配点有可能置换为u的匹配点,`
                    // `以腾出u的匹配点的空位`
                if(minh>slack[i])
                    minh=slack[i],nxt=i;// `点i下次将被访问`
            }
        //松弛
        for(int i=0;i<=n;++i)
            if(flag[i])lh[pair[i]]-=minh,rh[i]+=minh;
            else slack[i]-=minh;
        u=nxt;
    } while(pair[u]);// `直到找到未匹配点为止`
    // `置换匹配`
    while(u) {
        int p=pre[u];
        pair[u]=pair[p];
        u=p;
    }
}
int KM(int n) {
    for(int i=1;i<=n;++i) {
        int maxh=0;
        for(int j=1;j<=n;++j)
            maxh=std::max(maxh,w[i][j]);
        lh[i]=maxh;
    }
    reset(rh);
    reset(pair);
    for(int i=1;i<=n;++i)
        aug(i);
    int res=0;
    for(int i=1;i<=n;++i)
        res+=w[pair[i]][i];
    return res;
}
\end{lstlisting}
实测该方法比优化2的效率更高(2x)。
\index{*TODO!解释KM算法优化的合理性}
\subsection{二分图常见模型}
\subsubsection{最小点覆盖}
\index{K!König's theorem}
\begin{theorem}[König's Theorem]
	最小点覆盖数=最大匹配数。
\end{theorem}

使用反证法证明:如果有一条边两端顶点都不在最大匹配上,那么这条边可以进入最大匹配
成为一个更大的匹配边集,所以与最大匹配的假设矛盾。

\subsubsection{最大独立集}

\begin{theorem}
	最大独立集大小=顶点数-最小点覆盖数=顶点数-最大匹配数
\end{theorem}

证明:

容易发现去掉二分图中的最小点覆盖可得到一个独立集(若其不是独立集,则说明存在一条
边未被覆盖,与点覆盖的定义矛盾)。尝试以此独立集为基础扩展,可以发现若要使点覆盖
中的某个点变为独立集的点,由最小点覆盖数=最大匹配数可知,最小点覆盖的每个点都与$\geq 1$
的边相连,因此必须使不少于1个原独立集的点被删除。所以无论如何修改,最多得到与之大小
相等的独立集。

\subsubsection{DAG最小路径覆盖}

\paragraph{最小不相交路径覆盖}

将顶点拆成左右两点,若存在边$u\rightarrow v$则连边$Lu\rightarrow Rv$,求二分图最大匹配。

\begin{theorem}
	最小路径覆盖数=顶点数-二分图最大匹配数。
\end{theorem}

证明:二分图中每增加一个匹配,就意味着减少一条路径。

\paragraph{最小可相交路径覆盖}

先用Floyd求出传递闭包,转化为最小不相交路径覆盖问题。
因为如果要从a走到b,直接连边可以避开中间点的流量限制。

以上内容参考了罗茜\footnote{二分图详解及总结
	\url{https://www.cnblogs.com/alihenaixiao/p/4695298.html}},
justPassBy\footnote{有向无环图(DAG)的最小路径覆盖
	\url{https://www.cnblogs.com/justPassBy/p/5369930.html}}和
不可不戒\footnote{二分图:最大独立集\&最大匹配\&最小顶点覆盖
	\url{https://blog.csdn.net/lezg\_bkbj/article/details/9872189}}
的博客。
\subsection{Hall定理}
\index{H!Hall's Marriage Theorem}
Hall定理用于判断二分图是否存在完美匹配。
\begin{theorem}\label{Hall}
    二分图$G=\{V1,V2,E\},|V1|\leq|V2|$存在完美匹配当且仅当$V1$中任意$k$个顶点
    至少与$V2$中任意$k$个顶点相连。
\end{theorem}
\paragraph{证明}
充分性:假设二分图$G$不存在完美匹配,记$G$的最大匹配为$M$,$V1$上至少有一点
$u$不在$M$上。由条件可知点$u$有一条不在$M$上的边,记对面的点为$v$。若点$v$不在
$M$上,则与$M$为最大匹配矛盾;否则尝试使用匈牙利算法寻找增广路,记涉及到的$V1$的子集
为$S$,则右边至少有$|S|$个节点与其相连,因而存在增广路,与$M$为最大匹配矛盾。

必要性:由于二分图$G$有完美匹配,$V1$的$k$个顶点至少与各自的匹配相连。

还有一个比较有用的推论:
\begin{inference}
   对于二分图$G=\{V1,V2,E\},|V1|\leq|V2|$,若存在整数$t$,满足$V1$中
   任意节点的度数$\geq t$,$V2$中任意节点的度数$\leq t$,则$G$存在完美匹配。
\end{inference}

\paragraph{例题}
[POI2009]LYZ-Ice Skates

由定理~\ref{Hall}可以考虑枚举所有集合,但复杂度无法接受,考虑排掉一些显然不优的集合。
选出的集合可以分为3类:
\begin{itemize}
    \item 脚的大小连续;
    \item 脚的大小不连续但是鞋号区间连续,把中间未被选中的脚的大小选中,但是鞋号区间不变,
    可以有更充分的证据证明不存在完美匹配;
    \item 脚的大小不连续且鞋号区间不连续,这个集合可以根据鞋号区间的连续性分为
    前两种集合,每个集合是独立的子问题。
\end{itemize}
因此只需考虑脚的大小连续的集合。

记脚的大小为$i$的人数有$a_i$个,根据定理有$\displaystyle \sum_{i=l}^r{a_i}
\leq (r+d-l+1)*k$。让右端为常数,得$\displaystyle \sum_{i=l}^r{a_i-k}\leq d*k$,
可用线段树维护最大子段和。

代码:
\lstinputlisting{Source/Templates/Hall.cpp}

上述内容参考了Feynman1999的博客\footnote{
    Hall定理(二分图匹配问题,Hungary算法基础)
    \url{https://blog.csdn.net/feynman1999/article/details/76037603}
}。

\section{最大流}
Dinic与ISAP属于Ford-Fulkerson方法中的SAP(Shortest Augment Path)系。
而HLPP属于Push–Relabel算法。
\subsection{Dinic算法}
\index{D!Dinic}
个人比较喜欢使用Dinic算法\sout{(因为我只会这个)}。

Dinic的计算流程如下:
\begin{enumerate}
	\item BFS建分层图,若找不到增广路则退出;
	\item DFS在分层图上找增广路并修改流量,重复步骤1。
\end{enumerate}

时间复杂度证明:

\begin{enumerate}
	\item \begin{lemma}
		Dinic每次BFS后的阻塞流层数是递增的(即$d[t]$递增)。
	\end{lemma}
	\item 每次BFS的时间复杂度为$O(E)$。
	\item 每次DFS的时间复杂度为$O(VE)$。
\end{enumerate}

因此算法的时间复杂度为$O(V^2E)$。

在容量均为1的图上,Dinic的时间复杂度为$O(min \{ V^\frac{2}{3},E^\frac{1}{2} \} E)$,
证明:

留坑待填,参见\cite{NFTGC}。

做二分图最大匹配时Dinic跑得飞快,时间复杂度$O(\sqrt V E)$,证明:

留坑待填,参见\cite{DSNA}。

\index{*TODO!特殊图下Dinic的时间复杂度证明}

时间复杂度证明源自Wikipedia-EN\footnote{
	Dinic's algorithm - Wikipedia
	\url{https://en.wikipedia.org/wiki/Dinic\%27s\_algorithm}}以及
	permui的博客\footnote{ 最大流算法-ISAP - permui
		\url{https://www.cnblogs.com/owenyu/p/6852664.html}}
\subsubsection{优化}
\begin{itemize}
	\item 当前弧优化:每次从未遍历的边开始遍历,减少重复计算(就算前面的边没满,
	      下一次还可以增广)。
	\item 记录无法增广的点(将其深度设为-1),避免重复计算。
	\item (玄学,未测试)BFS找到一条增广路就退出,无法解释。
	\item 若图为分层图,在Dinic之前贪心预流(依旧玄学,未测试):
	      \begin{enumerate}
		      \item 从$s$开始逐层递推,计算能够流出节点$i$的流量$out[i]$;
		      \item 从$t$开始逐层倒推,计算每条边的实际流量。
	      \end{enumerate}
	      代码:

	      \lstinputlisting[title=PreFlow]{NetworkFlows/PreFlow.cpp}

	      该方法源自沐阳的博客。
	      \footnote{ZOJ-2364 Data Transmission 分层图阻塞流 Dinic+贪心预流 - 沐阳
		      \url{https://www.cnblogs.com/Lyush/p/3204099.html}}
\end{itemize}

\subsubsection{板子}

常规优化:
\lstinputlisting[title=DinicA]{Source/Templates/DinicA.cpp}

玄学优化(注意在随机数据下表现可能更差):

\begin{itemize}
	\item 伸缩操作:首先按照边的容量从大到小排序,然后按照
	$cap>=2^k,2^(k-1),\cdots,2^0$加边,每加一组边跑一次Dinic。
	时间复杂度$O(VE\lg C)$。
	\item 延迟加反向边:建图时仍然加正反向边,但是第一次Dinic
	时避开反向边,第二次Dinic时才考虑反向边。
	\item 不退流跑,一次性退流:BFS失败时才退流,若退流后仍然失败才退出迭代。
\end{itemize}

这些优化参见kczno1的博客\footnote{
	论如何用dinic ac 最大流 加强版
	\url{http://kczno1.blog.uoj.ac/blog/3375}}。

参考代码:

常规优化+伸缩操作+延迟加反向边(实践中还是这个比较好用):
\lstinputlisting[title=DinicB]{Source/Templates/DinicB.cpp}

kczno1的最新做法-不退流跑,一次性退流:
\lstinputlisting[title=DinicC]{Source/Templates/DinicC.cpp}

\subsubsection{当Dinic遇上LCT}

留坑待补。
\index{*TODO!Dinic with LCT}

\subsection{ISAP算法}
\index{I!Improved Shortest Augment Path}

Dinic每次BFS计算分层图的过程为找最短增广路的过程。每次BFS
重新计算层次编号$d$似乎有些浪费,因此ISAP在Dinic的基础上用
DFS直接修改层次编号的方式来优化算法。ISAP的时间复杂度仍然为$O(V^2E)$。
记数组$d[u]$为残存网络中点$u$到汇点的最短距离,为了编码方便让$d[T]=1$。

算法步骤如下:
\begin{itemize}
	\item 迭代DFS增广,若找不到满足$d[u]=d[v]+1$的可增广边则说明此时的最短路标号
	已经过时,为了让点$u$可增广,令$d[u]=min\{d[v]\}+1$。
	\item 若$d[S]>|V|$则说明已不存在简单增广路径,退出迭代。
\end{itemize}

\subsubsection{优化}
\begin{itemize}
	\item 若数组$d$被初始化为0,则DFS需要$O(n^2)$的时间来初始化
	数组$d$。可以在增广前从汇点开始BFS$O(n+m)$预处理数组$d$。
	\item gap优化:维护每种层次编号的数量$gap[d]$,若$gap[d]=0$则说明
	出现了断层,不存在新的增广路。此时简单地令$d[S]=n+1$结束算法。
	\item 类似Dinic可以使用当前弧优化,{\bfseries 但在层次标号被修改后要重置链头}。
	\item 层次标号的修改是连续的,每次增广完后$++d[u]$。
	\item 流量用完后直接退出。
\end{itemize}

板子(代码似乎比DinicA还短而且跑得比DinicB还快):
\lstinputlisting[title=ISAP]{Source/Templates/ISAP.cpp}

{\bfseries 注意$mf=0$时直接返回不要更新层次标号。}

ISAP算法参考了permui的博客\footnote{ 最大流算法-ISAP - permui
\url{https://www.cnblogs.com/owenyu/p/6852664.html}}。

\subsection{HLPP算法}
\index{H!Highest-label push–relabel\\ algorithm}

\sout{算法导论\cite{ITA3}~26.4节讲的推送-重贴标签算法是$O(V^3)$的。。。}

HLPP算法使用``推送-重贴标签''算法,其时间复杂度为$O(V^2\sqrt{E})$。虽然时间复杂度
比Dinic优,但由于HLPP算法上界较紧,在实践中往往跑不过Dinic(加了优化后表现还行)。

\subsubsection{推送-重贴标签算法}

以水流类比网络流,每条边都是一根有流量限制的水管,允许每个点暂时存储一些多余的水,
称为超额流。特别地,源汇点可以长期存储无限多的水。其它点需要伺机将自身的超额流推送
出去,这里给每个节点再引入一个``高度''参数,规定流量只能往低处走。固定源点的高度为$V$。
当某个节点高于源点时,它的超额流将退回给源点。{\bfseries 注意高度可以达到$2V-1$}

该算法由两个基本操作组成:
\begin{itemize}
	\item ``推送'':一个节点把自己的超额流推送给高度比自己低1的节点(源点无高度差限制)。
	\item ``重贴标签'':当一个节点无法推送完超额流时,将自身高度加到
	连边有残存流量的最低邻接点的高度+1。
\end{itemize}

首先令S的出边满流,然后维护超额流节点队列,每次取出节点对其进行推送或重贴标签操作。
直至不存在超额流节点。时间复杂度$O(V^2E)$。

\subsubsection{前置重贴标签算法}

每次重贴标签时将节点移至队首,可将时间复杂度优化至$O(V^3)$。

参见算法导论\cite{ITA3}~第26.5节。

\subsubsection{HLPP实现与优化}

使用优先队列以高度为关键字维护超额流节点,每次选取最高标号的节点进行``推送-重贴标签''。

优化:
\begin{itemize}
	\item gap优化:当一个点被重贴标签后,若没有其他点拥有其原来的高度,
	高于此高度的点就无法把流量推送到汇点。将这些点的高度全部设为$V+1$使其流量
	流回源点。
	\item 高度预计算(我因此而TLE多次):将$d$初始化为每个点到汇点的最短路径长。
	{\bfseries 注意源点的高度固定为$V$。}
	\item 使用桶维护优先队列:注意到高度值的范围不大,使用桶来维护较为快速。
\end{itemize}

板子:

优先队列版:
\lstinputlisting[title=HLPPA]{Source/Templates/HLPPA.cpp}

桶版(参考PM250的代码\footnote{
	R13845988 评测详情
	\url{https://www.luogu.org/record/show?rid=13845988}
},自己不会用vector然后就用set代替了,常数大好多):
\lstinputlisting[title=HLPPB]{Source/Templates/HLPPB.cpp}

HLPP算法参考了Mr\_Spade的博客\footnote{
	网络最大流——最高标号预流推进
	\url{https://www.cnblogs.com/Mr-Spade/p/9636935.html}
}。

\subsection{最大流与最小割}

\index{M!Max-flow min-cut theorem}
\begin{theorem}[Max-flow min-cut theorem]\label{MFMCT}
	最大流=最小割。
\end{theorem}

证明:
\begin{itemize}
	\item
	\begin{lemma}\label{MCA}
		最大流$\leq$最小割
	\end{lemma}
	由于流量被割边所限制,所以最大流$\leq$任意割,所以最大流$\leq$最小割。
	\item
	\begin{lemma}\label{MCB}
		最大流$\geq$最小割
	\end{lemma}
	证明:跑完最大流后残量网络内$s$与$t$不连通,所以得到了一个割,
	即最大流$\geq$最小割。
\end{itemize}

结合引理~\ref{MCA}与~\ref{MCB}可得最大流=最小割。
\subsection{无向图最小割}
\subsubsection{Stoer-Wagner Algorithm}
\index{S!Stoer-Wagner Algorithm}
若需要求全局最小割,使用Stoer-Wagner Algorithm。

算法步骤如下:
\begin{enumerate}
	\item 任意指定一个节点作为初始点集;
	\item 查询到点集内的点边权和的最大的点集外的点;
	\item 合并最后加入的两个节点$s,t$并更新最小割;
	\item 重复第一步直至整个图被合并。
\end{enumerate}
具体做法见代码。边权可用优先队列维护,时间复杂度$O(|V||E|\lg |E|)$。

模板(SP12056 FZ10B - Nubulsa Expo):
\lstinputlisting{Source/Templates/Stoer-Wagner.cpp}

这题$|V|$比较小所以可以用邻接矩阵存图,$O(|V|^3)$解决。

\index{*TODO!证明无向图最小割算法的正确性}
上述内容参考了Oyking的博客\footnote{
	全局最小割StoerWagner算法详解
	\url{https://www.cnblogs.com/oyking/p/7339153.html}
}。
\subsubsection{流量构造法}
若指定源汇点,连边时给正反向边的残余流量都初始化为割边代价,然后跑Dinic。

\section{费用流}
\section{带上下界网络流}
前置知识:最大流与费用流。
\subsection{无源汇有上下界可行流}\label{LFA}
最大流只能求带上界的网络流,考虑通过修改建图方式使边的流量下界为0,流量上界对应变为
上界-下界。若在修改后的图中,边$e$使流量从点$u$流出$f$个单位,则实际上边$e$通过
$f+low_e$单位的流量,这时需要额外给点$u$提供$low_e$单位的流量,同理流入流量时也需要
额外接收$low_e$单位的流量。考虑设立虚拟节点$S,T$与顶点连附加边额外补偿/接收顶点的额外
流量,最后做$S-T$的最大流使附加边满流。若附加边满流则说明可以满足每条边的流量下界限制,
存在可行解。此时每条边的实际流量即为图中流量+流量下界。
建图:对于一条边$(u,v,low,up)$,连边$(S,v,low),(u,T,low),(u,v,up-low)$。

\subsubsection{建图优化}

普通方法建图需要连$3E$条边,但我们发现如果顶点$u$同时连接$S,T$,两条附加边的流量
可以抵消掉一部分。因此可以在建图时预处理数组$d[i]$,维护$S$到$v$的流量与$v$
到$T$的流量之差,最后$O(n)$选择$S$或$T$连边。使用此方法可以将边数降到$V+E$。

\subsubsection{费用流}

如果按照常规方法连边,由于附加边的费用不相等,不能使用上述优化合并附加边。由于当且仅当
附加边满流时才存在可行解,可以先算上附加边的费用,连附加边时把费用当成0,就可以继续使用
建图优化了。

\subsection{有上下界带源汇点可行流}

若点$u$需要凭空生成流量$f$,从$S$向$u$连流量为$f$的边;
若点$u$需要凭空消灭流量$f$,从$u$向$T$连流量为$f$的边;
其余做法与子节~\ref{LFA}相同。

\subsection{有上下界带一组无限流量源汇点可行流}

易知$S$的流出等于$T$的流入,因此加边$(T,S,inf)$后做法与
子节~\ref{LFA}相同。最终该边的流量即为$S$到$T$的总流量。

\subsubsection{最大流}

\begin{enumerate}
    \item 先求出可行流;
    \item 在残量网络上跑$S\rightarrow T$的最大流;
    \item 答案即为{\bfseries 最终残量网络}中$T\rightarrow S$的流量+最大流流量。
\end{enumerate}

正确性证明:

\begin{property}\label{LFB}
    求解最大流时不会修改源汇点路径之外的边的流量。
\end{property}

因此步骤2仍然能够保证该流是一个可行流(附加边不变)。

\subsubsection{最小流}

\begin{enumerate}
    \item 先求出可行流;
    \item 在残量网络上跑$T\rightarrow S$的最大流;
    \item 答案即为{\bfseries 最终残量网络}中$T\rightarrow S$的流量-最大流流量。
    $T\rightarrow S$的最大流相当于将$S\rightarrow T$的流量减至最小。
    {\bfseries 注意得到答案为负的情况,此时应该令答案为0。}
\end{enumerate}

利用性质~\ref{LFB}我们仍然能保证该流是一个可行流。

上述内容参考了F.W.Nietzsche\footnote{有上下界的、有多组源汇的、网络流、费用流问题 - F.W.Nietzsche
\url{https://www.cnblogs.com/nietzsche-oier/p/8185805.html}}与
liu\_runda\footnote{有上下界的网络流学习笔记 - liu\_runda
\url{https://www.cnblogs.com/liu-runda/p/6262832.html}}的博客。

\section{常见网络流/最小割模型}
\subsection{平面图转对偶图}

平面图与对偶图的定义:
\begin{itemize}
	\item 平面图(Planar Graph):在平面上画出来可以使边与边只在顶点上相交的图。
	      \index{P!Planar Graph}
	\item 对偶图(Dual Graph):将平面图的每条边两边的区域连边而成的新平面图。
	      \index{D!Dual Graph}
\end{itemize}

记平面图$G$的对偶图为$G^*$,平面集合为$P_G$。

对偶图$G^*$有两个性质:
\begin{itemize}
	\item
	      \begin{property}
		      $G^*$中的环对应$G$中的一个割。
	      \end{property}
	\item
	      \begin{property}
		      $|P_G|=|V_{G^*}|,|E_G|=|E_{G^*}|$
	      \end{property}
\end{itemize}

实际应用时,首先连接$(s,t)$使得外部平面被分为两个平面,以获得源汇点$s',t'$(同时连
到一个点上并没有什么用),然后按照定义建图即可(注意不要加入边$s',t'$)。

那么$s,t$的最小割=$s'->t'$的最短路(即拆点前的最小环),时间复杂度降低不少。

\subsubsection{例题}

Luogu P4001 [BJOI2006]狼抓兔子\footnote{【P4001】[BJOI2006]狼抓兔子 - 洛谷
\url{https://www.luogu.org/problemnew/show/P4001}}

根据定理~\ref{MFMCT}转换为求最大流,将右上角当做起点,右下角当做终点,然后使用上述
方法连边即可。

\lstinputlisting[title=Luogu P4001]{Source/Unclassified/Done/4001.cpp}

\subsection{最大权闭合子图}

$S$向非负权点连容量为权值的边,负权点向$T$连容量为权值相反数的边,如果选择点$u$必须
选择点$v$,就从$u$向$v$连容量为$\infty$的边。

答案=正权值之和-最小割。

简单理解:如果割去正权点的权值,则说明舍弃该正权点,权值从答案中扣除;如果割去负权点
的权值,则说明选择之前的正权点并从答案扣除该负权值。

严格的正确性证明待补充。\index{*TODO!最大权闭合子图算法的正确性}

\subsubsection{板子}

Luogu P4174 [NOI2006]最大获利\footnote{【P4174】[NOI2006]最大获利 - 洛谷
\url{https://www.luogu.org/problemnew/show/P4174}}

\lstinputlisting[title=Luogu P4174]{Source/Source/'Network Flows'/4174.cpp}

\subsubsection{输出方案}

\begin{theorem}
    Dinic最后一次增广时可访问到的点就是最终方案。
\end{theorem}

简单理解:最后一次增广后BFS必然找不到增广路,此时割掉的边无法继续增广,对应的点无法被
访问到,剩余的点就是最终方案了。

上述内容参考了appgle\footnote{网络流算法基本模型 - appgle
	\url{https://www.cnblogs.com/hyl2000/p/6618519.html}},
MaxMercer\footnote{关于平面图到对偶图的转化 \\
	\url{https://blog.csdn.net/MaxMercer/article/details/77976666}}和
Cold\_Chair\footnote{网络流——最大权闭合子图 \\
	\url{https://blog.csdn.net/Cold\_Chair/article/details/52841351}}
的博客。

\section{本章注记}
\chapter{数据结构}
\minitoc
\section{树状数组}
\subsection{标号管辖范围}

\subsection{lowbit函数原理}

lowbit函数定义为:$lowbit(i)=i\&-i$。

由于$i$始终为正,所以$-i$的补码表示是$i$的位取反再加1。$i$末尾的0对应取反后的1,
再加1后就会变成$1000$的形式,1的位置就是$i$末尾1的位置,而$i$与$-i$之前的位均不同,
所以位与后为0,因此$i\&-i$仅保留末尾1的位。


\section{线段树}
\subsection{常规操作}
\subsection{技巧}
\subsubsection{全局最优值剪枝}
\subsubsection{标记永久化}
\subsection{zkw线段树}

\section{划分树}
划分树是一种类似于线段树但很少使用的数据结构,用来求解静态区间第k大问题。
划分树可用主席树代替,权当了解。
\subsection{构建}
和线段树类似,在下一层中将每一段区间分为两个子区间,以这段区间的中位数为划分依据,
并且被划分到同一边的数的相对次序不变。为了支持查询,还需要记录区间内每一个数及之前的数
有多少个划分到左区间中。
\subsubsection{注意事项}
\begin{itemize}
    \item 由于每一层都只有n个数,所以只要每一层开n个数的数组即可,记$A[d]$为
    第$d$层的划分情况,$cnt[d][i]$为第$d$层的第$i$个数及同区间之前的数有多
    少个数进入了左子树。
    \item 中位数可预先对数组排序获得,记排序后数组为$B$。
    \item 注意有多个中位数的情况(否则左区间的数将覆盖到右区间的存储区域上)。
\end{itemize}
\subsection{查询}

\begin{enumerate}
    \item 当到达叶子节点时直接返回$A[d][i]$或$B[i]$。
    \item 利用$cnt$数组差分可以查询到$[l,r]$之间有多少数进入了左子树,
    决定往哪棵子树走。
    \item 根据$cnt$数组重新计算目标范围在下一层的区间,递归查询。
\end{enumerate}

\subsection{板子}

SP3946 MKTHNUM - K-th Number

\lstinputlisting[title=PartitionedTree]
{DataStructure/PartitionedTree.cpp}

以上内容参考了hchlqlz\footnote{划分树讲解 - hchlqlz
\url{https://www.cnblogs.com/hchlqlz-oj-mrj/p/5744308.html}}的博客。

\section{二叉搜索树}
\index{B!Binary Search Tree}
\subsection{二叉搜索树性质}
\begin{property}
    向二叉搜索树中插入新节点$u$(节点关键字不重复),该节点
    要么是其前驱的右儿子,要么是其后继的左儿子。
\end{property}
\paragraph{证明} $u$的前驱后继在插入节点$u$前互为前驱后继,
若前驱有右儿子则说明前驱比后继高,后继没有左儿子,节点$u$成为后继的左儿子。
反之亦然。
\subsection{FHQTreap}\label{FHQTreap}
\index{F!FHQTreap}
FHQTreap是一种比较好写的二叉搜索树,虽然效率不太高
(不如子节~\ref{splay}\\的splay),但其易于理解,不需要旋转,
且对可持久化友好。
FHQTreap的使用基于两个基本操作:split与merge。

对于二叉搜索树的操作,由于FHQTreap本来就是一颗二叉搜索树,因此操作方法相同
(若对效率要求不高,可以灵活地使用split和merge以减少代码量)。

对于序列操作,使用子节~\ref{split}所述的splitKth完成区间提取。

\subsubsection{split}\label{split}

split函数的作用是将一棵树按照权值或位置划分成两棵子树。

\begin{itemize}
	\item
	      按位置划分:$split(rt,k,x,y)$表示将以$rt$为根的树分为以$x$为根的
	      左子树和以$y$为根的右子树,其中左子树内的节点是原树的前$k$个,需要维护
	      每个节点的子树大小$siz$。

	      代码如下:
	      \begin{lstlisting}[title=splitKth]
    void split(int u,int k,int& x,int& y) {
        if(u) {
            push(u);
            int lsiz=T[T[u].ls].siz;
            //`lsiz(+1)决定当节点u为第k个时被分到哪棵子树`
            if(k<=lsiz) {
                y=u;
                split(T[u].ls,k,x,T[u].ls);
            }
            else{
                x=u;
                split(T[u].rs,k-lsiz-1,T[u].rs,y);
            }
            update(u);
        }
        else x=y=0;
    }
\end{lstlisting}
	\item
	      按权值划分:$split(rt,k,x,y)$表示将以$rt$为根的树分为以$x$为根的
	      左子树和以$y$为根的右子树,其中左子树内的节点值均小于等于$k$。

	      代码如下:
	      \begin{lstlisting}[title=splitKey]
    void split(int u,int k,int& x,int& y) {
        if(u) {
            push(u);
            //`=决定当T[u].val==k时被分到哪棵子树`
            if(T[u].val<=k) {
                x=u;
                split(T[u].rs,k,T[u].rs,y);
            }
            else{
                y=u;
                split(T[u].ls,k,x,T[u].ls);
            }
            update(u);
        }
        else x=y=0;
    }
\end{lstlisting}
\end{itemize}

根据树的实际意义(二叉搜索树还是序列)以及实际需要来决定使用哪种split。

\subsubsection{merge}

merge将两棵树按照左右顺序(中序遍历)合并。
和treap一样,merge使用随机权重来保持树的平衡。

代码如下:

\begin{lstlisting}[title=merge]
    int merge(int u,int v) {
        if(u && v) {
            if(T[u].pri<T[v].pri) {
                push(u);
                T[u].rs=megre(T[u].rs,v);
                update(u);
                return u;
            }
            else{
                push(v);
                T[v].ls=merge(u,T[v].ls);
                update(v);
                return v;
            }
        }
        return u|v;
    }
\end{lstlisting}

\subsubsection{指示权重的伪随机数生成器}\label{WRG}

显然FHQTreap也是一个Treap,所以需要一个表现良好的伪随机数生成器
来指示该节点的权。

最易于实现的伪随机数生成算法就是线性同余法(LCG)了。
\index{L!Linear Congruential\\ Generator}
C++11中<random>的$std::linear\_congruential\_engine$给出
了两组预置的参数:
\begin{itemize}
	\item $minstd\_rand0:(a=16807, c=0, m=2147483647)$

	      Discovered in 1969 by Lewis, Goodman and Miller, adopted as
	      "Minimal standard" in 1988 by Park and Miller.
	\item $minstd\_rand:(a=48271, c=0, m=2147483647)$

	      Newer "Minimum standard", recommended by Park, Miller, and Stockmeyer in 1993.

\end{itemize}

\index{*Constant!线性同余随机数生成器\\$a=48271,c=0,$\\ $m=2147483647$}
通常选择第2个即$a=48271$,代码如下:

\begin{lstlisting}[title=minstd\_rand]
int getRand() {
    static int seed = 347;
    return seed = seed * 48271LL % 2147483647;
}
\end{lstlisting}

现在尝试说明它的优越性:

\begin{itemize}
	\item 根据节~\ref{PrimitiveRoot}所述,如果$g$是模数$P$的一个原根,
	      则$g$的幂模$P$可以取到$[1,P-1]$内的每一个数,且循环周期长度为$P-1$。
	      由于2147483647是梅森素数,所以它一定存在原根。
	      下列程序可证明48271是2147483647的一个原根:
	      \lstinputlisting[title=RandomTestA.cpp]{DataStructure/RandomTestA.cpp}
	\item 48271是2147483647的一个大原根,可以较早地使int溢出,从而避免出现因
	      $a$过小而导致产生``锯齿波''。

          下列程序证明了48271可以满足OI考试的需要:
          程序输出minc=7884 maxc=8515 except=8192 s2=88.492,
          可见该算法生成的随机数还是蛮均匀的。
    \item 模数为2147483647,随机数值域广。
\end{itemize}

参见cppreference\footnote{
	\url{https://en.cppreference.com/w/cpp/numeric/random/
		linear\_congruential\_engine}}与
Wikipedia-EN\footnote{
	\url{https://en.wikipedia.org/wiki/Linear\_congruential\_generator}}。

如果需要更均匀的随机数,可以使用如下方案(质量从低到高):
\begin{enumerate}
	\item 使用比LCG更好的梅森旋转算法
	      \footnote{std::mersenne\_twister\_engine - cppreference.com
		      \url{https://en.cppreference.com/w/cpp/numeric/random/
			      mersenne\_twister\_engine}};
	\item Intel指令集内置RDRAND;
	\item 使用由一些机构提供的真随机数生成器SDK,如\url{https://www.random.org/};
	\item 在需要蒙特卡洛采样的场合使用低差异序列如Halton,Sobol等。
\end{enumerate}

\subsection{splay}\label{splay}
\index{S!Splay}

由于Treap做LCT复杂度多一个log(而且我看不懂),splay仍然无法被完全替代。

splay主要由$rotate$和$splay$函数组成:

\subsubsection{rotate}

$rotate(u)$表示将节点$u$旋转到$u$的父亲上。

主要思想是提取出父亲所在的子树,然后把节点$u$提为子树的根,最后把一个原来的儿子
挂到原来的父亲下以保持二叉平衡树的性质。

具体步骤:
\begin{enumerate}
	\item 把自己挂到父亲的位置上;
	\item 根据相对位置把父亲挂到自己下方;
	\item 把父亲占用位置所在的节点(原来的儿子)挂到原来自己的位置上;
	\item 依次更新原父亲与自己的信息。
\end{enumerate}

$rotate$前需要提前将标记下传(在$splay$中$push$)。在实践中可使用辅助函数
$connect(u,p,c)$把节点$u$挂到节点$p$的位置$c$下,$getPos(u)$获得
节点$u$相对于父亲的位置。

代码如下:

\begin{lstlisting}[title=rotate]
int getPos(int u) {
    return u == T[T[u].p].c[1];
}
void connect(int u, int p, int c) {
    T[u].p = p;
    T[p].c[c] = u;
}
void rotate(int u) {
    int ku = getPos(u);
    int p = T[u].p;
    int kp = getPos(p);
    int pp = T[p].p;
    int t = T[u].c[ku ^ 1];
    connect(u, pp, kp);
    connect(t, p, ku);
    connect(p, u, ku ^ 1);
    update(p);
    update(u);
}
\end{lstlisting}

\subsubsection{splay}

$splay(u,p)$表示将节点$u$旋转到$p$的儿子的位置上。
$splay$操作有单旋与双旋之分,其中单旋会被卡(原因见下面所述)。

\paragraph{单旋}

单旋的思路很简单,就是不断地把自己旋转到父亲的位置上。

但是当原树是一条链时,单旋$splay$后仍然是一条链,树的高度并没有得到优化,
因此可以构造数据使得每次$splay$都达到最坏时间复杂度$O(n)$。

\paragraph{双旋}

当自己,父亲,爷爷在同一条直线上时,会多旋转一次来尽可能地减小树高。
这种做法在链上表现良好。

代码如下:

\begin{lstlisting}[title=splay]
void splay(int u,int p) {
    //push p->u
    while(T[u].p!=p) {
        int pu=T[u].p;
        if(T[pu].p!=p)rotate(getPos(u)==getPos(pu)?pu:u);
        rotate(u);
    }
}
\end{lstlisting}

\subsubsection{具体应用}

\paragraph{树根}

显式维护树根是一件麻烦事,容易发现$getPos(root)$总是返回0,所以$T[0].c[0]$
即为$root$。

\paragraph{二叉搜索树}

对于二叉搜索树的操作,同子节~\ref{FHQTreap}相同,直接当做二叉搜索树
进行操作。为了优化树高,每次执行完操作后对最近访问节点$u$调用$splay(u,0)$
(血泪史:[HNOI2002]营业额统计在$insert$重复时也要$splay$)。

\paragraph{序列操作}

对于序列操作,可使用splay来提取区间。如果需要操作区间$[l,r]$,则:

\begin{enumerate}
	\item 将节点$l-1 splay$到根节点;
	\item 将节点$r+1 splay$到$l-1$的右儿子上;
	\item 此时$r+1$的左子树就是目标区间。
\end{enumerate}

为了操作区间$[1,n]$,可以在头尾分别添加一个哨兵。

\paragraph{splay合并}

$join(T1,T2)$表示将树$T1$与$T2$合并为一棵树,其中$T1$的所有元素权值
均小于$T2$的任意元素权值。

做法很简单:把$T1$的最大节点$splay$到根,此时该节点没有右子树,然后把$T2$
的根挂到该节点右儿子的位置上。

上述内容参考了自为风月马前卒的博客\footnote{splay详解(一) - 自为风月马前卒
	\url{http://www.cnblogs.com/zwfymqz/p/7896036.html}}。

\paragraph{splay启发式合并}
维护splay树的size,合并时将size小的拆开插入size大的,时间复杂度$O(n\lg^2n)$。
\sout{据说拆树时使用中序遍历,插入复杂度均摊为$O(1)$,时间复杂度$O(n\lg n)$}。

{\bfseries 注意FHQTreap的merge必须保证两树无交且参数顺序按照合并后的序列顺序,不能简单地
将两树的根merge。}

\section{动态树}
\index{L!Link-Cut Tree}
\subsection{常规操作}
动态树是一堆splay组成的森林,主要以$access$和$makeRoot$操作为基础,
可以实现$link,cut,split,find$等功能。
LCT的splay用来维护以深度为关键字的重链信息。

\subsubsection{splay部分}
与常规splay的不同之处在于LCT中的splay的根也是有父亲的,指向另一棵splay的
节点,在此使用辅助函数$isRoot(u)$来判断节点$u$是不是splay的根。为了实现LCT
的$access$功能,需要支持区间翻转。

\begin{lstlisting}[title=splay]
int getPos(int u) {
    return u == T[T[u].p].c[1];
}
bool isRoot(int u) {
    int p = T[u].p;
    return T[p].c[0] != u && T[p].c[1] != u;
}
#define ls T[u].c[0]
#define rs T[u].c[1]
void pushDown(int u) {
    if (T[u].rev) {
        std::swap(ls, rs);
        T[ls].rev ^= 1;
        T[rs].rev ^= 1;
        T[u].rev = false;
    }
}
void update(int u);
void connect(int u, int p, int c) {
    T[u].p = p;
    T[p].c[c] = u;
}
void rotate(int u) {
    int ku = getPos(u);
    int p = T[u].p;
    int kp = getPos(p);
    int pp = T[p].p;
    int t = T[u].c[ku ^ 1];
    T[u].p = pp;
    if (!isRoot(p))
        T[pp].c[kp] = u;
    connect(t, p, ku);
    connect(p, u, ku ^ 1);
    update(p);
    update(u);
}
void push(int u) {
    if (!isRoot(u)) push(T[u].p);
    pushDown(u);
}
void splay(int u) {
    push(u);
    while (!isRoot(u)) {
        int p = T[u].p;
        if (!isRoot(p))
            rotate((getPos(u) == getPos(p)) ? p : u);
        rotate(u);
    }
}
\end{lstlisting}
\subsubsection{access}
$access(u)$的功能是使节点$u$与其LCT的根在同一棵$splay$内,
此时节点$u$是根节点重链中深度最大的点(已经和原来到儿子的边断开)。

具体操作如下:

\begin{enumerate}
    \item 将节点$u$翻转到其所在$splay$的根;
    \item 删除节点$u$的右子树,即断开与重儿子的连边,更新节点$u$;
    \item 跳到节点$u$的父亲节点,并翻转其至根;
    \item 把以$u$为根的链挂到父节点的右子树位置上,
        该操作自动断开父节点到原来重儿子的连边,更新;
    \item 重复步骤3合并重链直至合并到根节点为止。
\end{enumerate}

代码:
\begin{lstlisting}[title=access]
void access(int u) {
    int v = 0;
    do {
        splay(u);
        rs = v;
        update(u);
        v = u;
        u = T[u].p;
    } while(u);
}
\end{lstlisting}

\subsubsection{makeRoot}
$makeRoot(u)$的功能是把节点$u$翻转为整棵LCT的根。

具体操作如下:

\begin{enumerate}
    \item 打通节点$u$到根节点的路径,并使节点$u$成为splay的根;
    \item 由于节点$u$是splay中深度最大的节点,翻转整棵splay后就可以使
    节点$u$成为根了。
\end{enumerate}

代码:

\begin{lstlisting}[title=makeRoot]
void makeRoot(int u) {
    access(u);
    splay(u);
    T[u].rev ^= 1;
    pushDown(u);
}
\end{lstlisting}
\subsubsection{split}
若要提取$u-v$的路径,$makeRoot(u),access(v),splay(v)$后,节点$v$的子树
就保存了$u-v$的路径信息。
\subsubsection{find}\label{LCTFind}
首先$access(u),splay(u)$让$u$成为LCT根所在的splay的根。
根节点就是splay的最小节点(注意推送翻转标记)。

坑:有些毒瘤出题人可能会卡找最小节点的过程,因此在find找到根节点$u$后再执行一次
$splay(u)$。{\bfseries 注意如果$find$中使用了$splay(u)$并且$cut$中使用了$find$,
调用$find$后$u$在根位置,$cut$的后续代码需要稍微修改。}

该坑源自FlashHu的博客\footnote{
    LCT总结——概念篇+洛谷P3690[模板]Link Cut Tree(动态树)(LCT,Splay)
    \url{https://www.cnblogs.com/flashhu/p/8324551.html}
}。
\subsubsection{link}
$makeRoot(u)$后使$T[u].p=v$,注意在link前要进行连通性检测。
\subsubsection{cut}

\begin{itemize}
    \item 若保证$u,v$连接,则split后令$T[u].p=T[v].c[0]=0$
    (节点$v$深度最大,因此只需和左子树断开),注意要更新节点$v$。
    \item 若不保证$u,v$连接,则在$makeRoot(u)$后检查以下条件:
    \begin{itemize}
        \item 节点$v$所在LCT的根是$u$;
        \item 节点$u$的父亲是$v$(查根时执行完$find(v)$后节点$v$已经是
        节点$u$所在splay的根);
        \item 节点$v$的左儿子是$u$;
        \item 节点$u$没有右儿子(若有则说明$u-v$中有其他节点)。
    \end{itemize}
    由于$find(v)$隐式执行了$access(v),splay(v)$,只需令
    $T[u].p=T[v].c[0]=0$。
\end{itemize}

以上内容参考了Saramanda的博客\footnote{
    LCT(Link-Cut Tree)详解(蒟蒻自留地)
    \url{https://blog.csdn.net/saramanda/article/details/55253627}}。

\subsection{技巧/常见方法}
\subsubsection{DSU优化连通性检测}
如果可以保证处于同一连通块内的点不会再次分离(允许临时分离,比如动态MST在环上换边时),
可以使用本章~\ref{DSU}节所述的并查集代替~\ref{LCTFind}中昂贵的$find$。
\subsubsection{使用access进行从某节点到根的路径的染色}
$access(u)$后节点$u$到根节点的路径上的点在同一棵splay内,可用于
模拟染色过程。

参见[SDOI2017]树点涂色\footnote{【P3703】[SDOI2017]树点涂色 - 洛谷
    \url{https://www.luogu.org/problemnew/show/P3703}}。
\subsubsection{动态LCA}
注意link/cut操作不能换根,换根会使树的形态改变导致LCA不同。

求节点$u,v$的LCA时,首先$access(u)$开辟一条链,类似地$access(v)$再开辟一条链,
$access(v)$时会与$access(u)$的链相交,交点即为LCA。在实现时可以直接令$access$函数
返回$v$值作为与上一条链的交点(当$while(u)$为false时,说明在$u=T[u].p$之前的$u$是
链的交汇点,然而过程退出前将$u$赋给了$v$)。

板子:SP8791 DYNALCA
\lstinputlisting{Source/Source/LCT/SP8791.cpp}

事实上该方法的适用范围还可以再扩展。考虑每次$access$后将提取出的链打一个标记,那么在
下次$access$时,每个$u$都意味着这可能是与之前某个节点的LCA。检查标记可以得到对应的节点标号
(一般结合题目性质贪心保留某个特殊节点)。例题:「雅礼集训 2017 Day7」事情的相似度。
\subsubsection{LCT维护边权信息}
把边当做节点,与两端点相连,端点点权为0。边的节点编号最好为边编号+总端点数,
以便于快速得到原边编号并断开端点连接。
\subsubsection{LCT维护子树信息}
该方法主要用于解决{\bfseries 不断换根且子树无修改}的问题。若子树有修改则用TopTree
向轻儿子下放标记。

普通LCT只维护了一条链的信息,即splay内信息。维护子树信息的关键在于显式地将
虚子树(虚边儿子的子树)信息累加入自身信息中。每个节点维护2个数据,一个是
虚子树贡献,另一个是贡献总和。update时顺带累加虚子树贡献。

考虑哪些操作会改变虚实子树:
\begin{itemize}
    \item access:在$rs=v$处将虚子树贡献中$v$的贡献改为$rs$的贡献。
    \item makeRoot、find、split等变换树的形态的操作除access部分外无影响,
    无需特殊处理。
    \item link:令$T[u].p=v$会导致$v$及其祖先都需要更新贡献,可以``$split(u,v)$''
    后将$v$换到链顶后把$u$的贡献累加到$v$的虚子树贡献中,最后执行$T[u].p=v,update(v)$。
    如此只影响到了$v$的信息。
    \item cut:$u,v$在同一棵splay内,除access部分外无影响。
\end{itemize}

{\bfseries 注意节点与虚儿子一定连接,但不一定与实儿子连接。}对于简单的情况
不必考虑这个事实,比如「BJOI2014」大融合中的子树节点数。但像LOJ\#558
「Antileaf's Round」我们的 CPU 遭到攻击~这种题,子树黑点数的变化会导致子树
黑点距离和发生变化,因为实儿子不一定与自己相连。

接下来以此题为例讨论解决方法,下文的``链''指代splay序列:

在查询时需要$access(u),splay(u)$,此时$u$必定在splay链尾。虽然自己的实儿子与自己不一定相连,
但是自己的左实儿子和虚儿子连到到链的右端以及右实儿子和虚儿子到链的左端一定经过自己。考虑维护
自己的子树到splay中{\bfseries 实子树对应的链}的左端/右端的距离和。
\begin{itemize}
    \item 查询时把$access(u),splay(u)$后$u$自己就是链的右端,并且$u$在splay的根,子树是完整的,
    $u$到链右端的距离和就是答案。
    \item 维护子树信息仍然使用上述方法。
    \item 访问到左右端点的信息时注意提前下传翻转标记,下传翻转标记时同时把{\bfseries
    左右端信息交换}。

    {\bfseries 血泪史:用LCT刚「PA 2017」Banany 点分治题时,由于忘了下传翻转标记而
    加入错误信息,导致无法删除信息(用set维护虚子树信息)。在访问u的子节点的信息前一定要
    记得下传!!!一般在access替入rs到链左端的信息与update读取实儿子信息中需要使用。不过
    值得欣慰的是我用LCT拿了LOJ的rank1,时间是第二快的50\%,空间是第二小的30\%,代码长度是
    第二短的80\%。}
    \item 为了编码方便,可以将自身的贡献塞到虚子树贡献中。
    \item LCT初始化时使用常规DFS处理信息,而不是用link。
\end{itemize}

代码如下:
\lstinputlisting{Source/Source/LCT/LOJ558.cpp}

\subsubsection{LCT维护图上信息}
若动态图只加边不删边,仍旧可以使用LCT维护图上信息,关键是``可缩点''。

额外维护2个DSU,一个维护连通性,另一个维护{\bfseries 双连通性}。
当且仅当两个点双连通时才缩为一点,即第二个DSU指示该点的$id$。

每次访问父亲时都使用父亲的缩点$id$代替,其它操作不变。

关键在于$link$操作的处理,$link$前若两点未连通,执行普通$link$并标记第一个DSU。
否则$split(u,v)$拆出$u-v$的链,遍历整棵splay将其所有点的贡献累加到点$v$上,然后把
第二个DSU中的$id$置为$v$。{\bfseries 注意设置位置是原$id$,缩点后$v$的左右儿子要置0。}

上述内容参考了neither\_nor\footnote{
    LCT维护子树信息(子树信息LCT) LCT维护边权(边权LCT) 知识点讲解
    \url{https://blog.csdn.net/neither\_nor/article/details/52979425}
}和GuessYCB\footnote{
    LCT 模板及套路总结
    \url{https://www.cnblogs.com/GuessYCB/p/8330024.html}
}的博客。

\subsubsection{共价大爷游长沙技巧}
例题:UOJ\#207.共价大爷游长沙

需要不断给一棵树换边,判断给定路径是否全部经过某条边。

我原先的想法是每次增删路径时,在端点处打标记,查询时统计两边子树端点数。
这个想法是错的:考虑有两条路径都经过一条边,如果将这两条路径的一个端点交换,
在LCT上的标记还是一样的。

但是这个想法离正解已经很接近了。注意这里的问题是``判定‘'而不是``统计'',
考虑给每条路径随机一个权值,增删路径时在端点处异或该权值,查询任意一端子树异或和,
判断其是否为路径异或和。
\subsubsection{LCT维护黑白连通块}
该需求来自SP16549 QTREE6 - Query on a tree VI。
这题当然也可以使用LCT维护子树信息的做法,但是讨论比较多,性能不佳
(\CJKsout{树上单点修改+换根统计万金油倒是没错,复杂度保证全部交给LCT})。

考虑维护两个森林,一个森林上只有黑点,另一个森林上只有白点,更改颜色时暴力
断开原有连边,连到另一棵森林上,查询时只要询问其所在连通块的大小。大小比连通性
更容易维护。

但是如果这是一个菊花图,断开中间的点后,可能会导致$O(n)$的边修改量。那么考虑DFS
一遍给整棵树定型,每次只修改与父亲的连边。那么就保证了连通块内所有边的儿子都是该颜色
的点,只有连通块的根不是该颜色的点,找到深度最浅的点后返回右子树大小(不能返回子树大小-1
是因为左子树可能有其它连通块)。

因为整棵树被定型,不能使用makeRoot操作,需要对link和cut特殊处理。

\begin{itemize}
    \item $link(u,f_u)$:$f_u$要接受$u$子树大小的虚子树贡献,所以要
    $access(f_u),splay(f_u)$;为了将$u$挂在下面,要$splay(u)$;最后
    连接$fa[u]=f_u$、贡献$isiz[f_u]+=siz[u]$、更新$update(f_u)$。
    \item $cut(u,f_u)$:$access(u)$后要$splay(f_u)$而不是$splay(u)$,然后
    按照普通$cut$操作。若使用$splay(u)$,$f_u$并不一定是$u$的左儿子,不能令
    $fa[f_u]=0$,而是要令$fa[lson[u]]=0$。
\end{itemize}

该方法参考了cdsszjj的博客\footnote{
    bzoj3637: Query on a tree VI【LCT】\\
    \url{https://blog.csdn.net/cdsszjj/article/details/80332588}
}。

\section{并查集}\label{DSU}
\subsection{路径压缩}
\subsection{按秩合并}
\subsection{复杂度证明}


\section{K-D Tree}
K-D Tree是一棵二叉树,每一层按照某个轴将本空间内的所有点分为较为均匀的两部分,
该节点保存划分的中点。查询时依靠不断剪枝来提高查询速度。
\subsection{构树}
具体步骤如下:
\begin{enumerate}
	\item 对于当前子空间,选取一个轴来划分(使用$std::nth\_element$)出中点;
	\item 将中点存储在当前节点上;
	\item 递归建左右子树;
	\item 更新子树信息。
\end{enumerate}
$std::nth\_element$的复杂度为$O(n)$,因此构树的复杂度为$O(n\lg n)$。
\subsection{插入}
\subsubsection{离线标记}
构树时将所有点加入,记录每个点的id,然后加入点时打标记一路更新即可,
不过这样做影响了查询的复杂度。
\subsubsection{替罪羊树}
与二叉搜索树的插入相同,注意需要确定每一个节点的划分轴。当二叉树不平衡时会影响
查询复杂度,采用替罪羊树的策略,维护每棵子树的size,如果
$max(siz_l,siz_r)>=siz_u \cdot fac$则暴力重构子树(注意每次插入只要找到最高
的不平衡子树重构即可),一般$fac$取0.75。
\subsection{删除}
删除节点后的处理方法与插入相同。
注意被删除的节点可以gc。

\begin{lstlisting}[title=gc]
    std::vector<int> pool;
    int newNode() {
        static int cnt=0;
        int id;
        if(pool.size()) {
            id=pool.back();
            pool.pop_hack();
        }
        else id=++cnt;
        return id;
    }
    void freeNode(int u) {
        pool.push_back(u);
    }
\end{lstlisting}

\subsection{查询}
\begin{enumerate}
	\item 如果整棵子树均不满足要求,就直接返回;
	\item 如果整棵子树均满足要求且可以不需要继续递归,就记录答案(或者打标记)后返回;
	\item 计算当前节点;
	\item 递归左右子树。
\end{enumerate}
在随机数据下,查询的时间复杂度是$O(\lg n)$,在构造数据下复杂度约
是$O(n^\frac{d-1}{d})$。
证明待补充。
\index{TODO!K-D Tree查询复杂度证明}
\subsection{估值}
下列为一些常见估值函数:

由于每个方向上是的独立的,对每个方向贪心后加起来即可。
\subsubsection{曼哈顿距离最小}
$w=\sum_{i=1}^d{max(mind_i-p_i,0)+max(p_i-maxd_i,0)}$
当$p_i$在区域内时估值为0,在一边时估值为到最近一边的值(另一边由于符号问题值为0)。
\subsubsection{曼哈顿距离最大}
$w=\sum_{i=1}^d{max(abs(mind_i-p_i),abs(maxd_i-p_i))}$
选择距离最大的一边。
\subsubsection{欧几里得距离最小}
$w=\sum_{i=1}^d{max(mind_i-p_i,p_i-maxd_i,0)^2}$
\subsubsection{欧几里得距离最大}
$w=\sum_{i=1}^d{max((mind_i-p_i)^2,(p_i-maxd_i)^2)}$
\subsection{技巧}
\subsubsection{全局最优值剪枝}
如果通过该节点维护的子树信息可以确定子树内不存在更优解,搜索该子树
已经没有意义了。还可以搭配另一个优化:先求两棵子树的估价函数值,
选择最优的先进入(更有可能获得最优值然后减少在另一棵子树上的计算)。
\subsubsection{预处理降维}
如果插入与查询离线,则可以对某一维排序,边插入边查询,降低kd-Tree查询复杂度。

以上内容参考了n+e的课件\emph{K-D Tree 在信息学竞赛中的应用}\cite{kdTree}。

\section{堆}
\subsection{左偏树}
\index{L!Leftist Tree}
左偏树(Leftist Tree)也是一种二叉堆,核心操作是$merge$函数,
它可以以$O(\lg n)$合并两棵左偏树。

定义外节点为没有左子树或右子树的节点。对于左偏树的每一个节点,维护其到子树外节
点的最近距离,其中外节点的$dist=0$,$null$的$dist=-1$(其实没必要太严格,
差不多平衡就够了)。

左偏树具有左偏性质:
\begin{property}\label{LTC}
    $dist_l \geq dist_r$
\end{property}

由此定义可得到一个推论:

\begin{inference}
    $dist_u=min(dist_l,dist_r)+1=dist_r+1$
\end{inference}

考虑一棵距离为$k$的左偏树的最小节点数,得到以下定理:

\begin{theorem}
    一棵距离为$k$的左偏树为满二叉树时节点数最少,有$2^{k+1}-1$个节点。
\end{theorem}

由此得到推论~\ref{LTI}:

\begin{inference}\label{LTI}
    一棵节点数为$n$的左偏树,距离最大为$[lg(n+1)-1]$。
\end{inference}

先给出引理~\ref{LTL}:

\begin{lemma}\label{LTL}
    左偏树的最右链恰好有一个外节点。
\end{lemma}

证明:由于左偏树是一棵树,最右链至少有一个外节点;若存在两个及以上的外节点,则
对于某个非深度最深的点,必有右子树(否则链就断了),却没有左子树(由外节点定义可知),
与性质~\ref{LTC}矛盾。

由推论~\ref{LTI}与引理~\ref{LTL}可得如下定理:

\begin{theorem}\label{LTT}
    一棵由$n$个节点组成的左偏树最右链最多有$[lg(n+1)]$个节点。
\end{theorem}

以上证明参考了阿波罗2003的博客\footnote{
    浅谈左偏树 - 阿波罗2003
    \url{https://www.cnblogs.com/yyf0309/p/LeftistTree.html}
}。

我原来的简单理解:对于左偏树中的每一个节点,维护其子树高度。每次$merge$时先往右子树塞,
若右子树的深度比左子树的深度更大,就把左子树换过来塞。以此保证树的高度尽可能小。

$merge(u,v)$的操作如下:

\begin{enumerate}
    \item 如果$u$或$v$有一个为$null$则返回另一个节点;
    \item 若$v$应该在$u$的上一层则$swap(u,v)$;
    \item 递归将节点$v$的子树与$u$的右子树合并;
    \item 若$dist_l<dist_r$则$swap$左右子树;
    \item 更新节点$u$的距离$dist_u=dist_r+1$;
    \item 返回该树的根$u$。
\end{enumerate}

根据定理~\ref{LTT}可得$merge$的复杂度为$O(lgn)$。

代码如下(以大根堆为例):

\begin{lstlisting}[title=merge]
int merge(int u, int v) {
    if (u && v) {
        if (T[u].val < T[v].val)
            std::swap(u, v);
        T[u].r = merge(T[u].r, v);
        if (T[T[u].l].dis < T[T[u].r].dis)
            std::swap(T[u].l, T[u].r);
        T[u].dis = T[T[u].r].dis + 1;
    }
    return u | v;
}
\end{lstlisting}

\subsection{斜堆}
斜堆的操作与左偏树差不多,它们的区别是斜堆不维护到外节点的最近距离,
而是在每一次$merge$时简单地$swap$左右子树。

\section{可持久化数据结构}
可持久化数据结构的核心思想是{\bfseries Copy On Write}(写时复制),当一个对象将
被改变时,简单地复制其整体,未修改的部分仍引用原对象的数据,达到节省拷贝时间与
空间的目的。可持久化还容易支持历史查询操作。

可持久化数据结构有主席树(可持久化线段树),可持久化可并堆,可持久化Trie,
可持久化数组,可持久化并查集,可持久化平衡树等。

{\bfseries 要注意默认拷贝节点T[0]的初始值!!!}

\subsection{主席树}
用主席树做的经典模型有:
\begin{itemize}
    \item 差分
    \item 对于每一个节点为查询左节点,维护其右边节点为查询右节点时的答案
    \item 将某一维离散化后不断插入新元素,预处理出每一时刻的权值线段树,以支持后续操作
\end{itemize}

\subsubsection{区间覆盖单点查询}
需求:UOJ\#218 火车管理

区间覆盖操作有两种做法:
\begin{itemize}
    \item 将被完全覆盖的区间的节点的儿子设为nil,查询时返回最深的存在节点。
    \item 将被完全覆盖的区间的节点的儿子设为自己。
\end{itemize}
\subsection{可持久化Trie}
若遇到求区间xor最大值之类的问题,使用可持久化Trie。
对于and,or最大值问题,可以在插入完数后把整棵子树加到另一棵子树上去,
查询时只需考虑一边的子树。
\subsection{可持久化数组}
可持久化数组有两种实现:
\begin{itemize}
    \item 块状数组
    \item 主席树
\end{itemize}
可持久化并查集可使用可持久化数组实现。
\subsection{优化}
\subsubsection{标记永久化}
将对整个区间的操作记录在管理此区间的节点,标记不下传,统计时标记参与计算。
此法节约了$push$的时间且对可持久化友好。
\subsubsection{克隆开关}
若已知按照原方法有一部分数据不被任何时刻的数据结构引用时,直接在该数据上修改
(当然也可以gc,实现比较麻烦)。
可以在操作前设置一个$enableClone$开关,若为$false$则直接返回原节点。

代码如下:
\begin{lstlisting}[title=cloneA]
bool enableClone=true;
int cloneNode(int src) {
    if(enableClone) {
        int id=allocNode();
        T[id]=T[src];
        return id;
    }
    return src;
}
\end{lstlisting}

对于可持久化并查集,若使用路径压缩优化(实践中不太好用),则不好判断是否需要$clone$。
可以在每个节点上记录其被创建时的时间戳,与当前版本时间戳比较。

代码如下:
\lstinputlisting{Source/Templates/FDSU.cpp}
此法节约了复制节点时的时间与空间,但是路径压缩增加了修改的时间和空间,考场上最好只写
按秩合并。

\section{DLX舞蹈链}
\index{D!Dancing Links X}
DLX用来求解精确覆盖问题。

\paragraph{精确覆盖问题} 给定一个01矩阵,求使得每一列恰好有1个1的行集合。
\subsection{X算法}
X算法使用递归+回溯搜索可行解。

算法步骤如下:
\begin{enumerate}
	\item 从矩阵中选取一行;
	\item 将该行和该行所有1对应的列以及与该行冲突的行从矩阵中删除得到一个新矩阵。
	\item 若该矩阵为空矩阵,则跳到步骤4;否则递归求解新矩阵的精确覆盖,若返回false则
	      返回步骤1选取下一行;
	\item 若选取的行全部为1,则返回true,否则返回false。
\end{enumerate}
\subsection{DLX}
递归+回溯使得存储与维护矩阵既麻烦又费时。Donald E.Knuth使用双向链表
来维护矩阵,这个数据结构被称为Dancing Links。它利用了双向链表删除与恢复的方便性。

对于矩阵内的每一个1(此种矩阵一般为稀疏矩阵),维护其上下左右元素标号和自身坐标。
每个元素既是所属行的链表元素,又是所属列的链表元素。每个列的链表还有链头$C_i$(即0行元素),
这些链头又与总链头$head$串在一起,以便检查覆盖情况。

算法步骤如下:

记标示列链表链头$C$为将元素$C$所在列元素以及这些元素所在行元素删除,回标$C$为其逆操作。
\begin{enumerate}
	\item 检查$head.right$是否为自身,若是则覆盖完毕,输出答案栈内所有元素,返回true;
	\item 记$C=head.right$,标示$C$,枚举$C$所在链表内的行$D$:
	      \begin{enumerate}
		      \item 标示元素$D$所在链表行元素对应列链表链头。
		      \item 将其压入答案栈中;
		      \item 递归求解,若返回true则退出,否则逆序回标,枚举下一行。
	      \end{enumerate}
	\item 回标$C$,返回false。
\end{enumerate}

{\bfseries 为了提高搜索效率可以维护每列1的个数,每次选取1个数最少的列遍历。}

板子:
\lstinputlisting{Source/Templates/DLX.cpp}

上述内容参考了万仓一黍的博客\footnote{
	跳跃的舞者,舞蹈链(Dancing Links)算法——求解精确覆盖问题
	\url{http://www.cnblogs.com/grenet/p/3145800.html}
}。

\section{Hash表}
由于std::unordered\_set并不能满足实际性能需要,在此谈谈用于OI的Hash表设计。

事实上std::unordered\_set性能低下也是情有可原的,因为它需要面对的应用场合繁多。
但在OI应用中,我们能够预知全集的大小,元素一般为整数,且只需要支持插入、查询和全部删除操作,
而通用的std::unordered\_set并不知道这些信息。

下列方法的性能以LuoguP3727~曼哈顿计划E的评测结果为准(不幸的是\\std::unordered\_set
光荣地TLE了)。

\subsection{链接法}
链接法使用std::vector<int>当链表,以整数低位为关键字,查询时使用std::find暴力搜索,
插入时使用,清空时记录所有插入值,将插入值对应的桶清空。

测试结果:943ms。
\subsubsection{散列函数设计}
一个好的散列函数应该考虑Key的所有位,而简单地以低位为关键字是个糟糕的选择。模$2^p$和
$2^p-1$都不太好。应该选取不太接近2的幂的素数。

将模数从$2^{20}$改为$65537$后测试结果为364ms,优化效果显著。
\subsection{双重散列+开放寻址法}
开放寻址法存放一个较大的表,插入时取得Key的散列值$h(Key,0)$,然后查看该处是否有不同
元素,若有则移动到位置$h(Key,1)$,以此类推,直至找到一个空位或发现已插入为止。查询也是如此。

散列函数使用双重散列法,即$h(k,i)=(h_1(k)+ih_2(k))\bmod m$。表的大小$m$一般
取素数,$h_1(k)=k\bmod m,h_2(k)=c+k \bmod m',m'<m$。装载因子$\alpha$的散列表的平均
期望探查数为$\frac{1}{\alpha}\ln \frac{1}{1-\alpha}$。

测试结果:317ms。

\CJKsout{2019.3.2:为什么又是rank2。。。}
\subsection{完全散列}
完全散列适用于静态关键字集合的查询。适用于BSGS等场合。

留坑待补。
\index{*TODO!完全散列}

上述内容参考了算法导论\cite{ITA3}第11章~散列表。


\chapter{数论}
\minitoc
\section{辗转相除法GCD}
\subsection{裴蜀定理}
\index{B!Bézout's Theorem}
\begin{theorem}[Bézout's Theorem]\label{BT}
    对于任意$a,b\in \mathbb{Z}$,关于$x,y$的线性不定方程(裴蜀方程)
    $ax+by=c$有无穷多整数解$(x,y)$当且仅当$(a,b)|c$。特别地,
    一定存在$(x,y)$使得$ax+by=(a,b)$成立。
\end{theorem}

由此可得推论:

\begin{inference}
    $a,b$互质的充要条件是存在整数$(x,y)$使得$ax+by=1$。
\end{inference}

接下来证明一定存在$(x,y)$使得$ax+by=(a,b)$成立:

设$s$是$a$和$b$线性组合集中的最小正元素,对于某个整数组$(x,y)$有$ax+by=s$,
令$q=[a/s],r=a mod s=a-q(ax+by)=a(1-qx)+b(-qy)$,所以$r$也是一个线性组合。
因为$s$是线性组合集中的最小正元素,且$0\leq r \le s$,所以$r=0$,可得$s|a$。
同理$s|b$,因此$s$是$a,b$的公约数,可得$(a,b) \geq s$。因为$(a,b)|ax+by$
且$s>0$,所以$(a,b) \leq s$。结合$(a,b) \geq s$与$(a,b) \leq s$可得
$s=(a,b)$。

至于无穷多整数解嘛。。。拿最小公倍数调一调初始解$(x,y)$即可。

证明参考了霜刃未曾试的博客\footnote{关于裴蜀定理的一些证明\\
\url{https://blog.csdn.net/discreeter/article/details/69833579}}与
算法导论\cite{ITA3}第31.1节定理31.2的证明。
\subsection{exgcd}
由定理~\ref{BT}可知一定存在整数解$(x,y)$满足$ax+by=(a,b)$,如何构造
出一组解呢?

$exgcd$(扩展欧几里得算法)可求出一组特殊的整数解。

先上代码:

\begin{lstlisting}[title=exgcd]
    void exgcd(int a,int b,int& x,int& y,int& d) {
        if(b) {
            exgcd(b,a%b,y,x,d);
            y-=a/b*x;
        }
        else x=1,y=0,d=a;
    }
\end{lstlisting}

该算法由朴素$gcd$修改而来,因此同样讨论两种情况:

\begin{itemize}
    \item $b=0$时,$gcd$值为$a$,因此有$a=1*a+0*b$。
    \item $b\neq 0$时,考虑递归计算返回的一组解$(y',x')$满足
    $by'+(a-[\frac{a}{b}]b)x'=d$,可变形为$ax'+b(y'-[\frac{a}{b}]x')=d$,
    因此本次递归返回$(x',y'-[\frac{a}{b}]x')$。
\end{itemize}

\subsection{位运算gcd}

本节内容源自算法导论\cite{ITA3}思考题33-1。

\subsubsection{原理}

首先为了避免除法,将辗转相除法改为更相减损术,可以利用
$gcd$函数的以下性质:

\begin{itemize}
    \item \begin{character}\label{GCDC1}
        若$a,b$均为偶数,则$gcd(a,b)=2*gcd(a/2,b/2)$。
    \end{character}
    \item \begin{character}\label{GCDC2}
        若$a$是奇数,$b$是偶数,则$gcd(a,b)=gcd(a,b/2)$。
    \end{character}
    \item \begin{character}\label{GCDC3}
        若$a,b$均为奇数,则$gcd(a,b)=gcd((a-b)/2,b)$。
    \end{character}
\end{itemize}

特别地,当$gcd(x,y)$的参数中存在0,则返回$x|y$。

算法步骤如下:
\begin{enumerate}
    \item 特判$x,y$中存在0的情况;
    \item 根据性质~\ref{GCDC1},先消去$a,b$的2的幂的公因子,记录幂次$k$;
    \item 使$a$变为奇数以便于利用性质~\ref{GCDC2},同时去除2的幂的
    公因子避免重复计算;
    \item 利用性质~\ref{GCDC2}使$b$变为奇数;
    \item 保持$a<b$以便利用性质~\ref{GCDC3};
    \item 利用性质~\ref{GCDC3}使$b'=b-a$;
    \item 若此时$b$为0,则返回$a*2^k$,否则重复第4步。
\end{enumerate}

实际上就是利用性质~\ref{GCDC1}与~\ref{GCDC2}对更相减损术进行了优化。

\subsubsection{位扫描优化}

如果某个数末尾有多个0,则可以直接使用右移k位代替不断右移1位。
下面是统计末尾0的个数k的方法:

\begin{itemize}
    \item GCC自带了对位扫描指令的封装,即$\_\_builtin\_$系列函数,
    直接使用$\_\_builtin\_ctz$函数即可。
    \item Sean Eron Anderson 的\emph{Bit Twiddling Hacks}
    \footnote{\url{http://graphics.stanford.edu/~seander/bithacks.html}}
    中Counting consecutive trailing zero bits (or finding bit indices)
    一节介绍了计算末尾0个数的多种方法,这里只给出multiply and lookup法:
    \begin{lstlisting}[title=countTZ]
    int countTZ(unsigned int x) {
        static const int LUT[32] = {
            0, 1, 28, 2, 29, 14, 24, 3,
            30, 22, 20, 15, 25, 17, 4, 8,
            31, 27, 13, 23, 21, 19, 16, 7,
            26, 12, 18, 6, 11, 5, 10, 9
        };
        return LUT[(((v & -v) * 0x077CB531U)) >> 27];
    }
    \end{lstlisting}
    这里的神奇常数{\bfseries 0x077CB531U}与De Bruijn Sequence
    \index{D!De Bruijn Sequence}有关,在此不详细介绍。
    LUT可预处理得出(但是不太好记住0x077CB531U)。
\end{itemize}

\subsubsection{实现}

设$conutTZ(x)$返回$x$末尾0的个数,代码如下:

\begin{lstlisting}[title=gcdX]
int gcdX(int a,int b) {
    if(a && b) {
        int off=countTZ(a|b);
        a>>=countTZ(a);
        do {
            b>>=countTZ(b);
            if(a>b)std::swap(a,b);
            b-=a;
        }while(b);
        return a<<off;
    }
    return a|b;
}
\end{lstlisting}

\section{欧拉定理}
\subsection{费马小定理}\label{FLTS}
\index{F!Fermat's Little Theorem}
\begin{theorem}[Fermat's Little Theorem]\label{FLT}
	~\\
	$\forall p \textrm{ is a prime number},a\in \mathbb{Z},a^p \equiv a \pmod{p}$
\end{theorem}

该定理是定理~\ref{ET}的特殊化,不证。

由定理~\ref{FLT}可得:

\begin{inference}
	$a^{-1} \equiv a^{p-2} \pmod{p}$
\end{inference}

可使用快速幂在$O(lgp)$的复杂度下求某个数的逆元。

\subsection{线性推逆元}

如果需要获得$a\in [1,p)$内模$p$的逆元,复杂度为$O(plgp)$逐个快速幂的方法
并不是最优的。

首先有$1^{-1}\equiv 1 \pmod{p}$。

令$p=qa+r$,其中$q=[\frac{p}{a}],r=p \bmod a$。

再把$p \equiv 0 \pmod{q}$中的$p$用$qa+r$代替,两边同时乘上$(ar)^{-1}$,
移项得$a^{-1}\equiv -qr^{-1} \pmod{q}$,即
$a^{-1}\equiv -[\frac{p}{a}](p \bmod a)^{-1} \pmod{p}$。

代码如下:
\begin{lstlisting}[title=inv]
inv[1]=1;
for(int i=2;i<=n;++i)
    inv[i]=asInt64(mod-mod/i)*inv[mod%i]%mod;
\end{lstlisting}

以上内容参考了Miskcoo的博客\footnote{[数论]线性求所有逆元的方法 – Miskcoo's Space\\
	\url{http://blog.miskcoo.com/2014/09/linear-find-all-invert}}

Update:还有更一般的做法,可以在$O(n+\lg p)$内推出任意$n$个非0数的逆元:

首先计算出前$i$个数的前缀积$M_i$,然后快速幂计算$M_n^{-1}$,最后从后往前倒推计算每个数的
逆元。比如要计算$A_i$的逆元,倒推维护$X=M_n^{-1}\prod_{j=i+1}^n{A_j}$,那么
$A_i^{-1}=M_{i-1}X$。

该方法源自WAAutoMaton的博客\CJKsout{(知识都是在乱翻他人博客中学到的)}\footnote{
	[loj ???] 乘法逆元2 题解
	\url{https://wa-am.com/2019/03/08/loj-乘法逆元2-题解}
}。
\subsection{欧拉定理}
\index{E!Euler's Theorem}
\begin{theorem}[Euler's Theorem]\label{ET}
	~\\
	对于任意互质正整数对$(a,n)$,有$a^{\varphi(n)} \equiv 1 \pmod{n}$
\end{theorem}
证明:

令$S=\{[x]_n\in Z_n|(a,n)=1\}$(由与$n$互质的模$n$剩余类组成的集合),
它与$\cdot_n$构成整数模$n$乘法群,$(S,\cdot_n)$的阶为$\varphi(n)$。

接着有两种证明思路:
\begin{itemize}
	\item 对于任意一个与$n$互质的正整数$a$,$a$的幂模$n$的值$a,a^2,\cdots,a^k$
	      构成了一个子群,其中$a^k\equiv 1 \pmod{n}$。

	      根据定理~\ref{LT},有$k|\varphi(n)$,令$M=\varphi(n)/k$,有
	      $a^{\varphi(n)}=a^{kM}=(a^k)^M\equiv 1^M\equiv 1 \pmod{n}$。
	\item 根据定义得对于$[x]_n\in S$和$S$中的所有元素$[a_1]_n,[a_2]_n,\cdots,
		[a_{\varphi(n)}]_n$,$[x]_n \cdot_n [a_i]_n$\\组成的集合仍然是$S$,
		因此有$x^{\varphi(n)}[a_1]_n[a_2]_n\cdots[a_{\varphi(n)}]_n=
		(x[a_1]_n)(x[a_2]_n)\cdots\\(x[a_{\varphi(n)}]_n)\equiv[a_1]_n
		[a_2]_n\cdots[a_{\varphi(n)}]_n\pmod{n}$,两边消去可得
		$x^{\varphi(n)}\equiv 1\pmod{n}$。
\end{itemize}

上述证明源自Wikipedia-EN\footnote{Euler's theorem - Wikipedia
	\url{https://en.wikipedia.org/wiki/Euler's\_theorem}}和Eden Harder
的博客\footnote{RSA 加密周边 - Eden Harder
	\url{http://edenharder.is-programmer.com/posts/43247.html}}。
\subsection{扩展欧拉定理}
\begin{theorem}\label{exEuler}
	$\forall a\in \mathbb{Z},x,m\in \mathbb{Z^+},x\geq \varphi(m)
		,a^x\equiv a^{x \bmod \varphi(m)+\varphi(m)} \pmod{m}$
\end{theorem}

\begin{lemma}\label{EEL1}
	\begin{displaymath}
		\left\{
		\begin{array}{l}
			x\equiv y \pmod{m_1} \\
			x\equiv y \pmod{m_2}
		\end{array}
		\right.
		\Rightarrow x\equiv y \pmod{lcm(m_1,m_2)}
	\end{displaymath}
\end{lemma}

证明:
\begin{displaymath}
	\left\{
	\begin{array}{l}
		x\equiv y \pmod{m_1} \\
		x\equiv y \pmod{m_2}
	\end{array}
	\right.
	\Rightarrow
	\left\{
	\begin{array}{l}
		x+c_1m_1=y \\
		x+c_2m_2=y
	\end{array}
	\right.
\end{displaymath}
\begin{displaymath}
	\Rightarrow
	c_1m_1=c_2m_2=k\cdot lcm(m_1,m_2)
	\Rightarrow
\end{displaymath}
\begin{displaymath}
	x \equiv y \pmod{lcm(m_1,m_2)}
\end{displaymath}

\begin{inference}\label{EEL1I}
	当$a,b$互质时,$x\equiv y \pmod{ab}$
\end{inference}

\begin{inference}
	\begin{displaymath}
		\left\{
		\begin{array}{l}
			x\equiv y \pmod{m_1} \\
			\cdots               \\
			x\equiv y \pmod{m_n}
		\end{array}
		\right.
		\Rightarrow x\equiv y \pmod{lcm(m_1,\cdots,m_n)}
	\end{displaymath}
\end{inference}

\begin{lemma}\label{EEL2}
	\begin{displaymath}
		\forall p\textrm{ is a prime number},q\in \mathbb{Z^+},q>1,
		\varphi(p^q)\geq q.
	\end{displaymath}
\end{lemma}

证明:首先有$\varphi(p^q)=(p-1)p^{q-1}$,当$p$固定时,$q$取2使得$\varphi(p^q)-q$
最小,但该值仍非负。当且仅当$p=2,q=2$时,$\varphi(p^q)=q$。

接下来证明定理~\ref{exEuler}:

首先证明当$m$为素数$p$的幂$(m=p^q)$时成立:
\begin{itemize}
	\item 若$gcd(a,p)=1$,则$gcd(a,p^q)=1$,根据欧拉定理可证在该情况下成立;
	\item 若$gcd(a,p)=p$,由适用范围可知$x\geq q$,由引理~\ref{EEL2}可知
	      $x \bmod \varphi(p^q) + \varphi(p^q) \geq q$,因此
	      $a^x\equiv 0 \equiv a^{x \bmod \varphi(p^q)+\varphi(p^q)} \pmod{p^q}$
\end{itemize}

对于任意$m$,可根据算术基本定理将其分解为素数幂之积。因为$\varphi(p^q)|\varphi(m)$,所以有
$a^x\equiv a^{x \bmod \varphi(p^q)+\varphi(p^q)}
	\equiv a^{x \bmod \varphi(m)+\varphi(m)} \pmod{p^q}$。
根据引理~\ref{EEL1}及其推论合并这些式子可证明该定理。

以上证明源自后缀自动机·张的文章\footnote{微小的欧拉定理EXT证明
	\url{https://zhuanlan.zhihu.com/p/24902174}}。

\section{素性测试}
\subsection{Miller Rabin随机性素性测试}
\index{M!Miller–Rabin Primality Test}
\subsubsection{朴素算法}
根据~\ref{FLTS}中所述的费马定理,若要测试$p$是否为素数,
选取基数$a\in [2,p)$,检查$a^{p-1}\equiv 1 \pmod{p}$是否对
所有的$a$均成立。
\subsubsection{Miller-Rabin}
考虑每次随机选取多个$a$,当有$k$个$a$满足时,出错的概率最多为$2^{-k}$。证明详见
算法导论\cite{ITA3}第31.8节定理31.39。
Miller-Rabin沿用了朴素算法的思路,并且使用$witness$测试来代替朴素检查算法以尽可能避免把
{\bfseries Carmichael}数当做素数。

$witness(x,base)$返回$true$当$base$可以证明$x$是合数。
\begin{lstlisting}[title=witness]
bool witness(int x, int base) {
    int end = x - 1;
    int c = countTZ(end);
    int t = powm(base, end >> c, x);
    while(c--) {
        int ct = mulm(t, t, x);
        if(t != 1 && t != x - 1 && ct == 1)
            return true;//case 1
        t = ct;
    }
    return t != 1;//case 2
}
\end{lstlisting}
接下来证明正确性:
\begin{itemize}
	\item 如果从case~1处返回$true$,则说明找到了模$x$意义
            下$1$的一个非平凡平方根。

          \begin{theorem}\label{WITNESST}
              如果p是一个奇素数且$e\geq 1$,则方程

              \begin{equation}\label{sqrm}
                   x^2 \equiv 1 \pmod{p^e}
              \end{equation}
              只有两个解$x=\pm 1$。
	      \end{theorem}

          证明:方程~\ref{sqrm}等价于$p^e|(x-1)(x+1)$。因为$p>2$,所以
          $p|(x-1)$与$p|(x+1)$仅有一个成立(否则$p|((x+1)-(x-1))=2$),
          两个解为$x=\pm 1$。

	      \begin{inference}
		      如果模$n$意义下存在$1$的非平凡平方根,则$n$为合数。
	      \end{inference}

	      证明:定理~\ref{WITNESST}的逆否命题也成立,所以$n$不可能为奇素数的幂,且对于
	      $n=1,2$均不存在非平凡平方根,因此$n$必为合数。

	      根据该推论可得case~1有证据证明$x$为合数。
	\item 如果从case~2处返回$true$,则说明$x$不满足费马定理。
\end{itemize}
以上内容参考了算法导论\cite{ITA3}第31.8节。
\subsubsection{实现细节}
\begin{itemize}
    \item 在MillerRabin之前可先用前几个素数筛掉大部分的合数。
    \item 如果数据范围在4759123141内,即在$uint32\_t$范围内,
    只用2,7,61为基数判断。
    \item 如果数据范围在$10^{16}$内,使用2,3,7,61,24251作为基数,
    唯一的强伪素数为46856248255981。
\end{itemize}

以上内容参考了Matrix67的博客\footnote{
	数论部分第一节:素数与素性测试 | Matrix67: The Aha Moments
	\url{http://www.matrix67.com/blog/archives/234}}。
\subsection{Baillie–PSW素性测试}
\index{L!Baillie–PSW Primality Test}
这个方法在WC2019朱震霆的讲课课件《\CJKsout{简单}数论算法》中被提及。它在$2^{64}$次方内的
结果完全正确,适用于一般情况。\CJKsout{现在仍然没有发现确定的合数能够通过这个测试,不过这样
的数确实是存在的。}试验表明它比MillerRabin的效率更低,代码更长。

该算法结合了基于2的强Fermat测试(即Miller-Rabin的子过程witness)与强Lucas测试。
虽然这两个测试的伪素数十分多,但是这两个伪素数集合的交的大小(集合大小仍然是正无穷,
这里的大小可以理解为分布密度)要小得多。\CJKsout{unsigned long long内靠谱就行。}

\subsubsection{强Lucas测试}
\index{S!Strong Lucas Probable\\ Prime Test}
\paragraph{雅可比符号}
\index{J!Jacobi Symbol}
雅可比符号是勒让德符号(\ref{Legendre})的推广。它不再要求$p$是奇素数,而仅要求$p$是
大于1的奇数。为了防止误解$p$,这里的$p$使用$n$代替。

雅可比符号具有以下性质:
\begin{eqnarray}
    a\equiv b\pmod{n}&\Rightarrow& \Jacobi{a}{n}=\Jacobi{b}{n}\label{JSR1}\\
    \Jacobi{a}{n}&=&\left\{
    \begin{array}{lr}
        0&(a,n)\neq 1\\
        \pm 1&(a,n)=1
    \end{array}
    \right.\label{JSR2}\\
    \Jacobi{a}{bc}&=&\Jacobi{a}{b}\Jacobi{a}{c}\label{JSR3}\\
\end{eqnarray}

此外,二次互反律,完全积性函数,以及$a=2$时的勒让德符号的性质对于雅可比符号仍然成立。

根据性质~\ref{JSR3}可以得出若$\displaystyle n=\prod{p_i^{e_i}}$,
则$\displaystyle \Jacobi{a}{n}=\prod{\Legendre{a}{p_i}^{e_i}}$。

雅可比符号的计算可在$O(\lg a\lg n)$的时间内解决,记过程为$jacobi(a,b)$:

\begin{enumerate}
    \item 根据性质~\ref{JSR1}可以等价调用$jacobi(a\%b,b)$。
    \item 根据完全积性函数与$a=2$时的性质把$a$的因子2消去。
    \item 若$b=1$,根据完全积性函数的性质,返回1。
    \item 若$(a,b)\neq 1$,根据性质~\ref{JSR2},返回0。
    \item 否则$(a,b)=1$,使用二次互反律调用子过程$jacobi(b,a)$。
\end{enumerate}

这个过程与欧几里得算法很像。

\paragraph{Lucas序列生成}
\index{L!Lucas Sequence}
Lucas序列有参数$P,Q,D$,可由数列$(U,V)$组合而成,在此只单独研究这两个数列。

它们满足以下递推关系:

\begin{eqnarray*}
    U_0&=&0\\
    V_0&=&2\\
    U_{2k}&=&U_kV_k\\
    V_{2k}&=&V_k^2-2Q^k\\
    U_{2k+1}&=&(PU_{2k}+V_{2k})/2\\
    V_{2k+1}&=&(DU_{2k}+PV_{2k})/2
\end{eqnarray*}

除法操作中如果分子为奇数则再加上模数???

由递推关系可以从高位到低位构造出$(U_n,V_n)$。

\paragraph{判定}
若$U_{n-\Jacobi{D}{n}}\not \equiv 0\pmod{n}$,则$n$必为合数。
对于$\Jacobi{D}{n}=-1$,该条件改为$U_{n+1}\not \equiv 0\pmod{n}$。

此外可以检查$V_{n+1}\not \equiv 2Q \pmod{n}$,满足任意一个条件就判定它为合数,
这个几乎不增加计算代价的操作提高了判定合数的概率。
\subsubsection{实现}

算法步骤如下:
\begin{enumerate}
    \item 预筛:使用小素数试除可以筛去绝大多数合数(然而素数判定板子题生成的合数
    基本上是大质数之积)。
    \item 执行基于2的强Fermat测试。也可以使用其它的基,不过2已经经过了大量测试。
    \item 从数列A157142\footnote{参见\url{http://oeis.org/A157142}}
    ($1,-3,5,-7,9,-11,13,-15\cdots$)的第三项开始,找到第一个整数$D$满足
    $\Jacobi{D}{n}=-1$。选择这个数列的原因是易于生成且平均测试次数约为
    3.147755149。注意如果$n$是一个完全平方数,那么找不到满足条件的$D$,可以使用
    二分法或牛顿法快速开平方根预先判断。
    \item 以参数$D,P=1,Q=(1-D)/4$执行强Lucas测试。
\end{enumerate}

模板代码如下:
\lstinputlisting{Source/Templates/BailliePSW.cpp}

正确性证明留坑待补。
\index{*TODO!Baillie-PSW素性测试正确性证明}

上述内容参考了Wikipedia-EN\footnote{
    Jacobi Symbol
    \url{https://en.wikipedia.org/wiki/Jacobi\_symbol}

    Baillie–PSW primality test
    \url{https://en.wikipedia.org/wiki/Baillie-PSW\_primality\_test}

    Lucas pseudoprime
    \url{https://en.wikipedia.org/wiki/Lucas\_pseudoprime}

    Lucas sequence
    \url{https://en.wikipedia.org/wiki/Lucas\_sequence}
}。英文论文参见Lucas Pseudoprimes\cite{BPSW}
(\url{http://mpqs.free.fr/LucasPseudoprimes.pdf})

\section{Pollard Rho启发式因子分解}
\index{P!Pollard Rho}
Pollard Rho算法可以期望在$\Theta(\sqrt{p})$次算术运算内得到$n$的一个小因子$p$。

\subsection{利用Birthday Trick提高效率}
\subsubsection{朴素随机算法}
考虑每次随机选择一个数$x\in [2,n-1]$,若$x|n$,则$x$为$n$的一个因子,最坏情况下
($n$为两质数之积)期望测试次数为$\frac{n-2}{2}$。
\subsubsection{Birthday Trick}
\index{B!Birthday Trick}
可以把问题转换为选取$k$个数$x$,每次询问是否存在$(x_i-x_j)|n$。
根据生日悖论,当$k$达到$\Theta(\sqrt{n})$级别时,可以期望得到一组$(i,j)$满足该条件。
具体证明参见算法导论\cite{ITA3}第5.4.1节中采用指示器随机变量的分析。

询问是否存在$(x_i-x_j)|n$仍然不够高效,可以转换为询问是否存在$gcd(x_i-x_j,n)>1$。
那么只要选取约$\sqrt[4]{n}$个数即可。

所以Pollard Rho使用如下策略:
\begin{enumerate}
    \item 随机在区间$[2,n-1]$中选取$k$个数;
    \item 询问是否存在$gcd(x_i-x_j,n)>1$。
\end{enumerate}

在实践中把$k$个数存下是不可能的,可以每次把相邻的随机数当做$x_i,x_j$来测试。
有一个简单有效的伪随机数生成函数:$f(x)=x^2+c \bmod n$。

\subsection{Floyd判圈法}
对于某些数据,可能会导致在找到因子前伪随机数生成函数却掉入死循环的情况,此时将
永远找不到因子。遇到这种情况,应该及时退出迭代,修改随机数种子$x_0$与偏移$c$。

如何判断是否掉入死循环呢?这里使用Floyd发明的方法:令$A,B$同时迭代生成随机数,
$A_{i+1}=f(A_i),B_{i+1}=f(f(B_i))$,当$A_i=B_i$时,$B$至少比$A$多走了
一圈,说明已经掉入了循环。

代码实现:
\begin{lstlisting}[title=Pollard Rho]
int f(int x,int c,int n) {
    return (asInt64(x)*x+c)%n;
}
int pollardRhoImpl(int n,int seed) {
    int c=rand()%n;
    int a=f(seed,n),b=f(a,c,n);
    while(a!=b) {
        int d=gcd(iabs(a-b),n);
        if(d!=1 && d!=n)
            return d;
        a=f(a),b=f(f(b));
    }
    return 0;
}
int pollardRho(int x) {
    int d=0;
    do {
        d=pollardRhoImpl(x,rand()%n);
    } while(d);
    return d;
}
\end{lstlisting}

以上内容参考了\emph{A Quick Tutorial on Pollard's Rho Algorithm}
\footnote{原文地址
\url{www.cs.colorado.edu/~srirams/classes/doku.php/pollard\_rho\_tutorial}
\\中文翻译
\url{http://files.cnblogs.com/files/Doggu/Pollard-rho
\%E7\%AE\%97\%E6\%B3\%95\%E8\%AF\%A6\%E8\%A7\%A3.pdf}
}。

\section{RSA算法}
\subsection{原理}
\begin{theorem}[素数定理]\label{PT}
	$\displaystyle \lim_{n\rightarrow\infty}\frac{\pi(n)}{n/\ln n}=1$
\end{theorem}
RSA的安全性基于以下事实:寻找大素数很容易(根据定理~\ref{PT},素数密度挺大的),
但把两个大质数之积质因数分解却很难。

RSA算法的基本步骤如下:
\begin{enumerate}
	\item 随机选取两个大素数$p,q$,使得$p\neq q$,令$n=pq$;
	\item 选取一个与$\varphi(n)=(p-1)(q-1)$互质的小奇数$e$,
	      计算模$\varphi(n)$意义下$e$的乘法逆元$d$(由于$e$与$n$互质,根据定理
	      ~\ref{BT}得存在唯一解$d$);
	\item 将$P(e,n)$公开,作为{\bfseries RSA公钥};\\
	      将$S(d,n)$保密,作为{\bfseries RSA私钥}。
\end{enumerate}

对于消息$M$,公钥持有者可进行运算:$P(M)=M^e \bmod n$;
私钥持有者可进行运算:$S(M)=M^d \bmod n$。
对于用公/私钥加密$M$得到的密文$C$,只有使用私/公钥解密才能得到$M$(加解密操作相同)。
由于结果要$\bmod n$,消息$M$的域为$Z_n$。

下面证明RSA算法的正确性,即:
\begin{displaymath}
	P(S(M))\equiv S(P(M))\equiv M^{ed}\equiv M \pmod{n}
\end{displaymath}

因为$e,d$是模$\varphi(n)$意义下的乘法逆元,所以有$ed=1+k(p-1)(q-1)$。

\begin{itemize}
	\item 若$M\not\equiv 0 \pmod{p}$,则有
	      \begin{eqnarray*}
		      M^{ed}&\equiv& M^{1+k(p-1)(q-1)} \pmod{p}\\
		      &\equiv& M\cdot (M^{p-1})^{k(q-1)} \pmod{p}\\
		      &\equiv& M\cdot 1^{k(q-1)} \pmod{p}\\
		      &\equiv& M \pmod{p}
	      \end{eqnarray*}
	\item 若$M\equiv 0 \pmod{p}$,上述等式仍成立。
\end{itemize}

同样地,对于$q$有$M^{ed}\equiv M \pmod{q}$。根据推论~\ref{EEL1I},有
$M^{ed}\equiv M \pmod{n}$,证毕。

\subsection{应用}
\subsubsection{消息加密}
发送方使用接收方的公钥$P$把消息$M$加密得到密文$C$,将密文$C$发送给
接收方。接收方使用自己的私钥$P$解密得到消息$M$。
\paragraph{快速无公钥加密系统}
若消息过长,则仅用$P$加密对称加密算法的随机密钥$K$,同时用密钥$K$加密
$M$得到密文$C$,把$(P(K),C)$发送给接收方。接收方使用$P$解密得到$K$,
再用$K$对$C$解密。
\subsubsection{数字签名}
发送方使用自己的私钥$S$把消息$M$签署得到签名$C$,将消息$M$与签名$C$
发送给接收方。接收方使用发送方的公钥$P$解密$C$得到消息$M$,验证消息是否正确。
\paragraph{快速数字签名}
与快速无公钥加密系统类似,把对称加密算法的密钥改为快速散列函数的值。
\paragraph{证书链}
以一个可信根为起点,大家都信任它并且知道它的公钥。下一级将自己的公钥和被上一级
签署后的公钥作为自己的签名证书,接收方自上而下验证证书链上每一级的正确性,从而验证链尾端
消息的正确性。

以上内容参考了算法导论\cite{ITA3}第31.7节。

\section{中国剩余定理CRT}
\subsection{CRT}
\index{C!Chinese Remainder Theorem}
\begin{theorem}[Chinese Remainder Theorem]
    对于模线性方程组:
    \begin{displaymath}
        \left\{\begin{array}{l}
            x \equiv a_1 \pmod{n_1}\\
            x \equiv a_2 \pmod{n_2}\\
            \ldots\\
            x \equiv a_k \pmod{n_k}
        \end{array}\right.
    \end{displaymath}\\
    其中$n_1,n_2,\ldots,n_k$两两互质,令$N=\prod_{i=1}^k{n_i}$,
    该模线性方程组在$[0,N)$内有唯一解。
\end{theorem}
如何求解该线性方程组呢?和拉格朗日插值法的思路相同,对于每一个方程都给最终的解
贡献一个$x_i$,满足
\begin{displaymath}
x_i \bmod n_j =
\left\{\begin{array}{ll}
0 & \textrm{if $i\neq j$}\\
a_i & \textrm{if $i=j$}
\end{array}\right.
\end{displaymath}
答案即为$\sum_{i=1}^n{x_i} \bmod N$。
考虑$i\neq j$时$x_i$应该整除$n_j$,因此$x_i$应该有系数$M=N/n_i$;当$i=j$时,
$x_i$应该有系数$a_i$,为了抵消$M$带来的影响,再乘上$M$模$n_i$的乘法逆元即可(
由于$n$两两互质,$M$与$n_i$也互质,根据定理~\ref{ET},保证其乘法逆元存在)。
\subsection{ExCRT}
当$n$不满足两两互质的条件时,可能会找不到其乘法逆元。
所以我们采用另一种思路求解方程:每次选择两个方程将其合并,直到只剩一个方程为止。

考虑两个方程组成的方程组:
\begin{displaymath}
    \left\{\begin{array}{l}
        x \equiv a_1 \pmod{n_1}\\
        x \equiv a_2 \pmod{n_2}\\
    \end{array}\right.
\end{displaymath}
等价于
\begin{eqnarray}
    x=a_1+k_1n_1\label{CRTE}\\
    x=a_2+k_2n_2
\end{eqnarray}
移项得$k_1n_1-k_2n_2=a_2-a_1$,可以使用
$exgcd$求出$c_1n_1+c_2n_2=gcd(n_1,n_2)$的各项参数。根据定理~\ref{BT},
若$gcd(n_1,n_2)\not\mid(a_2-a_1)$则该方程组无解。等比例缩放方程求出$k1$,
带入方程~\ref{CRTE}反推出$x0$,得到新的模线性方程$x \equiv x0
\pmod{lcm(n_1,n_2)}$。

\section{积性函数与线性筛}
\subsection{定义}
\index{A!Arithmetic Function}
\paragraph{数论函数(Arithmetic Function)}
若函数$f:\mathbb{Z^+}\rightarrow\mathbb{C}$,则称函数$f$为数论函数。

\index{M!Multiplicative Function}
\paragraph{积性函数(Multiplicative Function)}
若函数$f$为数论函数,且$f(1)=1$,对于任意互质的正整数$a,b$都有$f(ab)=f(a)f(b)$,
则称函数$f$为积性函数。

\index{C!Completely Multiplicative\\ Function}
\paragraph{完全积性函数(Completely Multiplicative Function)}
若函数$f$为积性函数且对于任意正整数$a,b$都有$f(ab)=f(a)f(b)$,
则称函数$f$为完全积性函数。
\begin{property}\label{MFC}
	若$f$为积性函数,对于正整数$\displaystyle n=\prod_{i=1}^m{{p_i}^{c_i}}$,有
	$\displaystyle f(n)=\prod_{i=1}^m{f({p_i}^{c_i})}$
\end{property}
\begin{property}
	若$f$为完全积性函数,对于正整数$\displaystyle n=\prod_{i=1}^m{{p_i}^{c_i}}$,
	有$\displaystyle f(n)=\prod_{i=1}^m{f(p_i)^{c_i}}$
\end{property}
\subsection{常见积性函数}
\subsubsection{积性函数}
\begin{itemize}
	\item 除数函数$\displaystyle \sigma_k(n)=\sum_{d|n}{d^k}$,\\
	      根据性质~\ref{MFC}可得
	      $\displaystyle \sigma_k(n)=\prod_{i=1}^m{\sum_{j=0}^{c_i}{p_i^{jk}}}$
	\item 约数个数函数$\tau(n)=\sigma_0(n)$
	\item 约数和函数$\sigma(n)=\sigma_1(n)$
	\item \index{E!Euler Totient Function}
	      欧拉函数(Euler Totient Function)
	      $\displaystyle \varphi(n)=\sum_{i=1}^n{[(n,i)=1]}=n\prod_{p|n}{(1-\frac{1}{p})}$,
	      且有$\displaystyle \sum_{i=1}^n{[(n,i)=1]*i}=\frac{n\varphi(n)+[n=1]}{2}$
	\item \index{M!Möbius function}
	      莫比乌斯函数定义为:
	      \begin{displaymath}
		      \mu(d)=
		      \left\{
		      \begin{array}{ll}
			      1      & \textrm{if $d=1$}                                \\
			      (-1)^k & \textrm{if $\displaystyle d=\prod_{i=1}^k{p_i}$} \\
			      0      & \textrm{otherwise}
		      \end{array}
		      \right.
	      \end{displaymath}

	      简单来说就是如果存在平方因子则$\mu(n)$为0,否则$\mu(n)=(-1)^\textrm{质因子数}$。
\end{itemize}
\begin{theorem}\label{MobiusT}
	\begin{displaymath}
		[n=1]=\sum_{d|n}{\mu(d)}
	\end{displaymath}
\end{theorem}
证明:当$n=1$时,该等式成立。
对于$n>1$的情况,将$n$分解为$\displaystyle \prod_{i=1}^m{{p_i}^{c_i}}$,令
$\displaystyle X=\prod_{i=1}^m{p_i}$,
仅考虑$\mu(d)\neq 0$的部分,$\mu(d)$有贡献当且仅当$d|X$,因此$d$可表示为一个长度为$m$
的01向量。
由二项式定理可知选取奇数个1的向量方案数等于选取偶数个1的向量方案数,即正负贡献抵消。
\begin{theorem}\label{PhiT}
	\begin{displaymath}
		n=\sum_{d|n}{\varphi(d)}
	\end{displaymath}
\end{theorem}
证明:将$n$个分数$\frac{1}{n},\frac{2}{n},\cdots,\frac{n}{n}$化为最简分数,
$\varphi(x)$即表示分母为$x$的最简分数个数。
\begin{theorem}\label{SigmaT}
	\begin{eqnarray*}
		\sum_{i=1}^n{\tau(i)}&=&\sum_{i=1}^n{[\frac{n}{i}]}\\
		\sum_{i=1}^n{\sigma(i)}&=&\sum_{i=1}^n{i\cdot[\frac{n}{i}]}
	\end{eqnarray*}
\end{theorem}
证明:枚举因子$i$,$n$以内有$[\frac{n}{i}]$个因子。
\subsubsection{完全积性函数}
\begin{itemize}
	\item \index{U!Unit Function}
	      元函数(Unit Function)~$\epsilon(n)=[n=1]$
	\item \index{C!Constant Function}
	      恒等函数(Constant Function)~$1(n)=1$
	\item 单位函数$id(n)=n$
	\item 幂函数$id^k(n)=n^k$
\end{itemize}
以上内容参考了skywalkert的博客\footnote{浅谈一类积性函数的前缀和\\
	\url{https://blog.csdn.net/skywalkert/article/details/50500009}}与
Wikipedia-EN\footnote{Arithmetic function - Wikipedia
	\url{https://en.wikipedia.org/wiki/Arithmetic\_function}}。
\subsection{线性筛}
主要思路是每次拿当前的数和已经筛出的素数构造成新的合数并将其筛去。

代码如下:
\begin{lstlisting}[title=Euler]
int prime[size/log(size)],pcnt=0;
bool flag[size];
void pre(int n) {
    for(int i=1;i<=n,++i) {
        if(!flag[i])
            prime[++pcnt]=i;
        for(int j=1;j<=pcnt && prime[j]*i<=n;++j) {
            flag[prime[j]*i]=true;
            if(i%prime[j]==0)
                break;//case 1
        }
    }
}
\end{lstlisting}
注意到case 1中的优化,它保证了每个合数最多被筛1次,从而使时间复杂度变为$O(n)$,
并且增加了一个性质:合数只被其最小质因子筛去。
接下来证明该优化的正确性:当$i\bmod p_j=0$时,有$i=kp_j$,
若要用$p_{j+x}$筛去后面的合数$ip_{j+x}=kp_jp_{j+x}$,可知该合数未来将被合数$kp_{j+x}$与素数
$p_j$筛去,直接跳出不会影响结果,且保证合数被最小质因子筛除,便于质因数分解。

如果仅仅要筛$10^8$以内的素数,$O(n\lg\lg n)$的埃氏筛法更为Cache-Friendly,比线性筛
更为高效。
\subsection{积性函数筛}
\subsubsection{欧拉函数}
\begin{itemize}
	\item $\varphi(1)=1$;
	\item 若$i$为素数,则$\varphi(i)=i-1$;
	\item 若$i \bmod p_j=0$,则说明$ip_j$存在至少两个因子$p_j$,因此
	      $\varphi(ip_j)=\varphi(i)p_j$;
	\item 若$i \bmod p_j\neq 0$,则根据积性函数性质可得
	      $\varphi(ip_j)=\varphi(i)(p_j-1)$。
\end{itemize}
\subsubsection{莫比乌斯函数}
\begin{itemize}
	\item $\mu(1)=1$;
	\item 若$i$为素数,则$\mu(i)=-1$;
	\item 若$i \bmod p_j=0$,则说明$ip_j$存在至少两个因子$p_j$,因此
	      $\mu(ip_j)=0$。注意若数组已清零则不赋值;
	\item 若$i \bmod p_j\neq 0$,则根据积性函数性质可得
	      $\mu(ip_j)=-\mu(i)$。
\end{itemize}
\subsubsection{约数个数}
记数组$A_i$为$i$中最小质因子的次数。
\begin{itemize}
	\item $\tau(1)=1,A_1=0$;
	\item 若$i$为素数,则$\tau(i)=2,A_i=1$;
	\item 若$i \bmod p_j=0$,则说明$ip_j$存在至少两个因子$p_j$,因此
	      $\tau(ip_j)=\tau(i)\cdot\frac{A_i+2}{A_i+1}$且$A_{ip_j}=A_i+1$;
	\item 若$i \bmod p_j\neq 0$,则根据积性函数性质可得
	      $\tau(ip_j)=2\tau(i)$且$A_{ip_j}=1$。
\end{itemize}
\subsubsection{约数和}
由性质~\ref{MFC}可得
\begin{displaymath}
	\sigma(n)=\prod_{i=1}^m{\sum_{j=0}^{c_i}{p_i^j}}
\end{displaymath}
记数组$low_i$为$i$中最小质因子的幂,$sum_i$为$i$中最小质因子的贡献。
\begin{itemize}
	\item $\sigma(1)=1,low_1=1,sum_1=1$;
	\item 若$i$为素数,则$\sigma(i)=i+1,low_i=i,sum_i=i+1$;
	\item 若$i \bmod p_j=0$,则说明$ip_j$存在至少两个因子$p_j$,因此
	      $\sigma(ip_j)=\sigma(i)\cdot\frac{sum_{ip_j}}{sum_i}$且
	      $low_{ip_j}=low_i*p_j,sum_{ip_j}=sum_i+low_{ip_j}$;
	\item 若$i \bmod p_j\neq 0$,则根据积性函数性质可得
	      $\sigma(ip_j)=(p_j+1)\sigma(i)$且
	      $low_{ip_j}=p_j,sum_{ip_j}=p_j+1$。
\end{itemize}
\subsubsection{普通积性函数}
同约数和的思想,记数组$sum_i$为$i$中最小质因子的贡献。
要求能够快速推出$f({p_i}^{c_i})$的值。
\begin{itemize}
	\item $f(1)=1$;
	\item 若$i$为素数,则$f(i)=sum_i=\cdots$;
	\item 若$i \bmod p_j=0$,则说明$ip_j$存在至少两个因子$p_j$,因此
	      $f(ip_j)=f(i)\cdot\frac{sum_{ip_j}}{sum_i}$;
	\item 若$i \bmod p_j\neq 0$,则根据积性函数性质可得
	      $f(ip_j)=f(p_j)f(i)$。
\end{itemize}
以上内容参考了租酥雨的博客\footnote{积性函数与线性筛 - 租酥雨
	\url{https://www.cnblogs.com/zhoushuyu/p/8275530.html}}。
\subsection{因子分解}
通过在每次筛除时记录其最小质因子,可以于$O(\lg n)$复杂度内分解因子。

\section{狄利克雷卷积,狄利克雷逆与莫比乌斯反演}
\subsection{狄利克雷卷积}
\index{D!Dirichlet Convolution}
对于数论函数$f,g$,定义狄利克雷卷积
\begin{displaymath}
	(f*g)(n)=\sum_{d|n}{f(d)g(\frac{n}{d})}=\sum_{ab=n}{f(a)g(b)}
\end{displaymath}
由积性函数集合与狄利克雷卷积组成的群的乘法单位元为元函数$\epsilon$。

狄利克雷卷积有如下性质:
\begin{eqnarray*}
	\textrm{结合律} & (f*g)*h=f*(g*h)\\
	\textrm{分配律} & f*(g+h)=f*g+f*h\\
	\textrm{交换律} & f*g=g*f;\\
	\textrm{单位元} & f*\epsilon=\epsilon*f=f。
\end{eqnarray*}
\subsection{狄利克雷逆}
\index{D!Dirichlet Inverse}
已知数论函数$f$,求$g=f^{-1}$,满足$f*g=\epsilon$。
\begin{itemize}
	\item 当$n=1$时,有$(f*g)(1)=f(1)g(1)=\epsilon(1)=1$,
	      解得$g(1)=\frac{1}{f(1)}$。
	\item 当$n>1$时,
	      有$\displaystyle (f*g)(n)=\sum_{ab=n}{f(a)g(b)}=\epsilon(n)=0$,
		解得
		\begin{displaymath}
		g(n)=\frac{-1}{f(1)}\sum_{d|n,d<n}{f(\frac{n}{d})g(d)}
		\end{displaymath}
\end{itemize}
\subsubsection{狄利克雷逆性质}
\begin{property}
	积性函数的狄利克雷逆仍然是积性函数。
\end{property}
\begin{property}
	若数论函数$f,g$为积性函数,则$(f*g)^{-1}=f^{-1}*g^{-1}$。
\end{property}
\begin{property}\label{CMFP}
	积性函数$f$为完全积性函数当且仅当$f^{-1}(n)=\mu(n)f(n)$。
\end{property}
\subsubsection{常见数论函数及其狄利克雷逆}
\begin{itemize}
	\item $1*\mu=\epsilon$\\
	      参见定理~\ref{MobiusT}的证明。
	\item $id^\alpha*(\mu\cdot id^\alpha)=\epsilon$\\
	      根据性质~\ref{CMFP}可证明。
	\item $\displaystyle \varphi*(\sum_{d|n}{\mu(d)d})=\epsilon$\\
	      由定理~\ref{PhiT}可得$id=\varphi*1$,两边同时乘上$\mu$
	      可得$id*\mu=\varphi$,所以$\varphi^{-1}=id^{-1}*\mu^{-1}=id^{-1}*1$。
	\item $\sigma_\alpha*(\sum_{d|n}{\mu(d)\mu(\frac{n}{d})d^\alpha})=\epsilon$

	      $\sigma_\alpha=id^\alpha*1$可推出
	      $(\sigma_\alpha)^{-1}=(id^\alpha)^{-1}*\mu$
\end{itemize}
以上内容参考了Wikipedia-EN\footnote{Dirichlet convolution - Wikipedia\\
	\url{https://en.wikipedia.org/wiki/Dirichlet\_convolution}}。
\subsection{莫比乌斯反演}
\index{M!Möbius Inversion}
\begin{theorem}
	对于数论函数$f,g$,满足$\displaystyle g(n)=\sum_{d|n}f(d)$,则有
	\begin{displaymath}
		f(n)=\sum_{d|n}\mu(d)g(\frac{n}{d})
	\end{displaymath}
\end{theorem}
莫比乌斯反演可表示为若$g=f*1$则$f=\mu*g$。
证明:将$g=f*1$两边同时乘上$\mu$可证。
证明源自Wikipedia-EN\footnote{Möbius inversion formula - Wikipedia\\
	\url{https://en.wikipedia.org/wiki/Mobius_inversion}}。
\subsection{常见技巧}
\begin{itemize}
	\item
	      对于数论函数$g,f$,
	      \begin{displaymath}
		      g(n)=\sum_{n|d}{f(d)}\Rightarrow
		      f(n)=\sum_{n|d}{\mu(d)g(\frac{d}{n})}
	      \end{displaymath}
	\item
	      若$\displaystyle n=\prod_{i=1}^m{{p_i}^{c_i}},g(n)=\sum_{d|n}{f(d)}$
	      且$f$为积性函数,将$g$看做$f*1$可知$g$也是积性函数,则$\displaystyle
		      g(n)=\prod_{i=1}^m{\sum_{j=0}^{c_i}{f(p_i^j)}}$。
	\item 交换内外求和顺序。
	\item 枚举倍数、因子、最大公约数等有共性的值并换元。
	\item 在化简前缀和函数时可能会遇到如下式子:
	      \begin{eqnarray*}
		      ans(n)&=&\sum_{i=1}^n{f(i)}\\
		      &=&A(n)+B(n)\sum_{i=1}^n{\sum_{d|i}{f(d)}}\\
		      &=&A(n)+B(n)\sum_{\frac{i}{d}=1}^n{\sum_{j=1}^{\left[\frac{n}{\frac{i}{d}}\right]}{f(j)}}\\
		      &=&A(n)+B(n)\sum_{t=1}^n{\sum_{j=1}^{[\frac{n}{t}]}{f(j)}}\\
		      &=&A(n)+B(n)\sum_{t=1}^n{ans([\frac{n}{t}])}
	      \end{eqnarray*}
	      线性筛预处理一部分前缀和(一般预处理规模为$n^{2/3}$,最终时间复杂度
	      $O(n^{2/3})$,大规模前缀和使用根号分块法递归计算。

	      注意这里可以使用存储Trick来Cache计算结果(多次询问使用map或时间戳数组
	      清零,下面只讨论单次询问的情况)。设预处理了前$k$个前缀和,其中$k\geq \sqrt{n}$。
		那么$[\frac{n}{t}]>k$的值不超过$\sqrt{n}$个,并且$t$不同对应的$[\frac{n}{t}]$值也不同。
		所以可以以$t$为下标把计算结果存入另一个数组中。
	\item 同时除以最大公约数使其互质,然后套用$\varphi$。
	\item $\displaystyle [gcd(i,j)=1]=\sum_{k|gcd(i,j)}\mu(k)=\sum_{k|i,k|j}\mu(k)$
	\item $\displaystyle gcd(i,j)=\sum_{k|gcd(i,j)}{\varphi(k)}=\sum_{d=1}^{min(i,j)}{\varphi(d)[\frac{i}{d}][\frac{j}{d}]}$

		遇到该式时不要枚举gcd值将其转换为上一个式子,从而陷入更复杂的化简。

	\item $\displaystyle \sum_{i=1}^n{i}=
		      \sum_{i=1}^n{\sum_{d|i}\varphi(d)}=
		      \sum_{d=1}^n{\varphi(d)\cdot[\frac{n}{d}]}$
	\item $(id\cdot\varphi)*id=id^2$
	\item 「CQOI2015」选数:

	将$[L,H]$内的$K$的倍数除以$K$得到区间$[l,r]$,
	记$G_i$为$[l,r]$中$i$的倍数个数,使用前文所述技巧推出答案为
	$\displaystyle \sum_{d=1}^r{\mu(d)G_d^n}$。但是$r$仍然是$1e9$级别的,
	可以考虑整除分块枚举相同的$G_d$再使用杜教筛求$\mu$的前缀和解决。

	注意这里还有个条件$H-L\leq 1e5$,infinity\_edge的博客\footnote{
		「BZOJ3930」「CQOI2015」选数\\
		\url{https://blog.csdn.net/infinity\_edge/article/details/78829630}
	}中提到存在一个性质:
	{\bfseries 若所有数字不全相同,则数字的极差不小于它们的最大公约数。}

	接下来只处理数字不全相同的情况,由于选取的数在$[l,r]$内,因此只需枚举到
	$r-l$,同时选取方案要扣除数字全相同的情况,即$G_d^n-G_d$。这种讨论还有个
	例外:数字全为$1$时是可行解,最后需要特判$l$是否为1。时间复杂度
	$O((r-l)+\sqrt{r-l}\lg N)$。

	\item 「LibreOJ β Round \#4」求和:

	$\displaystyle \sum_{d|n}{\mu^2(d)\mu(\frac{n}{d})}$当且仅当$n$
	为完全平方数时值为$\mu(\sqrt{n})$,其余情况为0。

	证明:考虑$n$的质因数分解,若存在质因数幂次$>2$,则必有一个$\mu$值为0,该式的值为0;
	若存在质因数幂次为1,不考虑其它质因数是否使$\mu$为0,$\mu(\frac{n}{d})$项必然抵消。

	综上所述,$n$为完全平方数时才有贡献。
	\item $\mu(ab)=\mu(a)\mu(b)[(a,b)=1]$

	证明:当$(a,b)\neq 1$时,$ab$有平方因子,值为0。否则根据积性函数的性质,
	$\mu(ab)=\mu(a)\mu(b)$。
	\item LOJ\#6027. 「from CommonAnts」质数计数 I

	考虑构造一个积性函数使得$f(p)=[p\equiv 1\pmod{4}],f(ab)=f(a)f(b)$。只令
	$f(x)=[p\equiv 1\pmod{4}]$是错误的,因为两个在模4余3的剩余系的元素之积在余1
	剩余系,这样设计积性函数仅会漏统计这种情况。也就是说要同时统计余1和余3,但是必须
	将他们区分。注意积性函数的值域是复数域,那么可以将积性函数值表示为$a+bi$的形式,
	当$x\equiv 1\pmod{4}$时值为1,当$x\equiv 1\pmod{4}$时值为$i$,其余情况为0。
	再讨论$f(ab)=f(a)f(b)$的性质,当取$i^2=1$时$f(x)$恰好完全积性(不管$i$取什么
	值都可以构成群,同样满足积性函数性质)。最后套min\_25筛解决。事实上还可以将复数
	扩展为多个基。
	\item
	\begin{displaymath}
		\sum_{d=1}^n{\sum_{k=1}^{\frac{n}{k}}{F(k)G(\frac{n}{kd})}}\Rightarrow kd=q \Rightarrow \sum_{q=1}^n{G(q)\sum_{k|q}F(k)}
	\end{displaymath}
	\item SPOJ DIVCNT2:

	记$\omega(d)$为$d$的质因子个数。
	\begin{displaymath}
		\tau(n^2)=\sum_{d|n}{2^{\omega(d)}}
	\end{displaymath}

	证明:考虑$n^2$有而$n$没有的因子,这些因子必定满足其质因子分解中某个质因子指数
	超过$n$的对应指数。对于$n$的某个因子$d$,其质因子次数加上$n$的对应次数,就可以
	成为$n^2$的独有因子,方案数为$2^{\omega(d)}$,且这种计数方法不会重复或遗漏。

	\begin{displaymath}
		2^{\omega(n)}=\sum_{d|n}{|\mu(d)|}=\sum_{d|n}{\mu^2(d)}
	\end{displaymath}

	证明:$n$的所有无平方因子的因子都可以表示为一个长度为$\omega(n)$的01向量,表示选
	/不选这个质因子。枚举所有这种因子等价于枚举所有向量,其集合大小为$2^{\omega(n)}$。

	\begin{displaymath}
		\mu^2(i)=\sum_{j^2|i}{\mu(j)}
	\end{displaymath}

	证明:左式的意义是$i$是否含有非平凡平方因子。若$i$不含平方因子,右式的值
	为$\mu(1)=1$。若$i$含有平方因子,记$i=a^2b$,$b$不含平方因子,那么右式变形为
	$\displaystyle\sum_{j|a}{\mu(j)}$,因为$a\neq 1$,所以该式的值为0。

	这些方法参考了Candy?的博客\footnote{
		SPOJ DIVCNT2 [我也不知道是什么分类了反正是数论]\\
		\url{https://www.cnblogs.com/candy99/p/6715013.html}
	}。
	\item SDOI2018 旧试题:
	\begin{eqnarray*}
		\tau(ij)&=&\sum_{x|i}{\sum_{y|j}{[(x,y)=1]}}\\
		\tau(ijk)&=&\sum_{x|i}{\sum_{y|j}{\sum_{z|k}{[(x,y)=1][(x,z)=1][(y,z)=1]}}}
	\end{eqnarray*}

	第一个式子证明:考虑枚举$i$的因子$x$,枚举$j$的因子$y$,这样$xy$可以组成$ij$的因子。
	但是$xy$可能会有重复,强制$z=xy$只能被最小的$x$枚举,那么对于更大的$x'|z,x|x'$,
	对应的$y'$与$y$相比少了因子$\frac{x'}{x}$,注意到$\frac{j}{y'}$多了因子
	$\frac{x'}{x}$,那么$\frac{j}{y'}$与$x'$不互质。而对于最小的$x$,若它与
	$\frac{j}{y}$不互质,则令$x'=\frac{x}{(x,\frac{j}{y})}$可以得到更小的方案,
	与$x$最小矛盾。那么答案为
	$\displaystyle \sum_{x|i}{\sum_{y|j}{[(x,\frac{j}{y})=1]}}$,变形后得到原等式。
	\item 尽可能早地找到递归结构
	\item $[(ab,c)=1]=[(a,c)=1][(b,c)=1]$
\end{itemize}

更多技巧待补充。

\section{低于线性时间复杂度筛法}
这类筛法(杜教筛)主要用于计算大数据规模积性函数求和。
\subsection{约数函数前缀和}
求$\displaystyle \sum_{i=1}^n{\sigma(i)},n\leq 10^{12}$。
\begin{eqnarray*}
    \sum_{i=1}^n{\sigma(i)}&=&\sum_{i=1}^n{\sum_{d|i}d}\\
    &=&\sum_{d=1}^n{d[\frac{n}{d}]}
\end{eqnarray*}
由于$[\frac{n}{d}]$存在许多连续相同的值,使用整除分块法可做到$O(\sqrt{n})$。
\subsection{欧拉函数前缀和}
求$\displaystyle \sum_{i=1}^n{\varphi(i)},n\leq 10^{11}$。
由定理~\ref{PhiT}可得
$\displaystyle \varphi(n)=n-\sum_{d|n,d<n}{\varphi(d)}$。
\begin{eqnarray*}
    ans(n)&=&\sum_{i=1}^n{\varphi(i)}\\
    &=&\sum_{i=1}^n{\left(i-\sum_{d|i,d<i}{\varphi(d)}\right)}\\
    &=&\frac{n(n+1)}{2}-\sum_{i=2}^{n}{\sum_{d|i,d<i}{\varphi(d)}}\\
    &=&\frac{n(n+1)}{2}-\sum_{\frac{i}{d}=2}^n
    {\sum_{d=1}^{[\frac{n}{\frac{i}{d}}]}{\varphi(d)}}\\
    &=&\frac{n(n+1)}{2}-\sum_{t=2}^n
    {\sum_{d=1}^{[\frac{n}{t}]}{\varphi(d)}}\\
    &=&\frac{n(n+1)}{2}-\sum_{t=2}^n{ans([\frac{n}{t}])}
\end{eqnarray*}
同理使用分块+递归询问区间和来计算答案。为了降低复杂度,应该先线性筛预处理前一部分值。
当预处理$k=n^\frac{2}{3}$时可以取到复杂度$T(n)=O(n^\frac{2}{3})$。
\subsection{莫比乌斯函数前缀和}
求$\displaystyle \sum_{i=1}^n{\mu(i)},n\leq 10^{11}$。
由定理~\ref{MobiusT}可得
$\displaystyle \mu(n)=[n=1]-\sum_{d|n,d<n}{\mu(d)}$。
\begin{eqnarray*}
    ans(n)&=&\sum_{i=1}^n{\mu(i)}\\
    &=&\sum_{i=1}^n{\left([i=1]-\sum_{d|i,d<i}{\mu(d)}\right)}\\
    &=&1-\sum_{i=1}^n{\sum_{d|i,d<i}{\mu(d)}}\\
    &=&1-\sum_{t=2}^n{ans([\frac{n}{t}])}
\end{eqnarray*}
\subsection{其它函数前缀和}
主要思路是使用狄利克雷卷积构造出一个简单的前缀和函数,且用于卷积的另一个函数也容易计算。

令$\displaystyle A(n)=\sum_{i=1}^n\frac{i}{(n,i)}$,求
$\displaystyle \sum_{i=1}^n{A(n)},n\leq 10^{9}$。

先化简$A(n)$:
\begin{eqnarray*}
    A(n)&=&\sum_{i=1}^n\frac{i}{(n,i)}\\
    &=&\sum_{d|n}{\sum_{i=1}^n{[(n,i)=d]\cdot\frac{i}{d}}}\\
    &=&\sum_{d|n}{\sum_{\frac{i}{d}=1}^{\frac{n}{d}}
    {[(\frac{n}{d},\frac{i}{d})=1]\cdot\frac{i}{d}}}\\
    &=&\frac{1}{2}\left(1+\sum_{d|n}{d\cdot\varphi(d)}\right)
\end{eqnarray*}

那么答案即为$\displaystyle \frac{1}{2}\left(n+\sum_{t=1}^n
    {\sum_{d=1}^{[\frac{n}{t}]}{d\cdot\varphi(d)}}\right)$。

考虑计算$\displaystyle \sum_{d=1}^n{d\cdot\varphi(d)}$的值:

易知$(id\cdot\varphi)*id=id^2$,因为\begin{displaymath}
    \sum_{d|n}d\cdot\varphi(d)\cdot\frac{n}{d}=
    n\cdot\sum_{d|n}\varphi(d)=n^2
\end{displaymath}

所以有\begin{eqnarray*}
    \frac{n(n+1)(2n+1)}{6}&=&\sum_{i=1}^n{(id\cdot\varphi)*id}\\
    &=&\sum_{t=1}^n{t\cdot\sum_{d=1}^{[\frac{n}{t}]}{d\cdot\varphi(d)}}
\end{eqnarray*}

\subsection{复杂度分析}
为了得到较优的复杂度,需要设置合适的预处理大小。
\index{*TODO!筛法复杂度分析}

以上例题来自skywalkert的博客\footnote{浅谈一类积性函数的前缀和\\
    \url{https://blog.csdn.net/skywalkert/article/details/50500009}}。

\subsection{min\_25筛}
这里求和的积性函数$F$满足$F(p)$是一个关于$p$的低阶多项式且能够快速求出$F(p^k)$。
据说min\_25筛踩爆洲阁筛,那我就不学洲阁筛了。在此附上洲阁筛教程\footnote{
    洲阁筛学习 | \_\_debug's Home\\
    \url{http://debug18.com/posts/calculate-the-sum-of-multiplicative-function/}
}。

\subsubsection{预处理}
首先考虑求$\displaystyle \sum_{p\leq n}{F(p)}$。

记$g(n,j)$为满足$x$为$n$以内素数,或者$x$的最小质因子$>p_j$的$F(x)$之和,
所求值即为$g(n,|P|)$。考虑$g(n,j)$如何从$g(n,j-1)$转移。易知最小质因子为
$p_j$的合数为$p_j^2$,若其$>n$,则$g(n,j)$与$g(n,j-1)$都只求素数的积性函
数值之和,所以$g(n,j)=g(n,j-1)$。若$p_j^2\leq n$,则转移时会损失掉一些
$F(x)$,满足$x$的最小质因子为$p_j$。考虑提出$x$的$p_j$,满足$\frac{x}{p_j}$
的最小质因子$\geq p_j$,计算$\frac{x}{p_j}$的积性函数和,发现$g(\frac{n}{p_j},j-1)$
包括了它们,又因为$\frac{n}{p_j}\geq p_j > p_{j-1}$,所以要扣除
$\displaystyle \sum_{p<p_j}F(p)$。{\bfseries 由于积性函数$F$的特殊性,
把不同次数的项拆开算,单项为完全积性函数,乘上$F(p_j)$即为需要减去的值。}

综上,有\begin{displaymath}
    g(n,j)=\left\{\begin{array}{lr}
        g(n,j-1)        & p_j^2>n   \\
        g(n,j-1)-F(p_j)(g(\frac{n}{p_j},j-1)-\displaystyle \sum_{p<p_j}{F(p)}) & p_j^2\leq n \\
    \end{array}\right.
\end{displaymath}

预处理素数时只需要筛$\sqrt{n}$内的素数,边界$g(n,0)$是所有数按照素数的处理方式
计算的值之和,由于最后只需要$g(n,|P|)$,无需考虑$g(n,0)$的意义。

实质上$g(n,j)$就是埃氏筛法筛完$p_j$后未被筛的合数以及素数的积性函数值之和。

接下来尝试求出所有的$g(x,|P|),x=\lfloor \frac{n}{i}\rfloor$。
这里有一个存储上的trick:由于$\lfloor \frac{n}{i}\rfloor$有连续重复项,
最多$2\sqrt{n}$个,对于$x=\lfloor \frac{n}{i}\rfloor>\sqrt{n}$,把它
映射到$\lfloor \frac{n}{x}\rfloor$上存储,这样保证了空间复杂度为$O(\sqrt{n})$。

由于最后只要$g(x,|P|)$,$g$数组只要开1维滚动更新。

伪代码如下:
\begin{lstlisting}
int g[2][sqsiz],q[2*sqsiz];
int& getG(int x) {
    if(x<=sqr) return g[0][x];
    return g[1][n/x];
}
void calcG() {
    int m=0,i=1;
    while(i<=n) {
        int val=n/i;
        q[++m]=val;
        getG(val)=f(val);
        i=n/val+1;
    }
    for(int i=1;i<=psiz;++i) {
        int cp=p[i],cp2=cp*cp;
        for(int j=1;j<=m && cp2<=q[j];++j) {
            int k=q[j],&val=getG(k);
            val=sub(val,mul(f(cp),getG(k/cp)-sumf[i-1]));
        }
    }
}
\end{lstlisting}
\subsubsection{求和}
记$S(n,j)$为$n$以内最小质因子大于等于$p_j$的积性函数值和。
所求答案即为$S(n,1)+f(1)$。

把$S(n,j)$分为素数和合数求解:
\begin{itemize}
    \item 对于素数部分,$g(n,|P|)$代表了素数积性函数值和,再扣去不满足
    最小质因子要求的素数,最终贡献为$g(n,|P|)-\displaystyle \sum_{p<p_j}F(p)$。
    \item 对于合数部分,枚举其最小质因子$p_k$及其幂次$c$,单独贡献为\\
    $F(p_k^c)S(\frac{n}{p_k^c},k+1)+F(p_k^{c+1})$。注意此处的$F\cdot S$直接利用
    了积性函数的定义,因为$S$部分无$p_k$因子。由于$S$不处理$n=1$的部分,需要另外加上
    $F(p_k^{c+1})$。
\end{itemize}

递归的边界条件为$n\leq 1 \vee n<p_j$,无需记忆化。

时间复杂度为$O(\frac{n^\frac{3}{4}}{\lg n})$,空间复杂度为$O(\sqrt{n})$。

模板(LOJ\#6053. 简单的函数):
\lstinputlisting{Source/Templates/min_25.cpp}

上述内容参考了小蒟蒻yyb\footnote{
    min\_25筛
    \url{https://www.cnblogs.com/cjyyb/p/9185093.html}
}和租酥雨\footnote{
    Min\_25 筛
    \url{https://www.cnblogs.com/zhoushuyu/p/9187319.html}
}的博客。
\subsection{Powerful~Number}
定义Powerful~Number为所有质因子的指数都$\geq 2$的数,那么每个Powerful~Number都可以被表示为
$a^2b^3$的形式(若指数为奇数则分配一个立方给$b$,其余分给$a$)。{\bfseries 注意1也是Powerful~Number。}

\begin{theorem}
    $n$以内的Powerful~Number个数为$O(\sqrt{n})$。
\end{theorem}

证明:枚举$a$,将$b$的个数累积,可得式子
\begin{displaymath}
    \sum_{i=1}^{\lfloor\sqrt{n}\rfloor}{\lfloor\sqrt[3]{\frac{n}{i^2}}\rfloor}
\end{displaymath}

使用积分近似求出其上界为$\int_1^{\sqrt{n}+1}{\sqrt[3]{\frac{n}{(x-1)^2}} \ud x}=O(\sqrt{n})$。
根据Wikipedia-EN\footnote{Powerful~number
    \url{https://en.wikipedia.org/wiki/Powerful\_number}}的描述,其上界常数为
    $\frac{\zeta(\frac{3}{2})}{\zeta{3}}\approx 2.173$。

对于某个复杂的积性函数$f(x)$,若$f(p^e)$易于计算且存在一个简单(易于求$g(p^e)$与前缀和)的积性函数
$g(x)$,满足对于所有素数$p$,有$f(p)=g(p)$,称函数$g$拟合了函数$f$。

设$h=f/g$,这里的除法是狄利克雷除法,等价于$h=f*g^{-1}$。由于狄利克雷逆$g^{-1}$是积性函数,
狄利克雷卷积$h$也是积性函数。那么对于所有素数$p$,有$f(p)=h(1)g(p)+h(p)g(1)$,
由于$h(1)=1,f(p)=g(p)$,可得$h(p)=0$。由于$h(x)$是积性函数,$h(x)$可能非0当且仅当$x$是
Powerful~Number。

现在要求$\displaystyle Ans=\sum_{i=1}^n{f(n)}$,由于$f=h*g$,有
$\displaystyle Ans=\sum_{ab\leq n}{h(a)g(b)}$。由上文的推导可知$h(x)$仅在Powerful~Number处
有贡献,且Powerful~Number的个数是$O(\sqrt{n})$的,可以$O(\sqrt{n})$DFS暴力枚举质因子组合得到$a$。
记$n$以内的Powerful~Number组成的集合为$S$,原式变为
$\displaystyle Ans=\sum_{a\in S}{h(a)\sum_{b=1}^{\lfloor \frac{n}{a} \rfloor}{g(b)}}$。
易求$g(x)$的前缀和,问题在于如何快速推得$h(a)$的值。

由于$h(x)$是积性函数且$a$是Powerful~Number,在DFS时仅需计算$h(p^e),e>1$的值。
使用$f(p^e)$展开式,快速求得$f(p^e)$与$g(p^e)$,再根据历史信息$h(p^e'),e<e'$,
可以快速得到$h(p^e)$。如果$g(x)$是完全积性函数,可以对$f(p^e)$展开式平移得到$f(p^(e+1))$的展开式。
由于$e$很小,$h(p^e)$的求值不是瓶颈。不过预处理可以节省DFS时的重复计算。

DFS递归时会遇到大量的0次项,这些不必要的递归会导致实际运行缓慢。可以DFS钦定一些质因子必选,使得每层DFS
都对最终的$a$有贡献,杜绝爆栈。

具体实现参考Project~Euler~484:
\lstinputlisting{Source/Source/'Number Theory'/PE484.cpp}

{\bfseries 记得要计算$a=1$时的贡献!!!}
 
上述内容参考了fjzzq2002的博客\footnote{
利用powerful~number求积性函数前缀和
    \url{https://www.cnblogs.com/zzqsblog/p/9904271.html}
}。

Min_25使用Powerful Number得到新的做法,参见Sum~of~Multiplicative~Function~on~Powerful~Numbers
(\url{https://min-25.hatenablog.com/entry/2018/11/11/172228})。

\section{离散对数问题}
\index{D!Discrete Logarithm}
离散对数问题解决的是形如$a^x\equiv b\pmod{p}$的问题。
\subsection{BSGS}\label{BSGS}
\index{B!Baby Step Giant Step}
\subsubsection{BSGS}
普通BSGS仅考虑$P$为素数的情况。

以下为求解最小非负整数解的方法:

首先根据定理~\ref{ET}可知$x$的最小非负整数解小于$\varphi(P)=P-1$。将x表示为
$\sqrt{P}$进制数,分别用$O(\sqrt{P})$的复杂度枚举值,一半存入HashTable,另一半
查询是否有匹配值。注意参数的枚举顺序。

\begin{enumerate}
    \item 若$a$为$P$的倍数,则特判$b$是否为$0$,算法结束;
    \item 令$m=\lceil\sqrt{P}\rceil,x=im-j$,移项得$a^{im}\equiv ba^j\pmod{P}$;
    \item 枚举$ba^j$的值,按$j$从小到大{\bfseries 覆盖}存入HashTable;
    \item 枚举$(a^m)^i$的值,按$i$从小到大在HashTable中查询,存在则返回$im-j$;
    \item 返回无解。
\end{enumerate}

\subsubsection{ExBSGS}
ExBSGS可解决$a,P$不互质的问题。主要思路是将原方程化为普通BSGS可解决的方程。

记化简后方程为$Aa^{x-B}\equiv b\pmod{P}$,化简步骤如下:
\begin{enumerate}
    \item 将$A,B$初始化为$1,0$;
    \item 令$d=(a,P)$,
    \begin{itemize}
        \item 若$d\mid b$,则提出一个因子$d$,即$A*=a/d,b/=d,P/=d,++B$;
        \item 若$d\nmid b$,则特判$b$是否为$A$,$b=A$则$x=B$,$b\neq A$则无解;
    \end{itemize}
    \item 重复第2步直至$d=1$。
\end{enumerate}
令$x=im-j+B$转化为普通BSGS,{\bfseries 注意在BSGS前要暴力检查$x\in[0,B)$是否可行}。

代码如下:
\lstinputlisting{Source/Templates/exBSGS.cpp}

以上内容参考了ZigZagK的博客\footnote{BSGS及扩展BSGS\\
\url{https://blog.csdn.net/zzkksunboy/article/details/73162229}}。
\subsection{单有原根模数多询问离散对数问题}
对于这类问题,每次都计算一次HashTable十分浪费。
考虑求出原根$g$,令$a=g^A,b=g^B$,原式转化为$Ax\equiv B\pmod{\varphi(n)}$,
使用exgcd快速解决。由于求$a,b$的对数时的底数固定为$g$,可以预处理$g$的幂,也可以
预处理$g$的BSGS哈希表(哈希表的大小不必开根号,可以根据模数与询问规模决定)。

若模数$p$非常大,参见LOJ6542\url{https://loj.ac/problem/6542},留坑待补。

\index{*TODO!单模数多询问离散对数}
\subsection{Pollard rho算法}
\index{P!Pollard's rho Algorithm for Logarithms}

该算法的主要思路在于找到两组指数$(y_1,y_2),(z_1,z_2)$,使得
$a^{y_1}b^{y_2}\equiv a^{z_1}b^{z_2}$。然后将其转化为求解模线性方程组
$(z_2-y_2)x\equiv y_1-z_1\pmod{N}$,这里的$N$是循环群的阶。exgcd轻松解决,
不过需要注意的是要检查每个周期的解的合法性(最小非负整数解不一定是可行解)。

若模数有原根且不知道循环群的阶,可以求出$n$的原根$g$,那么阶$N=\varphi(n)$。

找指数的过程类似于因子分解的PollardRho算法,即使用随机函数来生成两组指数,
同时使用Folyd判圈法终止算法。

具体做法是将当前乘积$v$的集合划分到3个集合,不同集合的行为不同,分别为$a$的指数+1,$b$
的指数+1,$a,b$的指数翻倍。

该算法的期望复杂度是$O(\sqrt{n})$,代码量小,但是慢于BSGS。不过$O(1)$的空间使得它
在求解$n$较大的情况时较BSGS有绝对优势。

代码如下:
\lstinputlisting{Source/Templates/PollardRhoDL.cpp}

上述内容参考了Wikipedia-EN\footnote{
    Pollard's rho algorithm for logarithms
    \url{https://en.wikipedia.org/wiki/Pollard's\_rho\_algorithm\_for\_logarithms}
}。
\subsection{Pohlig–Hellman算法}
\index{P!Pohlig–Hellman Algorithm}
该算法的思路基于群论,需要知道循环群的阶。如果模数有原根,仍然可以使用上面的做法。
记循环群的阶$n=\displaystyle \prod{p_i^{e^i}}$,该算法可以将复杂度降到
$O(\displaystyle \sum{e_i(\lg n+\sqrt{p_i})})$,空间复杂度$O(\sqrt{p_{max}})$。

\subsubsection{阶为素数幂的情况}

记素数幂为$p^e$,算法步骤如下:

\begin{enumerate}
    \item 初始化$x_0=0$。
    \item 计算$G=g^{p^{e-1}}$,根据定理~\ref{LT},生成元$G$对应的循环群的阶为$p$。
    \item 对于$k=0,1,\cdots,e-1$,计算$b_k=(g^{-x_k}b)^{p^{e-1-k}}$,BSGS求解
    $G^{d_k}\equiv b_k\pmod{p}$,最后累加$x_{k+1}=x_k+p^kd_k$。
    \item 返回$x_e$。
\end{enumerate}

算法正确性留坑待补。
\index{*TODO!Pohlig-Hellman算法正确性证明}

\subsubsection{一般情况}
首先将阶$n$因式分解,求出模数的原根$g$。然后逐个计算$g_i=g^{n/p_i^{e_i}}$,构造出
阶为$p^e$形式的群,对应地$b_i=b^{n/p_i^{e_i}}$,使用上述方法求出
$g^{x_i}\equiv b_i\pmod{p_i^{e^i}}$的解,最后使用CRT合并答案。

代码如下(针对单模数多询问重新平衡,以空间换时间):
\lstinputlisting{Source/Templates/PohligHellman.cpp}

上述内容参考了Wikipedia-EN\footnote{
    Pohlig–Hellman algorithm
    \url{https://en.wikipedia.org/wiki/Pohlig-Hellman\_algorithm}
}。

\section{原根}\label{PrimitiveRoot}
\index{P!Primitive Root}
\subsection{基本定义与定理}
\subsubsection{数论阶}
设$n>1,(a,n)=1$,记$\delta_n(a)$为使得$a^r\equiv 1 \pmod{n}$
成立的最小正整数$r$,称其为$a$模$n$的阶。

\begin{theorem}
	设$n>1,(a,n)=1$,若$a^x\equiv 1 \pmod{n}$,则有$\delta_n(a)\mid x$。
\end{theorem}

\subsubsection{原根}
若$\delta_n(a)=\varphi(n)$,则称$a$为模$n$的一个原根。

若$a$为模$n$的原根,根据定理~\ref{ET},对于$0\leq i< \varphi(n)$,$a^i\bmod{n}$两两不同。

\begin{theorem}
	如果模$n$有原根,则它一共有$\varphi(\varphi(n))$个原根。
\end{theorem}

\begin{theorem}
	$n=2,4,p^i,2p^i\Leftrightarrow$模$n$有原根,其中$p$为奇素数。
\end{theorem}

\subsection{求模n的原根}

对$\varphi(n)$进行质因数分解,
对于$\displaystyle \varphi(n)=\prod_{i=1}^m{p_i^{c_i}}$,若
恒有$g^\frac{\varphi(n)}{p_i}\not\equiv 1 \pmod{n}$,则$g$为模$n$的原根。

以上内容参考了mosquito\_zm的博客\footnote{原根
	\url{https://blog.csdn.net/mosquito\_zm/article/details/77227570}}。
\subsection{原根的应用}
\begin{itemize}
	\item 在NTT中用于推算主单位根。
	\item 将乘积恒定转换为幂次和恒定后NTT。
	\paragraph{例题} [SDOI2015]序列统计\footnote{【P3321】[SDOI2015]序列统计 - 洛谷
	\url{https://www.luogu.org/problemnew/show/P3321}}

	将$x$映射为$g^i$,使用生成函数推导出多项式幂的形式,最后对答案进行逆映射。

\end{itemize}

\lstinputlisting[title=luogu P3321]{Source/Source/'FFT NTT'/3321.cpp}

\section{二次剩余与三次剩余}
留坑待补。
\subsection{二次剩余}
\subsection{三次剩余}

\section{类欧几里得算法}
已知$k_1,k_2,a,b,c,n$,求
\begin{displaymath}
    F(k_1,k_2,a,b,c,n)=\sum_{x=0}^n{x^{k_1}
    {\left \lfloor\frac{ax+b}{c}\right \rfloor}^{k_2}}
\end{displaymath}

其中$k_1,k_2$较小。

以下除法操作均指向下取整除法,$\lambda$为易求得的常数。讨论多种情况的处理方法:
\begin{itemize}
    \item 若$an+b<c \wedge k_2\neq 0$,直接返回0。
    \item 若$k_2=0 \vee a=0$,答案为$\lambda \displaystyle \sum_{x=0}^n{x^{k_1}}$。
    利用~\ref{psum}所述方法求解。
    \item $a\geq c \vee b\geq c$:

    令$\displaystyle F(k_1,k_2,a,b,c,n)=\\\sum_{x=0}^n{x^{k_1}
    ((a/c)x+(b/c)+((a \bmod c) \cdot x+(b\bmod c))/c)^{k_2}}$,
    对指数为$k_2$的部分进行多项式展开,可以拆出类似$\sum{\lambda F}$的形式,
    递归求解。

    \item 否则$a<c \wedge b<c \wedge k_2 \neq 0$,
    使用类似求期望的技巧,枚举指数为$k_2$的幂并差分,记$m=(an+b)/c$,
    有\begin{eqnarray*}
        F(k_1,k_2,a,b,c,n)&=&\sum_{i=0}^{m-1}{\left(
            ((i+1)^{k_2}-i^{k_2})\sum_{x=0}^n{
                x^{k_1}[(ax+b)/c>i]
            }
        \right)}\\
        &=&\sum_{i=0}^{m-1}{\left(
            ((i+1)^{k_2}-i^{k_2})\sum_{x=0}^n{
                x^{k_1}[x>(ic+c-b-1)/a]
            }
        \right)}\\
        &=&\sum_{i=0}^{m-1}{((i+1)^{k_2}-i^{k_2})}
        \cdot \sum_{x=0}^n{x^{k_1}}\\
        & &-\sum_{i=0}^{m-1}{\left(
            ((i+1)^{k_2}-i^{k_2})\sum_{x=0}^{(ic+c-b-1)/a}{
                x^{k_1}
            }
        \right)}
    \end{eqnarray*}
    前一部分可以插值求出,后一部分把多项式$\displaystyle \sum_{i=0}^n{i^k}=
    \sum_{i=0}^{k+1}{c_i\cdot n^i}$
    展开,化为$\sum{\lambda F}$的形式递归求解。
\end{itemize}

实现时可以令函数返回$g(n,a,b,c)=R^{k_1\cdot k_2}$存储所有需要的值,避免重复计算。
递归层数为$O(\lg n)$。

模板(LOJ\#138):
\lstinputlisting{Source/Templates/Euclidlike.cpp}

上述内容参考了fjzzq2002的博客\footnote{
    类欧几里得算法
    \url{https://www.cnblogs.com/zzqsblog/p/8904010.html}
}。

WC2019营员交流中有营员(忘了是谁)提出了``万能欧几里得''算法,可以简单解决类欧几里得问题。


\chapter{集合论~群论}
\section{集合论定理}
\begin{theorem}
	\begin{displaymath}
		A\oplus B=A\cup B - A\cap B
	\end{displaymath}
\end{theorem}
\index{D!De Morgan's Laws}
\begin{theorem}[De Morgan's Laws]\label{DML}
	\begin{eqnarray*}
		\overline{A\cup B}=\overline{A}\cap \overline{B} \\
		\overline{A\cap B}=\overline{A}\cup \overline{B}
	\end{eqnarray*}
\end{theorem}
\index{I!Inclusion–exclusion Principle}
\begin{theorem}[Inclusion–exclusion Principle]\label{IEP}
	\begin{displaymath}
		\left|\bigcup_{i=1}^n{A_i}\right|=
		\sum_{\emptyset \neq J\subseteq \{1,2,\ldots,n\}}{(-1)^{|J|-1}
			\left|\bigcap_{j\in J}{A_j}\right|}
	\end{displaymath}
\end{theorem}

由定理~\ref{DML}与~\ref{IEP}可得

\begin{theorem}
	\begin{displaymath}
		\left|\bigcap_{i=1}^n\overline{A_i}\right|=
		\left|\overline{\bigcup_{i=1}^n{A_i}}\right|=
		|U|+\sum_{\emptyset \neq J\subseteq \{1,2,\ldots,n\}}{(-1)^{|J|}
			\left|\bigcap_{j\in J}{A_j}\right|}
	\end{displaymath}
\end{theorem}

以上内容参考了Wikipedia-EN\footnote{Inclusion–exclusion principle - Wikipedia
	\url{https://en.wikipedia.org/wiki/Inclusion\%E2\%80\%93exclusion\_principle}

	De Morgan's laws - Wikipedia
    \url{https://en.wikipedia.org/wiki/De\_Morgan\%27s\_laws}}。

\section{拉格朗日定理}
\index{L!Lagrange's Theorem}
\begin{theorem}[Lagrange's Theorem]\label{LT}
	若$(S,\oplus)$是一个有限群,$(S',\oplus)$是$(S,\oplus)$的子群,则
	$|S'|$是$|S|$的约数。
\end{theorem}
证明留坑待补。
\index{*TODO!拉格朗日定理证明}

\section{置换群}
\index{P!Permutation Groups}
{\bfseries 置换}是从$[1,n]$到$[1,n]$的一一映射。

置换可以分解为多个循环,计算循环相关数据的方法为:枚举每一个节点
\begin{enumerate}
    \item 若该节点已被访问,则跳过;
    \item 顺着该节点对应的目标节点不断跳跃,标记已访问,直至跳跃到已访问点(即出发点)为止。
    \item 这个环就是一个循环。
\end{enumerate}
\begin{theorem}
    若对于一个置换有$n$个循环,长度分别为$l_1,l_2,\ldots,l_n$,
    则该置换的循环节长度为$lcm(l_1,l_2,\ldots,l_n)$。
\end{theorem}
\paragraph{不动点}
若一个状态$S$经由置换$f$置换后的状态与原状态相同,则状态$S$为$f$的不动点。
\paragraph{等价关系}
对于一个置换集合$F$,若状态$S$能经由$F$中的置换变为状态$S'$,则称$S$与$S'$等价。
\paragraph{等价类}
满足等价关系的状态属于同一等价类。

\subsubsection{Burnside引理}
\index{B!Burnside's Lemma}
\begin{lemma}[Burnside's Lemma]
    等价类数目为置换群$G$中所有置换的不动点数目的平均值。
    \begin{displaymath}
        |X/G|=\frac{1}{|G|}\sum_{g\in G}|X^g|
    \end{displaymath}
\end{lemma}
上述定理证明留坑待补。
\index{*TODO!证明Burnside引理}
\subsubsection{Polya定理}
\index{P!Pólya Enumeration Theorem}

\begin{theorem}[Pólya Enumeration Theorem]
    若对每一个节点进行$m$染色,置换$g$有$c(g)$个循环,则染色方案
    等价类数目为$\displaystyle \frac{1}{|G|}\sum_{g\in G}m^{c(g)}$。
\end{theorem}

证明:一个循环内所有的节点颜色相同,不同循环颜色的选择是独立的,每一个循环颜色选择
方案对应一个不动点,根据乘法原理可知$|X^g|=m^{c(g)}$。

以上内容参考了QAQqwe的博客\footnote{Burnside引理与Polya定理
\url{https://blog.csdn.net/liangzhaoyang1/article/details/72639208}}与
Wikipedia-EN\footnote{
    Burnside's lemma - Wikipedia
    \url{https://en.wikipedia.org/wiki/Burnside\%27s\_lemma}

    Pólya enumeration theorem - Wikipedia
    \url{https://en.wikipedia.org/wiki/P\%C3\%B3lya\_enumeration\_theorem}
}。

\subsubsection{常见题型}
题型来自My\_ACM\_Dream的博客\footnote{polya|burnside定理的一些总结\\
\url{https://blog.csdn.net/My\_ACM\_Dream/article/details/45312365}}。

\paragraph{正方形旋转}
n*n正方形染色:
\begin{itemize}
    \item 旋转0度,循环节数$n^2$。
    \item 旋转90/270度,若$n$为偶数,循环节数$\frac{n^2}{4}$;若$n$为奇数,
    循环节数$\frac{n^2-1}{4}+1$。
    \item 旋转180度,若$n$为偶数,循环节数$\frac{n^2}{2}$;若$n$为奇数,循环
    节数$\frac{n^2-1}{2}+1$。
\end{itemize}
奇偶循环节数不同的原因是因为$n$为奇数时中间的点自成一个循环节。
\paragraph{环形旋转}
\paragraph{环形翻转}
\paragraph{正方体旋转}


\chapter{组合数学}
\section{Catalan数}
\subsection{性质}
\index{C!Catalan Numbers}
Catalan数是组合数学中的常见数列
\footnote{A000108 - OEIS \url{http://oeis.org/A000108}},其前几项为
\begin{displaymath}
	1, 1, 2, 5, 14, 42, 132, 429, 1430, \ldots
\end{displaymath}

Catalan数(记为$C_n$)满足如下关系:
\begin{eqnarray}
	C_0&=&C_1=1\\
	C_{n+1}&=&\sum_{i=0}^n{C_iC_{n-i}}\label{CT2}\\
	&=&\sum_{i=1}^n{C_iC_{n+1-i}}\label{CT3}\\
	C_n&=&\frac{4n-2}{n+1}C_{n-1}\\
	C_n&=&{2n \choose n}-{2n \choose n+1}=\frac{1}{n+1}{2n \choose n}\\
	C_n&=&\prod_{k=2}^n\frac{n+k}{k}
\end{eqnarray}
根据Striling近似公式
\begin{displaymath}
	n!\sim\sqrt{2\pi n}\left(\frac{n}{e}\right)^n
\end{displaymath}
可得
\begin{displaymath}
	C_n\sim\frac{4^n}{\sqrt{\pi} n^\frac{3}{2}}
\end{displaymath}
\subsection{常见应用}
\subsubsection{括号序列,出栈序列,网格行走}
\paragraph{括号序列} 给定$2n$个位置填上左右括号使括号匹配(对于每一个位置之前的
左括号必须不少于右括号)。
\paragraph{出栈序列} 求将$n$个元素入栈一次(限制顺序)并出栈一次(不限制顺序)
的方案数(对于每一次操作都要保证栈不出现下溢,即入栈元素不少于出栈元素)。
\paragraph{网格行走} 在一个$n*n$的网格内从左下角移动到右上角,纵坐标必须不少于
横坐标,求方案数。
\paragraph{分析}
这三个问题是同构的,都满足操作数为$2n$且限制任意时刻操作A的数目不少于操作B的数目。
它们的答案都是$C_n$,以括号序列问题为例,通过等式~\ref{CT2}理解:
将括号序列看做由一个可分割的序列加上一个不可分割的序列(即最外层有一对配对括号)得来,
左边为$n_1+1$对,右边为$n_2$对,满足$n_1+n_2=n-1$,这种方案的贡献为
$C_{n_1}C_{n_2}$。
\subsubsection{二叉树构型计数}
\paragraph{有$n$个节点的二叉树}
通过等式~\ref{CT2}理解:枚举左右子树大小,满足左右子树节点数为$n-1$。
\paragraph{有$n+1$个叶子节点的满二叉树}
通过等式~\ref{CT3}理解:枚举左右子树叶子节点数,满足其和为$n+1$。
\subsubsection{阶梯填充}
用$n$个长方形填充$n*n$的阶梯的方案数为$C_n$。

不严格证明:填充一个以直角顶点与阶梯顶点为对顶点的长方形,使其分为大小为
$n_1*n_1,n_2*n_2$的两个小阶梯,满足$n_1+n_2=n-1$,分别分配$n_1,n_2$
个长方形的份额,就成为子问题了。该分析满足等式\ref{CT2}。
\subsubsection{凸包分割}
将$n+2$个顶点的凸包分为三角形的方案数为$C_n$。

猜想:最终将分为$n$个三角形。

证明留坑待补。
\subsubsection{圆上点连线}
将圆上的$2n$个点两两配对连线,所连$n$条线段不相交的方案数为$C_n$。

证明留坑待补。

\index{*TODO:Catalan应用证明}
上述内容参考了Wikipedia-EN\footnote{Catalan number - Wikipedia
	\url{https://en.wikipedia.org/wiki/Catalan\_number}}。

\section{Stirling数}

\section{Lucas/ExLucas}
\index{L!Lucas's Theorem}
\subsection{Lucas定理}
\begin{theorem}[Lucas's Theorem]
	对于非负整数$n,m$以及质数$p$,若
	\begin{eqnarray*}
		n&=&\sum_{i=0}^k{n_ip^i}\\
		m&=&\sum_{i=0}^k{m_ip^i}
	\end{eqnarray*}
	则
	\begin{displaymath}
		\binomial{n}{m}\equiv\prod_{i=0}^k\binomial{n_i}{m_i} \pmod{p}
	\end{displaymath}
\end{theorem}
Nathan Fine的证明:
对于质数$p$与整数$n$满足$1\leq n <p$,有
\begin{lemma}
	\begin{displaymath}
		p|\binomial{p}{n}=\frac{p\cdot(p-1)\cdots(p-n+1)}{n\cdot(n-1)\cdots 1}
	\end{displaymath}
\end{lemma}
证明:注意到$p$是质数且与分母的每一个数互质,不可被分母的因子约去,所以最终值必有因子$p$。
那么可用普通生成函数表达为:
\begin{displaymath}
	(1+x)^p\equiv 1+x^p \pmod{p}
\end{displaymath}
可归纳推广为
\begin{inference}\label{LucasI}
	\begin{displaymath}
		(1+x)^{p^i}\equiv 1+x^{p^i} \pmod{p},i\in \mathbb{N}
	\end{displaymath}
\end{inference}
利用生成函数证明:
\begin{eqnarray*}
	\sum_{m=0}^n{\binomial{n}{m}x^m}&=&(1+x)^n\\
	&=&\prod_{i=0}^k{((1+x)^{p^i})^{n_i}}\\
	&\equiv&\prod_{i=0}^k{(1+x^{p^i})^{n_i}} \textrm{~(根据推论~\ref{LucasI})}\\
	&=&\prod_{i=0}^k{\left(\sum_{m_i=0}^{n_i}
	\binomial{n_i}{m_i}x^{p_im_i}\right)}\\
	&=&\prod_{i=0}^k{\left(\sum_{m_i=0}^{p-1}
	\binomial{n_i}{m_i}x^{p_im_i}\right)}\\
	&=&\sum_{m=0}^n{\left(\prod_{i=0}^k\binomial{n_i}{m_i}\right)x^m} \pmod{p}
\end{eqnarray*}
以上内容参考了Wikipedia-EN\footnote{Lucas's theorem - Wikipedia
	\url{https://en.wikipedia.org/wiki/Lucas\%27s\_theorem}}
\subsection{ExLucas}
对于模数为合数的情况,将$p$质因数分解,即
$\displaystyle p=\sum_{i=1}^k{p_i^{c_i}}$。
然后用求出$\binomial{n}{m} \bmod{p_i^{c_i}}$的值,最后使用CRT合并。

首先将求组合数转换为求阶乘,但要从$n!$提出p的倍数最后处理,
即\begin{displaymath}
	n!=\prod_{i=1}^n{i^{[(n,i)=1]}}\cdot p^{[\frac{n}{p}]}\cdot
	\prod_{i=1}^{[\frac{n}{p}]}i
\end{displaymath}
\begin{itemize}
	\item 第一部分:前$[\frac{n}{p_i^{c_i}}]$块的答案相等,计算整块后快速幂,
	      末尾不完整的块暴力计算。
	\item 第二部分:由于组合数中分子分母都有因子$p$,单独拆出来算。

	统计次数代码:
	\begin{lstlisting}
		int cnt = 0;
		for(int i=n/p;i;i/=p)
			cnt+=i;
	\end{lstlisting}
	\item 第三部分:成为了一个新阶乘,递归解决。
\end{itemize}
对于单个质数幂计算复杂度$O(p_i^{c_i})$。

以上内容参考了Candy?的博客\footnote{[Lucas定理]【学习笔记】 - Candy?
	\url{https://www.cnblogs.com/candy99/p/6637629.html}}。

\section{康托展开}
\index{C!Cantor Expansion}
康托展开可用于将一个排列映射到一个自然数(在所有排列中的字典序排名,从0开始)。

排列$P$的康托展开为$\displaystyle \sum_{i=2}^n{(i-1)!a_i}$,
其中$a_i$为位置$i$后小于$P_i$的数的个数,即$\displaystyle \sum_{j=1}^{i-1}[P_i>P_j]$。

相反已知字典序排名可以将其映射到一个排列,这就是逆康托展开。首先根据排名值计算出$a$数组,
数组$a$是唯一确定的,因为$a_i<i$而前一项的系数$i!> (i-1)!a_i$,从前到后取模确定。
计算出数组$a$后,维护一个支持删除和询问第$k$大的数据结构,从前到后确定每一位数。

\section{其它公式}
\begin{theorem}
    \begin{displaymath}
        \sum_{i=0}^n\binomial{i}{k}=\binomial{n+1}{k+1}
    \end{displaymath}
\end{theorem}
\begin{itemize}
    \item 证明:将等式加上$\binomial{0}{k+1}=0$,左边不断合并即为右式。
    \item 证明:将右式不断展开即为左式。
    \item 理解:在$n+1$个中选$k+1$个点,第$k+1$个点为哨兵,剩下$k$个在它之前
    的元素中选择。
\end{itemize}
\begin{theorem}
    \begin{displaymath}
        \sum_{i=0}^n{\binomial{n}{i}i}=n2^{n-1}
    \end{displaymath}
\end{theorem}
\begin{theorem}
    \begin{displaymath}
        \int_0^1{\binomial{n}{k}x^k(1-x)^{n-k}\ud x}=\frac{1}{n+1}
    \end{displaymath}
\end{theorem}
\begin{theorem}
    \begin{displaymath}
        (-1+1)^n=\sum_{k=0}^n{(-1)^k\binomial{n}{k}}=[n=0]
    \end{displaymath}
\end{theorem}
更一般地,可以得到
\begin{theorem}
    \begin{displaymath}\label{BSum}
        \sum_{k=0}^m{(-1)^k\binomial{n}{k}}=\binomial{m-n}{m}
    \end{displaymath}
\end{theorem}
证明:
\begin{lemma}[上指标反转]\label{BSL}
    \begin{displaymath}
        \binomial{n}{m}=(-1)^m\binomial{m-n-1}{m}
    \end{displaymath}
\end{lemma}
将组合数的分子取反即可证明。
由于$(-1)^i=(-1)^{-i}$,该式也可表述为
\begin{inference}
    \begin{displaymath}
        (-1)^m\binomial{n}{m}=\binomial{m-n-1}{m}
    \end{displaymath}
\end{inference}
接下来证明定理:
\begin{eqnarray*}
    \sum_{k=0}^m{(-1)^k\binomial{n}{k}}&=&
    \sum_{k=0}^m{\binomial{k-n-1}{k}}\\
    &=&\binomial{m-n}{m} \textrm{(展开后不断合并最左边两项))}
\end{eqnarray*}

\chapter{多项式}
\minitoc
\section{快速傅里叶变换FFT}
\index{F!Fast Fourier Transformation}
\subsection{FFT原理}
FFT求多项式卷积的过程为:$\Theta(n\lg n)$求值$\Rightarrow \Theta(n)$点值乘法
$\Rightarrow \Theta(n\lg n)$插值。

$\Theta(n\lg n)$求值/插值的复杂度是在单位复数根上计算得到的。

\subsubsection{单位复数根}

定义{\bfseries $n$次单位复数根}是满足$\omega^n=1$的复数$\omega$,恰好有$n$个,即
$\omega_n^k=e^{2\pi ik/n},k=0,1,\cdots,n-1$。

定义{\bfseries 主$n$次单位根}$\omega_n=e^{2\pi i/n}$。

下面是关于$n$次单位复数根的性质:

\begin{lemma}[消去引理]\label{FFTL1}
	对于任意整数$n\geq 0,k \geq 0,d>0$,
	\begin{displaymath}
		\omega_{dn}^{dk}=\omega_n^k
	\end{displaymath}
\end{lemma}
证明:
\begin{displaymath}
	\omega_{dn}^{dk}=e^{2\pi i dk/dn}=e^{2\pi i k/n}=\omega_n^k
\end{displaymath}

\begin{inference}\label{FFTI2}
	对于任意偶数$n>0$,有
	\begin{displaymath}
		\omega_n^{n/2}=\omega_2=-1
	\end{displaymath}
\end{inference}

\begin{lemma}[折半引理]
	对于偶数$n>0$,$n$个$n$次单位复数根的平方组成的集合为$n/2$个$n/2$
	次单位复数根的集合。
\end{lemma}
证明:根据引理~\ref{FFTL1}可得$(\omega_n^k)^2=(\omega_n^{k+n/2})^2=
	\omega_{n/2}^k$,每个$n/2$次单位复数根恰好被得到2次。

\begin{lemma}[求和引理]\label{FFTL4}
	对于任意整数$n\geq 1$与不能被$n$整除的非负整数$k$,有
	\begin{displaymath}
		\sum_{i=0}^{n-1}{(w_n^k)^i}=0
	\end{displaymath}
\end{lemma}
证明:
\begin{displaymath}
	\sum_{i=0}^{n-1}{(w_n^k)^i}=\frac{(w_n^k)^n-1}{w_n^k-1}=0
\end{displaymath}
$n$不整除$k$保证了分母不为0。

\subsubsection{DFT}
\index{D!Discrete Fourier Transform}
对于次数界为$n$的多项式
\begin{displaymath}
	A(x)=\sum_{i=0}^{n-1}{a_ix^i}
\end{displaymath}
其DFT为
\begin{displaymath}
	DFT_n(a)=(y_0,y_1,\cdots,y_{n-1})=
	(A(\omega_n^0),A(\omega_n^1),\cdots,A(\omega_n^{n-1}))
\end{displaymath}

\subsubsection{FFT}
FFT采用分治策略,假设$n$是2的幂(不足补0),其步骤如下:
\begin{enumerate}
	\item 若次数界为1,则返回$a_0$。
	\item 定义新的次数界为$n/2$多项式
	      \begin{eqnarray*}
		      A^{[0]}(x)&=&a_0+a_2x+\cdots+a_{n-2}x^{n/2-1}\\
		      A^{[1]}(x)&=&a_1+a_3x+\cdots+a_{n-1}x^{n/2-1}
	      \end{eqnarray*}
	      递归计算其在点$(\omega_n^0)^2,(\omega_n^1)^2,\cdots,(\omega_n^{n-1})^2$
	      的值(实际上递归只求了前一半)。
	\item 该多项式满足等式\begin{equation}\label{RFFTE}
		      A(x)=A^{[0]}(x^2)+xA^{[1]}(x^2)
	      \end{equation}
	      可利用递归计算的值合并。
	      对于$k=0,1,\cdots,n/2-1$,
	      \begin{eqnarray*}
		      y_k&=&y_k^{[0]}+\omega_n^ky_k^{[1]}\\
		      y_{k+n/2}&=&y_k^{[0]}-\omega_n^ky_k^{[1]}
	      \end{eqnarray*}
	      正确性证明:
	      \begin{eqnarray*}
		      y_k&=&y_k^{[0]}+\omega_n^ky_k^{[1]}\\
		      &=&A^{[0]}(\omega_{n/2}^k)+\omega_n^kA^{[1]}(\omega_{n/2}^k)\\
		      &=&A^{[0]}(\omega_n^{2k})+\omega_n^kA^{[1]}(\omega_n^{2k})
		      \textrm{~(根据引理~\ref{FFTL1})}\\
		      &=&A(\omega_n^k) \textrm{~(根据式~\ref{RFFTE})}\\
		      y_{k+n/2}&=&y_k^{[0]}-\omega_n^ky_k^{[1]}\\
		      &=&A^{[0]}(\omega_{n/2}^k)+\omega_n^{k+n/2}A^{[1]}(\omega_{n/2}^k)
		      \textrm{~(根据推论~\ref{FFTI2})}\\
		      &=&A^{[0]}(\omega_n^{2k+n})+\omega_n^{k+n/2}A^{[1]}(\omega_n^{2k+n})
		      \textrm{~(根据引理~\ref{FFTL1}与$\omega_n^n=1$)}\\
		      &=&A(\omega_n^{k+n/2}) \textrm{~(根据式~\ref{RFFTE})}\\
	      \end{eqnarray*}
\end{enumerate}
\subsubsection{逆DFT}
\begin{theorem}
	对于范德蒙德矩阵$V_n$满足$v_{ij}=\omega_n^{ij}$,
	其逆矩阵$V_n^{-1}$满足$v^{-1}_{ij}=\omega_n^{-ij}/n$(矩阵下标从0开始)。
\end{theorem}
证明:
\begin{eqnarray*}
	[V_nV_n^{-1}]_{ij}&=&\sum_{k=0}^{n-1}{\omega_n^{ik}\omega_n^{-kj}/n}\\
	&=&\sum_{k=0}^{n-1}{\omega_n^{k(i-j)}/n}\\
	&=&[i=j]\textrm{~(根据引理~\ref{FFTL4})}\\
	&\Rightarrow& V_nV_n^{-1}=I_n
\end{eqnarray*}
所以逆DFT的矩阵就是DFT矩阵的$\frac{1}{n}$,用$\omega_n^{-1}$代替$\omega_n$,
对多项式的DFT再做一次DFT后除以$n$就能得到原多项式。

以上内容来自算法导论\cite{ITA3}第30章 多项式与快速傅里叶变换。
\subsection{迭代FFT实现}
\subsubsection{单位复数根预处理}
一般可以确定FFT所需最大规模,因此可以在FFT前预处理。
\begin{lstlisting}
typedef double FT;
typedef std::complex<FT> Complex;
int tot;
Complex root[size],invR[size];
void init(int n) {
	tot=n;
	FT base=2.0*acos(-1.0)/n;
	for(int i=0;i<n;++i) {
		root[i]=Complex(cos(base*i),sin(base*i));
		invR[i]=conj(root[i]);
	}
}
\end{lstlisting}
根据引理~\ref{FFTL1},$root[tot/n*k]=w_n^k$。
\subsubsection{离线位逆序置换}
为了做迭代FFT,我们需要置换原多项式的系数,第$i$个系数被换到基于$2^k$的$i$的位逆序上。
若FFT的规模不变,比如求多项式乘法之类的简单问题,可预处理位逆序置换数组后,按照数组下标
置换(规定$i<rev[i]$保证只置换1次)。
\begin{lstlisting}
int rev[size];
void initRev(int k) {
	rev[0]=0;
	int end=1<<k;
	for(int i=1;i<end;++i)
		rev[i]=rev[i>>1]>>1|(i&1)<<(k-1);
}
\end{lstlisting}
\subsubsection{在线位逆序置换}
在分治FFT时,需要不同规模的位逆序置换,预处理置换数组不太适用。这里介绍的是时间复杂度
$O(n)$,空间复杂度$O(1)$的在线算法。
\begin{lstlisting}
void rev(int* A,int n) {
	for(int i=0,j=0;i<n;++i) {
		if(i>j)std::swap(A[i],A[j]);
		int l=n>>1;
		while((j^=l)<l)l>>=1;//key
	}
}
\end{lstlisting}
在每轮循环开始时,$i$与$j$互为位逆序。
代码中key部分是整个算法的关键,$i$的递增就是正向二进制加法,$j$就要
进行反向二进制加法,即从高到低找到第1个0,从高到低将这一段取反。

时间复杂度证明:设$n=2^k$,按$while$迭代次数分类,有
\begin{eqnarray*}
	T(n)&=&\sum_{i=1}^k{i\cdot 2^{k-i}}\\
	&=&n\sum_{i=1}^k{i\cdot 2^{-i}}\\
	&=&n\sum_{i=1}^k{\sum_{j=i}^k{2^{-j}}}\\
	&<&n\sum_{i=1}^k{2^{1-i}}\\
	&<&2n\Rightarrow T(n)=O(n)
\end{eqnarray*}
\subsubsection{迭代FFT}
考虑每一层递归树,发现最底层的递归树是按位逆序顺序排的(每一次递归相当于一次01划分)。
若系数数组按照这种顺序存储,则可以很快地找到每一层对应的位置。

迭代FFT的计算顺序是自底向上的,代码如下:

\begin{lstlisting}
void FFT(int n, Complex *A, Complex *w) {
    for (int i = 0, j = 0; i < n; ++i) {
        if (i > j) std::swap(A[i], A[j]);
        int l = n >> 1;
        while ((j ^= l) < l) l >>= 1;
    }
    for (int i = 2; i <= n; i <<= 1) {
        int m = i >> 1, fac = tot / i;
        for (int j = 0; j < n; j += i)
            for (int k = 0; k < m; ++k) {
                Complex t = A[j + k + m] * w[k * fac];
                A[j + k + m] = A[j + k] - t;
                A[j + k] += t;
            }
    }
}
\end{lstlisting}
首先做位逆序置换,然后从倒数第二层开始向上递推,$i$表示当前层每一块的长度,$m$是下一层
每一块的长度,$fac$是单位复数根缩放系数,$j$是块编号,内层循环与递归FFT合并答案的方法
相同,注意蝴蝶操作的求值顺序。

以上内容参考了Miskcoo的博客\footnote{从多项式乘法到快速傅里叶变换 – Miskcoo's Space
\url{http://blog.miskcoo.com/2015/04/polynomial-multiplication-and-fast-fourier-transform}}。
\subsection{实序列DFT}\label{RDFT}
对于多项式乘法这类问题,常规方法需要对每个序列做1次DFT。但由于我们
只在实数域上做FFT,可以考虑将两个实序列合并为一个复序列,
直接使用1次DFT以及一点后处理得到两个实序列的DFT。

记大小为$N$的序列$a$的DFT为$A$,考虑序列$A$的共轭为$\overline{A}$:
\begin{displaymath}
	\overline{A(n)}=\sum_{k=0}^{N-1}{\overline{a(k)}\cdot\omega_N^{-nk}}=
	\sum_{k=0}^{N-1}{\overline{a(k)}\cdot\omega_N^{K(N-n)}}
	n=0,1,\cdots,N-1
\end{displaymath}

对于实序列$a$,有$a(n)=\overline{a(n)}$,由此可得$\overline{A(n)}=A(N-n)$,
即实序列的DFT共轭对称。

令$z(i)=x(i)+i\cdot y(i)$,则$Z(n)=X(n)+i\cdot Y(n)$,
记$Z(n)$的实部与虚部分别为$X(n)$与$Y(n)$,有
\begin{eqnarray*}
	\overline{Z(N-n)}&=&\overline{X(N-n)+i\cdot Y(N-n)}\\
	&=&\overline{A(N-n)+i\cdot B(N-n)}\\
	\Rightarrow X(n)-i\cdot Y(n)&=&A(N-n)-i\cdot B(N-n)
\end{eqnarray*}

联立解得
\begin{eqnarray*}
	2X(n)&=&(A(n)+A(N-n))+i\cdot (B(n)-B(N-n))\\
	2Y(n)&=&(B(n)+B(N-n))-i\cdot (A(n)-A(N-n))
\end{eqnarray*}

对复序列$z(n)$的DFT后处理一下就可以得到$X(n)$与$Y(n)$(注意要对$n=0$
进行特别处理)。

代码如下:
\begin{lstlisting}
FFT(p,A,root);
X[0] = A[0].real();
Y[0] = A[0].imag();
for (int i = 1; i < p; ++i) {
	FT x1 = A[i].real(), y1 = A[i].imag();
	FT x2 = A[p - i].real(), y2 = A[p - i].imag();
	X[i] = Complex((x1 + x2) * 0.5, (y1 - y2) * 0.5);
	Y[i] = Complex((y1 + y2) * 0.5, (x2 - x1) * 0.5);
}
\end{lstlisting}

以上内容参考了Miskcoo的博客\footnote{实序列离散傅里叶变换的快速算法 – Miskcoo's Space
\url{http://blog.miskcoo.com/2018/01/real-dft}}。
\subsection{MTT之拆系数FFT}
MTT主要用来解决模数不满足NTT条件(模数无原根或者欧拉函数值中2的幂次比数据范围小)的卷积问题,可以使用
拆系数FFT的实数运算来绕开限制。

设模数为$M$,解决方法是找一个$k\geq\sqrt{M}$,将序列$a(n)$拆为2个新序列
$a_0(n)=a(n)/k,a_1(n)=a(n)\% k$,两两相乘后按照系数合并。共4次DFT,4次IDFT。

优化:
\begin{itemize}
	\item 注意到这4个多项式均为实序列,所以可按照~\ref{RDFT}节中所述方法仅使用2次DFT。
	\item 合并时有两个多项式的系数均为$k$,可以加在一起IDFT,仅使用3次IDFT。
\end{itemize}

\section{快速数论变换NTT}
\subsection{NTT原理}
NTT的原理与FFT类似,即找到单位根$x$满足$x^n\equiv 1 \pmod{p}$。
NTT模数$p$需满足$p$为素数且$p=r\cdot 2^k+1$。

根据定理~\ref{FLT}可知若模数$p$为素数则有$x^{p-1}\equiv 1 \pmod{p}$,
所以当$n|(p-1)$时才能进行NTT。

根据~\ref{PrimitiveRoot}节所述,$p$必有原根,设$p$的原根为$g$,则
$g^\frac{p-1}{n}$就是{\bfseries 主$n$次单位根},$n$个单位根即为
$w_n^k=g^{k\cdot \frac{p-1}{n}}$。

其余部分与FFT相同。

\subsection{NTT实现}
NTT仅预处理单位复数根部分不同,以模998244353为例:
\begin{lstlisting}
int tot, root[size], invR[size];
void init(int n) {
    const int g = 3;
    tot = n;
    Int64 base = powm(g, (mod - 1) / n);
    Int64 invBase = powm(base, mod - 2);
    root[0] = invR[0] = 1;
    for (int i = 1; i < n; ++i)
        root[i] = root[i - 1] * base % mod;
    for (int i = 1; i < n; ++i)
        invR[i] = invR[i - 1] * invBase % mod;
}
\end{lstlisting}
\subsection{NTT常见模数}
\begin{itemize}
    \item $469762049=7*2^{26}+1$。
    \item $998244353=119*2^{23}+1$。
    \item $1004535809=479*2^{21}+1$,加起来不爆int。
    \item $2281701377=17*2^{27}+1$,平方恰好不爆long long。
\end{itemize}
\index{*Constant!NTT模数P=\{469762049,\\998244353,1004535809,\\2281701377\},g=3}
它们的原根均为3。
\subsection{MTT之三模数NTT}
选取3个模数,比如\{469762049,998244353,1004535809\},要求它们的乘积大于卷积
过程中最大的数,分别以这三个数为模数求NTT,最后使用CRT求解同余方程组。

但是使用CRT求解会爆long long,因此先合并前两项,得到
\begin{eqnarray*}
    x&\equiv&n_1 \pmod{p_1}\\
    x&\equiv&n_2 \pmod{p_2}
\end{eqnarray*}
设$x=k_1p_1+n_1=k_2p_2+n_2$,由于$k_1<p_2$,我们可以求解
$k_1p_1\equiv n_2-n_1 \pmod{p_2}$得到$k_1$,带入原式求出$x \bmod{p}$的值。

该方法来自AntiLeaf\footnote{COGS2294 释迦 - AntiLeaf
\url{http://www.cnblogs.com/hzoier/p/6441967.html}}

\section{快速沃尔什变换FWT}
\index{F!Fast Walsh-Hadamard Transform}
\subsection{FWT原理}
FWT主要用来求下列卷积:
\begin{displaymath}
    z_n=\sum_{i\oplus j=n}{a_ib_j}
\end{displaymath}
其中$\oplus$为二元位运算符。

其原理与FFT相同,关键在于蝴蝶操作。

序列可表示为$A=(A_0,A_1)$,01表示当前处理的位,只需找到对于$C=A\otimes B$,
有$FWT(C)=FWT(A)*FWT(B)$即可,由$UFWT(FWT(A))=A$可推出其逆变换。

现场推导时可以仅考虑由倒数第二层合并至倒数第三层的情况,其余自底向上可以归纳证明。

\subsubsection{and}
由$C=(A_0*B_0+A_0*B_1+A_1*B_0,A_1*B_1)$
可构造出
\begin{eqnarray*}
    FWT_{and}(A)&=&(FWT_{and}(A_0)+FWT_{and}(A_1),FWT_{and}(A_1))\\
    UFWT_{and}(A)&=&(UFWT_{and}(A_0)-UFWT_{and}(A_1),UFWT_{and}(A_1))
\end{eqnarray*}
\subsubsection{or}
\begin{eqnarray*}
    FWT_{or}(A)&=&(FWT_{or}(A_0),FWT_{or}(A_1)+FWT_{or}(A_0))\\
    FWT_{or}(A)&=&(UFWT_{or}(A_0),UFWT_{or}(A_1)-UFWT_{or}(A_0))
\end{eqnarray*}
\subsubsection{xor!!!}
\begin{eqnarray*}
    FWT_{xor}(A)&=&(FWT_{xor}(A_0)+FWT_{xor}(A_1),FWT_{xor}(A_0)-FWT_{xor}(A_1))\\
    FWT_{xor}(A)&=&\left(\frac{UFWT_{xor}(A_0)+UFWT_{xor}(A_1)}{2},
    \frac{UFWT_{xor}(A_0)-UFWT_{xor}(A_1)}{2}\right)
\end{eqnarray*}
\subsection{nand,nor,nxor}
求出$\oplus$为and,or,xor时的FWT后按取反交换即可。

以上内容参考了picks的博客
\footnote{Fast Walsh-Hadamard Transform « Picks's Blog
    \url{http://picks.logdown.com/posts/179290-fast-walsh-hadamard-transform}}。
\subsection{FWT实现}
\lstinputlisting[title=FWT]{Source/Templates/FWT.cpp}

\section{快速莫比乌斯变换/反演}\label{FMT}
\index{F!Fast Mobius Transformation}
快速莫比乌斯变换用于计算集合并卷积。
快速莫比乌斯反演其实就是其逆变换。
实际上这就是FWT的非递归形式。

给定数组$f,g$,求$f*g$的集合并卷积$\displaystyle h[k]=\sum_{i|j==k}{f[i]g[j]}$,
数组规模为$2^n$。

记数组$F$为$f$的莫比乌斯变换,满足$\displaystyle F[i]=\sum_{i\&j==j}{f[j]}$
(实际上就是子集和)。同理计算数组$g$的莫比乌斯变换$G$。令$H[i]=F[i]*G[i]$,会发现
$\displaystyle H[i]=\sum_{i\&j==j}{h[j]}$。

接下来考虑如何从$f$快速推出$F$:枚举每一个比特位,对每一比特位做前缀和,这样就可以保证
得到正确结果。时间复杂度$O(2^nn)$。
\begin{lstlisting}
int end = 1 << n;
for(int i = 1; i < end; i <<= 1)
    for(int j = 0; j < end; ++j)
        if(j & i)
            A[j] += A[j ^ i];
\end{lstlisting}

注意到所有的$j$都是$i$的父集,且$A$值改变不传递,因此可以枚举$i$的父集去掉
判断。不过实际优化效果并达不到2倍,可能性能瓶颈在于存取。

\begin{lstlisting}
int end = 1 << n;
for(int i = 1; i < end; i <<= 1)
    for(int j = i; j < end; j = (j + 1) | i)
        A[j] += A[j ^ i];
\end{lstlisting}

Update:似乎FWT比FMT更快些,可能它更Cache-Friendly吧。

逆变换只要将其倒着做就可以了(从高到低枚举与从低到高枚举等价,$A$值改变
的影响不会传递,所以不必控制枚举顺序)。
\begin{lstlisting}
int end = 1 << n;
for(int i = 1; i < end; i <<= 1)
    for(int j = 0; j < end; ++j)
        if(j & i)
            A[j] -= A[j ^ i];
\end{lstlisting}

用容斥可得$\displaystyle h[i]=\sum_{i\&j==j}{(-1)^{|I|-|J|}H[j]}$,其中$|I|$
表示$i$对应的集合$I$的大小。

证明:该式的$H[j]$子集中含有$h[i]$当且仅当$i==j$,贡献为$h[i]$。考虑其余$h[k]$对
$H[j]$的贡献,以及它们在式中最终对$h[i]$的贡献,设$d=|I|-|K|$,$h[k]$对$h[i]$的贡献
为$\displaystyle \sum_{i=0}^{2^d-1}{(-1)^{d-|i|}}$。由于选择奇数位与选择偶数位的
方案数相等,贡献为0。

\paragraph{注意事项} 单次FMT的实际意义很重要,经常用于辅助容斥预处理,
即使对集合幂级数的操作无定义。

\paragraph{例题}
Luogu P3175 [HAOI2015]按位或\footnote{
    【P3175】[HAOI2015]按位或 - 洛谷
    \url{https://www.luogu.org/problemnew/show/P3175}
}

记$a_i$为走$i$步到达目标状态的概率,答案为$\displaystyle \sum_{k=1}^
\infty{k(a_k-a_{k-1})}=-\sum_{k=0}^\infty a_k$。令概率数组为集合幂级数$f$,
定义乘法运算为集合幂卷积,可知答案为$f$的无限递减几何级数第$2^n-1$项的倒数,
若其收敛则其答案为$\left(\frac{1}{f-1}\right)_{2^n-1}$。计算$f$的莫比乌斯变换$F$,
那么函数$T(f)$对应序列${T(F_i)}$。记$G_i={\frac{1}{F_i-1}}$,
最后使用容斥计算出$g_{2^n-1}$,不必做反演。

若有解则$F_{2^n-1}=1$,此时对应的$G_{2^n-1}$应该为0,所以容斥时不考虑目标状态项。

代码:
\lstinputlisting{Source/Templates/FMT.cpp}

上述内容参考了liu\_runda的博客\footnote{
    bzoj4036[HAOI2015]set 按位或
    \url{https://www.cnblogs.com/liu-runda/p/6443577.html}
    不会FWT的选手计算集合并卷积的方法
    \url{http://liu-runda.blog.uoj.ac/blog/2360}
}。

\subsection{子集卷积}
有时按照1的个数将原数组分解也是一个比较好的思路。

例题:LOJ161 子集卷积

分析后可得到要求的是$h[k]=\displaystyle \sum_{i\&j=0 \land i|j=k}{f[i]g[j]}$。
考虑除去约束$i\&j=0$,由于我们能够做约束$i|j=k$的卷积,在满足这个约束的情况下,约束
$i\&j=0$可以等价为$bitcount(i)+bitcount(j)=bitcount(k)$。那么将原数组按照1的个数
分解,然后对每类$bitcount$卷积,最后只要输出$bitcount(k)$的$k$的值。

\section{多项式高级算法}
{\bfseries 警告:慎卡常导致编码/调试困难。}

Update:多项式算法模块化封装已完成,参见~\ref{Module}节与~\ref{Optmize}节的内容。

\subsection{牛顿迭代法}
已知函数$G(z)$,求函数$F(z) \bmod{z^n}$满足$G(F(z))\equiv 0 \pmod{z^n}$。

当$n=1$时,多项式退化为常数,直接求解。

记$h=\left\lfloor\frac{n+1}{2}\right\rfloor$,
若已知$G(F_0(z)) \equiv 0\pmod{z^h}$,
尝试计算$G(F(z))$在$F_0(z)$处的泰勒展开:
\begin{displaymath}
    G(F(z))=\sum_{i=0}^\infty{\frac{G^{(i)}(F_0(z))}{i!}\cdot (F(z)-F_0(z))^i}
\end{displaymath}
可以发现$F(z)$与$F_0(z)$的后$h$项均相等,所以两多项式之差的$n\geq 2$次方多项式的
最小非0项次数$\geq n$,模$z^n$意义下无贡献,因此仅前两项展开有效,即
\begin{displaymath}
    G(F(z))\equiv G(F_0(z))+G'(F_0(z))(F(z)-F_0(z)) \pmod{z^n}
\end{displaymath}
结合$G(F(z))\equiv 0 \pmod{z^n}$可得到新的$F(z)$:
\begin{displaymath}
    F(z)\equiv F_0(z)-\frac{G(F_0(z))}{G'(F_0(z))} \pmod{z^n}
\end{displaymath}
这就是牛顿迭代法。

\subsubsection{注意事项}
\begin{itemize}
    \item 使用FFT加速卷积后及时取模,即把不需要的位置0。否则循环卷积性质会干扰低次项。
    当然也可以直接将规模对齐到2的幂次,甚至可以利用循环卷积优化,最后一次再取模。
    \item 卷积时使用$>2$倍模次数的2的幂作为卷积规模,因为乘法操作至少需要这么多项
    才足够确定多项式系数。
    \item 若要求$\bmod{z^n}$意义下的结果,求导/积分这类导致多项式次数变化的操作,
    显然仅把次数设为$n$会导致信息丢失,而且讨论每种操作的最高次数也很麻烦。不妨全部求
    $\bmod{z^{n+c}}$意义下的结果,$c$为足够大的小常数,最后直接截断输出。
\end{itemize}

\subsection{多项式开方}
对于给定的$A(z)$,求$F(z) \pmod{z^n}$,使得$F^2(z)\equiv A(z)\pmod{z^n}$。

构造方程$F^2(z)-A(z)\equiv 0\pmod{z^n}$,
同理可得
\begin{eqnarray*}
    F(z)&\equiv& F_0(z)-\frac{F_0(z)^2-A(z)}{2F_0(z)} \pmod{z^n}\\
    &\equiv& \frac{F_0(z)^2+A(z)}{2F_0(z)} \pmod{z^n}
\end{eqnarray*}

注意当$t=0$时可能需要用二次剩余在模意义下开根。

\subsection{多项式求逆}
{\bfseries 多项式求逆是基于牛顿迭代法的多项式算法的基本工具。}

对于给定的$A(z)$,求$F(z) \pmod{z^n}$,使得$F(z)\cdot A(z)\equiv 1\pmod{z^n}$。

构造方程$F(z)\cdot A(z)-1\equiv 0\pmod{z^n}$,
同理可得
\begin{eqnarray*}
    F(z)&\equiv& F_0(z)-\frac{F_0(z)A(z)-1}{A(z)} \pmod{z^n}\\
    &\equiv& F_0(z)-(F_0(z)A(z)-1){F(z)} \pmod{z^n}\\
    &\equiv& F_0(z)-(F_0(z)A(z)-1){F_0(z)} \pmod{z^n}~(F_0(z)A(z)-1\equiv 0\pmod{z^h})\\
    &\equiv& 2F_0(z)-F_0^2(z)A(z) \pmod{z^n}
\end{eqnarray*}
\subsection{多项式取模}
给定一个$n$次多项式$A(z)$与$m$次多项式$B(z)$($m\leq n$),
求多项式$D(z),R(z)$满足
\begin{displaymath}
    A(z)=D(z)B(z)+R(z),deg(D)\leq n-m,deg(R)<m
\end{displaymath}
记$n$次多项式$A(z)$的系数翻转$A^R(z)=z^nA(\frac{1}{z})$。

令$deg(D)=n-m,deg(R)=m-1$,不足高位补0。
将原方程的$z$全部换成$\frac{1}{z}$,然后乘上$z^n$,有
\begin{eqnarray*}
    z^nA(\frac{1}{z})&=&z^nD(\frac{1}{z})B(\frac{1}{z})+z^nR(\frac{1}{z})\\
    A^R(z)&=&D^R(z)B^R(z)+z^{n-m+1}R^R(z)
\end{eqnarray*}
由于$deg(D^R)\leq n-m$,所以可以在模$z^{n-m+1}$的情况下求解$D^R(z)$,
翻转后带入原方程求出$R(z)$。

{\bfseries 一定要想清楚每个多项式的次数,以及在模$x$的几次方下计算。}
\subsection{多项式求导与积分}
根据$(x^n)'=nx^{n-1}$可得
\begin{eqnarray*}
    F'(z)&=&\sum_{i=1}^{n-1}{ic_iz^{i-1}}\\
    \int F(z)\ud z&=&\sum_{i=1}^{n-1}{\frac{c_{i-1}}{i}\cdot z^i}
\end{eqnarray*}
时间复杂度$O(n)$。
\subsection{多项式ln}
考虑对$\ln A(z)$求导:
\begin{displaymath}
    (\ln A(z))'=\frac{A'(z)}{A(z)}
\end{displaymath}
所以有
\begin{displaymath}
    \ln A(z)=\int \frac{A'(z)}{A(z)}\ud z
\end{displaymath}
由于要求逆元,时间复杂度$O(n \lg n)$。
\subsection{多项式exp}
先进行如下变换:
\begin{displaymath}
    F(z)-e^{A(z)}\equiv 0 \pmod{z^n}
    \Rightarrow \ln F(z)-A(z)\equiv 0 \pmod{z^n}
\end{displaymath}
直接牛顿迭代法得
\begin{eqnarray*}
    F(z)&=&F_0(z)-(\ln F_0(z)-A(z))F_0(z)\\
    &=&F_0(z)(1-\ln F_0(z)+A(z))
\end{eqnarray*}

一般其输入的常数项为0,由麦克劳林级数展开式得常数项恒为1。
\subsection{多项式快速幂}
使用常规快速幂可以得到$O(n\lg n\lg k)$的复杂度。
但是通过如下变形:
\begin{displaymath}
    F^k(z)=e^{k \ln F(z)}
\end{displaymath}
使用多项式ln/exp可以得到$O(n\lg n)$的复杂度。

{\bfseries 注意此种情况只适用于常数项为1的情况,其它情况需要缩放多项式。}
\subsection{多项式三角函数}
由欧拉公式可得
\begin{displaymath}
    e^{F(z)i}=\cos F(z)+\sin F(z) i
\end{displaymath}
在复数域上做多项式exp。

上述内容基本上使用牛顿迭代法递归求解,下面的模板(LOJ150 挑战多项式)涵盖了大部分
操作:
\lstinputlisting{Source/Source/'FFT NTT'/LOJ150.cpp}

\subsection{进制转换}
将$A$进制数转换为$B$进制数。

$A$进制数可表示为$\displaystyle \sum_{i=0}^n{a_iA^i}$,求出其在
$B$进制下的值。

对其分治,即
\begin{displaymath}
    \sum_{i=0}^n{a_iA^i}=\sum_{i=0}^{\frac{n}{2}-1}{a_iA^i}
    +A^{\frac{n}{2}}\sum_{i=0}^{\frac{n}{2}}{a_{i+\frac{n}{2}}A^i}
\end{displaymath}

预处理出$A^i$对应的$B$进制数,然后分治,同时使用FFT优化右边的卷积。

时间复杂度$O(n \lg^2 n)$。

\subsection{多项式多点求值}
已知$A(z)$与$n$个点$z_0,z_1,\cdots,z_{n-1}$,
求$A(z_0),A(z_1),\cdots,A(z_{n-1})$。

将要求的$x$分为两半,那么左右两半对应的插值多项式的次数为$[\frac{n}{2}]$,
求出这两个插值多项式后递归计算。

对于左半部分,考虑多项式$\displaystyle B(z)=\prod_{i=0}^
{[\frac{n}{2}]}(z-z_i)$,满足$deg(B)=[\frac{n}{2}]$。
令$A(z)=D(z)B(z)+R(z)$,当$z$为左半部分中的点时,$D(z)B(z)$为0,即
$A(z)=R(z)$。那么可以两边同时模$B(z)$,达到次数减半的效果。
右边部分类似。

{\bfseries 注意$B(z)$要$O(n\lg^2 n)$分治预处理,空间$O(n\lg n)$,递归树根的
多项式不必计算。}。

时间复杂度$O(n\lg^2n)$。

代码:
\lstinputlisting{Source/Templates/Evaluator.cpp}
\subsection{多项式多点插值}

对待插值点分治,假设求出了左半部分插值多项式$A_{left}(z)$,构造出左半部分的$B(z)$
使其满足$A(z)=A_{left}(z)+A_{right}(z)B(z)$,那么左半部分的点都将在$A(z)$上。
仅考虑右半部分:
\begin{displaymath}
    \forall(x_i,y_i)\in P_{right},y_i=A_{left}(x_i)+A_{right}(x_i)B(x_i)
\end{displaymath}
化简得
\begin{displaymath}
    A_{right}(x_i)=\frac{y_i-A_{left}(x_i)}{B(x_i)}
\end{displaymath}
那么可利用多点求值得到一组新的插值点,递归求解。时间复杂度$O(n \lg^3 n)$。

\subsection{组合数取模}

求$\binomial{n}{m} \bmod{p}$的值,其中$p$为素数,$n,m\leq 1e9$。

\begin{itemize}
    \item $p\leq 1e6$:线性预处理逆元+lucas;
    \item $n<p$:可知$n!$与$p$互质,因此只要求$n!$。
    可构造多项式:
    \begin{eqnarray*}
        Q(x)&=&\prod_{i=1}^{\sqrt{n}}{(x+i)}\\
        n!&=&\prod_{i=0}^{\sqrt{n}-1}{Q(i\sqrt{n})}
    \end{eqnarray*}
    卷积出$Q(x)$后在$i\sqrt{n}=0,\sqrt{n},\cdots,(\sqrt{n}-1)\sqrt{n}$上
    多点求值,时间复杂度\\$O(\sqrt{n}\lg^2 n)$。(在考场上不太好写,可以用一些预处理时间
    每隔一段打表,查询时定位到对应位置暴力计算,适用于$n$较大而查询很少的情况。)
    \item $n\geq p$:结合上述两种做法,易知第二种做法最多只执行一次。
\end{itemize}

\subsection{CDQ分治FFT}
已知函数$g$在$[1,n)$上的值,求函数$f(x)=\displaystyle \sum_{y=1}^x{f(x-y)g(y)}$
在$[1,n)$上的值,其中$f(0)=1$。

对于这种自我依赖的卷积,可以考虑将待求值一分为二,分治求解。{\bfseries 分治到小规模时
直接$O(n^2)$暴力求解。}

记卷积过程为$conv(l,r)$,$m$为$l,r$中点,首先递归调用$conv(l,m)$求出区间的前一半,
然后计算使用卷积累加前一半对后一半的贡献,然后递归调用$conv(m+1,r)$补足区间后一半的剩余项。
时间复杂度$O(n\lg^2 n)$。

事实上这就是CDQ分治的过程,思路参见第\ref{CDQ}节。

CDQ分治FFT时,需要对整个多项式进行平移,可能导致不知道该从哪个位置,哪个方向取
卷积结果。这时使用端点来计算对应的位置,找出对应规律。
\subsubsection{优化}
将$F$与$G$看做序列$f,g$的OGF,令$g(0)=0$,计算$F(x)\otimes G(x)=F(x)-f(0)x^0$,
变形为$F(x)\equiv \frac{1}{1-G(x)} \pmod{x^n}$,使用多项式求逆$O(n\lg n)$解决。

\subsection{多项式乘积(普通分治FFT)}
求多个多项式之积(积的次数上界为$W$)可以使用分治乘法解决,时间复杂度
$O(W\lg W\lg n)$。

\subsubsection{优化A} 由于求的是多项式乘积,因此在IDFT时不用做除法,而是记录除法因子,
在最终答案中做1次除法。

\subsubsection{优化B} 使用优先队列维护多项式次数,然后按照类似赫夫曼编码的策略
贪心合并多项式。参见kczno1的代码\footnote{\url{https://loj.ac/submission/110667}}。

\subsubsection{优化C} 若它为多个一次多项式相乘,则会出现大量小规模卷积,使用暴力解决。

以上内容参考了picks
\footnote{
Newton's Method of Polynomial $\ll$ Picks's Blog
\\\url{http://picks.logdown.com/posts/209226-newtons-method-of-polynomial}

Positional Notation Conversion $\ll$ Picks's Blog
\\\url{http://picks.logdown.com/posts/208342-positional-notation-conversion}

Binomial Coefficient Modulo Prime $\ll$ Picks's Blog
\\\url{http://picks.logdown.com/posts/245545-binomial-coefficient-modulo-prime}

}
、Miskcoo\footnote{
牛顿迭代法在多项式运算的应用 – Miskcoo's Space
\\\url{http://blog.miskcoo.com/2015/06/polynomial-with-newton-method}

多项式除法及求模 – Miskcoo's Space
\\\url{http://blog.miskcoo.com/2015/05/polynomial-division}

多项式的多点求值与快速插值 – Miskcoo's Space
\\\url{http://blog.miskcoo.com/2015/05/polynomial-multipoint-eval-and-interpolation}
}和VictoryCzt\footnote{
    分治FFT学习笔记
    \url{https://blog.csdn.net/VictoryCzt/article/details/82939586}
}的博客。

\subsection{二元多项式卷积}
该需求源自2015年国家集训队论文集中金策的《生成函数的运算与组合计数问题》,
用于解决二元生成函数的卷积。

将二元函数的系数排为矩阵,对每一行做DFT,再对每一列做DFT,将点值相乘,
对每一列做IDFT,每一行做IDFT。整个过程与DFT计算卷积的过程类似。记多项式次数
为$n,m$,时间复杂度为$O(nm\lg nm)$。
\subsection{循环卷积}
该需求来自CTSC2010 性能优化,题意为求$ab^C$在模$x^n$意义下的循环卷积模$n+1$的值。

使用常用的基2NTT,必须把卷积规模开到$2n$,然后每次把溢出部分加回去。即使使用快速幂也
带来$O(\lg n)$的复杂度。

不过NTT可以使用混合积,例题有一个性质,即$n$是smooth number,可以分解为
$2^a3^b5^c7^d$的形式。那么可以按照当时FFT的推导写出基2,基3,基5,基7的蝴蝶操作,
然后使用递归形式求解。

\index{*TODO!混合基FFT的非递归形式}

由于是循环卷积,求出点值后可以不考虑溢出,即无需限制点值乘法的次数,允许直接在点值上
做快速幂。

参考代码:
\lstinputlisting{Source/Source/'FFT NTT'/P4191.cpp}

该内容参考了skywalkert的博客\footnote{
    BZOJ 1919 [Ctsc2010]性能优化\\
    \url{https://blog.csdn.net/skywalkert/article/details/51737272}
}。

对于没有特殊性质的$n$,若要求模$x^n$意义下的循环卷积,可以使用Bluestein’s Algorithm。
参见国家集训队2016论文集毛啸的《再探快速傅里叶变换》。

\index{*TODO!通用循环卷积}

\section{FFT封装}
牛顿迭代法/分治FFT时使用。

使用vector存储多项式。
\begin{lstlisting}
typedef std::vector<int> Poly;
\end{lstlisting}

封装一个工具函数计算做乘法所需最小规模。
\begin{lstlisting}
int getSize(int x) {
    x <<= 1;
    int p = 1;
    while(p < x)
        p <<= 1;
    return p;
}
\end{lstlisting}

封装一个工具函数用于拷贝多项式,参数顺序遵循memcpy。
\begin{lstlisting}
void copyPoly(Poly& dst, const Poly& src, int siz) {
    memcpy(dst.data(), src.data(), sizeof(int) * siz);
}
\end{lstlisting}

原来的函数名称是copy,但是编译器经常将其与std::copy在重载决议中混淆
(比如siz的实参类型为std::size时)。并且涉及模板和STL的代码并不好
根据编译器错误信息发现错误位置。

封装DFT、IDFT,注意再给IDFT一个参数指示模数,只对需要的系数做除法。
\begin{lstlisting}
void DFT(int n, Poly& A) {
    NTT(n, A.data(), root);
}
void IDFT(int n, Poly& A, int rn) {
    NTT(n, A.data(), invR);
    Int64 div = powm(n, mod - 2);
    for(int i = 0; i < rn; ++i)
        A[i] = A[i] * div % mod;
    memset(A.data() + rn, 0, sizeof(int) * (n - rn));
}
\end{lstlisting}

牛顿迭代法系列函数一般使用如下签名:
\begin{lstlisting}
void func(int n, const Poly& sf, Poly& g)
\end{lstlisting}

其中$sf$为原多项式,使用时拷贝,$g$尽量重复使用,空间由调用端分配。

以多项式求逆为例:
\begin{lstlisting}
void inv(int n, const Poly& sf, Poly& g) {
    if(n == 1)
        g[0] = powm(sf[0], mod - 2);
    else {
        int h = (n + 1) >> 1;
        inv(h, sf, g);
        int p = getSize(n);
        DFT(p, g);

        Poly f(p);
        copyPoly(f, sf, n);
        DFT(p, f);

        for(int i = 0; i < p; ++i) {
            g[i] = (2 - asInt64(g[i]) * f[i]) % mod *
                g[i] % mod;
            if(g[i] < 0)
                g[i] += mod;
        }
        IDFT(p, g, n);
#ifdef CHECK
        for(int i = 0; i < n; ++i) {
            int sum = 0;
            for(int j = 0; j <= i; ++j)
                sum =
                    (sum + asInt64(sf[j]) * g[i - j]) %
                    mod;
            if(i == 0 && sum != 1)
                throw;
            if(i != 0 && sum != 0)
                throw;
        }
#endif
    }
}
\end{lstlisting}

求逆/开方等操作在调试时最好$O(n^2)$验证其正确性。

简单多项式乘法根据规模使用暴力/FFT解决。

{\bfseries 注意std::vector<>::data()是c++11新增的用法,使用时注意程序可移植性。}

\section{本章注记}
Miskcoo的博客中有大量非常优秀的FFT系列文章
\footnote{\url{http://blog.miskcoo.com/tag/fft}}。

HocRiser对此系列做了全面的总结
\footnote{\url{http://www.cnblogs.com/HocRiser/p/8207295.html}}。

\chapter{线性代数}
\section{高斯消元}
\subsection{变换为上三角矩阵}
高斯消元的思路很简单:每次通过初等变换消去第$i$行第$i$列下面的项,
最终变换为一个上三角矩阵。
\begin{lstlisting}[title=gauss]
typedef double FT;
const FT eps=1e-8;
FT A[size][size];
bool gauss(int n) {
    for(int i=1;i<=n;++i) {
        int x=i;
        for(int j=i+1;j<=n;++j)
            if(fabs(A[j][i])>fabs(A[x][i]))
                x=j;
        if(fabs(A[x][i])<eps)return false;
        if(x!=i) {
            for(int j=i;j<=n;++j)
                std::swap(A[i][j],A[x][j]);
        }
        for(int j=i+1;j<=n;++j) {
            FT fac=A[j][i]/A[i][i];
            for(int k=i;k<=n;++k)
                A[j][k]-=fac*A[i][k];
        }
    }
    return true;
}
\end{lstlisting}
时间复杂度$O(n^3)$。
\subsubsection{精度}
对于实数运算,选择绝对值最大数作为主元,以减小误差。
\subsubsection{优化}
可以观察矩阵的特殊性来优化消元复杂度,对于方程间联系不大的情况使用
迭代解小规模线性方程组。
\subsection{求解线性方程组}\label{LSE}
即求解$Ax=b$的向量$x$。
注意到上三角矩阵的最后一行是一元方程,解出后倒数第二行还是是一元方程,所以不断逆推即可。
注意对矩阵$A$的操作也要应用到向量$b$上(两边同时乘上变换矩阵)。
时间复杂度$O(n^2)$。
\begin{lstlisting}
for(int i=n;i>=1;--i) {
    FT sum=B[i];
    for(int j=i+1;j<=n;++j)
        sum-=A[i][j]*X[j];
    X[i]=sum/A[i][i];
}
\end{lstlisting}
\subsection{求解逆矩阵}\label{InvMatGauss}
设将$A$变换为上三角矩阵的变换矩阵为$P$,求逆矩阵即求解方程$PAA^{-1}=PI$,
由矩阵乘法的定义可知,将$A^{-1}$与$PI$按列拆分,即得到$(PA)A_i^{-1}=(PI)_i$
的形式,按照求解线性方程组的方法做即可。

由于要额外维护$PI$,所以常数较LUP分解大。

\subsubsection{板子}

【P4783】【模板】矩阵求逆 - 洛谷
\footnote{\url{https://www.luogu.org/problemnew/show/P4783}}
\lstinputlisting[title=InvMatGauss]{LinearAlgebra/InvMatGauss.cpp}
时间复杂度$O(n^3)$。

\section{LUP分解}
LUP分解的数值稳定性较高斯消元法强。
\subsection{基本原理}
要求解线性方程组$Ax=b$,对系数矩阵$A$进行LUP分解:
\begin{displaymath}
	PA=LU
\end{displaymath}
其中$P$为置换矩阵,$L$为下三角矩阵,$U$为上三角矩阵。

将$Ax=b$左乘$P$,得$PAx=Pb$,然后用$PA=LU$代换得$LUx=Pb$。
设$y=Ux$,有$Ly=Pb$,可以使用类似于~\ref{LSE}节$O(n^2)$求解$y$,
然后再次使用该方法求出$x$。

\subsection{LUP分解}
\subsubsection{LU分解}
考虑不会出现不需要换主元的情况(比如对称正定矩阵),即$P=I$。

运用矩阵代数将A分解:
\begin{eqnarray*}
	A&=&\left[\begin{array}{c|ccc}
			a_{11} & a_{12} & \cdots & a_{1n} \\
			\hline
			a_{21} & a_{22} & \cdots & a_{2n} \\
			\vdots & \vdots & \ddots & \vdots \\
			a_{n1} & a_{n2} & \cdots & a_{nn}
        \end{array} \right]\\
    &=&\left[\begin{array}{cc}
        a_{11}&w^T\\
        v&A'
    \end{array}\right]\\
    &=&\left[\begin{array}{cc}
        1&0\\
        v/a_{11}&I_{n-1}
    \end{array}\right]
    \left[\begin{array}{cc}
        a_{11}&w^T\\
        0&A'-vw^T/a_{11}
    \end{array}\right]\\
    &=&\left[\begin{array}{cc}
        1&0\\
        v/a_{11}&I_{n-1}
    \end{array}\right]
    \left[\begin{array}{cc}
        a_{11}&w^T\\
        0&L'U'
    \end{array}\right] \textrm{~(递归分解子矩阵)}\\
    &=&\left[\begin{array}{cc}
        1&0\\
        v/a_{11}&L'
    \end{array}\right]
    \left[\begin{array}{cc}
        a_{11}&w^T\\
        0&U'
    \end{array}\right]\textrm{~(左右矩阵分别为L,U)}\\
    &=&LU
\end{eqnarray*}

求出矩阵的外围部分与子矩阵后将子矩阵递归分解。

\subsubsection{LUP分解实现}
设换主元时把第1行(因为要递归分解)与第$k$行交换的置换矩阵为$Q$,则$QA$
可以进行LU分解,即\begin{displaymath}
    QA=\left[\begin{array}{cc}
        1&0\\
        v/a_{k1}&I_{n-1}
    \end{array}\right]
    \left[\begin{array}{cc}
        a_{k1}&w^T\\
        0&A'-vw^T/a_{k1}
    \end{array}\right]
\end{displaymath}
设子矩阵满足$P'(A'-vw^T/a_{k1})=L'U'$,与$Q$相乘得到置换矩阵$P$,即
\begin{displaymath}
    P=\left[\begin{array}{cc}
        1&0\\
        0&P'
    \end{array}\right]Q
\end{displaymath}
继续变换:
\begin{eqnarray*}
    PA&=&\left[\begin{array}{cc}
        1&0\\
        0&P'
    \end{array}\right]
    \left[\begin{array}{cc}
        1&0\\
        v/a_{k1}&I_{n-1}
    \end{array}\right]
    \left[\begin{array}{cc}
        a_{k1}&w^T\\
        0&A'-vw^T/a_{k1}
    \end{array}\right]\\
    &=&\left[\begin{array}{cc}
        1&0\\
        P'v/a_{k1}&P'
    \end{array}\right]
    \left[\begin{array}{cc}
        a_{k1}&w^T\\
        0&A'-vw^T/a_{k1}
    \end{array}\right]\\
    &=&\left[\begin{array}{cc}
        1&0\\
        P'v/a_{k1}&I_{n-1}
    \end{array}\right]
    \left[\begin{array}{cc}
        a_{k1}&w^T\\
        0&P'(A'-vw^T/a_{k1})
    \end{array}\right]\\
    &=&\left[\begin{array}{cc}
        1&0\\
        P'v/a_{k1}&I_{n-1}
    \end{array}\right]
    \left[\begin{array}{cc}
        a_{k1}&w^T\\
        0&L'U'
    \end{array}\right]\\
    &=&\left[\begin{array}{cc}
        1&0\\
        P'v/a_{k1}&L'
    \end{array}\right]
    \left[\begin{array}{cc}
        a_{k1}&w^T\\
        0&U'
    \end{array}\right]\\
    &=&LU
\end{eqnarray*}

理解性记忆:按照《线性代数及其应用》的描述,LU分解的目的就是使用一系列行倍加变换把
$A$化为阶梯形$U$,同时构造单位下三角矩阵$L$使得对$L$施加相同的行变换后变为$I$。
记行变换矩阵为$P$,则$PL=I$且$PA=U$,由$L$可逆可得$LP=I$,继而得到$A=LU$。
行倍加变换的策略就是把每次把当前主元位置下方的元素消为0,由于$L$对应的主元位置为1,
该主元位置下方对应的就是行倍加的系数。

实际操作非常简单:选取非0主元置换到对角线上,然后令$l_{ji}=a_{ji}/a_{ii}$,
同时将其作为行倍加系数进行行倍加变换。{\bfseries 注意置换时要整行置换。}

\begin{lstlisting}
typedef double FT;
const FT eps=1e-8;
FT A[size][size],B[size];
bool LUP(int n){
    for(int i=1;i<=n;++i) {
        int x=i;
        for(int j=i+1;j<=n;++j)
            if(fabs(A[j][i])>fabs(A[x][i]))
                x=j;
        if(fabs(A[x][i])<eps)return false;
        if(i!=x){
            for(int j=1;j<=n;++j)
                std::swap(A[i][j],A[x][j]);
            std::swap(B[i],B[x]);
        }
        for(int j=i+1;j<=n;++j) {
            A[j][i]/=A[i][i];
            FT fac=A[j][i];
            for(int k=i+1;k<=n;++k)
                A[j][k]-=fac*A[i][k];
        }
    }
    return true;
}
\end{lstlisting}
{\bfseries 注意L与U同时覆盖于A数组上,即}
\begin{displaymath}
	a_{ij}=\left\{\begin{array}{ll}
		l_{ij} & \textrm{if $i>j$}     \\
		u_{ij} & \textrm{if $i\leq j$}
	\end{array}\right.
\end{displaymath}
\subsection{正向/反向替换}
原理同~\ref{LSE}节,代码如下:
\begin{lstlisting}
FT Y[size],X[size];
void solve(int n) {
    for(int i=1;i<=n;++i) {
        FT sum=B[i];
        for(int j=1;j<i;++j)
            sum-=A[i][j]*Y[j];
        Y[j]=sum;
    }
    for(int i=n;i>=1;--i) {
        FT sum=Y[i];
        for(int j=i+1;j<=n;++j)
            sum-=A[i][j]*X[j];
        X[i]=sum/A[i][i];
    }
}
\end{lstlisting}
\subsection{LUP分解求逆矩阵}
同~\ref{InvMatGauss}节所述,求解$n$次线性方程组得到逆矩阵。LUP分解的好处
在于只需维护置换矩阵$P$而不是维护变换矩阵$PI$(这两个矩阵的意义不同)。
\lstinputlisting[title=InvMatLUP]{Source/Templates/LUP.cpp}

以上内容参考了算法导论\cite{ITA3} 第28章 矩阵运算。

\section{行列式}
\subsection{定义与性质}
记$A_{[ij]}$为矩阵$A$去掉第$i$行第$j$列后的矩阵,$\tau(P)$为序列$P$的逆序对数
,$P=(p_1,p_2,\cdots,p_n)$为$1\ldots n$的排列。
那么$n*n$矩阵$A$的行列式定义为:
\begin{eqnarray}
	det(A)&=&\sum_{P=(p_1,p_2,\cdots,p_n)}
	{(-1)^{\tau(P)}\prod_{i=1}^n{a_{ip_i}}}\label{DetA}\\
	&=&\left\{
	\begin{array}{ll}
		a_{11}                                                    & \textrm{~if~$n=1$} \\
		\displaystyle \sum_{j=1}^n{(-1)^{1+j}a_{1j}det(A_{[1j]})} & \textrm{~if~$n>1$}
	\end{array}
	\right.\label{DetB}
\end{eqnarray}

式~\ref{DetA}可理解为选择$n$个不同行且不同列的元素,逆序对数为偶数的加,奇数的减。

式~\ref{DetB}是式~\ref{DetA}的递归形式。

记$A_{ij}=(-1)^{i+j}det(A_{[ij]})$为元素$a_{ij}$的{\bfseries 代数余子式}。

行列式拥有如下性质:
\begin{property}
	$det(A)=det(A^T)$
\end{property}

证明:根据式~\ref{DetA}可知,行和列是无关的,转置后$det(A)$不变。
\begin{property}
	若矩阵$A$的某一行/列为0,则$det(A)=0$。
\end{property}

证明:根据式~\ref{DetA}可知每个排列中矩阵元素必有一个0,所以$det(A)=0$。
\begin{property}
	矩阵$A$的任意一行/列乘以$\lambda$,$det(A)$乘以$\lambda$。
\end{property}

证明:根据式~\ref{DetA}可知每个排列中有一个矩阵元素乘以$\lambda$,则$det(A)$放大$\lambda$。
\begin{property}
    若矩阵$C$的第$i$行/列可分解为$c=a+b$的形式,
    则$det(C)$可分解为分别包含这两个向量,其余元素与矩阵$C$相同的矩阵的行列式之和。
\end{property}
\begin{inference}
	将矩阵$A$中某一行/列元素加$\lambda$倍到另一行/列上,$det(A)$不变。
\end{inference}

将$det(A')$拆分,有$det(A')=det(A)+det(A_{add})=det(A)$(因为$A_{add}$有线性相关向量)。
\begin{property}
	交换$A$任意两行/列,$det(A)$变号。
\end{property}

证明:交换会使逆序对数奇偶性改变(通过计算相邻交换步数可证明),$det(A)$变号。
\begin{property}
	det(AB)=det(A)det(B)
\end{property}
\begin{theorem}
	$det(A)=0\Leftrightarrow$矩阵$A$奇异
\end{theorem}

证明:$det(A)=0$意味着至少有2个系数向量是线性相关的,矩阵$A$是欠定方程组的系数矩阵,没有逆矩阵。

代数余子式有如下性质:
\begin{property}
    \begin{eqnarray*}
        det(A)&=&\sum_{j=1}^n{a_{ij}A_{ij}} ~i=1,2,\cdots,n\\
        &=&\sum_{i=1}^n{a_{ij}A_{ij}} ~j=1,2,\cdots,n\\
    \end{eqnarray*}
\end{property}
\begin{property}
    \begin{displaymath}
        \forall i!=j,\sum_{k=1}^n{a_{ik}A_{jk}}=0
    \end{displaymath}
\end{property}
\subsection{求行列式}
\begin{theorem}\label{LTAD}
	若矩阵$A$为上三角矩阵,则$det(A)$为主对角线上元素之积。
\end{theorem}

证明:唯一矩阵元素积可能非0的排列方案只有在主对角线上。

高斯消元后根据定理~\ref{LTAD}计算行列式,注意在初等变换操作中交换行时要变号
{\bfseries (尤其是在计算模意义下有向生成树计数时)}。

以上内容参考了算法导论~\cite{ITA3} 附录D 矩阵。

若矩阵比较特殊,有较多的非零项,可以选取0的个数最多的一行/列展开。

\section{基础设施}
以下内容主要讨论二维空间中的计算几何。
\subsection{点,向量,直线,半平面的表示}
点与向量由2个坐标表示;半平面和直线由直线上一点$ori$与
直线的方向向量$dir$表示;直线上的一点可表示为$ori+dir*t$,通过控制$t$的取值还可以
表示射线或线段;
可以人为规定半平面的顺时针/逆时针180$\circ$为半平面所在点集,
通过叉积来判断点在半平面的哪一边。
\begin{lstlisting}
typedef double FT;
struct Vec {
    FT x,y;
    //constructor
    //operator+-*
};
struct Line {
    Vec ori,dir;
    Vec operator()(FT t) const {
        return ori+dir*t;
    }
};
\end{lstlisting}
\subsection{点乘与叉乘}
向量点乘$dot(a,b)=a.x*b.x+a.y*b.y=|a||b|cos<a,b>$,一般用来
判断与法向量的夹角,以及辅助叉乘计算两向量夹角。点乘满足加法分配律和交换律。

向量叉乘$cross(a,b)=a.x*b.y-b.x*a.y=|a||b|sin<a,b>$,这是两向量
构成的平行四边形的有向面积。一般用来判断向量的相对方向以及计算多边形的面积。
叉乘满足$cross(a,b)=-cross(b,a)$和加法分配律。

\begin{theorem}[拉格朗日公式]
	cross(a,cross(b,c))=b*dot(a,c)-c*dot(a,b)
\end{theorem}

三维向量的叉乘计算了垂直于这两个向量的向量(两向量组成平面的法向量),即
\begin{displaymath}
	cross(a,b)=\left(\begin{array}{c}
		a.y*b.z-b.y*a.z \\
		a.z*b.x-b.z*a.x \\
		a.x*b.y-b.x*a.y
	\end{array}\right)
\end{displaymath}
其方向满足右手定则(右手四指与大拇指垂直,食指指向向量$a$,其余三指指向向量$b$,
大拇指方向即为叉乘方向),长度满足
$|cross(a,b)|=|a||b|sin<a,b>$。

结合点乘和叉乘可得点$A$与点$B,C,D$所组成的三棱锥$A-BCD$的有向体积为
\begin{displaymath}
	V=|\frac{1}{6}dot(\overrightarrow{BA},cross(\overrightarrow{BC},
	\overrightarrow{BD}))|
\end{displaymath}
\subsection{点到直线的距离}
设偏移向量$delta=p-ori$:
\begin{itemize}
	\item 计算$dot(delta,dir)/|dir|$可得到投影长度$d'$,根据
	      勾股定理得到$d^2=|delta|^2-d'^2$。
	\item 计算$|cross(delta,dir)|$可得偏移向量与方向向量构成的平行四边形的面积,
	      根据面积公式得到$d=|\frac{cross(delta,dir)}{|dir|}|$。
\end{itemize}
叉乘法的运算量少且精度较高,总是被选用。
\subsection{直线、线段的交点}
先上代码:
\begin{lstlisting}
Vec intersect(const Line& a,const Line& b) {
    Vec delta=a.ori-b.ori;
    FT t=cross(b.dir,delta)/cross(a.dir,b.dir);
    return a.ori+a.dir*t;
}
\end{lstlisting}
\paragraph{证明} 首先可以发现两直线的方向向量都已经被单位化了。
然后通过两个直角三角形中边与$sin$的关系可证。

线段相交也是如此,但首先要判断两线段是否相交。将该问题转换为
两线段是否互相平分。设线段为$a-b,c-d$,首先判断$a-b$平分$c-d$,
即$c,d$分别位于$a-b$两边,有$cross(c-a,b-a)*cross(d-a,b-a)\leq 0$,
同理对$c-d$做一遍即可。
\subsection{判定点是否在多边形内}
\subsubsection{随机射线法}
从点P开始随机引出一条射线,计算其与多边形的边的交点个数,若为奇数次则
在多边形内。注意射线恰好经过点时要重新选择方向。
\subsubsection{旋转角法}
从点P与多边形的一个点开始不断旋转到下一个点,直至转完一圈为止。
此时若点P旋转了0$\circ$,则在多边形外;若点$P$旋转了360$\circ$,
则在多边形内。旋转角可以使用$atan2(cross(a,b),dot(a,b))$计算,以
$\pi(180\circ)$为界进行比较。
\subsubsection{半平面法}
若该多边形为凸多边形,以每条边构造半平面,使用叉积判断是否在半平面内,
若点在所有半平面内则在多边形内。
\subsection{向量的旋转}
根据复数乘法的规律:模长相乘,幅角相加。构造逆时针旋转$\theta$单位旋转向量
$e^\theta=\cos \theta+\sin \theta i$,将原向量乘以该向量即可得到旋转后的向量:
$(x\cos \theta-y\sin \theta,x\sin \theta+y\cos \theta)$。
\subsection{坐标系的切换}
在同一$d$维坐标空间中的点$P$与单位正交向量$b_1,b_2,\ldots,b_d$,以
该向量组为新坐标空间的基向量,那么点$P$在新坐标空间的坐标值为它在
这些向量上的投影。

在三维空间中,新坐标空间的$+x$轴为$T$,$+y$轴为$B$,$+z$轴为$N$,将
$TBN$组成一个矩阵(从上到下)$A$,则$P'=AP$。因为$TBN$为单位正交向量,
所以有$AA^T=I$,可得$P=A^TP$。
\subsection{点、向量、法向量的坐标变换}
可以使用一个矩阵$R^{d*d}$来表示$d$维空间中的旋转,缩放,坐标系切换,
引入齐次坐标(即给向量再加一维,非透视投影时恒为1)可支持平移(仅对于点的变换有效)。

在三维空间下使用$4*4$的矩阵来表示对坐标的变换。

\subsubsection{平移}
将坐标平移$(x,y,z)$:
\begin{displaymath}
	\left(\begin{array}{cccc}
		1 & 0 & 0 & x \\
		0 & 1 & 0 & y \\
		0 & 0 & 1 & z \\
		0 & 0 & 0 & 1 \\
	\end{array}\right)
\end{displaymath}
\subsubsection{旋转}
以基于$z$轴旋转$\theta$为例:

思路是对$x,y$轴坐标进行二维旋转,$z$轴坐标不变。
\begin{displaymath}
	\left(\begin{array}{cccc}
		\cos \theta & -\sin \theta & 0 & 0 \\
		\sin \theta & \cos \theta  & 0 & 0 \\
		0           & 0            & 1 & 0 \\
		0           & 0            & 0 & 1 \\
	\end{array}\right)
\end{displaymath}
\subsubsection{缩放}
对坐标分别缩放$x,y,z$:
\begin{displaymath}
    \left(\begin{array}{cccc}
        x & 0 & 0 & 0 \\
        0 & y & 0 & 0 \\
        0 & 0 & z & 0 \\
        0 & 0 & 0 & 1 \\
    \end{array}\right)
\end{displaymath}
\subsubsection{向量变换}
注意平移不会影响向量的变换,因此将矩阵截断为3*3矩阵$T'=mat3(T)$。
\subsubsection{法向量变换}
若法向量$N$变换按照向量变换计算,若遇到缩放则会发生变换后不垂直于变换后平面
的情况。因为缩放矩阵的基向量不是单位向量。考虑一个与该法向量垂直的向量
$V$,满足$N^T\cdot V=0$。那么对于变换后的两向量仍然
要保持垂直,即$N^TT'^{-1} \cdot T'V=0$,对左边进行转置得到$N'=(T'^{-1})^TN$。
\subsection{反射与折射}
以下的入射向量与出射向量均为单位向量。
\subsubsection{反射}
根据反射定律,反射向量由水平方向的向量$I_x$减去垂直方向的向量$I_y$。
若法向量$N$为单位向量,有
\begin{eqnarray*}
    I&=&I_x+I_y\\
    I_y&=&N\cdot dot(I,N)\\
    O&=&I_x-I_y\\
\end{eqnarray*}
解得$O=I-2N\cdot dot(I,N)$。
\subsubsection{折射}
斯涅尔定律描述了折射率与角度的关系:
\begin{theorem}
    $\eta_1\sin \theta_1=\eta_2\sin \theta_2$
\end{theorem}
同样以法向量和切向量为基向量进行正交分解,
记$\eta=\frac{\eta_1}{\eta_2}$,有
\begin{eqnarray*}
    I&=&I_x+I_y\\
    I_y&=&N\cdot dot(I,N)\\
    I_x&=&\sin \theta_1T\\
    O&=&O_x+O_y=\sin \theta_2T-\cos \theta_2N
\end{eqnarray*}
化简:
\begin{eqnarray*}
    O&=&\sin \theta_2T-\cos \theta_2N\\
    &=&\frac{\sin \theta_2}{\sin \theta_1}I_x - \cos \theta_2 N\\
    &=&\frac{\eta_1}{\eta_2}(I-dot(I,N)\cdot N)-\cos \theta_2 N\\
    &=&\eta\cdot I-(\eta dot(I,N)+\cos \theta_2)N
\end{eqnarray*}

代码如下,注意全反射的情况(即$\sin \theta_2$超出值域):
\begin{lstlisting}
Vec refract(Vec I,Vec N,FT eta) {
    FT idn=dot(I,N);
    FT cosO2=1.0-eta*eta*(1-idn*idn);
    if(cosO2<0.0)return Vec();
    FT k=eta*idn+sqrt(cosO2);
    return I*eta-k*N;
}
\end{lstlisting}
上述内容参考了Milo Yip的文章\footnote{
    用 C 语言画光(五):折射
    \url{https://zhuanlan.zhihu.com/p/31127076}
}和glm库的代码\footnote{
    glm/func\_geometric.inl at master · g-truc/glm · GitHub
    \url{https://github.com/g-truc/glm/blob/master/glm/detail/func\_geometric.inl}
}。
\subsection{pick定理}
\index{P!Pick's Theorem}
\begin{theorem}[Pick's Theorem]
    若格点多边形内的点数为$a$,落在边上的点数为$b$,则
    该多边形的面积为$a-\frac{b}{2}+1$。
\end{theorem}
\subsection{切比雪夫距离}
切比雪夫距离是两点坐标之差的最小值。
分别考虑到原点曼哈顿距离和切比雪夫距离为1的点集$P,Q$,
发现将$P$绕原点45$\circ$后缩放$\sqrt{2}$倍后等于$Q$。

因此$P(x,y)\rightarrow Q(x+y,x-y)$,$Q(x,y)
\rightarrow P(\frac{x+y}{2},\frac{x-y}{2})$。
\subsection{精度处理}
一般引入$eps=1e-8$来避免精度问题。

常见问题:
\begin{itemize}
    \item 判断两个值相等:$fabs(a-b)<eps$。
    \item 要输出$1.00$却输出$0.99$或者要输出$0.0$却输出$-0.0$:
        对正值$+eps$,负值$-eps$。
    \item 已知$\sin \theta$求$\cos \theta$:$sqrt(1.0-cosTheta*cosTheta)$
    可能会因为传入$sqrt$的参数小于0而返回$nan$,在调用数学函数前要clamp到定义域内
    或特判。
    \item 若表示的范围越界,则可以使用对数之和表示数。
    \item 技巧:若要维护$set<FT>$,在比较器中引入$eps$自动完成去重工作。
\end{itemize}
该内容参考了Ac\_smile的博客\footnote{计算几何中的精度问题\\
    \url{https://www.cnblogs.com/acsmile/archive/2011/05/09/2040918.html}
}。

\section{最小二乘逼近}
\index{L!Least Squares}
该方法用来确定基函数的系数以拟合曲线。

设有$m$个数据点$(x_1,y_1),(x_2,y_2),\cdots,(x_m,y_m)$,
$n$个基函数$f_1,f_2,\cdots,f_n$
记向量$y=(y_1,y_2.\cdots,y_m)$,基函数值矩阵$A$
为\begin{displaymath}
    A=\left[
    \begin{array}{cccc}
    f_1(x_1)&f_2(x_1)&\cdots&f_n(x_1)\\
    f_1(x_2)&f_2(x_2)&\cdots&f_n(x_2)\\
    \vdots&\vdots&\ddots&\vdots\\
    f_1(x_m)&f_2(x_m)&\cdots&f_n(x_m)
    \end{array}
    \right]
\end{displaymath}
设系数向量为$c$,则有误差向量$\eta=Ac-y$。

使用最小二乘的思想,令$\|\eta\|^2$最小,
即最小化
\begin{displaymath}
    \|Ac-y\|=\sum_{i=1}^m{\left(\sum_{j=1}^n{a_{ij}c_j}-y_i\right)}^2
\end{displaymath}

对向量$c$中的每个元素微分并让结果为0,
即对于单个元素$c_k$,有
\begin{displaymath}
    \frac{\ud \|\eta\|^2}{\ud c_k}=\sum_{i=1}^m{2\left(
        \sum_{j=1}^n{a_{ij}c_j-y_i}\right)a_{ik}}=0
\end{displaymath}
将$n$个等式结合在一起,发现这是个向量*矩阵的形式,可得新的矩阵方程$(Ac-y)^TA=0$,
记$A^+=(A^TA)^{-1}A^T$为矩阵$A$的伪逆,有$c=A^+y$。

以上内容参考了算法导论\cite{ITA3} 28.3节。


\appendix
\chapter{资料推荐}
\section{个人博客}
\begin{itemize}
	\item Miskcoo \url{http://blog.miskcoo.com/} FFT与数论
	\item hzwer \url{http://hzwer.com/} 大量题解
    \item skywalkert \url{https://blog.csdn.net/skywalkert} 数论
    \item Candy? \url{https://www.cnblogs.com/candy99}
    \item 租酥雨 \url{http://www.cnblogs.com/zhoushuyu/}
    \item VFleaKing \url{http://vfleaking.blog.uoj.ac/}
    \item Picks \url{http://picks.logdown.com/} FFT
    \item MoebiusMeow \url{https://www.cnblogs.com/meowww}
    \item ACMaker \url{https://blog.csdn.net/ACMaker} 旋转卡壳
    \item ACdreamer \url{https://blog.csdn.net/ACdreamers}
    \item 小蒟蒻yyb \url{https://www.cnblogs.com/cjyyb/}
    \item fjzzq2002 \url{https://www.cnblogs.com/zzqsblog}
\end{itemize}
\section{好用的网站/工具}
\begin{itemize}
    \item Wikipedia-EN \url{https://en.wikipedia.org} 比百度不知道高到哪里去了
    \item cppreference \url{https://en.cppreference.com/w/} 查询C++知识必备
    \item OEIS \url{http://oeis.org/} 整数数列题找规律必备
    \item WolframAlpha \url{http://www.wolframalpha.com/} 高级计算器
    \item GeoGebra \url{https://www.geogebra.org/} 优秀的数学软件
    \item 3Blue1Brown \url{http://www.3blue1brown.com/} 脑洞有趣易懂的数学教程
    \item Graphviz \url{http://www.graphviz.org/} 图形可视化工具
    \item Project~Euler \url{https://projecteuler.net/} 数学题集
    \item BZOJ离线题库(BZOJCH)[By 阮行止] \\\url{http://ruanx.pw/bzojch/index.html}
    \item BZOJ题号查找器(BZOJNO)[By 阮行止]
    \url{http://ruanx.pw/bzojch/bzojno.html}
\end{itemize}
\section{优秀资料}
\begin{itemize}
    \item CodeForces优秀文章 \url{http://codeforces.com/blog/entry/57282}
    \item E-Maxx算法合集英文版 \url{https://cp-algorithms.com/}
    \item OI-Wiki \url{https://github.com/24OI/OI-wiki/}
    \item 知乎上OI相关文章 \url{https://www.zhihu.com/collection/213577780}
    \item 演算法筆記 江任捷\\\url{http://www.csie.ntnu.edu.tw/\~u91029/index.html}
\end{itemize}
\section{经典书籍}
\begin{itemize}
    \item 《算法导论(原书第3版)》 ISBN: 9787111407010
    \item 《线性代数及其应用(原书第5版)》 ISBN: 9787111602576
\end{itemize}
\section{其它}
\begin{itemize}
    \item THUSC2018wys发言稿 \\\url{https://wys.life/2018-06-THUSC-speech.html}
    \item 《中国面壁者》 中国青年报\\
    {\footnotesize \url{http://zqb.cyol.com/html/2018-04/17/nw.D110000zgqnb\_20180417\_1-04.htm}}
    \item 膜拜kczno1,代码常数超小
\end{itemize}

\printindex
\begin{thebibliography}{999}
	\bibitem{NFTGC} Even, Shimon; Tarjan, R. Endre (1975).
	"Network Flow and Testing Graph Connectivity".
    \emph{SIAM Journal on Computing}.
    \bibitem{DSNA} Tarjan, R. E. (1983).
    \emph{Data structures and network algorithms.}
    \bibitem{MCIOI}胡伯涛 《最小割模型在信息学竞赛中的应用》
    \bibitem{ITA3}  Thomas H.Cormen / Charles E.Leiserson /
     Ronald L.Rivest / Clifford Stein.
     \emph{Introduction to Algorithms Third Edition}
    \bibitem{kdTree}n+e 《K-D Tree 在信息学竞赛中的应用》
    \bibitem{huffman}M. J. Golin and G. Rote.
    \emph{A dynamic programming algorithm for constructing optimal
    prefix-free codes with unequal letter costs}
    \bibitem{LLH}L. L. Larmore and D. S. Hirschberg.
    \emph{A fast algorithm for optimal length-limited Huffman codes}
    \bibitem{chord} 陈丹琦《弦图与区间图》
    \bibitem{MMG}S. Micali and V. V. Vazirani,
    "An O($\sqrt{|v|}$|E|) algoithm for finding maximum matching
    in general graphs",
     21st Annual Symposium on Foundations of Computer Science (sfcs 1980)(SFCS),
     Syracuse, NY, USA USA, , pp. 17-27.
    doi:10.1109/SFCS.1980.12
    \bibitem{AGM} Anant Jindal and Gazal Kochar and Manjish Pal,
    "Maximum Matchings via Glauber Dynamics",CoRR
    \bibitem{GBT} Yang Zhe 《SPOJ375 QTREE解法的一些研究》
\end{thebibliography}

\backmatter
\chapter{后记}
\end{document}
